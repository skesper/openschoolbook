

\chapter{Differentialrechnung}\label{chap:diff}

TODO


\section{Überlegungen}

Das langfristige Ziel dieses Kapitels ist es, zu einer Funktion $f$, die noch zu bestimmenden Anforderungen genügt, eine Funktion $f'$ zu bestimmen, die an jeder Stelle den Wert der Steigung von $f$ annimmt. 

Es sei im folgenden $f : \mathbb{R} \longrightarrow \mathbb{R}$ immer eine Funktion. Wir betrachten eine Stelle $a\in \mathbb{R}$ und geben ein $0<\epsilon < 1$ vor. Die Sehne von $f$ zwischen Punkt $a$ und $a+\epsilon$ ist
\[
S = \frac{f(a+\epsilon)-f(a)}{\epsilon}
\]
Wir suchen nun eine Funktion, die zumindest an der Stelle $a$ den Funktionswert $S$ annimmt. Die einfachsten Funktionen, die wir kennen, sind die linearen Funktionen.



\section{Stetigkeit}



\chapter{Integralrechnung}


