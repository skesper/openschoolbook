

\chapter{Differentialrechnung}\label{chap:diff}

Es gibt sehr viele -- in erster Linie mathematische -- Gründe, warum die Steigung einer Funktion eine interessante und wissenswerte Information darstellt. Doch sehen wir uns zuerst eine Situation an, die in unserem täglichen Leben vorkommt.

Der Tacho eines Autos ist etwas, dem wir ständig begegnen. Sei es als Autofahrer oder als Beifahrer, oder gar als Fahrradfahrer. Der Tacho ist ein Messinstrument und gibt die aktuelle Geschwindigkeit des Fahrzeugs an. Geschwindigkeit ist ein physikalischer Begriff. Physiker verwenden Mathematik zur Beschreibung ihrer Beobachtungen und zur Entwicklung ihrer Theorien. Sie haben zum Beispiel herausgefunden, dass wenn man den Weg, den ein Fahrzeug zurücklegt, durch die Zeit teilt, die es dafür brauchte, erhält man ein Maß für die Geschwindigkeit, bzw. ein Maß für die Änderung der zurückgelegten Strecke. Diese recht grobe Angabe stellt nur die Durchschnittsgeschwindigkeit dar. Aber man ist natürlich auch an der Geschwindigkeit interessiert, die das Fahrzeug zu jedem Zeitpunkt inne hat. Teilen wir die Strecke in $n$ gleich große Teile und messen die Zeit, die das Fahrzeug für jeden einzelnen dieser Abschnitte brauchte. Dann bekommen wir die Durchschnittsgeschwindigkeit auf jedem dieser Bereiche. Machen wir nun die Bereiche, und damit auch die Zeitabschnitte, immer kleiner ($n \rightarrow \infty$), dann nähert sich der Wert, den wir in den einzelnen Bereichen messen, immer mehr der zu diesem Zeitpunkt gefahrenen Geschwindigkeit. 

Daraus folgt ein uns völlig bekannter Zusammenhang: Nämlich, dass der Fahrer über die Geschwindigkeit, direkte Kontrolle über die zurückgelegten Strecke besitzt. Fährt er schneller, so legt er in gleicher Zeit mehr Strecke zurück. 

Eine ähnlich bekannte Situation betrifft die Beschleunigung. Werden wir in die Sitze des Autos gedrückt, so gibt der Fahrer Gas und das Fahrzeug fährt schneller. Werden wir in den Gurt gedrückt, bremst der Fahrer und das Fahrzeug wird langsamer. Die Beschleunigung ist also ein Maß für die Änderung der Geschwindigkeit. Und zwar in genau der selben Art, wie die Geschwindigkeit ein Maß für die Änderung der zurückgelegten Strecke ist:

\begin{fancyquotes}
Die Beschleunigung verhält sich zur Geschwindigkeit, wie die Geschwindigkeit zur zurückgelegten Strecke. 
\end{fancyquotes}

\section{Definitionen}

\subsection{Dehnungsbeschränktheit}

\begin{definition}\label{def:lipschitz}
Eine Funktion $f : A \subseteq \mathbb{R} \longrightarrow \mathbb{R}$ heißt \textsl{dehnungsbeschränkt}, wenn es zu beliebigen Werten $x,y\in A$ eine Konstante $K$ gibt, so dass
\[
\left\vert f(x)-f(y)  \right\vert \le K\cdot \vert x-y \vert
\]
oder auch in dieser Form dargestellt:
\[
\left\vert \frac{f(x)-f(y)}{x-y}  \right\vert \le K
\]
Diese Definition ist auch als \textsl{Lipschitz-Stetigkeit}\footnote{benannt nach \textbf{Rudolf Otto Sigismund Lipschitz}, *14. Mai 1832 in Königsberg i. Pr.; \ding{61}7. Oktober 1903 in Bonn.} bekannt.
\end{definition}


\section{Überlegungen}


\subsection{Kritik an Zeichnungen}

Im Folgenden werden wir öfters anschaulich argumentieren, gegebenenfalls unterstützt durch eine Zeichnung. Solche Zeichnungen benötigen etwas Interpretation. Die Funktionen, die wir in diesem Teil untersuchen werden, sofern nicht ausdrücklich anders angegeben, sind immer reellwertige Funktionen, also $f:\mathbb{R} \longrightarrow \mathbb{R}$. Ihre Funktionswerte sind eindimensional. Ein Diagramm mit $x$ und $y$-Achse ist aber eine zweidimensionale Darstellung. Was also in diesem Moment gemacht wird ist, wir wandeln die Funktion $f$ in eine \textsl{Kurve} im $\mathbb{R}^2$ um:
\[
k : \mathbb{R} \longrightarrow \mathbb{R}^2
\]
indem wir 
\[
k(f,x) = \begin{pmatrix}
x \\
f(x)
\end{pmatrix}
\]
setzen. $k$ ist eine Vektor-wertige Funktion. Wir argumentieren also auf Basis einer ganz anderen Funktion $k$. Man muss also vorsichtig sein und überprüfen, ob die Argumente nicht nur aufgrund der Darstellung funktionieren, oder ob sie Darstellungsunabhängig sind und wirkliche Informationen über $f$ offenlegen. 

Es ist also immer eine gute Idee, rein formal zu argumentieren. Sobald man sich eine bildliche Vorstellung von einem mathematischen Sachverhalt macht, ist man oft schon in der Situation, dass die Abbildung nicht mehr viel mit der "`mathematischen Realität"' zu tun hat. Dieser Abschnitt soll erneut eine Motivation für die Lernenden sein, auf Basis der Formalismen zu argumentieren und mit Zeichnungen vorsichtig zu sein.

\subsection{Linearisierung}

Unser Begriff von der Steigung ist direkt an das Verständnis gebunden, dass die Steigung einer Funktion an einem gewissen Punkt direkte Information darüber gibt, wie sich die Funktion in der Nähe dieses Punktes verhält. Der Fehler zwischen der Funktion und einer linearen Annäherung sollte also kontrollierbar sein durch den Abstand zum Punkt, an dem wir die Steigung kennen. Wir definieren eine lineare Funktion, die im Punkt $a$ den Funktionswert von $f$  und eine Steigung $s$ besitzt über
\[
l(x) = s\cdot (x-a)+f(a)
\]
Dabei gehen wir davon aus, dass diese Annäherung "`in der Nähe"' von $a$ überhaupt sinnvoll ist. Das heißt, im Folgenden ist immer $|x-a|<1$.

Wir nehmen an, dass $s$ ein "`Kandidat"' für die Ableitung ist. Wir versuchen also $s$ möglichst gut zu wählen. 
Der Abstand von $f$ zu $l$ ist:
\begin{equation}\label{eq:lin}
\begin{split}
|f(x)-l(x)| &= |f(x)-(s(x-a)+f(a))| \\
&= |f(x)-f(a) -s\cdot (x-a)| \\
&\le |f(x)-f(a)| + |s||x-a| \quad \text{mit der Dreiecksungleichung} \\
&\le K\cdot |x-a| + |s||x-a| \quad \text{mit der Lipschitz-Stetigkeit von }f \\
&= \underbrace{(K+|s|)}_{=K_1}\cdot |x-a| \\
&= K_1\cdot |x-a|
\end{split}
\end{equation}

Das ist schon ein recht gutes Ergebnis. Es besagt, dass die Funktion $f$ und die \textbf{lineare} Funktion $l$ sich nicht weiter von einander entfernen, als ein Vielfaches des Abstandes von $|x-a|$. Und das haben wir ohne ernsthaften Aufwand unter Ausnutzung der Lipschitz-Stetigkeit und Einsatz der Dreiecks-Ungleichung erfahren. Auf der anderen Seite besagt Ungleichung (\ref{eq:lin}) aber auch das Folgende, wenn man in Zeile zwei $|x-a|$ ausklammert und auf beiden Seiten kürzt, dann erhält man
\begin{equation}\label{eq:const}
\left| \frac{f(x)-f(a)}{x-a} -s \right| \le K_1
\end{equation}
Da die Steigung $s$ an der Stelle $a$ unser Kandidat für die Ableitung ist, hätten wir das Problem, dass wir nicht garantieren könnten, dass die Sehne sich je an die Steigung annähert, denn mit dieser Abschätzung können wir nur sagen, dass der Abstand kleiner als eine Konstante sein muss. Sich also gegebenenfalls überhaupt nicht annähert.


Betrachten wir die Sehne von $f$
\begin{equation*}
S_f(x,a) = \frac{f(x)-f(a)}{x-a}
\end{equation*}

Es ist nicht verwunderlich, dass wir uns für den Grenzwert $\lim\limits_{x\rightarrow a} S_f(x,a)$ interessieren. Denn dieser sollte die Steigung im Punkt $a$ sein und damit die beste Wahl für die Ableitung von $f$ in $a$. Damit der Grenzwert existiert, brauchen wir an dieser Stelle die Vollständigkeit nach Definition \ref{def:voll}. Nennen wir 
\begin{equation}\label{eq:slim}
\lim\limits_{x\rightarrow a} S_f(x,a) = S_f
\end{equation}
wenn wir fordern, dass $f$ $\epsilon$-$\delta$-stetig ist, liefert und das Lemma \ref{lem:stetig}: 
\[
|S_f(x,a) - S_f| \le C\cdot |x-a|
\]
Das sollte uns nicht überraschen, wir verschärfen damit nur (\ref{eq:const}) und setzen um, dass $|S_f(x,a) - S_f| \rightarrow 0$ für $x\rightarrow a$, was ja durch (\ref{eq:slim}) schon dargelegt wurde.
Dann folgt aus (\ref{eq:lin})
\begin{equation}
\begin{split}
|f(x)-l(x)| &= |f(x)-(S_f(x-a)+f(a))| \\
&= |f(x)-f(a) -S_f\cdot (x-a)| \\
&= \left| \frac{f(x)-f(a)}{x-a} -S_f \right| \cdot |x-a| \\
&= \left| S_f(x,a) -S_f \right| \cdot |x-a| \\
&\le C\cdot |x-a| \cdot |x-a| \\
&= C\cdot |x-a|^2
\end{split}
\end{equation}

\begin{definition}\label{def:diff}
Wir nennen den Grenzwert der Sehnen $S_f$ die \textsl{Ableitung} von $f$ am Punkt $a$ und schreiben diesen als $f'(a)$. Der Sehengrenzwert wird auch als \textsl{Tangente} bezeichnet. 
\end{definition}

\bigskip

Fassen wir zusammen: Durch die Überlegung, dass die Sehnensteigung von $f$ sich unserem Kandidaten für die Ableitung annähern soll, wenn $x$ sich immer weiter an $a$ annähert, gibt uns Lemma \ref{lem:stetig} die Ungleichung
\[
\left| \frac{f(x)-f(a)}{x-a} - f'(a) \right| \le C \cdot |x-a|
\]
Dabei implizieren wir, dass die Annäherung $x\rightarrow a$ mittels einer Folge erreicht wird. Durch Multiplikation von $|x-a|$ auf beiden Seiten erhalten wir:
\[
|f(x)-f(a) -f'(a)\cdot (x-a)| \le C\cdot |x-a|^2
\]
Im Vergleich zu (\ref{eq:lin}) ist dies eine substantielle Verbesserung, denn für $|x-a|<1$ ist $|x-a|^2\ll 1$. Das bedeutet, dass in einer kleinen Umgebung von $a$ der Abstand zwischen der Funktion $f(x)$ und der linearen Annäherung $f'(a)\cdot (x-a)+f(a)$ sehr klein ist.

Nachdem wir auf diese umständliche Art eine Begründung für die Definition \ref{def:diff} formulierten, müssen wir nun noch nachweisen, dass diese auch eindeutig ist. 

\begin{lemma}
Die Ableitung einer Funktion $f$ an der Stelle $a$ ist eindeutig. 
\end{lemma}
\begin{proof}
Es sei vorausgesetzt, dass $s_1$ und $s_2$ verschiedene Steigungen von $f$ im Punkt $a$ sind. So würde in einer kleinen Umgebung von $a$ gelten:
\begin{equation*}
\begin{split}
|f(x)-f(a) -s_1\cdot (x-a)| &\le C_1\cdot |x-a|^2 \\
|f(x)-f(a) -s_2\cdot (x-a)| &\le C_2\cdot |x-a|^2
\end{split}
\end{equation*}
Wir subtrahieren diese Ungleichungen voneinander
\begin{equation}
\begin{split}
|f(x)-f(a) -s_1\cdot (x-a)|-|f(x)-f(a) -s_2\cdot (x-a)| &\le C_1\cdot |x-a|^2-C_2\cdot |x-a|^2 \\
|f(x)-f(a) -s_1\cdot (x-a)-f(x)-f(a) -s_2\cdot (x-a)| &\le (C_1+C_2)\cdot |x-a|^2 \\
|(s_1+s_2)\cdot (x-a)| &\le (C_1+C_2)\cdot |x-a|^2 \\
|s_1+s_2| &\le (C_1+C_2)\cdot |x-a|
\end{split}
\end{equation}
(Da $C_1,C_2 >0$, ist $C_1-C_2$ immer kleiner als $C_1+C_2$). Die rechte Seite wird in der Nähe von $a$ so klein, wie wir nur wollen, während die linke Seite der Ungleichung einen konstanten Wert besitzt. Daher ist die Ungleichung falsch, wenn 
\[
|x-a| < \left| \frac{s_1+s_2}{C_1+C_2} \right|
\]
Was im Widerspruch zur Voraussetzung steht, dass $s_1\ne s_2$, daher müssen diese gleich sein.
\end{proof}

\subsection{Rechenregeln}

Die Ableitung von verknüpften Funktionen unterliegt Rechenregeln, die wir hier nachweisen wollen. 


\chapter{Integralrechnung}


