
\chapter{Differentialgleichungen}\label{chap:diffgl}

Differentialgleichungen sind ein elementares Mittel zur Beschreibung von komplizierten Funktionen. Physiker nutzen Differentialgleichungen zur Beschreibung natürlicher Phänomene. Ihr Verständnis der Vorgänge ist dabei von elementarer Bedeutung.

\section{Ein einführendes Beispiel}

Betrachten wir zum Einstieg die folgende Gleichung:

\begin{equation}
f'(x) = f(x)
\end{equation}
Was sagt uns diese Gleichung? Zunächst einmal nur, dass die Ableitung von $f$ gleich $f$ ist. Aber wir haben keine Funktion $f$ definiert. Also ist diese Gleichung nur im Sinne einer Anforderung zu verstehen. Sprich: Wir \emph{suchen} eine Funktion $f$, deren Ableitung gleich $f$ ist. 

Erinnern wir uns an Abschnitt \ref{chap:diffexp}. Dort haben wir gelernt, dass die Exponentialfunktion invariant unter der Ableitung ist. 
\begin{equation}
\exp'(x) = \exp(x)
\end{equation}
