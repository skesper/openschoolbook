\part{Analysis}


\chapter{Grundlagen}

Der Begriff einer Abbildung ist uns schon öfter untergekommen. Eine Funktion ist eine Abbildung, die zwei Mengen (hier $X$ und $Y$) in Beziehung setzt:

\[ f: X \longrightarrow Y \]
$f$ ordnet jedem Element $x\in X$ ein Element $y\in Y$ zu, indem
\[f(x) =y\]
gilt. Die Umkehrung gibt es allgemein nicht, aber wenn sie existiert, so wird sie als $f^{-1}$ bezeichnet und es gilt:
\[ f^{-1}(y) =x\]


\section{Polynome}

Wir hatten in Kapitel \ref{chap:poly} Polynome bereits kennengelernt. 

\subsection{Ableitung von Polynomen}



\section{Kurvendiskussion}



\chapter{Folgen}


\section{Cauchy Folgen}


\section{Vollständigkeit}



\chapter{Differentiation}



\chapter{Integration}


