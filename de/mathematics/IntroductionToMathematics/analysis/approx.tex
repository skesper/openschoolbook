
\chapter{Approximation}

Eine sehr wichtige Anwendung der Differentialrechnung ist die Approximation von Funktionen. Das bedeutet, wir möchten uns anhand von wenigen Informationen Funktionen definieren, die "`so nah wie möglich"' an unbekannte oder nicht definierbare Funktionen heran kommen. Hierzu ist natürlich zunächst zu definieren, was die "`Nähe"' zwischen zwei Funktionen ausmacht. Gibt es so etwas wie einen Abstand zwischen Funktionen? 

Suggestiverweise kann diese Frage mit "`ja"' beantwortet werden. Doch dazu später mehr. 

\section{Bernsteinpolynome}

\begin{TODO}
\end{TODO}

\section{Taylor-Reihe}

\begin{TODO}
\end{TODO}

\section{Splines}

\begin{TODO}
Entscheidung, ob dies nicht über den Rahmen dieses Buches hinaus geht. Auf der anderen Seite sind Splines (besonders kubische) recht einfach zu berechnen. Daher als Beispiel für eine Anwendung der Differentialrechnung sehr hilfreich.
\end{TODO}