


\chapter{Eigenwerte}

Kommen wir auf die reellen $n\times n$ Matrizen zurück, wobei $n<\infty$ ist. Sie stellen lineare Abbildungen vom $\mathbb{R}^n$ in sich selbst dar. 
\begin{definition}
Die Menge aller invertierbaren $n\times n$-Matrizen bildet, zusammen mit der Matrixmultiplikation als Verknüpfung, eine Gruppe. Sie wird $\mathfrak{gl}_n(\mathbb{R})$ genannt. Die Abkürzung $\mathfrak{gl}$ kommt von der "`general linear group"', der allgemeinen linearen Gruppe, auch oft $GL(n,\mathbb{R})$ genannt. Dass diese Matrizen eine Gruppe bezüglich der Matrixmultiplikation bilden, wird in den Aufgaben nachgewiesen.
\end{definition}


\begin{definition}
Es seien $A\in \mathfrak{gl}_n(\mathbb{R})$ und $v\in \mathbb{R}^n$ ein Vektor. Falls weiter gilt
\[
Av = \lambda v
\]
mit $\lambda\in \mathbb{R}$, so heißt $v$ ein \textsl{Eigenvektor} von $A$. Sowie $\lambda$ ein \textsl{Eigenwert} zum Eigenvektor $v$.
\end{definition}

Diese Definition mag überraschend aussehen, denn schließlich ist $A$ ein vermeintlich "`kompliziertes"' und hochdimensionales Objekt, während auf der rechten Seite nur die Multiplikation einer Zahl mit einem Vektor steht, also eine Streckung des Vektors. Doch letztlich ist $A$ eine lineare Abbildung. Das bedeutet, $A$ bildet Vektoren im $\mathbb{R}^n$ wieder auf Vektoren ab. Dies kann im Grunde nur durch Drehungen, Verzerrungen und Spiegelungen geschehen. Erst nicht-lineare Abbildungen haben die Möglichkeit komplexeres Verhalten an den Tag zu legen. 

\begin{definition}
Seien $A,v,\lambda$ wie oben, des Weiteren sei $x\in \mathbb{R}$ ein Parameter und $I_n$ das neutrale Element der Multiplikation des $\mathfrak{gl}_n(\mathbb{R})$. Dann ist 
\[
p(x) = \det(A-x\cdot I_n) : \mathbb{R} \longrightarrow \mathbb{R}
\]
ein normiertes Polynom vom Grade $n$. Es wird \textsl{charakteristisches Polynom} genannt.
\end{definition}

\begin{theorem}
Die Nullstellen des charakteristischen Polynoms zur Matrix $A\in \mathfrak{gl}_n(\mathbb{R})$ sind die Eigenwerte von $A$.
\end{theorem}
\begin{proof}
TODO
\end{proof}

\section{Aufgaben}
TODO

\chapter{Numerische Lösungsverfahren}

\section{Gauß Elemination}

\section{QR Zerlegung}

\section{Iterative Lösungsmethoden}




