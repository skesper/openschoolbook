

\chapter{Exponentialrechnung und Polynome}


\section{Exponentialrechnung}

\subsection{Binomische Formeln}

Die Binomischen Formeln sind elementare Umformungen zweier Summanden zum Quadrat. Dabei gibt es die folgenden Varianten:

\begin{eqnarray}
(a+b)^2 &=& a^2 +2ab +b^2 \label{eq:binom1} \\
(a-b)^2 &=& a^2 -2ab +b^2 \label{eq:binom2} \\
(a+b)(a-b) &=& a^2 -b^2 \label{eq:binom3}
\end{eqnarray}

Interessant ist es, die Summe $(a+b)$ mit natürlichen Exponenten zu untersuchen. Dabei behalten wir auch Faktoren vom Wert 1 bei, um das darauf Folgende klarer darzustellen:

\begin{equation} \label{eq:bino}
\begin{split}
(a+b)^0 &= 1 \\
(a+b)^1 &= 1\cdot a+1\cdot b \\
(a+b)^2 &= 1\cdot a^2 +2\cdot ab +1\cdot b^2 \\
(a+b)^3 &= 1\cdot a^3 + 3\cdot a^2b + 3\cdot ab^2 +1\cdot b^3 \\
(a+b)^4 &= 1\cdot a^4 + 4\cdot a^3b + 6\cdot a^2b^2 +4\cdot ab^3 + 1\cdot b^4
\end{split}
\end{equation}


\begin{definition}
Schreiben wir die Faktoren der einzelnen Summanden in einer dreieckigen Struktur auf, wobei $n$ den Exponent der Summe darstellt:

\begin{center}
\begin{tabular}{rccccccccc} 
$n=0$:& & & & & 1\\
\noalign{\smallskip} $n=1$:& & & & 1 & & 1\\
\noalign{\smallskip} $n=2$:& & & 1 & & 2 & & 1\\
\noalign{\smallskip} $n=3$:& & 1 & & 3 & & 3 & & 1\\
\noalign{\smallskip} $n=4$:& 1 & & 4 & & 6 & & 4 & & 1
\end{tabular}
\end{center}
Wie wir sehen, besteht jede Zeile dieser Tabelle aus den Faktoren der einzelnen Summanden der Gleichungen in (\ref{eq:bino}). Desweiteren erkennen wir, dass der erste und letzte Faktor jeder Zeile immer 1 ist und alle anderen Faktoren sich als Summe der darüber befindlichen Faktoren bilden lässt, also $2=1+1$, $3=1+2$, $6=3+3$, usw.
\end{definition}

Dieses Dreieck kann für beliebige $n\in \mathbb{N}_0$ berechnet werden und stellt immer die Faktoren der Binomischen Formel $(a+b)^n$ dar. Es wird als meist \emph{Pascal'sches Dreieck}\index{Dreieck, Pascal} genannt. In China wird es als \emph{Yáng-Hu\=\i-Dreieck}\footnote{Nach \textbf{Yáng Hu\=\i}, *um 1238 in Hangzhou, Zhejiang; \ding{61} um 1298}\index{Dreieck, Yang-Hui} bezeichnet. In Deutschland ist die Bezeichnung nach Pascal\index{Pascal, Blaise}\footnote{\textbf{Blaise Pascal}, französischen Mathematiker *19. Juni 1623 in Clermont-Ferrand, \ding{61}19. August 1662 in Paris.} üblich.

\begin{definition}
Die \emph{Fakultät} einer natürlichen Zahl $n$ wird mit $n!$ bezeichnet und berechnet sich
\[
n! = 1\cdot 2\cdot 3 \dotsm n
\]
Es gilt per Definition $0! =1$.
\end{definition}

\begin{definition}
Sei der \emph{Binomialkoeffizient} definiert als 
\[
	\binom{n}{k} = \frac{n!}{k!\cdot (n-k)!}
\]
Er spricht sich "`$n$ über $k$"' aus.
\end{definition}

Des Weiteren gilt 
\[
\binom{n}{0} = \binom{n}{n} = 1
\]
Dann kann die allgemeine Binomische Formel geschrieben werden als:

\begin{equation}
(a+b)^n = \sum_{k=0}^{n} \binom{n}{k} \cdot a^{n-k}\cdot b^k
\end{equation}

\begin{claim}
Dass ein Koeffizient im Pascal'schen Dreieck durch die Summe der darüber befindlichen Koeffizienten gebildet wird, ist konsistent mit der Definition des Binomialkoeffizienten, denn es gilt:
\[
\binom{n+1}{k+1} = \binom{n}{k} + \binom{n}{k+1}
\]
\end{claim}

\begin{proof}
\begin{eqnarray*}
\binom{n+1}{k+1} &=& \frac{(n+1)!}{(k+1)!(n-k)!} = \frac{n!(n+1)}{(k+1)!(n-k)!} = \frac{n!(n+1+k-k)}{(k+1)!(n-k)!} \\
&=& \frac{n!(k+1) +n!(n-k)}{(k+1)!(n-k)!} = \frac{n!(k+1)}{(k+1)!(n-k)!}+\frac{n!(n-k)}{(k+1)!(n-k)!} \\
&=& \frac{n!}{k!(n-k)!} + \frac{n!}{(k+1)!(n-k-1)!} = \binom{n}{k}+\binom{n}{k+1}
\end{eqnarray*}
\end{proof}

\HandRight \qquad Im letzten Term der ersten Zeile wurde eine null der Form $0=+k-k$ hinzugefügt. Durch diesen Trick ist es möglich, die notwendigen Umformungen zu machen. Geschicktes hinzu addieren und gleichzeitiges subtrahieren ist ein vielseitig einsetzbares Mittel um notwendige Umformungen zu realisieren und sollte im Repertoire keines Mathematikers fehlen. 

\bigskip

Mit Hilfe der Binomialkoeffizienten ist die Darstellung des Pascal'schen Dreiecks auch auf folgende Weise möglich:

\begin{center}
\begin{tabular}{rccccccccc} 
$n=0$:& & & & & $\binom{0}{0}$\\
\noalign{\smallskip} $n=1$:& & & & $\binom{1}{0}$ & & $\binom{1}{1}$\\
\noalign{\smallskip} $n=2$:& & & $\binom{2}{0}$ & & $\binom{2}{1}$ & & $\binom{2}{2}$\\
\noalign{\smallskip} $n=3$:& & $\binom{3}{0}$ & & $\binom{3}{1}$ & & $\binom{3}{2}$ & & $\binom{3}{3}$\\
\noalign{\smallskip} $n=4$:& $\binom{4}{0}$ & & $\binom{4}{1}$ & & $\binom{4}{2}$ & & $\binom{4}{3}$ & & $\binom{4}{4}$
\end{tabular}
\end{center}


\section{Polynome}\label{chap:poly}

\begin{definition}
Eine reelle oder komplexe Funktion
\[ m_n(x) = a\cdot x^n  \]
wird als \emph{Monom}\index{Monom} bezeichnet. Die Summe verschiedener Monome als \emph{Polynom}\index{Polynom}:
\begin{equation}\label{eq:polynom}
p_n(x) = \sum_{i=0}^{n} a_i \cdot x^i
\end{equation}
Beachte, dass $x^0 = 1$ ist, somit gibt es in einem Polynom einen konstanten Wert $a_0$. Sei $a_n\ne 0$, dann wird $a_n$ als \emph{Leitkoeffizient}\index{Leitkoeffizient} bezeichnet. Als \emph{Rang}\index{Rang, Polynom} eines Polynoms wird der höchste Exponent $q$ bezeichnet, dessen zugehöriger Monom-Faktor $a_q \ne 0$ ist.

Ist $a_n = 1$, so wird das Polynom als \emph{normiert}\index{normiert} oder manchmal auch \emph{monisch}\index{monisch} bezeichnet.
\end{definition}

Aufgrund dessen, dass Polynome nur und ausschließlich über ihre Faktoren $a_i$ definiert werden, kann man jedes Polynom vom Grad $n$ als Punkt im $\mathbb{K}^{n+1}$ ($\mathbb{K}$ ist entweder $\mathbb{R}$ oder $\mathbb{C}$) auffassen. Sei $p_n$ wie in Gleichung (\ref{eq:polynom}), weiter ist $P_n$ die Menge aller Polynome vom Grad $n$. Dann gibt es eine isomorphe Abbildung 

\begin{equation}\label{eq:Pn}
\pi : P_n \longrightarrow \mathbb{K}^{n+1}
\end{equation}
derart, dass
\begin{equation}
\pi\left(p_n\right) = \begin{pmatrix}
a_0\\
a_1\\
\vdots\\
a_n
\end{pmatrix}
\end{equation}

Aufgrund der Isomorphie von $\pi$ wird die Vektorraumstruktur des $\mathbb{K}^{n+1}$ auf den $P_n$ übertragen. $P_n$ ist also ein ($n+1$)-dimensionaler Vektorraum.




\section{Aufgaben}

\begin{prob}
\label{poly.1.1}
Weise nach, dass $P_n$ aus Gleichung (\ref{eq:Pn}) die Anforderungen eines Vektorraums erfüllt. (Siehe Kapitel \ref{vectorspace}).
\end{prob}