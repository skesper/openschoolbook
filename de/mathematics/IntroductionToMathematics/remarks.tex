\chapter{Vorbemerkungen}

\section{Definitionssprint!}

Das Verständnis von bestimmten Sachverhalten in der Mathematik wird meist dadurch verbessert, in dem alle notwendigen Definitionen, die für den Sachverhalt bestimmend sind, an einer Stelle zusammengefasst sind. So können zum Beispiel Ringe nur dann verstanden werden, wenn der Lernende weiß, was Gruppen und im Besonderen abelsche Gruppen sind. Sowie es notwendig ist zu verstehen, was eine Menge zu einer Gruppe macht. 

Die zu einem bestimmten Thema gehörenden Definitionen werden in sogenannten "`Definitionssprints"' zusammengefasst und können dort schnell und übersichtlich nachvollzogen werden. In diesen Sprints stehen alle neuen und zum Thema gehörigen Definitionen. Solche aus vorherigen Kapiteln werden vorausgesetzt, sodass die Definitionssprints aufeinander aufbauen.

Definitionssprints können an den Überschriften erkannt werden die -- offensichtlicherweise -- "`Definitionssprint!"' genannt wurden.

