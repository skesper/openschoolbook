\chapter{Vorbemerkungen}

\section{Zur Schulmathematik}

In der Schule eingesetzte Mathematik-Bücher haben (in einem gewissen Sinne völlig zurecht) die Eigenschaft, die Vermittlung des Wissens vor das Wissen selbst zu stellen. Am Anfang der Entwicklung ist es wichtig, den Lernenden nicht zu demotivieren und ihn nicht vor einen Berg Wissen zu stellen, um ihm dann das Gefühl zu geben, diesen alleine besteigen zu müssen. Das ist ein gutes Vorhaben und es soll hier nicht daran gerüttelt werden. 

Dem gegenüber steht die Erfahrung, die ich während meiner Schulzeit machte, dass die Informationen, die man als Schüler bekam -- im Besonderen aus den Mathematik-Büchern -- nicht vollständig erschienen. Dass elementare Informationen fehlten, und dass das Abarbeiten von Textaufgaben nicht der zentrale Dreh- und Angelpunkt einer Mathematikerkarriere sein konnte (nicht, dass ich eine solche je anstrebte).

Im ersten Semester Analysis sowie Lineare Algebra an der Hochschule wurde von den Professoren dann genau dies angeprangert: Die Schulmathematik kümmert sich um viele Teilbereiche der Mathematik, aber um keins davon auf korrekte Art und Weise. Eine ernüchternde Frage des Analysis-Professors brachte seine Ansichten auf den Punkt: Er fragte die anwesenden Studenten (gute 500), ob sie in der Schule Differentialrechnung hatten. Alle, soweit ich sie sehen konnte, hoben die Hand. Nächste Frage: Welche hatten von der Vollständigkeit der reellen Zahlen gehört? Zwei oder drei Arme erhoben sich. Seine Reaktion war entsprechend. 

Mathematik ist eine reiche und lebendige Wissenschaft. Ihre Theorien mögen für Nicht-Mathematiker kompliziert und undurchdringlich erscheinen. Es wird oft vom viel beschworenen Elfenbein-Turm geredet. Aber letztlich ist es die Mathematik, die es Wissenschaftlern anderer Disziplinen erlaubt zu tun, was sie tun, indem sie die formalen Grundlagen und Werkzeuge zur Verfügung stellt, die ebendiese brauchen zur Formulierung ihrer Ideen.

Mathematik erschöpft sich nicht im Überprüfen des Kassenbons im Supermarkt. Ebenso wenig machen sich Mathematiker ständig Gedanken darüber, wie schnell Arbeiter einen Graben ausheben können und wie man dies beschleunigen könnte. 

Das führt natürlich direkt zur Frage: Was ist Mathematik? Eine umfassende Antwort auf diese Frage kann hier nicht gegeben werden. Jedoch soviel zumindest: Im Gegensatz zu dem, was viele glauben, besteht Mathematik zu einem großen Teil aus Argumentieren und dem Beweisen von Behauptungen. Da geschieht Mathematik! Nicht beim Ausrechnen, wie viele Quadratmeter eine Wohnung hat, nicht beim Bestimmen der Mehrwertsteuer oder beim Aufsummieren von gekauften Artikeln. Dies würde man mit dem Begriff \textsl{Rechnen} verbinden und das Fach Mathematik in den Schulen sollte entweder in "`Rechnen"' umbenannt werden, oder -- was meiner Vorstellung eher entspräche -- die Schüler sollten näher an das herangebracht werden, was Mathematik ausmacht: Verstehen, abstrahieren, erweitern, beweisen. Dann würden sie im ersten Semester an der Universität wissen, warum man die Vollständigkeit zum Verständnis des Differentialbegriffs benötigt. 

In diesem Sinne ist dieses Buch zu verstehen. Es konzentriert sich nicht auf die Art der Wissensvermittlung. Es konzentriert sich darauf konsistent zu sein. Nicht vollständig, denn das wäre gar nicht möglich. Jeder Teil dieses Buches könnte gleich mehrere Bücher füllen. Aber wenn etwas in dieses Buch aufgenommen wurde, dann sind alle zum Verständnis notwendigen Teile ebenfalls vorhanden.


\section{Definitionssprint!}

Das Verständnis von bestimmten Sachverhalten in der Mathematik wird meist dadurch verbessert, in dem alle notwendigen Definitionen, die für den Sachverhalt bestimmend sind, an einer Stelle zusammengefasst sind. So können zum Beispiel Ringe nur dann verstanden werden, wenn der Lernende weiß, was Gruppen und im Besonderen abelsche Gruppen sind. Sowie es notwendig ist zu verstehen, was eine Menge zu einer Gruppe macht. 

Die zu einem bestimmten Thema gehörenden Definitionen werden in sogenannten "`Definitionssprints"' zusammengefasst und können dort schnell und übersichtlich nachvollzogen werden. In diesen Sprints stehen alle neuen und zum Thema gehörigen Definitionen. Solche aus vorherigen Kapiteln werden vorausgesetzt, sodass die Definitionssprints aufeinander aufbauen.

Definitionssprints können an den Überschriften erkannt werden die -- offensichtlicherweise -- "`Definitionssprint!"' genannt wurden.

\section{Hinweise}

Es gibt zwei verschiedene Arten von Hinweisen: 

\begin{svgraybox}
Die grau hinterlegte Box beinhaltet wichtige Hinweise und sollten unbedingt beachtet werden. Falls die Hinweise für eine eingeschränkte Empfängergruppe bestimmt sind, ist dies am Anfang der Box bemerkt. Also z.B. "`Hinweis für Lehrer"'. 
\end{svgraybox}

\bigskip

\HandRight \qquad Die Hinweise mit "`Hand"' sind als Tipps zu verstehen. An diesen Stellen werden praktische Hinweise gegeben, die das Leben vereinfachen, oder auf einen Interessanten Sachverhalt aufmerksam machen wollen. Sie sind für das Verständnis nicht notwendig. 

\bigskip

Namen von historischen Personen, wie z.B. Blaise Pascal\footnote{\textbf{Blaise Pascal}, französischen Mathematiker *19. Juni 1623 in Clermont-Ferrand, \ding{61}19. August 1662 in Paris.}, werden im allgemeinen als Fußnote dargestellt. 