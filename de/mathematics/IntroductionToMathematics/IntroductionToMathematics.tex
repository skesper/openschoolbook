\documentclass[graybox,envcountchap,sectrefs,deutsch]{openschoolbook}

\usepackage[utf8]{inputenc}
\usepackage{mathptmx}
\usepackage{eurosym}
\usepackage{type1cm}         
\usepackage{booktabs}
\usepackage{array}
\usepackage{needspace}
\usepackage{makeidx}
\usepackage{graphicx}
\usepackage{multicol}
\usepackage[bottom]{footmisc}

\usepackage[T1]{fontenc}
\usepackage[ngerman]{babel}

\usepackage{color}
\usepackage{tikz}

% Farben für Konstruktionen
\definecolor{colPkt}{rgb}{0,0,1} % für gegebene Punkte
\definecolor{colPktKon}{rgb}{0.49,0.49,1} % für konstruierte Punkte
\definecolor{colWin}{rgb}{1,0,0} % für Winkel

\newcolumntype{L}[1]{>{\raggedright\let\newline\\\arraybackslash\hspace{0pt}}m{#1}}
\newcolumntype{C}[1]{>{\centering\let\newline\\\arraybackslash\hspace{0pt}}m{#1}}
\newcolumntype{R}[1]{>{\raggedleft\let\newline\\\arraybackslash\hspace{0pt}}m{#1}}

\usepackage{amssymb}
\usepackage{amsmath}

\def\currency{\officialeuro} % de
%\def\currency{\$} % us
% ...

\makeindex

\author{Stephan Kesper}
\title{Einführung in die Mathematik}
\subtitle{Zum Selbststudium oder als Grundlage für den Unterricht}

\begin{document}

\maketitle

\frontmatter

\preface

Dies ist ein sogenanntes "`offenes"' Schulbuch, das bedeutet, dass dieses Buch kostenlos jedem ohne Vorbehalt zur Verfügung gestellt wird. Es enthält Wissen, das von Freiwilligen ohne finanzielle Interessen zusammengetragen wurde. 

Es wird unter einer Commons-Creative-Lizenz veröffentlicht, wie unten angegeben. Lehrer sind herzlich willkommen Inhalte dieses Buches im Unterricht zu verwenden, auch in Teilen und Auszugsweise. Verlage können das Buch als solches Drucken und in gedruckter Form vertreiben, solange die Creative-Commons-Lizenz dadurch nicht beeinträchtigt wird.

Die Mathematik stellt zusammen mit der Sprache, eine Basis allen Wissens dar. Jedoch ist die Sprache grundlegender in dem Sinne, das sie Voraussetzung für die Erklärungen zum Verständnis der Mathematik ist. Folgerichtig ist dieses Buch erst dann zu verwenden, wenn der Lernende bereits ein gewisses Grundverständnis von Sprache besitzt. 

\bigskip

Mathematik ist eine der wenigen Wissenschaften, deren Elemente vollständig aus dem Geist von Menschen entstanden sind. Es gibt kein "`natürliches"' Vorbild für die mathematischen Elemente. Daher ist sie eine der Wissenschaften, deren Inhalte zumindest von einem Menschen bereits verstanden worden sein müssen, nämlich jenem, der die Inhalte aufschrieb und gegebenenfalls bewies.

Das bedeutet, dass die Mathematik vollständig verstehbar ist, sofern der Lernende genug Zeit und Willen aufbringt, sich mit ihr auseinander zusetzen. Diese Erkenntnis steht in direktem Widerspruch zur landläufigen Meinung, dass Mathematik schwer verständlich und unzugänglich ist. Warum dies so ist, wird von Fall zu Fall unterschiedlich sein und es gibt wohl kaum eine allgemeine Begründung. Doch viele Menschen lassen sich speziell von den Formeln abschrecken. So kompliziert diese Formeln auch im Einzelnen aussehen möchten, so sind sie doch unabdingbar in dem Sinne, dass sie vollkommen unzweideutig Sachverhalte darstellen können. Und zwar auf einem Niveau dem natürliche Sprachen weit hinterher hinken. 

Die mathematischen Formalismen\footnote{Im Sinne von Formel und nicht von einer Form-gerechten Vorgehensweise.} werden in Schulen meist auf eine Weise den Schülern beigebracht, dass sie sich davon abgeschreckt fühlen. Auf der anderen Seite sind mathematische Kenntnisse in vielen Bereichen des täglichen Lebens durchaus wichtig, und sei es nur um die Zahlenspielereien der Banken und Versicherungen auf Kredit- und Versicherungsverträgen zu durchschauen. 

Dieses Buch versucht in erster Linie Wissen zu vermitteln. Die Didaktik, wie es vermittelt werden soll, bleibt dabei -- schon aufgrund fehlender didaktischer Fachkenntnis -- auf der Strecke. Es ist also im Sinne einer Referenz zu verstehen, sodass sehr viel tieferes Wissen zur Verfügung steht, als dies in Schulen benötigt wird. Nichts desto trotz erscheint es dem Autor aber sinnvoll das Buch auf diese Weise zu schreiben, da es dann interessierten Lernenden die Möglichkeit gibt, sich auch weit über das Schulwissen hinaus zu bilden. In der Hoffnung, dass es zumindest für einige Menschen von Nutzen sein wird.


\vspace{\baselineskip}
\begin{flushright}\noindent
Koblenz, \today \hfill {\it Stephan Kesper}
\end{flushright}

\vfill

\noindent Dieser Inhalt ist unter der Creative-Commons-Lizenz vom Typ Namensnennung - Nicht-kommerziell - Weitergabe unter gleichen Bedingungen 3.0 Unported lizenziert. Um eine Kopie dieser Lizenz einzusehen, besuchen Sie

\bigskip
\begin{center}
\texttt{http://creativecommons.org/licenses/by-nc-sa/3.0/}
\end{center}

\bigskip

\noindent oder schreiben Sie einen Brief an Creative Commons, 444 Castro Street, Suite 900, Mountain View, California, 94041, USA.




\tableofcontents

\mainmatter


\chapter{Natürliche Zahlen}

\section{Was sind natürliche Zahlen?}

Der Mengenbegriff (im Sinne von Anzahl) birgt ein natürliches Verständnis dafür, was eine Zahl ist. Im Verständnis der meisten Menschen ist eine Zahl unmittelbar mit dem Begriff der Anzahl von Dingen verbunden --- zum Beispiel drei Äpfel oder fünf Orangen.

Um einem Obsthändler zu erklären, wie viele Äpfel man haben möchte, ist die Nutzung von Zahlen durchaus praktisch. Genauso wie beim Metzger Zahlen etwas abstrakter verwendet werden. So möchte man 500 Gramm Hackfleisch. Das Hackfleisch besteht nicht aus 500 Einzelteilen, sondern die Zahl bezeichnet das Gewicht dessen, was man bestellt.

Diese Zahlen werden in der Mathematik als \textbf{natürliche} Zahlen bezeichnet. Ihre Gesamtheit, d.h. alle natürlichen Zahlen inklusive etwas, das als "`unendlich"' bezeichnet wird -- darauf kommen wir später zurück -- wird mit dem Zeichen $\mathbb{N}$ abgekürzt.

Die folgenden Symbole bezeichnen die ersten neun natürlichen Zahlen. 

\[ 
1, 2, 3, 4, 5, 6, 7, 8, 9
\]

Sie bilden den Grundstock aller folgenden Zahlen, die aus diesen zusammengesetzt sind. Wie diese zusammen gesetzt werden, erfahren wir gleich.

\section{Operationen}

\subsection{Gleichheit, Ungleichheit, Vergleiche}

Mit dem Symbol "`="' wird die Anforderung beschrieben, dass alles, was auf der linken Seite des Symbols steht den selben Wert hat wie das, was auf der rechten Seite des Symbols steht. Folgendes ist demnach korrekt:

\begin{eqnarray*}
1 &=& 1 \\
5 &=& 5 \\
7 &=& 7
\end{eqnarray*}
Während das folgende falsch ist:
\begin{eqnarray*}
1 &=& 7 \\
5 &=& 3 \\
7 &=& 1
\end{eqnarray*}

Es liegt in der Verantwortung desjenigen, der die Gleichung aufstellt, dafür zu garantieren, dass die Gleichheit auch wirklich erfüllt ist. Papier ist geduldig! Man kann hinschreiben, was man will, ob ein hingeschriebenes Gleichheitszeichen auch wirklich Gleichheit bedeutet, kann nur der Hinschreibende wissen.

Möchte man ausdrücken, dass die linke und rechte Seite nicht übereinstimmt, so verwendet man das Symbol "`$\neq $"'. So werden oben angegebene Falschaussagen wieder korrekt indem man das Ungleichzeichen verwendet:
\begin{eqnarray*}
1 &\neq & 7 \\
5 &\neq & 3 \\
7 &\neq & 1
\end{eqnarray*}

Der Ungleichheit stehen noch qualifizierende Ungleichzeichen zur Seite. Dass 1 nicht gleich 7 ist stimmt zwar, ist aber weniger interessant, als die Aussage, dass 1 kleiner als 7 ist. Wenn Hans einen Apfel besitzt und Peter zwei, dann hat Hans weniger Äpfel als Peter, wie Peter mehr Äpfel hat als Hans. Die drückt man durch die folgenden Symbole aus:

\begin{eqnarray*}
1 & < & 7 \\
5 & > & 3 \\
7 & > & 1
\end{eqnarray*}

Hat man eine Ungleichung aufzustellen, bei der es akzeptabel ist, dass beide Seite auch den gleichen Wert haben können, so verwendet man die selben Symbole mit einem Unterstrich:
\begin{eqnarray*}
1 & \le & 7 \\
5 & \ge & 3 \\
7 & \ge & 1
\end{eqnarray*}

Das Symbol "`$\le$"' wird "`kleiner oder gleich"' ausgesprochen und das Symbol "`$\ge$"' "`größer oder gleich"'. Es ist zu beachten, dass folgende Ungleichungen ebenfalls \textbf{alle} korrekt sind:
\begin{eqnarray*}
4 & \le & 5 \\
5 & \le & 5 \\
6 & \ge & 5 \\
5 & \ge & 5
\end{eqnarray*}

Wenn für beiden Seiten einer Ungleichung sowohl das $\le$ als auch das $\ge$ korrekt ist, so gilt $=$ Gleichheit. Dies sollte im Kopf behalten werden, denn es gibt einige Beweise in der Mathematik, die genau dies ausnutzen.


\subsection{Addition und Multiplikation}

Um mit natürlichen Zahlen rechnen zu können, definieren wir zwei Operationen. Die Multiplikation, dargestellt durch das Zeichen "`$\cdot$"' und die Addition, dargestellt durch das Zeichen "`$+$"'.

Mit der Addition fügen wir einzelne Mengen zu größeren Mengen zusammen. Hat Hans $2$ Äpfel und Peter $3$ Äpfel, so haben sie zusammen $5$ Äpfel, oder einfacher formuliert:
\[
2+3=5
\]

\noindent Hätte Peter keinen Apfel
\[2+?=2\]
dann wäre eine Zahl hinzuzuzählen, die keinen Wert besitzt. Eine solche Zahl wird mit dem Symbol "`0"' bezeichnet. Sie ist eine Invariante bezüglich der Addition -- d.h. Additionen mit dieser Zahl ändern nicht den Wert der ursprünglichen Zahl. Man nennt sie "`Null"'. Demnach erweitern sich die Symbole der ersten zehn natürlichen Zahlen auf diese Weise:

\[0,1,2,3,4,5,6,7,8,9\]

Die Null ist per Definition nicht Teil der natürlichen Zahlen. Das hat historische Gründe -- so dachte man im Mittelalter, dass die Zahl Null vom Teufel erdacht worden sei\footnote{Historischer Beleg?}. Für unsere Betrachtung sollte sie aber Teil der natürlichen Zahlen sein. Daher werden wir von nun an mit einer Zahlenmenge umgehen, die mit $\mathbb{N}_0$ bezeichnet wird, sie besteht aus allen natürlichen Zahlen, inklusive der 0.

\textsl{Was kommt nach der 9?}

Die Zahl, die nach der 9 kommt, ist definiert durch die Summe $9+1$. Es wurde in frühen Zeiten einmal festgelegt\footnote{Historischer Beleg?}, dass wir ein Zahlensystem bestehend aus zehn Ziffern verwenden. Die zehn Ziffern beinhalten die Null, daher ist für die zehn kein eigenes Symbol mehr übrig. Daher wurden die zusammengesetzten Zahlen erfunden. So ist die Symbolfolge
\[10\]
diejenige Zahl, die nach der 9 kommt, also
\[9+1=10\]
Das ist eine Definition die bedeutet, dass wenn einem die Symbole ausgehen, kann man eine zusätzliche Stelle verwenden um diese darzustellen. Zusätzliche -- im Sinne von höherwertigen -- Stellen werden links an die Zahl angehängt. Das funktioniert auch mit drei, vier und allen weiteren Stellen:
\begin{eqnarray*}
99+1 &=& 100 \\
999+1 &=& 1000 \\
\cdots
\end{eqnarray*}

\noindent \textsl{Wie werden aber dann Zahlen zusammengesetzt?}

\noindent Sehen wir uns ein Beispiel an:
\[9+5 = ?\]
Wir wissen, dass 
\[9+1 = 10\]
ist, und dass
\[1+4 = 5\]
So können wir schreiben:
\[9+1+4 = 10 + 4 = 14\]
An dieser Stelle haben wir eine Eigenschaft verwendet, die das \textbf{Assoziativgesetz} genannt wird. Wir haben 5 durch die Summe von zwei Zahlen ersetzt und die gesamte Summe anders kombiniert. Sehen wir uns das mit Klammern an:
\[9+5 = 9+(1+4) = (9+1)+4 = 10+4 = 14\]
Klammern bevorzugen eine Operation. So ist
\[9+(1+4)=9+5\]
dass wir die 1 von $1+4$ wegnehmen und zu $9+1$ hinzufügen dürfen ist in der Anschauung vollkommen klar, ob nun Hans 9 Äpfel und Peter 5 Äpfel besitzen, oder Hans 10 und Peter 4, macht für die Gesamtanzahl keinen Unterschied. Aber dieser Operation liegt das obengenannte Assoziativgesetz zugrunde, das wir im restlichen Buch immer wieder verwenden werden. Dass wir es verwenden dürfen, liegt an der "`Harmlosigkeit"' der natürlichen Zahlen $\mathbb{N}$. Wir werden später noch Konstrukte kennen lernen, die das Assoziativ- und im Besonderen das Kommutativgesetz nicht erfüllen. Trotzdem kann man mit diesen Konstrukten genauso rechnen, wie wir das mit den natürlichen Zahlen tun, man darf mit ihnen nur nicht alles machen, was man mit Zahlen tun kann.

Die nächste Operation, die wir kennenlernen, ist die Multiplikation:
\[3\cdot 4 = ?\]
Diese entspricht ebenfalls der Anschauung. Lägen in drei Körben jeweils vier Äpfel, so hätte man insgesamt
\[\underbrace{4+4+4}_{3 \, \mathrm{mal}} = 12\]
Äpfel. Also sind 
\[3\cdot 4 = 12\]
Das Ergebnis bliebe gleich, wenn man vier Körbe mit jeweils drei Äpfeln hätte:
\[3\cdot 4 = 4\cdot 3 = 12\]
Das ist das \textbf{Kommutativgesetz}.

Gehen wir jetzt mal davon aus, wir hätten drei Körbe mit jeweils 2 grünen und 4 roten Äpfeln. Wie viele grüne Äpfel, wie viele rote und wie viele Äpfel insgesamt hätten wir dann?

Die einzelnen Summen können wir leicht bestimmen: 
\[3\cdot 2 = 6\]
grüne Äpfel,
\[3\cdot 4 = 12 \]
rote Äpfel und somit
\[6+12 = 18\]
Äpfel insgesamt. Dabei haben wir zunächst nur die Anzahl der grünen, dann die der roten Äpfel berechnet. Wir haben also folgendes gemacht:
\[3\cdot \underbrace{(2+4)}_{\mathrm{Inhalt\ eines\ Korbes}} = 3\cdot 2 + 3\cdot 4 = 18\]
Zur Berechnung der Gesamtsumme hätten wir aber auch gleich die roten und grünen Äpfel pro Korb summieren können:
\[3\cdot (2+4) = 3\cdot (6) = 3\cdot 6 = 18 \]

Dass wir zuerst die Anzahlen aller grünen Äpfel und dann die aller roten Äpfel berechnen konnten, wird als  \textbf{Distributivgesetz} bezeichnet.

\subsection{Subtraktion und Division}

Wenn Hans drei Äpfel besitzt und zwei davon isst, bleibt ihm nur einer übrig. Die Frage, die sich der Lernende stellen sollte ist:

\textsl{Welche Zahl erfüllt folgende Gleichung?}

\[3 + ? = 1\]







Der Grund warum Subtraktion und Division ein eigenes Unter-Kapitel bilden liegt darin begründet, dass sie nicht das Kommutativgesetz beachten. 

\subsection{Bruchrechnung}

\subsection{Potenzrechnung}



% % Insert some special declarations 

\part{Lineare Algebra}


\part{Geometrie}



\part{Analysis}


\chapter{Funktion}

Der Begriff einer Abbildung ist uns schon öfter untergekommen. Eine Funktion ist eine Abbildung, die zwei Mengen (hier $X$ und $Y$) in Beziehung setzt:

\[ f: X \longrightarrow Y \]
$f$ ordnet jedem Element $x\in X$ ein Element $y\in Y$ zu, indem
\[f(x) =y\]
gilt. Die Umkehrung gibt es allgemein nicht, aber wenn sie existiert, so wird sie als $f^{-1}$ bezeichnet und es gilt:
\[ f^{-1}(y) =x\]




\chapter{Grundlagen}

Die Algebra konzentriert sich -- im Gegensatz zur Arithmetik -- auf die Verallgemeinerung der Begriffe der Analyse bestimmter Sachverhalte. Der Arithmetik Teil dieses Buches begann mit Berechnungen auf Basis von Äpfeln. Diese geben einen unmittelbaren Zugang zu den Begriffen des Rechnens. In der Algebra wird es solches nicht geben. Die Ansätze hier sind rein abstrakt zu verstehen. Auch wenn direkte, anschauliche Beispiele zu bestimmten algebraischen Sachverhalten existieren, so sollte der Lernende versuchen, nicht anhand dieser sein Verständnis auszubilden, sondern rein in der Sache selbst. Das führt letztlich dazu, dass er vollständig ohne konkrete Anschauung Sachverhalte analysieren und Probleme lösen kann. Vollständig unabhängig davon, ob diese Probleme einen realen, abstrakten oder mit dem Verstand nicht nachvollziehbaren Hintergrund haben. Wenn z.B. im Raum der Polynome zwei Funktionen die Bedingung erfüllen
\[f^2(x) + g^2(x) = 1 \]
so stellt dies genauso den Satz des Pytagoras dar, als wenn ein Dreieck betrachtet würde. 
\[a^2 +b^2 = c^2\]

\section{Zeichen}

In der Algebra verwendet man anstelle von Zahlen im allgemeinen Buchstaben. Es immer vom Kontext abhängig, welcher Buchstabe was bedeutet. Aber es haben sich bestimmte Dinge eingebürgert. So sind Konstanten meistens mit den Buchstaben
\[a, b, c, \dots \]
bezeichnet. Unbekannte in Gleichungen meist mit 
\[x, y, z, p, q, \dots \]
Sowie Indizes mit 
\[i, j, k, \dots \]
Es können aber auch griechische Buchstaben auftauchen. Hier eine Übersicht

\bigskip

\begin{tabular}{c|c|l}
\hline
\textbf{Kleiner Buchstabe} & \textbf{Großbuchstabe} & \textbf{Bezeichnung} \\
\hline
$\alpha $ & $A $ & Alpha \\
$\beta $ & $B $ & Beta \\
$\gamma $ & $\Gamma $ & Gamma \\
$\delta $ & $\Delta $ & Delta \\
$\epsilon $ & $E $ & Epsilon \\
$\zeta $ & $Z $ & Zeta \\
$\eta $ & $H $ &  Eta\\
$\theta $ & $\Theta $ & Theta \\
$\iota $ & $I $ & Iota \\
$\kappa $ & $K $ & Kappa \\
$\lambda $ & $\Lambda $ & Lambda \\
$\mu $ & $M $ & Mu \\
$\nu $ & $N $ & Nu \\
$\xi $ & $\Xi $ &  Xi \\
$\omicron $ & $O $ & Omicron \\
$\pi $ & $\Pi $ & Pi \\
$\rho $ & $P $ & Rho \\
$\sigma $ & $\Sigma $ & Sigma  \\
$\tau $ & $T $ & Tau \\
$\upsilon $ & $\Upsilon $ & Ypsilon \\
$\phi $ & $\Phi $ & Phi \\
$\chi $ & $X $ & Chi \\
$\psi $ & $\Psi $ & Psi \\
$\omega $ & $\Omega $ & Omega \\
\hline
\end{tabular}

\section{Gesetze}

Da diese im Arithmetikteil nicht abstrakt definiert wurden, seien sie hier noch einmal wiederholt:

\bigskip

\noindent \textsl{Kommutativgesetz}
\begin{eqnarray*}
a+b &=& b+a \\
a\cdot b &=& b\cdot a
\end{eqnarray*}

\noindent \textsl{Assoziativgesetz}
\begin{eqnarray*}
a+(b+c) &=& (a+b)+c \\
a\cdot (b\cdot c) &=& (a\cdot b)\cdot c
\end{eqnarray*}

\noindent \textsl{Distributivgesetz}
\begin{eqnarray*}
a\cdot (b+c) &=& a\cdot b+ a\cdot c
\end{eqnarray*}


\chapter{Mengen, Gruppen, Ringe, Körper}

Kategorisierung spielt in der Algebra eine sehr wichtige Rolle. Kennt man die Eigenschaften eines Dinges, so kann man damit umgehen.



\appendix
\part{Anhang}


\chapter{Lösungen}


\section{Ganze Zahlen}
\begin{sol}{arith.1.1}
\begin{center}
\begin{tabular}{C{2cm}C{2cm}C{2cm}C{2cm}}
falsch & richtig & richtig & richtig \\
richtig & richtig & falsch & richtig \\
richtig & richtig & richtig & falsch
\end{tabular}
\end{center}
\end{sol}


\begin{sol}{arith.1.2}

\begin{eqnarray*}
2+3 &=& 5 \hskip 1cm | +2 \\
2+3+2 &=& 5+2 \hskip 1cm | \cdot 3 \\
(2+3+2) \cdot 3 &=& (5+2)\cdot 3 \\
2\cdot 3 + 3\cdot 3 + 2\cdot 3 &=& 5\cdot 3 +2 \cdot 3 \\
6+9+6 &=& 15 + 6 \\
21 &=& 21
\end{eqnarray*}

\end{sol}

\begin{sol}{arith.1.3}
\begin{center}
\begin{tabular}{C{2cm}C{2cm}C{2cm}C{2cm}}
$8+9=17$ & $8-9 =-1$ & $3\cdot 4=12$ & $8/2=4$ \\
$13+5=18$ & $13-5=8$ & $13\cdot 5=65$ & $27/3=9$ \\
$17+3=20$ & $17-3=14$ & $17\cdot 3=51$ & $25/5=5$
\end{tabular}
\end{center}
\end{sol}

\begin{sol}{arith.1.4}
\[-(2-3) = (-1)\cdot (2-3) = (-1)\cdot 2 + (-1)\cdot (-1) \cdot 3 = -2+3\]
Weil $(-1)\cdot (-1)=1$ und $1\cdot 3 = 3$.
\end{sol}

\section{Vektor, Matrix, Tensor}

\begin{sol}{matrix.1}

Seien $A,B,C\in \mathbb{R}^{m\times n}$ und $\alpha , \beta \in \mathbb{R}$. Der Nachweis wird ausschließlich auf den einzelnen Komponenten der Matrix geführt, da Addition und skalare Multiplikation ausschließlich auf den Komponenten der Matrix ausgeführt werden. Daraus folgt praktisch schon die Behauptung, dass $\mathbb{R}^{m\times n}$ ein Vektorraum ist. Aber hier sei der Nachweis noch vorgerechnet:

\begin{description}
\item[(A1)] $A+(B+C) = (a_{i,j}+(b_{i,j}+c_{i,j}))_{i,j} = ((a_{i,j}+b_{i,j})+c_{i,j})_{i,j} = (A+B)+C$. Hierbei wird die Assoziativität der reellen Zahlen verwendet.
\item[(A2)] $ (a_{i,j}+0)_{i,j} = (0+a_{i,j})_{i,j} = (a_{i,j})_{i,j}$
\item[(A3)] Da $\mathbb{R}$ ein Körper ist, gibt es zu jedem $a_{i,j}\in \mathbb{R}$ ein $ -a_{i,j}\in \mathbb{R}$, sodass $a_{i,j} +(-a_{i,j}) = -a_{i,j} + a_{i,j} = 0$ daraus folgt, dass $A+(-A) = -A +A = 0$.
\item[(A4)] $ A+B = (a_{i,j}+b_{i,j})_{i,j} = (b_{i,j}+a_{i,j})_{i,j} = B+A$
\item[(M1)] $ (\alpha \beta) A = ((\alpha \beta)a_{i,j})_{i,j} = (\alpha (\beta a_{i,j})_{i,j} = \alpha (\beta A) $
\item[(M2)] $ I\cdot A = ( \sum_{k=1}^{n} I_{i,k}a_{k,j})_{i,j} $, da $I_{i,k}=0$ für alle $i\ne k$, bleibt als Ergebnis der Summe nur $I_{i,i}a_{i,j} = a_{i,j} $ und damit ist $I\cdot A = A$
\item[(M3)] $ (\alpha +\beta )A = ((\alpha + \beta)a_{i,j})_{i,j} = (\alpha a_{i,j} + \beta a_{i,j})_{i,j} = \alpha A + \beta A $
\item[(M4)] $ \alpha (A+B) = (\alpha(a_{i,j} + b_{i,j})_{i,j} = (\alpha a_{i,j} + \alpha b_{i,j})_{i,j} = \alpha A + \alpha B $
\end{description}

\end{sol}


\backmatter
% bibliography, glossary and index would go here.

\end{document}