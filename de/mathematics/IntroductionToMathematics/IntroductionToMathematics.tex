\documentclass[]{memoir}
%\documentclass[graybox,envcountchap,sectrefs,deutsch]{openschoolbook}

\usepackage[utf8]{inputenc}
\usepackage{mathptmx}
\usepackage{eurosym}
\usepackage{type1cm}         
\usepackage{booktabs}
\usepackage{array}
\usepackage{needspace}
\usepackage{makeidx}
\usepackage{graphicx}
\usepackage{multicol}
\usepackage[bottom]{footmisc}

\usepackage[T1]{fontenc}
\usepackage[ngerman]{babel}

\usepackage{stmaryrd}

\usepackage{pifont} % \ding{61} = Death Symbol, cross

\usepackage{bbding} % Hands and Point Symbols

\usepackage{lgreek}

\usepackage{color}
\usepackage{tikz}
\usetikzlibrary{calc}
\usetikzlibrary{intersections}


% Farben für Konstruktionen
\definecolor{colPkt}{rgb}{0,0,0} % für gegebene Punkte
\definecolor{colPktKon}{rgb}{0,0,0} % für konstruierte Punkte
\definecolor{colWin}{rgb}{0,0,0} % für Winkel

\newcolumntype{L}[1]{>{\raggedright\let\newline\\\arraybackslash\hspace{0pt}}m{#1}}
\newcolumntype{C}[1]{>{\centering\let\newline\\\arraybackslash\hspace{0pt}}m{#1}}
\newcolumntype{R}[1]{>{\raggedleft\let\newline\\\arraybackslash\hspace{0pt}}m{#1}}

\usepackage{amssymb}
\usepackage{amsmath}
\usepackage{amsthm}

\usepackage{fouriernc}

\newtheorem{definition}{Definition}
\newtheorem{prob}{Problem}
\newtheorem{claim}{Behauptung}
\newtheorem{theorem}{Theorem}
\newtheorem{lemma}{Lemma}
\newtheorem{remark}{Bemerkung}
\newtheorem{sol}{Lösung}


\def\currency{\officialeuro} % de
%\def\currency{\$} % us
% ...

\makeindex


\begin{document}
\frontmatter

% --------------- TITLE PAGE
\thispagestyle{empty}
\begin{flushright}
{\Large Dipl.-Math. Stephan Kesper}
\vskip 3cm
{\Huge Einführung in die Mathematik}
\vskip 1cm
{\Large \textsl{Zum Selbststudium oder als Grundlage für den Unterricht}}
\vskip 5cm
\today
\vfill
\includegraphics{../../../License/by-nc-sa.png}
\end{flushright}
% --------------- TITLE PAGE
\newpage


%\preface

\chapter{Vorwort}

Dies ist ein sogenanntes "`offenes"' Schulbuch, das bedeutet, dass dieses Buch kostenlos jedem ohne Vorbehalt zur Verfügung gestellt wird. Es enthält Wissen, das von Freiwilligen ohne finanzielle Interessen zusammengetragen wurde. 

Es wird unter einer Commons-Creative-Lizenz veröffentlicht, wie unten angegeben. Lehrer sind herzlich willkommen Inhalte dieses Buches im Unterricht zu verwenden, auch in Teilen und Auszugsweise. Verlage können das Buch als solches Drucken und in gedruckter Form vertreiben. Hierfür ist eine Ausnahme Lizenz zu beantragen. 

\bigskip

Die Mathematik stellt zusammen mit der Sprache, eine Basis allen Wissens dar. Jedoch ist die Sprache grundlegender in dem Sinne, das sie Voraussetzung für die Erklärungen zum Verständnis der Mathematik ist. Folgerichtig ist dieses Buch erst dann zu verwenden, wenn der Lernende bereits ein gewisses Grundverständnis von Sprache besitzt. 

Mathematik ist eine der wenigen Wissenschaften, deren Elemente vollständig aus dem Geist von Menschen entstanden sind. Es gibt kein "`natürliches"' Vorbild für die mathematischen Elemente. Daher ist sie eine der Wissenschaften, deren Inhalte zumindest von einem Menschen bereits verstanden worden sein müssen, nämlich jenem, der die Inhalte aufschrieb und gegebenenfalls bewies.

Das bedeutet, dass die Mathematik vollständig verstehbar ist, sofern der Lernende genug Zeit und Willen aufbringt, sich mit ihr auseinander zusetzen. Diese Erkenntnis steht in direktem Widerspruch zur landläufigen Meinung, dass Mathematik schwer verständlich und unzugänglich ist. Warum dies so ist, wird von Fall zu Fall unterschiedlich sein und es gibt wohl kaum eine allgemeine Begründung. Doch viele Menschen lassen sich speziell von den Formeln abschrecken. So kompliziert diese Formeln auch im Einzelnen aussehen möchten, so sind sie doch unabdingbar in dem Sinne, dass sie vollkommen unzweideutig Sachverhalte darstellen können. Und zwar auf einem Niveau dem natürliche Sprachen weit hinterher hinken. 

Die mathematischen Formalismen\footnote{Im Sinne von Formel und nicht von einer Form-gerechten Vorgehensweise.} werden in Schulen meist auf eine Weise den Schülern beigebracht, dass sie sich davon abgeschreckt fühlen. Auf der anderen Seite sind mathematische Kenntnisse in vielen Bereichen des täglichen Lebens durchaus wichtig, und sei es nur um die Zahlenspielereien der Banken und Versicherungen auf Kredit- und Versicherungsverträgen zu durchschauen. 

Dieses Buch versucht in erster Linie Wissen zu vermitteln. Die Didaktik, wie es vermittelt werden soll, bleibt dabei -- schon aufgrund fehlender didaktischer Fachkenntnis -- auf der Strecke. Es ist also im Sinne einer Referenz zu verstehen, sodass sehr viel tieferes Wissen zur Verfügung steht, als dies in Schulen benötigt wird. Nichts desto trotz erscheint es dem Autor aber sinnvoll das Buch auf diese Weise zu schreiben, da es dann interessierten Lernenden die Möglichkeit gibt, sich auch weit über das Schulwissen hinaus zu bilden. In der Hoffnung, dass es zumindest für einige Menschen von Nutzen sein wird.


\vspace{\baselineskip}
\begin{flushright}\noindent
Koblenz, \today \hfill \textit{Stephan Kesper}
\end{flushright}

\vfill

\noindent Dieser Inhalt ist unter der Creative-Commons-Lizenz vom Typ Namensnennung - Nicht-kommerziell - Weitergabe unter gleichen Bedingungen 3.0 Unported lizenziert. Um eine Kopie dieser Lizenz einzusehen, besuchen Sie

\bigskip
\begin{center}
\texttt{http://creativecommons.org/licenses/by-nc-sa/3.0/}
\end{center}

\bigskip

\noindent oder schreiben Sie einen Brief an Creative Commons, 444 Castro Street, Suite 900, Mountain View, California, 94041, USA.




\tableofcontents

\newpage

\listoffigures


\mainmatter

\chapterstyle{veelo}

\chapter{Vorbemerkungen}

\section{Definitionssprint!}

Das Verständnis von bestimmten Sachverhalten in der Mathematik wird meist dadurch verbessert, in dem alle notwendigen Definitionen, die für den Sachverhalt bestimmend sind, an einer Stelle zusammengefasst sind. So können zum Beispiel Ringe nur dann verstanden werden, wenn der Lernende weiß, was Gruppen und im Besonderen abelsche Gruppen sind. Sowie es notwendig ist zu verstehen, was eine Menge zu einer Gruppe macht. 

Die zu einem bestimmten Thema gehörenden Definitionen werden in sogenannten "`Definitionssprints"' zusammengefasst und können dort schnell und übersichtlich nachvollzogen werden. In diesen Sprints stehen alle neuen und zum Thema gehörigen Definitionen. Solche aus vorherigen Kapiteln werden vorausgesetzt, sodass die Definitionssprints aufeinander aufbauen.

Definitionssprints können an den Überschriften erkannt werden die -- offensichtlicherweise -- "`Definitionssprint!"' genannt wurden.




\chapter{Natürliche Zahlen}

\section{Was sind natürliche Zahlen?}

Der Mengenbegriff (im Sinne von Anzahl) birgt ein natürliches Verständnis dafür, was eine Zahl ist. Im Verständnis der meisten Menschen ist eine Zahl unmittelbar mit dem Begriff der Anzahl von Dingen verbunden --- zum Beispiel drei Äpfel oder fünf Orangen.

Um einem Obsthändler zu erklären, wie viele Äpfel man haben möchte, ist die Nutzung von Zahlen durchaus praktisch. Genauso wie beim Metzger Zahlen etwas abstrakter verwendet werden. So möchte man 500 Gramm Hackfleisch. Das Hackfleisch besteht nicht aus 500 Einzelteilen, sondern die Zahl bezeichnet das Gewicht dessen, was man bestellt.

Diese Zahlen werden in der Mathematik als \textbf{natürliche} Zahlen bezeichnet. Ihre Gesamtheit, d.h. alle natürlichen Zahlen inklusive etwas, das als "`unendlich"' bezeichnet wird -- darauf kommen wir später zurück -- wird mit dem Zeichen $\mathbb{N}$ abgekürzt.

Die folgenden Symbole bezeichnen die ersten neun natürlichen Zahlen. 

\[ 
1, 2, 3, 4, 5, 6, 7, 8, 9
\]

Sie bilden den Grundstock aller folgenden Zahlen, die aus diesen zusammengesetzt sind. Wie diese zusammen gesetzt werden, erfahren wir gleich.

\section{Operationen}

\subsection{Gleichheit, Ungleichheit, Vergleiche}

Mit dem Symbol "`="' wird die Anforderung beschrieben, dass alles, was auf der linken Seite des Symbols steht den selben Wert hat wie das, was auf der rechten Seite des Symbols steht. Folgendes ist demnach korrekt:

\begin{eqnarray*}
1 &=& 1 \\
5 &=& 5 \\
7 &=& 7
\end{eqnarray*}
Während das folgende falsch ist:
\begin{eqnarray*}
1 &=& 7 \\
5 &=& 3 \\
7 &=& 1
\end{eqnarray*}

Es liegt in der Verantwortung desjenigen, der die Gleichung aufstellt, dafür zu garantieren, dass die Gleichheit auch wirklich erfüllt ist. Papier ist geduldig! Man kann hinschreiben, was man will, ob ein hingeschriebenes Gleichheitszeichen auch wirklich Gleichheit bedeutet, kann nur der Hinschreibende wissen.

Möchte man ausdrücken, dass die linke und rechte Seite nicht übereinstimmt, so verwendet man das Symbol "`$\neq $"'. So werden oben angegebene Falschaussagen wieder korrekt indem man das Ungleichzeichen verwendet:
\begin{eqnarray*}
1 &\neq & 7 \\
5 &\neq & 3 \\
7 &\neq & 1
\end{eqnarray*}

Der Ungleichheit stehen noch qualifizierende Ungleichzeichen zur Seite. Dass 1 nicht gleich 7 ist stimmt zwar, ist aber weniger interessant, als die Aussage, dass 1 kleiner als 7 ist. Wenn Hans einen Apfel besitzt und Peter zwei, dann hat Hans weniger Äpfel als Peter, wie Peter mehr Äpfel hat als Hans. Die drückt man durch die folgenden Symbole aus:

\begin{eqnarray*}
1 & < & 7 \\
5 & > & 3 \\
7 & > & 1
\end{eqnarray*}

Hat man eine Ungleichung aufzustellen, bei der es akzeptabel ist, dass beide Seite auch den gleichen Wert haben können, so verwendet man die selben Symbole mit einem Unterstrich:
\begin{eqnarray*}
1 & \le & 7 \\
5 & \ge & 3 \\
7 & \ge & 1
\end{eqnarray*}



\subsection{Addition und Multiplikation}

Um mit natürlichen Zahlen rechnen zu können, definieren wir zwei Operationen. Die Multiplikation, dargestellt durch das Zeichen "`$\cdot$"' und die Addition, dargestellt durch das Zeichen "`$+$"'.

Mit der Addition fügen wir einzelne Mengen zu größeren Mengen zusammen. Hat Hans $2$ Äpfel und Peter $3$ Äpfel, so haben sie zusammen $5$ Äpfel, oder einfacher formuliert:
\[
2+3=5
\]

\noindent Hätte Peter keinen Apfel
\[2+?=2\]
dann wäre eine Zahl hinzuzuzählen, die keinen Wert besitzt. Eine solche Zahl wird mit dem Symbol "`0"' bezeichnet. Sie ist eine Invariante bezüglich der Addition -- d.h. Additionen mit dieser Zahl ändern nicht den Wert der ursprünglichen Zahl. Man nennt sie "`Null"'. Demnach erweitern sich die Symbole der ersten zehn natürlichen Zahlen auf diese Weise:

\[0,1,2,3,4,5,6,7,8,9\]

Die Null ist per Definition nicht Teil der natürlichen Zahlen. Das hat historische Gründe -- so dachte man im Mittelalter, dass die Zahl Null vom Teufel erdacht worden sei\footnote{Historischer Beleg?}. Für unsere Betrachtung sollte sie aber Teil der natürlichen Zahlen sein. Daher werden wir von nun an mit einer Zahlenmenge umgehen, die mit $\mathbb{N}_0$ bezeichnet wird, sie besteht aus allen natürlichen Zahlen, inklusive der 0.

\textsl{Was kommt nach der 9?}

Die Zahl, die nach der 9 kommt, ist definiert durch die Summe $9+1$. Es wurde in frühen Zeiten einmal festgelegt\footnote{Historischer Beleg?}, dass wir ein Zahlensystem bestehend aus zehn Ziffern verwenden. Die zehn Ziffern beinhalten die Null, daher ist für die zehn kein eigenes Symbol mehr übrig. Daher wurden die zusammengesetzten Zahlen erfunden. So ist die Symbolfolge
\[10\]
diejenige Zahl, die nach der 9 kommt, also
\[9+1=10\]
Das ist eine Definition die bedeutet, dass wenn einem die Symbole ausgehen, kann man eine zusätzliche Stelle verwenden um diese darzustellen. Zusätzliche -- im Sinne von höherwertigen -- Stellen werden links an die Zahl angehängt. Das funktioniert auch mit drei, vier und allen weiteren Stellen:
\begin{eqnarray*}
99+1 &=& 100 \\
999+1 &=& 1000 \\
\cdots
\end{eqnarray*}

\noindent \textsl{Wie werden aber dann Zahlen zusammengesetzt?}

\noindent Sehen wir uns ein Beispiel an:
\[9+5 = ?\]
Wir wissen, dass 
\[9+1 = 10\]
ist, und dass
\[1+4 = 5\]
So können wir schreiben:
\[9+1+4 = 10 + 4 = 14\]
An dieser Stelle haben wir eine Eigenschaft verwendet, die das \textbf{Assoziativgesetz} genannt wird. Wir haben 5 durch die Summe von zwei Zahlen ersetzt und die gesamte Summe anders kombiniert. Sehen wir uns das mit Klammern an:
\[9+5 = 9+(1+4) = (9+1)+4 = 10+4 = 14\]
Klammern bevorzugen eine Operation. So ist
\[9+(1+4)=9+5\]
dass wir die 1 von $1+4$ wegnehmen und zu $9+1$ hinzufügen dürfen ist in der Anschauung vollkommen klar, ob nun Hans 9 Äpfel und Peter 5 Äpfel besitzen, oder Hans 10 und Peter 4, macht für die Gesamtanzahl keinen Unterschied. Aber dieser Operation liegt das obengenannte Assoziativgesetz zugrunde, das wir im restlichen Buch immer wieder verwenden werden. Dass wir es verwenden dürfen, liegt an der "`Harmlosigkeit"' der natürlichen Zahlen $\mathbb{N}$. Wir werden später noch Konstrukte kennen lernen, die das Assoziativ- und im Besonderen das Kommutativgesetz nicht erfüllen. Trotzdem kann man mit diesen Konstrukten genauso rechnen, wie wir das mit den natürlichen Zahlen tun, man darf mit ihnen nur nicht alles machen, was man mit Zahlen tun kann.

Die nächste Operation, die wir kennenlernen, ist die Multiplikation:
\[3\cdot 4 = ?\]
Diese entspricht ebenfalls der Anschauung. Lägen in drei Körben jeweils vier Äpfel, so hätte man insgesamt
\[\underbrace{4+4+4}_{3 \, \mathrm{mal}} = 12\]
Äpfel. Also sind 
\[3\cdot 4 = 12\]
Das Ergebnis bliebe gleich, wenn man vier Körbe mit jeweils drei Äpfeln hätte:
\[3\cdot 4 = 4\cdot 3 = 12\]
Das ist das \textbf{Kommutativgesetz}.

Gehen wir jetzt mal davon aus, wir hätten drei Körbe mit jeweils 2 grünen und 4 roten Äpfeln. Wie viele grüne Äpfel, wie viele rote und wie viele Äpfel insgesamt hätten wir dann?

Die einzelnen Summen können wir leicht bestimmen: 
\[3\cdot 2 = 6\]
grüne Äpfel,
\[3\cdot 4 = 12 \]
rote Äpfel und somit
\[6+12 = 18\]
Äpfel insgesamt. Dabei haben wir zunächst nur die Anzahl der grünen, dann die der roten Äpfel berechnet. Wir haben also folgendes gemacht:
\[3\cdot (2+4) = 3\cdot 2 + 3\cdot 4 = 18\]
Zur Berechnung der Gesamtsumme hätten wir aber auch gleich die roten und grünen Äpfel pro Korb summieren können:
\[3\cdot (2+4) = 3\cdot (6) = 3\cdot 6 = 18 \]

Dass wir zuerst die Anzahlen aller grünen Äpfel und dann die aller roten Äpfel berechnen konnten, ist das sogenannte \textbf{Distributivgesetz}.

\subsection{Subtraktion und Division}

Wenn Hans drei Äpfel besitzt und zwei davon isst, bleibt ihm nur einer übrig. Die Frage, die sich der Lernende stellen sollte ist:

\textsl{Welche Zahl erfüllt folgende Gleichung?}

\[3 + ? = 1\]







Der Grund warum Subtraktion und Division ein eigenes Unter-Kapitel bilden liegt darin begründet, dass sie nicht das Kommutativgesetz beachten. 

\subsection{Bruchrechnung}

\subsection{Potenzrechnung}


\chapter{Ganze Zahlen}

Der Mengenbegriff (im Sinne von Anzahl) birgt ein natürliches Verständnis dafür, was eine Zahl ist. Im Verständnis der meisten Menschen ist eine Zahl unmittelbar mit dem Begriff der Anzahl von Dingen verbunden --- zum Beispiel drei Äpfel oder fünf Orangen.

Um einem Obsthändler zu erklären, wie viele Äpfel man haben möchte, ist die Nutzung von Zahlen durchaus praktisch. Genauso wie beim Metzger Zahlen etwas abstrakter verwendet werden. So möchte man 500 Gramm Hackfleisch. Das Hackfleisch besteht nicht aus 500 Einzelteilen, sondern die Zahl bezeichnet das Gewicht dessen, was man bestellt.

\section{Natürliche Zahlen}

Dem Ziel, zu erklären, was ganze Zahlen sind, nähern wir uns über den Umweg der natürlichen Zahlen, die einen wesentlichen Bestandteil der ganzen Zahlen bilden. 


\begin{definition}
Zahlen -- im Sinne von Anzahl -- werden in der Mathematik als \emph{natürliche} Zahlen bezeichnet. Ihre Gesamtheit, d.h. alle natürlichen Zahlen inklusive etwas, das als "`unendlich"' bezeichnet wird -- darauf kommen wir später zurück -- wird mit dem Zeichen $\mathbb{N}$ \index{Natürliche Zahlen} abgekürzt.
\end{definition}

Die folgenden Symbole bezeichnen die ersten neun natürlichen Zahlen. 

\[ 
1, 2, 3, 4, 5, 6, 7, 8, 9
\]

Sie bilden den Grundstock aller folgenden Zahlen, die aus diesen zusammengesetzt sind. Wie diese zusammen gesetzt werden, erfahren wir gleich.

\section{Grundlagen}

\subsection{Gleichheit, Ungleichheit, Vergleiche}

\begin{definition}
Mit dem Symbol "`="' wird die Anforderung beschrieben, dass alles, was auf der linken Seite des Symbols steht den selben Wert hat wie das, was auf der rechten Seite des Symbols steht. \index{Gleichheit =}
\end{definition}
Folgendes ist demnach korrekt:

\begin{eqnarray*}
1 &=& 1 \\
5 &=& 5 \\
7 &=& 7
\end{eqnarray*}
Während das folgende falsch ist:
\begin{eqnarray*}
1 &=& 7 \\
5 &=& 3 \\
7 &=& 1
\end{eqnarray*}

Es liegt in der Verantwortung desjenigen, der die Gleichung aufstellt, dafür zu garantieren, dass die Gleichheit auch wirklich erfüllt ist. Papier ist geduldig! Man kann hinschreiben, was man will, ob ein hingeschriebenes Gleichheitszeichen auch wirklich Gleichheit bedeutet, kann nur der Hinschreibende wissen.

Für Gleichungen gilt im allgemeinen, dass Gleichheit erhalten bleibt, wenn auf beiden Seiten der Gleichung die selben Operationen ausgeführt werden. Multipliziert man beide Seiten mit der selben Zahl, oder addiert auf beiden Seiten die selbe Zahl, so bleibt die Gleichheit erhalten.

\begin{eqnarray*}
5 &=& 5  \hspace{1cm}| +1\\
5+1 &=& 5+1 \\
6 &=& 6
\end{eqnarray*}

Das, was man auf beiden Seiten einer Gleichung macht, kann durch einen abgesetzten senkrechten Strich dargestellt werden. Alles, was auf der rechten Seite des Strichs steht, wird auf beiden Seiten der Gleichung angewendet. Hier also jeweils eine 1 addiert.

\begin{definition}
Möchte man ausdrücken, dass die linke und rechte Seite nicht übereinstimmt, so verwendet man das Symbol "`$\neq $"'. 
\end{definition}
So werden oben angegebene Falschaussagen wieder korrekt indem man das Ungleichzeichen verwendet: \index{Ungleichheit $\ne$}

\begin{eqnarray*}
1 &\neq & 7 \\
5 &\neq & 3 \\
7 &\neq & 1
\end{eqnarray*}

Der Ungleichheit stehen noch qualifizierende Ungleichzeichen zur Seite. Dass 1 nicht gleich 7 ist stimmt zwar, ist aber weniger interessant, als die Aussage, dass 1 kleiner als 7 ist. Wenn Hans einen Apfel besitzt und Peter zwei, dann hat Hans weniger Äpfel als Peter, wie Peter mehr Äpfel hat als Hans. Dies drückt man durch die folgenden Symbole aus: \index{Kleiner als $<$} \index{Größer als $>$}

\begin{eqnarray*}
1 & < & 7 \\
1 & < & 2 \\
2 & > & 1
\end{eqnarray*}

Hat man eine Ungleichung aufzustellen, bei der es akzeptabel ist, dass beide Seite auch den gleichen Wert haben können, so verwendet man die selben Symbole mit einem Unterstrich:
\begin{eqnarray*}
1 & \le & 7 \\
5 & \ge & 3 \\
7 & \ge & 1
\end{eqnarray*}

Das Symbol "`$\le$"' wird "`kleiner oder gleich"' \index{Kleiner oder gleich $\le$} ausgesprochen und das Symbol "`$\ge$"' "`größer oder gleich"'\index{Größer oder gleich $\ge$}. Es ist zu beachten, dass folgende Ungleichungen ebenfalls \textbf{alle} korrekt sind:
\begin{eqnarray*}
4 & \le & 5 \\
5 & \le & 5 \\
6 & \ge & 5 \\
5 & \ge & 5
\end{eqnarray*}

Wenn für beiden Seiten einer Ungleichung sowohl das $\le$ als auch das $\ge$ korrekt ist, so gilt $=$ Gleichheit. Dies sollte im Kopf behalten werden, denn es gibt einige Beweise in der Mathematik, die genau dies ausnutzen.

Auch für Ungleichungen gilt, dass Operationen, die auf beiden Seiten ausgeführt werden, die Ungleichung erhalten. Das gilt bei Ungleichungen allerdings nur in etwas eingeschränkter Form, nämlich die Multiplikation mit einer negativen Zahl ($<0$) führt dazu, dass sich das Ungleichzeichen umdreht. Aber dazu später mehr.

\subsection{Addition und Multiplikation}

\begin{definition}
Um mit natürlichen Zahlen rechnen zu können, definieren wir zwei Operationen. Die Multiplikation\index{Multiplikation}, dargestellt durch das Zeichen "`$\cdot$"' und die Addition\index{Addition}, dargestellt durch das Zeichen "`$+$"'. Diese Operationen werden als \emph{Verknüpfung} bezeichnet, weil sie zwei Zahlen miteinander "`verknüpfen"' zu einer neuen Zahl.
\end{definition}

Mit der Addition fügen wir einzelne Mengen zu größeren Mengen zusammen. Hat Hans $2$ Äpfel und Peter $3$ Äpfel, so haben sie zusammen $5$ Äpfel, oder einfacher formuliert:
\[
2+3=5
\]

\noindent Hätte Peter keinen Apfel
\[2+?=2\]
dann wäre eine Zahl hinzuzuzählen, die keinen Wert besitzt. 
\begin{definition}
Eine Zahl ohne Wert wird mit dem Symbol "`0"' dargestellt.\index{Null "`0"'}
\end{definition}

Sie ist eine Invariante bezüglich der Addition -- d.h. Additionen mit dieser Zahl ändern nicht den Wert der ursprünglichen Zahl. Man nennt sie "`Null"'. 

\begin{definition}
Etwas wird als \emph{Invariante}\index{Invariante} einer Verknüpfung bezeichnet, wenn sie den Wert einer Zahl mit der sie Verknüpft wird, nicht verändert.
\end{definition}

Demnach erweitern sich die Symbole der ersten zehn natürlichen Zahlen auf diese Weise:

\[0,1,2,3,4,5,6,7,8,9\]

Die Null ist per Definition nicht Teil der natürlichen Zahlen. Das hat historische Gründe -- so dachte man im Mittelalter, dass die Zahl Null vom Teufel erdacht worden sei\footnote{Historischer Beleg?}. Für unsere Betrachtung sollte sie aber Teil der natürlichen Zahlen sein. Daher werden wir von nun an mit einer Zahlenmenge umgehen, die mit $\mathbb{N}_0$ bezeichnet wird, sie besteht aus allen natürlichen Zahlen, inklusive der 0.

\textsl{Was kommt nach der 9?}

Die Zahl, die nach der 9 kommt, ist definiert durch die Summe $9+1$. Es wurde in frühen Zeiten einmal festgelegt\footnote{Historischer Beleg?}, dass wir ein Zahlensystem bestehend aus zehn Ziffern verwenden. Die zehn Ziffern beinhalten die Null, daher ist für die zehn kein eigenes Symbol mehr übrig. Daher wurden die zusammengesetzten Zahlen erfunden. So ist die Symbolfolge
\[10\]
diejenige Zahl, die nach der 9 kommt, also
\[9+1=10\]
Das ist eine Definition die bedeutet, dass wenn einem die Symbole ausgehen, kann man eine zusätzliche Stelle verwenden um diese darzustellen. Zusätzliche -- im Sinne von höherwertigen -- Stellen werden links an die Zahl angehängt. Das funktioniert auch mit drei, vier und allen weiteren Stellen:
\begin{eqnarray*}
99+1 &=& 100 \\
999+1 &=& 1000 \\
\cdots
\end{eqnarray*}

Interessant an der Zehn ist, dass sie aus einer 1 sowie der 0 zusammengesetzt ist. Also hatten die Menschen früher kein Problem mit der 0, solange sie als Teil einer anderen Zahl auftrat. Nur wenn sie alleine stand, wurde sie als verwirrend angesehen.

Aufgrund dessen, dass unser Zahlensystem auf der Zahl 10 basiert, sind gerade die Multiplikationen mit 10 besonders einfach: 
\begin{eqnarray*}
3\cdot 10 &=& 30 \\
30\cdot 10 &=& 300 \\
300\cdot 10 &=& 3000\\
\cdots
\end{eqnarray*}
Man hängt einfach nur eine 0 an die Zahl an.

\bigskip

\noindent \textsl{Wie werden aber dann Zahlen zusammengesetzt?}

\noindent Sehen wir uns ein Beispiel an:
\[9+5 = ?\]
Wir wissen, dass 
\[9+1 = 10\]
ist, und dass
\[1+4 = 5\]
So können wir schreiben:
\[9+1+4 = 10 + 4 = 14\]
An dieser Stelle haben wir eine Eigenschaft verwendet, die das \emph{Assoziativgesetz} genannt wird. Wir haben 5 durch die Summe von zwei Zahlen ersetzt und die gesamte Summe anders kombiniert. Sehen wir uns das mit Klammern an:
\[9+5 = 9+(1+4) = (9+1)+4 = 10+4 = 14\]
Klammern bevorzugen eine Operation. So ist
\[9+(1+4)=9+5\]
dass wir die 1 von $1+4$ wegnehmen und zu $9+1$ hinzufügen dürfen ist in der Anschauung vollkommen klar, ob nun Hans 9 Äpfel und Peter 5 Äpfel besitzen, oder Hans 10 und Peter 4, macht für die Gesamtanzahl keinen Unterschied. Aber dieser Operation liegt das obengenannte Assoziativgesetz\index{Assoziativgesetz} zugrunde, das wir im restlichen Buch immer wieder verwenden werden. Dass wir es verwenden dürfen, liegt an der "`Harmlosigkeit"' der natürlichen Zahlen $\mathbb{N}$. Wir werden später noch Konstrukte kennen lernen, die das Assoziativ- und im Besonderen das Kommutativgesetz nicht erfüllen. Trotzdem kann man mit diesen Konstrukten genauso rechnen, wie wir das mit den natürlichen Zahlen tun, man darf mit ihnen nur nicht alles machen, was man mit Zahlen tun kann.

Die nächste Operation, die wir kennenlernen, ist die Multiplikation:
\[3\cdot 4 = ?\]
Diese entspricht ebenfalls der Anschauung. Lägen in drei Körben jeweils vier Äpfel, so hätte man insgesamt
\[\underbrace{4+4+4}_{3 \, \textnormal{mal}} = 12\]
Äpfel. Also sind 
\[3\cdot 4 = 12\]
Das Ergebnis bliebe gleich, wenn man vier Körbe mit jeweils drei Äpfeln hätte:
\[3\cdot 4 = 4\cdot 3 = 12\]
Das ist das \emph{Kommutativgesetz}\index{Kommutativgesetz}.

Gehen wir jetzt mal davon aus, wir hätten drei Körbe mit jeweils 2 grünen und 4 roten Äpfeln. Wie viele grüne Äpfel, wie viele rote und wie viele Äpfel insgesamt hätten wir dann?

Die einzelnen Summen können wir leicht bestimmen: 
\[3\cdot 2 = 6\]
grüne Äpfel,
\[3\cdot 4 = 12 \]
rote Äpfel und somit
\[6+12 = 18\]
Äpfel insgesamt. Dabei haben wir zunächst nur die Anzahl der grünen, dann die der roten Äpfel berechnet. Wir haben also folgendes gemacht:
\[3\cdot \underbrace{(2+4)}_{\textnormal{Inhalt eines Korbes}} = 3\cdot 2 + 3\cdot 4 = 18\]
Zur Berechnung der Gesamtsumme hätten wir aber auch gleich die roten und grünen Äpfel pro Korb summieren können:
\[3\cdot (2+4) = 3\cdot (6) = 3\cdot 6 = 18 \]

Dass wir zuerst die Anzahlen aller grünen Äpfel und dann die aller roten Äpfel berechnen konnten, wird als  \emph{Distributivgesetz}\index{Distributivgesetz} bezeichnet.

\subsection{Subtraktion und Division}

Der Grund warum Subtraktion und Division ein eigenes Unter-Kapitel bilden liegt darin begründet, dass sie nicht das Kommutativgesetz beachten, wie wir gleich sehen werden.

Wenn Hans drei Äpfel besitzt und zwei davon isst, bleibt ihm nur einer übrig. Die Frage, die sich der Lernende stellen sollte ist:

\textsl{Welche Zahl erfüllt folgende Gleichung?}

\[3 + ? = 1\]

Die Zahl, die diese Gleichung erfüllt, muss kleiner als 0 sein, denn $3+0 > 1$. Was also ist kleiner als 0? Solche Zahlen werden als negative Zahlen bezeichnet. Addiert man eine natürliche Zahl mit einer negativen Zahl, so ist das Ergebnis kleiner, als die natürliche Zahl selbst. 

\begin{definition}
Eine negative Zahl ist eine natürliche Zahl, die mit einem "`$-$"'\index{Subtraktion} Zeichen als negativ gekennzeichnet wird. Sie ist identisch zu einer Zahl, die mit $-1$ multipliziert wird:
\[ -4 = (-1) \cdot 4 \]
Des Weiteren gilt
\begin{eqnarray*}
(-1)\cdot 1 &=& 1\cdot (-1) = -1 \\
(-1)\cdot (-1) &=& 1
\end{eqnarray*}

\end{definition}

Zur Vereinfachung der Darstellung gilt:

\begin{eqnarray*}
3 + -2 &=& 1 \hspace{1cm}\textnormal{falsch}\\
3 + (-2) &=& 1 \hspace{1cm}\textnormal{richtig, aber umständlich}\\
3 - 2  &=& 1 \hspace{1cm}\textnormal{ok}
\end{eqnarray*}

Die Menge aller negativen Zahlen wird mit dem Zeichen $-\mathbb{N} $ dargestellt. Die Vereinigung der Mengen $-\mathbb{N} $ und $\mathbb{N}_0$ wird als die Menge der \emph{ganzen Zahlen} bezeichnet und $\mathbb{Z}$ genannt. Also ist 

\[
\mathbb{Z} = -\mathbb{N} \cup \mathbb{N}_0 =  \{  \dots, -3, -2, -1, 0, 1, 2, 3, \dots \}
\]

Bei der Multiplikation mit negativen Zahlen, muss man das Assoziativgesetz anwenden:

\[ (-3)\cdot 6 = ((-1)\cdot 3) \cdot 6 = (-1)\cdot (3\cdot 6) = (-1)\cdot (18) = -18 \]

In dem gleichen Sinn, wie sich die Subtraktion umgekehrt zur Addition verhält, versuchen wir uns nun vorzustellen, dass es auch zur Multiplikation eine Zahl gibt, die sich umgekehrt zu dieser verhält. Kommen wir auf das vorher erwähnte Beispiel zurück, dass wir 3 Körbe mit jeweils 4 Äpfeln haben. Insgesamt haben wir also 12 Äpfel, wie wir festgestellt hatten. Ständen wir vor dem Problem, 12 Äpfel auf drei Körbe gleichmäßig zu verteilen, wüssten wir also sofort, dass wir 4 Äpfel in jeden Korb legen müssten. Wir haben also die 12 Äpfel in drei Körbe aufgeteilt und etwas zu teilen ist Gegenstand der Division. Sie wird in der folgenden Form dargestellt:

\[ \frac{12}{3} = 4 \] 
Manchmal (z.B. wenn nicht so viel Platz ist) auch
\[ 12 / 3 = 4 \]
oder
\[ 12 \div 3 = 4 \]
Diese Gleichungen bedeuten alle das selbe, auch wenn für die Division hier drei verschiedene Zeichen (---, /, $\div$) verwendet wurden. Leider hat sich keins davon als Standard durchgesetzt, es werden in verschiedenen Situationen immer wieder diese Zeichen auftauchen. Das sollte den Lernenden nicht verwirren, denn alle bedeuten dasselbe. Wir werden hier in der Regel nur den waagerechten Stich verwenden, bis auf wenige Ausnahmen. Das $\div$ Zeichen nur beim Umformen von Gleichungen um anzuzeigen, dass durch eine Zahl geteilt werden soll.

Division\index{Division} ist in diesem Kapitel nur unzureichend zu erklären, da es nur in bestimmten Sonderfällen möglich ist, eine Zahl durch eine andere Zahl zu dividieren, um dann wieder eine ganze Zahl herauszubekommen. In solchen Fällen spricht man davon, dass eine Zahl ein \emph{Teiler} einer anderen Zahl ist. In obigem Beispiel ist 3 ein Teiler von 12, da 3 die Anzahl der Körbe ist, auf die wir unsere Äpfel aufteilen. Genauso wie 4 ein Teiler von 12 ist, da dies die Anzahl der Äpfel ist, die wir in die Körbe aufteilen. Wir werden im nächsten Kapitel darauf eingehen, was passiert, wenn wir z.B. nur 11 Äpfel hätten, aber trotzdem diese auf 3 Körbe aufteilen möchten. 

Wir hatten vorher gesehen, dass es für die Multiplikation keine Rolle spielt, ob in vier Körben jeweils drei Äpfel liegen, oder in drei Körben jeweils vier Äpfel. In beiden Fällen ist die Gesamtanzahl der Äpfel zwölf. 

Bei der Division ist dem nicht so. Wenden wir das Kommutativgesetz auf eine Division an, passiert das Folgende:
\[ 12 / 3 = 3/12 \]
Diese Gleichung würde bedeuten, dass wenn man zwölf Äpfel auf drei Körbe aufteilt, in jedem Korb genauso viele Äpfel lägen, als würde man drei Äpfel auf zwölf Körbe aufteilen, was offensichtlich falsch ist. Richtig ist:
\[ 12 / 3 \ne 3/12 \]
Also gilt das Kommutativgesetz \textbf{\underline{nicht}} für die Division. Bei der Subtraktion gilt es "`fast"':

\[ 3-2 \ne 2-3 \]
aber
\[ 3-2 = 3+(-1)\cdot 2 = \underbrace{(-1)\cdot ((-1)\cdot 3}_{\textsl{weil $(-1)\cdot (-1)=1$}} + 2) = (-1)\cdot(2-3) = -(2-3)\]
also
\[ 3-2 = -(2-3) \]

Bei der Subtraktion gilt das Kommutativgesetz, bis auf das sogenannte Vorzeichen.
\begin{definition}
Als \emph{Vorzeichen} wird entweder das $+$ oder das $-$ bezeichnet, das vor eine Zahl steht. Es bestimmt, ob die Zahl positiv ($+$) oder negativ ($-$) ist.
\end{definition}

Spätestens hier sollte es dem Lernenden klar sein, dass die Subtraktion identisch ist mit der Addition bei umgekehrtem Vorzeichen.
\[ 3-2 = 3+(-1)\cdot 2 = 3+(-2) \]

\begin{fancyquotes}
Eine Randbemerkung: Im Grunde wird hier das Kommutativgesetz der Addition angewendet, denn
\[ 3-2 = 3+(-1)\cdot 2 = (-1)\cdot 2+3 = -2+3 \]
Demnach ist $-(2-3) = -2+3$, dies wird in den Aufgaben nachgewiesen.
\end{fancyquotes}

Wie wir schon bei der Multiplikation gesehen haben, ist die Division mit 10 ebenfalls besonders einfach. Im Gegensatz zur Multiplikation werden anhängende Nullen einfach weggenommen:

\begin{eqnarray*}
3000 / 10 &=& 300 \\
300 / 10 &=& 30 \\
30 / 10 &=& 3
\end{eqnarray*}

Die 3 ist in den ganzen Zahlen nicht weiter durch 10 teilbar. Das Ergebnis von $3/10$ können Sie berechnen, nachdem Sie die Rationalen Zahlen kennen gelernt haben. 

\section{Reihenfolge bei Operationen}

Sind in einer Berechnung sowohl Multiplikationen, Divisionen, Additionen und Subtraktionen durchzuführen, so gibt es eine vorrangige Reihenfolge, in der diese durchgeführt werden müssen. Es gilt der Satz

\begin{quote}
Punktrechnung geht vor Strichrechnung
\end{quote}

\noindent Als Punktrechnung werden Multiplikation und Division bezeichnet aufgrund der Zeichen $\cdot$ und $\div$, als Strichrechnung Addition und Subtraktion aufgrund der Zeichen + und --.

Folgende Berechnung soll dies veranschaulichen:
\begin{eqnarray*}
3+5\cdot 2-4\div 2 &=&  11 \\
&\ne & 6
\end{eqnarray*}
Das Ergebnis 6 bekommt man, wenn man keine Reihenfolge beachtet und einfach von links nach rechts alle Berechnungen durchführt, also 3+5=8, multipliziert mit 2 ergibt 16, minus 4 ergibt 12, geteilt durch 2 ergibt 6. Die korrekte Berechnung ist die folgende: Zuerst berechnet man die multiplikativen Teile, $5\cdot 2=10$ und $4\div 2=2$, also bleibt 3+10-2=11.

Die einzige Möglichkeit diese Reihenfolge aufzuheben ist es Klammern zu verwenden. Klammern gehen vor Punktrechnung und somit auch vor Strichrechnung. Wäre die Aufgabe also gewesen
\[
(3+5)\cdot 2-4\div 2
\]
so ist zuerst die Klammer auszurechnen 3+5=8, multipliziert mit 2 ergibt 16. Nun geht aber trotzdem die Division vor die Subtraktion, also muss zuerst $4\div 2=2$ berechnet werden und von 16 abgezogen werden, also ergibt dies 14.

\section{Primzahlen}

Betrachten wir die Zahl 12. Sie ist -- das hatten wir bereits gesehen -- durch 4 und durch 3 teilbar, denn wir wissen, dass $3\cdot 4 = 12$ ist. Und wir wissen, dass die 2 die 4 teilt. Aber wir kennen keine Zahl, die die 3 teilt. Wie sieht es mit der 5 aus? Und wie mit der 11?

Offensichtlich gibt es Zahlen, die Teiler haben und Zahlen, die nicht teilbar sind. Nun könnte man sich fragen, ob dies unter Umständen daran liegt, dass wir sehr kleine Zahlen betrachten? Sind größere Zahlen leichter durch andere Zahlen teilbar? 

Diese Frage ist nur sehr schwer zu beantworten. Und für sehr große Zahlen gibt es immer noch Geheimnisse zu entdecken. Fakt ist aber: Egal wie groß die Zahlen werden, es gibt immer noch dazwischen welche, die durch keine der vorherigen Zahlen teilbar sind. Daher die folgende Definition:

\begin{definition}
Zahlen, die nur durch 1 oder sich selbst teilbar sind, nennen wir \emph{Primzahlen}\index{Primzahl}. Tabelle \ref{tab:primes} zeigt die Primzahlen $p$ mit $1 < p < 1000$.
\end{definition}


\begin{table}
%\centering
\begin{tabular}{ccccccccccccc}
2 & 3 & 5 & 7 & 11 & 13 & 17 & 19 & 23 & 29 & 31 & 37 &  \\
41 & 43 & 47 & 53 & 59 & 61 & 67 & 71 & 73 & 79 & 83 & 89 &  \\
97 & 101 & 103 & 107 & 109 & 113 & 127 & 131 & 137 & 139 & 149 & 151 &  \\
157 & 163 & 167 & 173 & 179 & 181 & 191 & 193 & 197 & 199 & 211 & 223 &  \\
227 & 229 & 233 & 239 & 241 & 251 & 257 & 263 & 269 & 271 & 277 & 281 &  \\
283 & 293 & 307 & 311 & 313 & 317 & 331 & 337 & 347 & 349 & 353 & 359 &  \\
367 & 373 & 379 & 383 & 389 & 397 & 401 & 409 & 419 & 421 & 431 & 433 &  \\
439 & 443 & 449 & 457 & 461 & 463 & 467 & 479 & 487 & 491 & 499 & 503 &  \\
509 & 521 & 523 & 541 & 547 & 557 & 563 & 569 & 571 & 577 & 587 & 593 &  \\
599 & 601 & 607 & 613 & 617 & 619 & 631 & 641 & 643 & 647 & 653 & 659 &  \\
661 & 673 & 677 & 683 & 691 & 701 & 709 & 719 & 727 & 733 & 739 & 743 &  \\
751 & 757 & 761 & 769 & 773 & 787 & 797 & 809 & 811 & 821 & 823 & 827 &  \\
829 & 839 & 853 & 857 & 859 & 863 & 877 & 881 & 883 & 887 & 907 & 911 &  \\
919 & 929 & 937 & 941 & 947 & 953 & 967 & 971 & 977 & 983 & 991 & 997 
\end{tabular}
\caption{Die Primzahlen zwischen 1 und 1000}\label{tab:primes}
\end{table}

Besonders zu beachten ist, dass die zwei eine Sonderrolle hat: Sie ist die einzige gerade Primzahl. Das liegt daran, dass alle geraden Zahlen durch zwei teilbar sind, aber alleine die zwei nur noch einen einzigen weiteren Teiler hat, nämlich die 1, als einzige kleinere Zahl.

Die Menge der Primzahlen liegt in den natürlichen Zahlen $\mathbb{N}$. Die Frage, die wir uns stellen sollten ist: Gibt es eine größte Primzahl? Oder anders formuliert: Gibt es unendlich viele Primzahlen? Diese Frage beantworten wir im Anhang in Kapitel \ref{chap:proofprime}.

Das besondere an den Primzahlen ist, dass wir diejenigen Zahlen, die keine Primzahlen sind, durch das Produkt von Primzahlen darstellen können. Und das wiederum ist so besonders, weil dieses Produkt eindeutig ist. Das heißt, jede natürliche Zahl kann eindeutig in ein Produkt von Primzahlen zerlegt werden.

\begin{definition}
Die Zerlegung einer Zahl in das Produkt von Primzahlen heißt \emph{Primzahlzerlegung}\index{Primzahlzerlegung}.
\end{definition}

Ebenfalls im Anhang werden wir mit Lemma \ref{lem:prim} den Beweis kennenlernen, dass Primzahlzerlegungen für jede natürliche Zahl $>1$ existieren und eindeutig sind. Solche Primzahlzerlegungen sind in erster Linie für Mathematiker interessant, auf der anderen Seite hat jeder schon einmal einen Internet Browser mit dem \texttt{https}-Protokoll Benutzt. Also zum Beispiel \texttt{https://www.google.de}. Die Basis für dieses verschlüsselte Protokoll ist der sogenannte RSA-Algorithmus. Benannt nach seinen Erfindern Rivest, Shamir und Adleman\footnote{\textbf{Ronald Linn Rivest}, *1947 in Schenectady, New York. \textbf{Adi Shamir}, *6. Juli 1952 in Tel-Aviv. \textbf{Leonard Adleman}, *31. Dezember 1945 in San Francisco}. Die Sicherheit dieses Algorithmus wird dadurch garantiert, dass es selbst mit den größten und leistungsfähigsten Computern sehr, sehr lange dauert, die Primzahlzerlegung einer Zahl zu berechnen, die größer ist als eine 1 mit 300 Nullen im Falle eines 1024 Bit Schlüssels, bzw. eine 1 mit über 600 Nullen im Falle eines 2048 Bit Schlüssels.

Sehen wir uns einige Beispiele an: 

\begin{align*}
2\cdot 2\cdot 3 &= 12 & 3\cdot 7 &= 21 & 2\cdot 3\cdot 5\cdot 7 &= 210 \\
11\cdot 13\cdot 17\cdot 19 &= 46189 & 2\cdot 2\cdot 2 &= 8 & 7\cdot 7\cdot 7 &= 343
\end{align*}
Wie wir sehen, kommen in Primzahlzerlegungen die Primzahlen nicht notwendigerweise einfach vor. Primzahlen können beliebig oft in Zerlegungen vorkommen.

\section{Aufgaben}

\begin{prob}
\label{arith.1.1}

Welche der folgenden Gleichungen und Ungleichungen sind richtig oder falsch?
\begin{center}
\begin{tabular}{C{2cm}C{2cm}C{2cm}C{2cm}}
$1=2$ & $3<5$ & $9<10$ & $10>9$ \\
$8=8$ & $3<6$ & $10<10$ & $11>10$ \\
$5=5$ & $10<10$ & $10<11$ & $999 \ge 1000$ 
\end{tabular}
\end{center}
\end{prob}

\begin{prob}
\label{arith.1.2}
Ausgehend von der Gleichung $2+3=5$ führe folgende Operationen auf beiden Seiten der Gleichung durch, ohne zu vereinfachen, bis es ausdrücklich gewünscht ist (letzter Punkt):
\begin{enumerate}
\item addiere auf beiden Seiten eine 2
\item multipliziere mit 3
\item wende das Distributiv Gesetz auf beiden Seiten der Gleichung an
\item vereinfache soweit, bis auf beiden Seiten nur noch eine Zahl steht.
\end{enumerate}
\end{prob}

\begin{prob}
\label{arith.1.3}

Bestimme die rechten Seiten der folgenden Aufgaben:
\begin{center}
\begin{tabular}{C{2cm}C{2cm}C{2cm}C{2cm}}
$8+9=$ & $8-9=$ & $3\cdot 4=$ & ${8/ 2}=$ \\
$13+5=$ & $13-5=$ & $13\cdot 5=$ & ${27/ 3}=$ \\
$17+3=$ & $17-3=$ & $17 \cdot 3=$ & ${25/ 5}=$ 
\end{tabular}
\end{center}
\end{prob}

\begin{prob}
\label{arith.1.4}
Warum ist folgende Gleichung richtig?
\[ -(2-3) = -2+3 \]
\end{prob}



\chapter{Rationale und Reelle Zahlen}

In diesem Kapitel wird der Begriff der Zahl erweitert. Im letzten Kapitel haben wir die ganzen Zahlen kennengelernt. Diese sind aber nur ein Teil der Zahlen, mit denen wir täglich umgehen. Beim Einkaufen zum Beispiel besorgen wir zwar Äpfel und Milchtüten in ganzen Einheiten, aber schon beim Brot lassen wir uns oft nur die "`Hälfte"' geben. Die Preise der Dinge, die wir im Supermarkt kaufen, haben -- im allgemeinen absichtlich -- nie einen ganzzahligen Preis, da \currency 1,99 vermeintlich "`viel billiger"' klingt, denn \currency 2,00.

Wir werden uns im Besonderen mit der Teilbarkeit von Zahlen auseinander setzten, wie auch der Unteilbarkeit (Primzahlen), als auch mit den irrationalen und den komplexen Zahlen.

\section{Rationale Zahlen}

Wir hatten im vorhergehenden Kapitel immer wieder das Beispiel der drei Körbe mit den vier Äpfeln erwähnt und auch gleich die Frage gestellt, was passiert, wenn man nicht zwölf, sondern nur elf Äpfel hätte.
\[ \frac{11}{3} = ? \]
Versuchen wir uns zunächst dem Problem zu nähern, indem wir solange Äpfel in Körbe legen, bis wir zu wenig Äpfel haben, um dies gleichmäßig zu tun. Es ist klar, dass wir in jeden Korb drei Äpfel legen können und danach zwei übrig haben. Wenn wir die Äpfel nicht zerschneiden wollten, würden wir sagen, dass wir drei Äpfel in die Körbe legen konnten und zwei übrig behalten haben. 

\begin{definition}
Dies bezeichnen wir als \textsl{Division mit Rest}. \index{Division mit Rest}
\end{definition}
In unserem Beispiel würden man sagen:
\[ \frac{11}{3} = 3 \textsl{ Rest } 2 \]

Aber ist das wirklich einfacher? Sicher nicht. Das Ergebnis $11/3$ ist, vom mathematischen Standpunkt aus betrachtet, ein vollkommen korrektes Ergebnis und benötigt keine andere Darstellung. Daher tendiert man dazu, solche Ergebnisse in der Teilerform zu behalten und nicht weiter umzuformen. Division mit Rest wurde in früheren Zeiten verwendet und entspricht sehr der Anschauung. Heute würden schwierigere Rechenaufgaben grundsätzlich mit einem Taschenrechner oder Computer durchgeführt. Diese rechnen nie mit Brüchen und schon gar nicht mit einer Division mit Rest. Sie verwenden Dezimalzahlen, die später erklärt werden. So hat die Division mit Rest -- im Rechnen mit Zahlen -- ausgedient und soll hier nur der Vollständigkeit halber erwähnt sein. 

Es sei aber trotzdem noch gesagt, dass in der linearen Algebra endliche Zahlenkörper eine Rolle spielen und dort Divisionen mit Rest sehr wohl eine Bedeutung haben und intensiv genutzt werden. Diese endlichen Zahlenkörper treten u.a. in der Kryptographie auf und sind daher auch z.B. beim Online Banking von Bedeutung, wenn ein Internet Browser eine verschlüsselte Verbindung zum Bankserver auf nimmt (Stichwort https-Protokoll).

\begin{definition}
Eine Teilerform nennt man einen \textsl{Bruch}\index{Bruch}, sowie das Rechnen mit Brüchen \textsl{Bruchrechnung}.\index{Bruchrechnung}
\end{definition}

\subsection{Bruchrechnung}

\begin{definition}
Ein Bruch besteht aus drei Teilen. Oben steht eine Zahl, die als \textsl{Zähler} bezeichnet wird, unten eine Zahl, die \textsl{Nenner} genannt wird und dazwischen ist der \textsl{Bruchstrich}:
\[ \frac{\textsl{Zähler}}{\textsl{Nenner}} \]
\end{definition}

Brüche können, wie andere Zahlen auch, addiert, subtrahiert, multipliziert und dividiert werden. Dies geschieht nach folgenden Regeln:

\begin{definition}
Brüche können nur dann addiert oder subtrahiert werden, wenn ihr Nenner übereinstimmt. 
\end{definition}

Das liegt daran, dass man eine Gleichung einfach mit dem Nenner multiplizieren kann, um so die bekannten Operationen der ganzen Zahlen zu verwenden. Beispiel:

\begin{eqnarray*}
\frac{3}{8}+\frac{5}{8} &=& \frac{3}{8}+\frac{5}{8} \hskip 1cm | \cdot 8 \\
3 +5 &=& 3+5 \\
3+5 &=& 8 \hskip 1cm | \div 8\\
\frac{3}{8}+\frac{5}{8} &=& \frac{8}{8} \\
\frac{3}{8}+\frac{5}{8} &=& 1
\end{eqnarray*}
Auf der rechten Seite haben wir uns in der vorletzten Zeile zunutze gemacht, dass ein Bruch seinen Wert nicht ändert, falls Nenner und Zähler mit der selben Zahl multipliziert werden. So ist 

\[ 1 = \frac{a}{a} \textsl{ für alle } a\in \mathbb{Z} \]
wie auch 
\begin{equation}
\label{relation}
\frac{x}{y} = \frac{a\cdot x}{a\cdot y} \textsl{ für alle } a\in \mathbb{Z}
\end{equation}

\begin{definition}
Wir bezeichnen Zahlen der Form 

\[ \frac{a}{b} \textsl{ für alle } a,b \in \mathbb{Z}, b\ne 0\]
als \textsl{Rationale Zahlen} und stellen sie mit dem Symbol $\mathbb{Q}$ dar. 
\end{definition}
Also ist 

\[ \mathbb{Q} = \left\{ \frac{a}{b} \middle\vert a,b \in \mathbb{Z}, b\ne 0 \right\} \]
Die rationalen Zahlen beinhalten die ganzen Zahlen auf eine natürliche Weise, da für $n,a\in \mathbb{Z}$ gilt
\[ n = \frac{n}{1} = \frac{n \cdot a}{1\cdot a}\]
In diesem Fall ist $a$ ein Teiler der Zahl $n\cdot a$. 

\begin{quote}
\textsl{Hinweis für Lehrer:}

Es gibt Mathematiker, die bei der Definition der rationalen Zahlen einen axiomatischen Ansatz wählen, der die rationalen Zahlen auf Basis von Äquivalenzrelationen definieren. So wäre dann eine rationale Zahl immer durch genau eine Äquivalenzklasse dargestellt. Da Äquivalenzklassen im allgemeinen durch ihr Erzeugendes Element dargestellt werden, würden nach diesem Ansatz nur diejenigen Brüche zu den rationalen Zahlen gehören, die die Erzeugenden Elemente ihrer Äquivalenzklasse wären. Der Autor hält dies für sehr verwirrend und kaum vermittelbar, warum der Bruch 
\[ \frac{3}{11}\]
zu den rationalen Zahlen zählen soll, der Bruch
\[ \frac{9}{33}\]
aber nicht, sondern nur zur Äquivalenzklasse von $\frac{3}{11}$.
\end{quote}

\subsection{Weitere Begriffe}

\begin{definition}
Es seien Zähler $Z$ und Nenner $N$ eines Bruchs durch die selbe Zahl $a$ teilbar. Das bedeutet, $Z=a\cdot z$ sowie $N=a\cdot n$ mit $a,z,n\in \mathbb{Z}$. Also gilt:
\[
\frac{Z}{N} = \frac{a\cdot z}{a\cdot n} = \frac{a}{a}\cdot \frac{z}{n} = 1\cdot \frac{z}{n} = \frac{z}{n}
\]
In diesem Fall sagt man, dass man den Bruch um $a$ \textsl{kürzen}\index{kürzen} kann.
\end{definition}

Des Weiteren ist mit $Z$ und $N$ wie oben
\[
\frac{Z}{N} = \frac{\frac{Z}{a}}{\frac{N}{a}} = \frac{Z}{a\cdot \frac{N}{a}} = \frac{Z \cdot a}{a\cdot N}
\]
Das bedeutet, ist im Zähler ein Bruch, so kann dessen Nenner zum Nenner multipliziert werden. Steht im Nenner ein Bruch, so wird dessen Nenner zum Zähler multipliziert. 

\subsection{Dezimalzahlen}

\begin{definition}
Eine \textsl{Dezimalzahl} ist eine Zahl der folgenden Form:\index{Dezimalzahl}
\[ z_m z_{m-1} \dots z_1 z_0, z_{-1} z_{-2} \dots z_{-n} \]
\end{definition}
Die $z_i$ sind Ziffern und somit Zahlen aus der Menge $\{0,1,2,3,4,5,6,7,8,9\}$.

Zur Erläuterung betrachten wir zunächst die Zahlen zwischen 0 und 1. Dies sind die Zahlen 
\[ \frac{1}{n}\] für alle $n\in \mathbb{N}$. 

\begin{eqnarray*}
1/2 &=& 0,5 \\
1/4 &=& 0,25 \\
1/5 &=& 0,2 \\
1/8 &=& 0,125 \\
\dots
\end{eqnarray*}

Die Brüche $1/3$, $1/6$ und $1/7$ haben eine besondere Eigenschaft, sie haben nämlich keine endliche Dezimaldarstellung:

\begin{eqnarray*}
1/3 &=& 0,3333333333\dots \\
1/6 &=& 0,1666666666\dots \\
1/7 &=& 0,142857142857142857142857142857\dots
\end{eqnarray*}
Die sich wiederholenden Teile der Nachkommastellen werden mit einem Strich darüber gekennzeichnet:

\begin{eqnarray*}
1/3 &=& 0,\overline{3} \\
1/6 &=& 0,1\overline{6} \\
1/7 &=& 0,\overline{142857}
\end{eqnarray*}
Natürlich gibt es auch Dezimalzahlen, die größer sind als 1:

\begin{eqnarray*}
97/3 &=& 32,\overline{3} \\
21/5 &=& 4,2 \\
177/7 &=& 25,\overline{285714}
\end{eqnarray*}
Dezimalzahlen treten immer wieder als Ergebnisse von Berechnungen von Taschenrechnern und Computern auf, daher sind sie sehr wichtig. 

Größere Dezimalzahlen werden der Übersichtlichkeit halber mit Trennzeichen gegliedert. So werden 
\[100000\]
auch als
\[100.000\]
geschrieben. Sind Kommastellen zu berücksichtigen, ist zwischen Punkt und Komma strikt zu unterscheiden:
\[100.000.00 \ne 100.000,00\]
Als Trennzeichen zu den Nach-"'Komma"'-Stellen, ist immer das Komma zu verwenden. Vermutlich aufgrund zunehmender Amerikanisierung und dem Gebrauch von Taschenrechnern hat sich auch der Punkt als Trennzeichen für die Nachkommastellen eingebürgert. Aber dies ist falsch im deutschen Zahlengebrauch. Die hier vorgestellte Darstellung gilt in dieser Form nur in Deutschland. 

\bigskip

\begin{tabular}{L{4cm}R{4cm}}
Deutschland & 1.000.000,00 \\
Amerika/England & 1,000,000.00 \\
Schweiz & 1'000'000,00 \\
ISO 31-0 & 1\,000\,000,00
\end{tabular}

\bigskip

Die Internationale Organisation für Normung (ISO) hat im ISO 31-0 Standard festgelegt, dass als Tausender-Trennzeichen ein schmales Leerzeichen zu verwenden sei. Man sieht dies vereinzelt in Veröffentlichungen aber ansonsten hat sich diese Schreibweise nur wenig durchgesetzt. 

\section{Potenzrechnung}

Einleitend sei hier bemerkt, dass sich das Potenzieren mit natürlichen Zahlen zur Multiplikation verhält, wie die Multiplikation zur Addition:

\[ 4\cdot 3 = \underbrace{4+4+4}_{\text{3 mal}} = 12 \]

\[ 4^3 = \underbrace{4\cdot 4\cdot 4}_{\text{3 mal}} = 64 \]

\begin{definition}
Die Zahl, die potenziert werden soll, nennt man \textsl{Grundzahl}\index{Grundzahl}, und die Zahl mit der potenziert werden soll \textsl{Hochzahl}\index{Hochzahl}, oder \textsl{Exponent}\index{Exponent}. Den Wert der Potenz einer Grundzahl ($a$) mit einer natürlichen Zahl ($n \in \mathbb{N}$) als Exponent erhält man durch:
\[ a^n = 1\cdot \underbrace{a\cdot a \cdot \dots \cdot a}_{\text{n mal}} \]
Den Wert einer Potenz mit einer negativen natürlichen Zahl durch:
\[ a^{-n} = 1 \div \underbrace{a\div a \div \dots \div a}_{\text{n mal}} = \frac{1}{a^n} \]
Des Weiteren gilt immer
\[ a^0 = 1,\ a^1=a \]
\end{definition}
Das Kommutativgesetz gilt nicht:
\[ 64 = 4^3 \ne 3^4 = 81 \]


\section{Reelle und irrationale Zahlen}\label{chap:realbegin}

Die reellen Zahlen sind -- zusammen mit den komplexen Zahlen -- mit die gebräuchlichsten Zahlenkörper in der Mathematik. Die über diesen Körpern gebildeten, dreidimensionalen Vektorräume entsprechen in direkter Weise unserer räumlichen Vorstellung (Euklidischer Raum). 

Auch wenn es seit Euklid (etwa 300 v. Chr.) ein Verständnis für die reellen Zahlen im Sinne eines "`Kontinuums"' gibt, wurde eine erste, formale Konstruktion der reellen Zahlen zum ersten Mal durch Karl Weierstraß\footnote{\textbf{Karl Theodor Wilhelm Weierstraß}, *31. Oktober 1815 in Ostenfelde; \ding{61}19. Februar 1897 in Berlin} dargelegt. Die verschiedenen Konstruktionsarten der reellen Zahlen darzulegen, würde weit über das Ziel hinaus gehen, ein Buch für Schüler zu schreiben. Daher können wir nur versuchen, argumentativ ein Gefühl für die reellen Zahlen zu entwickeln. 
\begin{quote}
Zu je zwei reellen Zahlen, egal wie nahe sie beieinander liegen, gibt es immer eine dritte dazwischen.
\end{quote}
Dieser Satz ist leicht daher gesagt, stellt aber eine sehr zentrale Erkenntnis im Zusammenhang mit den reellen Zahlen dar, denn dies gilt für keine andere Zahlenmenge, die wir bisher kennen gelernt haben.

Die rationalen Zahlen waren schon "`ziemlich"' vollständig in dem Sinne, dass wir immer zu je zwei Zahlen aus $\mathbb{Q}$ folgendes tun können: Seien $a=\frac{a_z}{a_n}$ und $b=\frac{b_z}{b_n}$ rationale Zahlen. Dann kann immer die Zahl $c$ gefunden werden mit
\[
c = \frac{a_z\cdot b_n + a_n\cdot b_z}{2\cdot a_n\cdot b_n}
\]
$c$ liegt damit immer zwischen den Zahlen $a$ und $b$. Da $a_z,a_n,b_z$ und $b_n$ natürliche Zahlen sind, ist der Zähler und Nenner von $c$ jeweils wieder eine natürliche Zahl und deshalb ist $c\in \mathbb{Q}$. 

Es gibt allerdings Zahlen, die nicht durch einen Bruch dargestellt werden können, egal wie groß die Zahlen in Zähler und Nenner sind. Trotzdem bleibt immer ein Unterschied. Beispiele sind das Verhältnis eines Kreises zu seinem Durchmesser. Diese Zahl wird mit $\pi$ abgekürzt. Genauso wie der Durchmesser eines Quadrates mit Kantenlänge 1, dies ist die $\sqrt{2}$. Keine solche Zahl kann mit Brüchen "`genau genug"' angenähert werden und diese sind nicht in den rationalen Zahlen enthalten. Solche Zahlen werden irrational genannt

Die Vereinigung aller irrationalen Zahlen mit den rationalen Zahlen gelten als die reellen Zahlen und werden mit dem Symbol $\mathbb{R}$ bezeichnet.

Im Anhang \ref{chap:realfinal} werden die reellen Zahlen genauer eingeführt. 

\section{Intervalle}

\begin{definition}
Teilmengen der reellen Zahlen, die einen zusammenhängenden Bereich abdecken, werden auch als \textsl{Intervalle} bezeichnet. Sie haben eine obere und eine untere Grenze. Alles, was dazwischen liegt, gehört zu dieser Teilmenge. Sei $a<b$, dann wird mit
\[
\lbrack a,b \rbrack
\]
die Menge aller Zahlen $x$ bezeichnet, für die gilt $a\le x\le b$
\end{definition}

\begin{definition}
Gehört die obere und untere Zahle nicht zum Intervall, wenn also gilt:
\[
a<x<b
\]
dann heißt das Intervall \textsl{offen}. In diesem Fall werden runde Klammern verwendet
\[
x\in (a,b)
\]
\end{definition}

\begin{definition}
Natürlicherweise gilt dies auch für nur einseitig offene Intervalle, 
\[
\lbrack a,b) \text{ oder } (a,b\rbrack
\]
Solche Intervalle werden auch als \textsl{halboffen} bezeichnet und es gilt
\[
a\le x <b \text{ bzw. } a<x \le b
\]
\end{definition}

\section{Betrag und Vorzeichen}

Jede Zahl in $r\in \mathbb{R}_+$ hat eine Entsprechung im Negativen $-r$. Das führt uns zu der Überlegung, dass der Wert einer Zahl unabhängig von ihrem Vorzeichen von Interesse sein könnte.

\begin{definition}
Der \textsl{Betrag}, oder \textsl{Absolutbetrag} ist eine Funktion $\vert . \vert : \mathbb{R} \longrightarrow \mathbb{R}_+$
\begin{equation}
\vert r \vert = \begin{cases}
r & \text{falls } r\ge 0 \\
-r & \text{falls } r<0
\end{cases}
\end{equation}
Da $\mathbb{Z} \subset \mathbb{Q} \subset \mathbb{R}$ sind, gilt diese Definition auch für die ganzen und rationalen Zahlen. Für die natürlichen Zahlen ist sie sinnlos, da es keine negativen natürlichen Zahlen gibt. Aber da $\mathbb{N} \subset \mathbb{Z}$, ist trivialer Weise $\vert n \vert = n$ für alle $n\in \mathbb{N}$.
\end{definition}

Das meist weggelassene $+$ vor den Zahlen, sowie das $-$ wird als \textsl{Vorzeichen} bezeichnet. Die folgende Definition ist daher naheliegend, wenn auch zu diesem Stadium noch nicht weiter von Interesse:

\begin{definition}
Die Funktion 
\[
\text{sign} : \mathbb{R} \longrightarrow \lbrace -1,1\rbrace
\]
bestimmt das Vorzeichen für jede reelle Zahl. Sei $x \in \mathbb{R}$
\begin{equation}
\text{sign}(x) = \begin{cases}
1 & \text{falls } x\ge 0 \\
-1 & \text{falls } x < 0
\end{cases}
\end{equation}
\end{definition}
Seien $a,b\in \mathbb{R}$, dann gilt:
\begin{equation}
\begin{split}
|a\cdot b| &= |a|\cdot |b| \\
|a\div b| &= |a|\div |b| \\
|a+b| &\le |a|+|b| \\
|a-b| &\le |a|+|b| \\
|a|-|b| &\le |a-b| \\
\end{split}
\end{equation}


\section{Aufgaben}
TODO


\chapter{Komplexe Zahlen}

Um die komplexen Zahlen zu erklären, wollen wir uns zunächst eine Gleichung ansehen. 

\[ ?^2 +1 = 0 \]
Welche Zahl kann diese Gleichung erfüllen? Wir formen um:
\begin{eqnarray*}
?^2 +1 &=& 0 \hskip 1cm | -1 \\
?^2 +1 -1 &=& -1 \hskip 1cm | \sqrt{.} \\
\sqrt{?^2} &=& \sqrt{-1} \\
? &=& \sqrt{-1}
\end{eqnarray*}
Stimmt das? Wir setzen ein:
\begin{eqnarray*}
\sqrt{-1}^2 +1 &=& 0 \\
(-1)^{\frac{1}{2}\cdot 2} +1 &=& 0 \\
(-1)^{\frac{2}{2}} +1 &=& 0 \\
(-1)^{1} +1 &=& 0 \\
-1+1 &=& 0 \\
0 &=& 0
\end{eqnarray*}
Die Gleichung stimmt offensichtlich, doch hatten wir gelernt, dass Wurzeln aus negativen Zahlen nicht definiert sind. Kann man also einfach mit dem Quadrieren einer negativen Zahl rechnen und erwarten, dass etwas sinnvolles dabei heraus kommt? Die einfache Antwort ist: "`Ja"', man kann. Denn Fakt ist: Als wir die Gleichung mit einem $?$ umgeformt hatten, wussten wir noch nicht, dass das Fragezeichen für die Wurzel aus $-1$ steht. Wendet man also die Rechenschritte korrekt auf eine Zahl (oder Variable, wie wir im späteren sehen werden) an, so bleibt das Resultat korrekt. 

Die Definition der komplexen Zahlen fußt auf dieser Erkenntnis. Sie verwendet die negativen Wurzeln indem ein Buchstabe $i$ \index{$i = \sqrt{-1}$} den Wert $\sqrt{-1}$ annimmt und so zum Rechenelement wird.
\begin{eqnarray*}
\sqrt{-17} &=& \sqrt{-1\cdot 17}\\
&=& \sqrt{-1}\cdot \sqrt{17}\\
&=& i \cdot \sqrt{17}
\end{eqnarray*}
oder allgemein:

\begin{eqnarray*}
\sqrt{-x^2} &=& \sqrt{-1\cdot x^2}\\
&=& \sqrt{-1}\cdot \sqrt{x^2}\\
&=& i \cdot x
\end{eqnarray*}
wobei $x$ jede reelle Zahl sein kann, also $x\in \mathbb{R}$. Allgemein werden die komplexen Zahlen definiert als 

\[ \mathbb{C} = \{ a+i\cdot b \ |\ a,b \in \mathbb{R} \} = \mathbb{R}+i\mathbb{R} \]
Das bedeutet, dass komplexe Zahlen aus zwei reellen Zahlen bestehen. Sie können nicht miteinander addiert werden, da das $i$ dies verhindert. Weil die komplexen Zahlen dadurch aus zwei reellen Komponenten gebildet werden, nennt man sie auch oft zweidimensionale Zahlen. Als Ausblick sei erwähnt, dass in der linearen Algebra die komplexen Zahlen mit einem zwei dimensionalen reellen Vektorraum identifiziert werden können.

\section{Die Grundrechenarten der komplexen Zahlen}

Der Einfachheit halber wird im Folgenden $i\cdot b$ durch $ib$ ersetzt. Der Multiplikationspunkt kann weggelassen werden. Und alle $a,b,c,d \in \mathbb{R}$ sowie alle $p,q \in \mathbb{C}$ mit $p = a+ib$ und $q = c+id$. Beachte, dass $i^2=\sqrt{-1}^2 = -1$.

\begin{definition} Addition
\[ p+q = (a+c)+i(b+d)\]
\end{definition}

\begin{definition} Subtraktion
\[ p-q = (a-c)+i(b-d)\]
\end{definition}

\begin{definition} Multiplikation
\[ p\cdot q = a(c+id)+ib(c+id) = (ac-bd)+i(ad+bc)  \]
\end{definition}

\begin{definition} Division
\[ \frac{p}{q} = \frac{a+ib}{c+id} = \frac{(a+ib)(c-id)}{(c+id)(c-id)} = \frac{ac+bd}{c^2+d^2} + i\frac{bc-ad}{c^2+d^2}\]
\end{definition}

\section{Historische Bemerkung} 

Gerolamo Cardano\index{Cardano, Gerolamo} (1501-1576) behandelte in seinem 1545 erschienen Buch \textit{Artis magnae sive de regulis algebraicis liber unus} die Aufgabe zwei Zahlen zu finden, deren Produkt 40 und deren Summe 10 sei. Er setzte dafür die Gleichung an:

\[ x^2-10x+40=0 \]

Er erkannte, dass diese Gleichung keine Lösung hat, fügte aber die Bemerkung hinzu, dass falls die entsprechenden Umformungen sinnvoll und erlaubt wären, dass dann $5+\sqrt{-15}$ sowie $5-\sqrt{-15} $ in der Tat Lösungen der Gleichung wären. 

In diesem Sinne hatte Cardano bereits den ersten Schritt in die richtige Richtung getan. Doch dauerte es noch eine ganze Weile, bis komplexe Zahlen sich durchsetzten.


\section{Unendlich $\infty$}

\index{Unendlich $\infty$}
Der Begriff \textsl{Unendlich} bezeichnet keine Zahl, sondern im Grunde eher einen Zustand. Unendlich ist etwas -- wie der Begriff nahe legt --, wenn es kein Ende besitzt. So haben zum Beispiel die natürlichen Zahlen kein oberes Ende:

\[
\mathbb{N} = \left\lbrace 1,2,3, \dots \right\rbrace
\]
So unscheinbar die "`$\dots$"' auch sind, so bezeichnen sie den allergrößten Teil der Menge $\mathbb{N}$. Denn auch die Zahl 
\[9.287.375.864.825.551.256.365.751.255\]
ist eine natürliche Zahl, genauso wie 
\[2^{9.287.375.864.825.551.256.365.751.255}\]
oder
\begin{equation} \label{eq:huge}
10^{9.287.375.864.825.551.256.365.751.255}
\end{equation}
oder sogar
\begin{equation}\label{eq:reallyhuge}
9.287.375.864.825.551.256.365.751.255^{9.287.375.864.825.551.256.365.751.255}
\end{equation}

Es gibt Zahlen in der Mathematik, die so groß sind, dass selbst wenn man alle Atome im Universum in Papier und Tinte umwandelte, sie nicht ausreichen würden, um die Zahl aufzuschreiben. Die letzte oben angegebene Zahl ist ein gutes Beispiel dafür. Astronomen schätzen, dass es etwa $10^{77}$ Atome im Weltall gibt. Und das ist wesentlich weniger, als die in (\ref{eq:huge}) darstellte Zahl, geschweige denn (\ref{eq:reallyhuge}).

Der Punkt ist: Es gibt keine größte natürliche Zahl. Immer wenn man glaubt, eine gefunden zu haben, gibt es eine noch größere. Dieser Zustand, nämlich dass es keine größte Zahl gibt, sondern immer weitere, wird mit dem Zeichen $\infty$ abgekürzt. 

Dementsprechend ist es korrekter, wenn man die natürlichen Zahlen in dieser Form schreibt:
\[
\mathbb{N} = \left\lbrace 1,2,3, \dots, \infty \right\rbrace
\]

Aufgrund dessen, dass $\infty$ keine Zahl ist, gelten auch sämtliche Rechenregeln, die wir kennen gelernt haben, nicht für $\infty$. So ist zum Beispiel $\infty +1$ unsinnig, genauso wie $\infty-1$. Das selbe gilt für die Multiplikation und Division. Es gibt allerdings einige Regeln, die es zu wissen gilt. Denn es passiert, dass man beim Rechnen oder umformen auf $\infty$ stößt und dann damit umgehen können muss. 

Für $a\in \mathbb{R}$ (beachte, dass $\mathbb{N} \subset \mathbb{Z} \subset \mathbb{R}$) mit $-\infty < a < \infty$ gilt

\begin{eqnarray*}
\infty + a &=& \infty \\
\infty \cdot a &=& \infty \\
\frac{a}{\infty} &=& 0
\end{eqnarray*}

Für $a=\infty$ oder $a=-\infty$ sind alle oben angegebenen Regeln hinfällig. Alle Operationen mit diesen Werten sind unzulässig und nicht definiert!

Eine der erstaunlichsten und gleichzeitig verwirrendsten Eigenschaften von unendlich großen Mengen ist, dass sie echte Teilmengen besitzen, die ebenfalls wieder unendlich groß sind. Also "`gleich viele"' Elemente haben, nämlich unendlich viele. Betrachten wir die Primzahlen in $\mathbb{N}$:

\begin{eqnarray*}
1. & 2 \\
2. & 3 \\
3. & 5 \\
4. & 7 \\
5. & 11 \\
\dots & \dots
\end{eqnarray*}
Dass es unendlich viele Primzahlen gibt, muss bewiesen werden, was wir hier tun wollen, auch wenn die Beweisführung erst später erklärt wird. Es wird dabei vorausgesetzt, dass jede Zahl, die größer als 1 ist und keine Primzahl, eine eindeutige Zerlegung in Primfaktoren besitzt. Dies müsste natürlich vorher auch bewiesen werden, aber die Beweise würden hier zu weit führen. Daher müssen wir uns darauf verlassen, dass dies richtig ist. 

\begin{lemma}
Es gibt unendlich viele Primzahlen.
\end{lemma}
\begin{proof}
Der Beweis wird durch einen Widerspruch geführt. Behauptung: Es gibt nur endlich viele Primzahlen. Wenn dem so wäre, dann gäbe es eine größte Primzahl $N$. Sei

\[ M = 2\cdot 3\cdot 5\dots \cdot N +1 \]
eine Zahl gebildet aus dem Produkt aller Primzahlen (es gibt ja nur endlich viele) addiert mit 1.

Wir wissen, dass $M$ eine eindeutige Zerlegung in Primfaktoren besitzt. Da aber $M$ bereits aus dem Produkt aller bekannten Primzahlen erzeugt wurde, und somit nicht durch eine davon teilbar ist (weil M um 1 größer ist, als das Produkt aller Primzahlen), bleiben nur zwei Möglichkeiten: a) $M$ ist selbst eine Primzahl, es ist aber $M>N$, was gegen die Voraussetzung verstößt, dass $N$ die größte Primzahl ist. Oder b) Die Zerlegung von $M$ in Primfaktoren enthält eine Primzahl, die größer ist, als $N$, was wiederum gegen die Voraussetzung verstößt. Also ist die Behauptung, dass es nur endlich viele Primzahlen gibt, falsch und somit stimmt die eigentliche Aussage, die wir beweisen wollten, nämlich dass es unendlich viele Primzahlen gibt. 

\qed
\end{proof}

Somit hat die Aufzählung der Primzahlen kein Ende. Sie ist also in unserem vorher beschriebenen Sinne "`unendlich"'. Wenn also die Aufzählung unendlich ist, dann kann jeder natürlichen Zahl eine Primzahl zugewiesen werden. Daraus folgt, dass es genauso viele natürliche Zahlen gibt, wie es Primzahlen gibt. 

Unserer Anschauung nach, ist dies natürlich falsch, denn wir denken immer in endlichen Mengen. Und für endliche Mengen, wie z.B. die Zahlen zwischen 1 und 100 oder zwischen 1000 und 2000, stimmt es sicher, dass sich in diesen weniger Primzahlen befinden, als natürliche Zahlen. Aber im Unendlichen wird es richtig, weil es keinen Mengenbegriff im Unendlichen mehr gibt. Unendlich ist Unendlich. Mengen

\begin{definition}
Eine unendliche Menge, deren Elemente durchgezählt werden können, also in denen jedem Element eine natürliche Zahl zugeordnet werden kann, nennt man \textsl{abzählbar unendlich}. 
\end{definition}


\section{Polarkoordinaten der Komplexen Zahlen}
TODO

\section{Riemannsche Zahlenkugel}
TODO

\section{Aufgaben}
TODO


\chapter{Angewandte Mathematik}

In diesem Kapitel beschäftigen wir uns ausschließlich mit den Anwendungen der Mathematik im Alltag. 


\section{Prozentrechnung}

\subsection{Beispiel Mehrwertsteuer}

Ein Problem mit dem wir es täglich zu tun bekommen, ist die Mehrwertsteuer. Sie ist auf alle Handelswaren zu erheben und an das Finanzamt abzuführen. Sie beträgt allgemein 19\%, auf Lebensmittel 7\%. 

Das Zeichen "`\%"' wird "`Prozentzeichen"' genannt. Der Begriff kommt aus dem lateinisch-italienischen "`per cento"', "`vom Hundert"', und erklärt bereits, worum es dabei geht. Man teilt eine beliebige Menge in 100 Teile und nimmt sich (das Finanzamt) 19 Teile davon. 

Der aktuelle VW Golf\footnote{Preis von Dezember 2013} kostet 14.264,70 \officialeuro\  ohne Mehrwertsteuer. Dieser Preis in 100 Teile geteilt ergibt 142,647 \officialeuro, mit 19 multipliziert ergibt 2710,293 \officialeuro. Daher müssen Kunden, die den VW Golf kaufen möchten, 16.975,- \officialeuro\  bezahlen. Der Händler führt dann 2710,29 \officialeuro\  an das Finanzamt als Mehrwertsteuer ab.

\subsection{Allgemeine Prozentrechnung}

Sei im Folgenden $p$ der Prozentsatz, also z.B. $p=19\%$, und $w$ der Prozentwert, im obigen Beispiel die 2710,29 \officialeuro. Des Weiteren sei $K$ die Menge, von der uns die Prozent interessieren. Dann gilt:
\[
w = K\cdot \frac{p}{100}
\]
Interessiert nur der Gesamtwert $K' = K+w$, dann kann folgendes berechnet werden:
\[
K' = K+w = K+K\cdot \frac{p}{100} = K\cdot \left(1+\frac{p}{100}\right)
\]
Wenn $p=19\%$ weiterhin die Mehrwertsteuer ist, dann würde der letzte Teil der Gleichung ausgerechnet folgendes ergeben:
\[
K' = K\cdot 1,19
\]
Die 19\% tauchen also hinter dem Komma der Zahl auf. Dies ist die einfachste Möglichkeit, zu einer beliebigen Zahl einen beliebigen Prozentwert hinzu zurechnen.

Betrachten wir den Fall eines Rabatt-Angebots. Hier werden vom Preis Prozente abgezogen. Das heißt, der Prozentwert $w$ ändert sich nicht. Nur wird er
\[
K' = K-w
\]
von der Menge abgezogen.
\[
K' = K-w = K-K\cdot \frac{p}{100} = K\cdot \left(1-\frac{p}{100}\right) 
\]
Gäbe ein Geschäft einen Rabatt von 30\%, so wäre 
\[
K' = K\cdot \left(1-\frac{30}{100}\right) = K\cdot (1-0,3) = K\cdot 0,7
\]

Um es noch etwas komplizierter zu machen gibt das Geschäft 30\% Rabatt auf den sogenannten Netto-Preis, d.h. der Preis ohne Mehrwertsteuer. Es sei nun $K$ der Preis der Ware ohne Mehrwertsteuer. Das bedeutet, wir ziehen zunächst 30\% ab und rechnen dann 19\% Mehrwertsteuer dazu, damit wir wissen, was wir endlich bezahlen müssen:
\begin{eqnarray*}
K' &=& (K-30\% ) + 19\% = \left( K\cdot \left( 1-\frac{30}{100} \right) \right)\cdot \left( 1+\frac{19}{100} \right)\\
&=& K\cdot \left( 0,7 \cdot 1,19 \right)\\
&=& K\cdot 0,833
\end{eqnarray*}
Das bedeutet, dass der Unterschied zwischen dem ursprünglichen Netto-Preis und dem, was wir bezahlen müssen, 16,7\% beträgt.

\subsection{Prozentrechnung bei Krediten}

Wenn wir einen Kredit nehmen, verschulden wir uns bei unserer Bank. Das bedeutet, die Bank gibt uns Geld, das wir in Raten wieder zurückzahlen müssen. Natürlich macht die Bank dies nicht kostenlos. Sie verlangt dafür Kreditzinsen in Form von Prozenten. Aktuell übliche Kreditzinsen liegen bei 8\% - 11\%. Aber worauf werden die Zinsen verlangt?

Sehen wir uns das genauer an: Stellen wir uns vor, wir möchten einen Studien-Urlaub zu den verwunschenen Orten der Maya nach Süd-Amerika machen. Ein Reiseveranstalter verlangt für eine solche Reise \currency 5.000,-. Wir haben nicht genug Geld auf unserem Konto, möchten diese Reise aber unbedingt machen. Also fragen wir die Bank nach einem Kredit. Nach einer Prüfung willigt diese ein und leiht uns das Geld zu einem Zinssatz von 8\%. Das bedeutet, dass die Bank jeden Monat ein zwölftel dieses Zinssatzes auf das Geld aufschlägt, das wir zum Anfang des Monats der Bank noch schulden. Das bedeutet, die Bank erhöht den Betrag, den wir ihr schulden jeden Monat um $\frac{8}{12}$\%.

Gleichzeitig tilgen\footnote{Tilgung = Rückzahlung} wir den Gesamtbetrag mit unseren Monatsraten um einen bestimmten Teilbetrag. Damit wir den Kredit überhaupt irgendwann zurückzahlen können, muss demnach der Teilbetrag, den wir jeden Monat zurückzahlen, höher sein, als der Betrag, der durch die Verzinsung dazu kommt.

Die Berechnung dazu sieht folgendermaßen aus, es sei $p=\frac{8}{12}\% $, $q = 1+\frac{p}{100}$ und $t$ unsere Tilgung. Dann entspricht der Restbetrag unseres Kredits $K_i$ in Monat $i$ nach beginn der Rückzahlung der folgenden Berechnung:
\[
K_i = \dots (((K-t)\cdot q-t)\cdot q-t)\cdot q \dots
\]
Die Punkte bedeuten, dass wir die Operation $(. -t)\cdot q$ genau $i$-Mal durchführen müssen.

Die Bank gewährt uns diesen Kredit bei einer monatlichen Rate von \currency 220,-. Die Bank hat die Rate so gewählt, dass wir in 24 Monaten den Kredit abbezahlt haben. Der ersten zwei Monate berechnen sich wie folgt:

\begin{eqnarray*}
K_1 &=& (5000-t)\cdot q = (5000 -220)\cdot 1,00\bar{6} = 4.811,87 \\
K_2 &=& (4.811,87 -t) \cdot q = (4.811,87 - 220)\cdot 1,00\bar{6} = 4.622,48
\end{eqnarray*}
und so weiter. 

Der sogenannte Tilgungsplan, also die Aufstellung aller Restbeträge für jeden der 24 Monate, ist in der folgenden Tabelle dargestellt:

\begin{center}
\begin{tabular}{C{4cm}C{4cm}}
\hline
\textbf{Monat} & \textbf{Restbetrag} \\
\hline
1& 4.811,87 \currency  \\
2&	 4.622,48 \currency  \\
3&	 4.431,83 \currency  \\
4&	 4.239,91 \currency  \\
5&	 4.046,71 \currency  \\
6&	 3.852,22 \currency  \\
7&	 3.656,43 \currency  \\
8&	 3.459,34 \currency  \\
9&	 3.260,94 \currency  \\
10&	 3.061,21 \currency  \\
11&	 2.860,15 \currency  \\
12&	 2.657,75 \currency  \\
13&	 2.454,01 \currency  \\
14&	 2.248,90 \currency  \\
15&	 2.042,42 \currency  \\
16&	 1.834,57 \currency  \\
17&	 1.625,34 \currency  \\
18&	 1.414,71 \currency  \\
19&	 1.202,67 \currency  \\
20&	 989,22 \currency  \\
21&	 774,35 \currency  \\
22&	 558,05 \currency  \\
23&	 340,30 \currency  \\
24&	 121,10 \currency  \\
\hline
\end{tabular}
\end{center}

\bigskip

\noindent Die Restrate beträgt nicht mehr die vollen \currency 220,-, sondern nur noch \currency 121,10. 

Weil der monatliche Betrag, den wir abbezahlen immer konstant ist -- bis auf den letzten Monat --, die Zinsen aber abnehmen, da der verzinste Betrag abnimmt, erhöht sich der Betrag, den wir zurückzahlen in jedem Monat. Es ist üblich, die Rückzahlung konstant zu halten. Es wäre unpraktisch, die Rate monatlich anzupassen, da der Kunde dann nie genau wüsste, wie viel er zu überweisen hätte (ungeachtet der Tatsache, dass Kreditraten im allgemeinen automatisch abgebucht werden).



\section{Dreisatz}

Der Dreisatz ist kein Mathematischer Satz, wie wir sie später kennenlernen werden. Er beschreibt eine Vorgehensweise zum Lösen von Verhältnisaufgaben. Diese Verhältnisse treten in zwei Variationen auf:

\begin{enumerate}
\item $a\div b = c\div x$, mit $a,b,c$ gegeben und $x$ gesucht.
\item $a\cdot b = c\cdot x$ mit $a,b,c$ gegeben und $x$ gesucht.
\end{enumerate}

Als eindrucksvolles Beispiel sei hier das \emph{Kartoffel-Paradoxon}\index{Kartoffel-Paradoxon} erwähnt, welches den Dreisatz mit der Prozentrechnung verbindet. 

\begin{quote}
Ein Bauer erntet 100kg Kartoffeln und lagert sie über den Winter ein. Zum Zeitpunkt der Ernte bestehen sie zu 99\% aus Wasser. Im Frühjahr möchte er sie verkaufen und bringt sie zum Markt. Zu diesem Zeitpunkt bestehen sie nur noch zu 98\% aus Wasser, während sich der Trockenanteil nicht verändert hat. Wieviel kg Kartoffeln bringt er zum Markt?
\end{quote}

Die meisten Menschen schätzen, wenn sie die Aufgabe zum ersten Mal hören, dass sich das Gesamtgewicht der Kartoffeln nicht wesentlich verändert hat. Also auf knapp 99kg. Sehen wir uns das genauer an und berechnen die Trockenmasse, also den Wert, der sich nicht verändert:

\begin{equation}\label{eq:3satz1}
\text{Trockenmasse: } 1\text{kg} = 100\text{kg} \cdot 1\% = 100\text{kg} \cdot \frac{1}{100}
\end{equation}
Die Trockenmasse der Kartoffeln kann sich über den Winter nicht ändern, also bleibt diese erhalten und wir suchen das Gesamtgewicht von dem 2\% gerade 1kg sind:
\begin{equation}
\text{Trockenmasse: } 1\text{kg} = ?\text{kg} \cdot 2\% = ?\text{kg} \cdot \frac{2}{100}
\end{equation}
Wir formen um zu 
\begin{equation}\label{eq:3satz3}
\frac{1\text{kg}\cdot 100}{2} = 50\text{kg}
\end{equation}
das bedeutet, dass sich das Gesamtgewicht der Kartoffeln halbiert hat, obwohl sich der Wasseranteil nur um ein Prozent verringert hat. Auf der anderen Seite wird es einsichtiger, wenn man bedenkt, dass sich der Anteil der Trockenmasse verdoppelt hat, obwohl das Gewicht der Trockenmasse gleich geblieben ist.

Die Interpretation, warum der Dreisatz "`Drei"'-Satz heißt, sind uneinheitlich. Favorisierte Ideen sind, a) dass es drei Werte und eine Unbekannte gibt, und b) dass drei Schritte zur Lösung notwendig sind. Wir schließen uns hier der Interpretation b) an. Die Gleichungen (\ref{eq:3satz1}) bis (\ref{eq:3satz3}) stellen somit die drei Schritte des Dreisatzes dar. Hier ein einfacheres Beispiel zu einem Dreisatz erster Form:

\begin{quote}
Wir kaufen ein Netz mit 25 Orangen, es wiegt 5kg. Wie viele Orangen sind in einem Netz mit 3kg?
\end{quote}
Um die Anzahl der Orangen in einem 3kg Netz herauszubekommen, müssen wir wissen, wie viel eine Orange wiegt, denn wir haben nur das Gewicht 3kg als Anhaltspunkt. Deshalb müssen wir die 5kg durch die 25 Orangen teilen um das Gewicht einer Orange zu bestimmen. Haben wir dies, können wir die Gleichung aufstellen, dass 3kg geteilt durch eine uns unbekannte Anzahl von Orangen das gleiche ergeben muss, wie die Division von 5kg durch die 25 Orangen. Somit wird die Berechnung wieder in drei Schritten dar gestellt:
\begin{equation*}
\begin{split}
\frac{5kg}{25} &=  200g \\
\frac{3kg}{?} &= 200g \\
\frac{3000g}{200g} &= 15
\end{split}
\end{equation*}

Und noch ein Beispiel für einen Dreisatz zweiter Form:
\begin{quote}
Die Arbeiter in einer Fabrik für Autozubehör nehmen sich aus einem Regal Kisten mit Einzelteilen, die zu Armlehnen zusammengebaut werden. Die Zufuhr von Kisten ist nicht beschränkt, das bedeutet, es sind immer genug Kisten für alle Arbeiter vorhanden. 

Zehn Arbeiter können eine vorgegebene Menge an Armlehnen innerhalb eines Arbeitstages (acht Stunden) zusammenbauen. Eines Tages meldet sich morgens bei Schichtbeginn der Kunde und möchte noch am selben Tag eine Tagesration an Armlehnen haben. Um die Armlehnen zum Kunden zu bringen, braucht ein LKW sechs Stunden. Wie viele Arbeiter muss der Zulieferer zusätzlich einsetzen, um den Kunden zufrieden zu stellen?
\end{quote}
Die Arbeitsleistung der Arbeiter ist nicht bekannt, wir können also nicht wie mit den Orangen vorgehen und zunächst die Arbeitsleistung eines einzelnen Arbeiters ausrechnen. Wir müssen die Gleichung sofort ansetzen. Erschwerend kommt hinzu, dass wir nicht wissen, wie lange Zeit den Arbeiten bleibt. Daher müssen wir uns überlegen, was die Länge des Tages beeinflusst. Es bleibt nur die Lieferung per LKW, die mit sechs Stunden zu Buche schlägt. Demnach bleiben nur zwei Stunden für die Herstellung von soviel Armlehnen, wie 10 Arbeiter sonst in acht Stunden herstellen.
\begin{equation*}
\begin{split}
10\cdot 8 &= ? \cdot 2 \\
\frac{10\cdot 8}{2} &= ? = 40
\end{split}
\end{equation*}
In diesem Fall gibt es auch drei Gleichungen, nämlich die linke Seite und die rechte Seite der ersten Gleichung entsprechen den Schritten 1 und 2 des Dreisatzes. Und die zweite Gleichung dem Schritt 3. Das ergibt die Lösung 40, sprich der Zulieferer muss 40 Arbeiter einsetzen, damit diese in zwei Stunden vollbringen, was sonst 10 Arbeiter in acht Stunden erbringen.

Allgemein geht man beim Dreisatz wie folgt vor:
\begin{enumerate}
\item Aufstellung der Gleichung mit den bekannten Werten.
\item Aufstellung der Gleichung mit einem unbekannten Wert.
\item Umformung und Lösung.
\end{enumerate}
Bei der Aufstellung der Gleichung gibt es keine allgemeine Vorgehensweise. Der Lernende muss erkennen können, ob der Dreisatz von erster oder zweiter Form ist. 

\section{Textaufgaben}

Textaufgaben spielen in der Schul-Mathematik eine sehr große Rolle. Anhand von textuell dargelegten Problemen soll Schülern ein intuitiver Einstieg in die Mathematik gegeben werden. Im Gegensatz dazu erwartet Studierende in der universitären Mathematik im allgemeinen keine Textaufgaben. Sie gelten somit lediglich als didaktisches Mittel zum Zweck der Motivation von Schülern. 

Wie im vorhergehenden Abschnitt bereits gesehen, sind Dreisatzaufgaben oft in Form von Textaufgaben vorgegeben, sodass der Lernende sich mit Alltagsproblemen konfrontiert sieht. 

Aufgrund der Tatsache, dass Textaufgaben keinen einheitlichen Lösungsweg haben und sie direkt von der Formulierung abhängen, ist in diesem Abschnitt auch nicht viel Weiteres darüber zu sagen, als dass sie existieren und Schüler sich damit auseinander setzen müssen. Beispielaufgaben sind im folgenden Abschnitt zu finden.

\section{Aufgaben}

TODO

% % Insert some special declarations 

\part{Lineare Algebra}


\chapter{Grundlagen}

Die Analysis, auch Infinitesimalrechnung genannt, wurde von Gottfried Wilhelm Leibnitz\footnote{\textbf{Gottfried Wilhelm Leibnitz}, *1. Juli 1646 in Leibzig; \ding{61}14. November 1716 in Hannover} und Isaac Newton\footnote{\textbf{Sir Isaac Newton}, *4. Januar 1643 in Woolsthorpe-by-Colsterworth in Lincolnshire; \ding{61}31. März 1727 in Kensington} unabhängig voneinander entwickelt. Sie behandelt Grenzwerte von Folgen und Reihen, Funktionen reeller Zahlen, deren Stetigkeit, Differenzierbarkeit und Integrierbarkeit. 

\section{Folgen}

Folgen sind ein zentraler Begriff in der Mathematik, im Besonderen der Analysis. Die Betrachtung von Folgen und ihrem Verhalten, hat viele Bereiche der Mathematik inspiriert, so wäre die Funktionentheorie ohne die Banach\footnote{\textbf{Stefan Banach}, polnischer Mathematiker,  *30. März 1892 in Krakau, \ding{61}31. August 1945 in Lemberg}-Räume kaum vorstellbar. \index{Banach Raum}\index{Banach, Stefan} Banachräume sind bestimmte Vektorräume, in denen jede Cauchy-Folge gegen einen im Raum enthaltenen Wert konvergiert.

\begin{definition}
Eine \textsl{Folge} ist eine -- wie der Name suggeriert -- Abfolge von Mathematischen Objekten. Dies können zum Beispiel Zahlen, Vektoren oder Funktionen sein, oder jedes andere Mathematische Objekt, solange sich diese in eine Reihenfolge bringen lassen. Bei Folgen ist eine Reihenfolge der Elemente festgelegt. Sind $a_i$ die Elemente einer Folge, so ist immer $a_{i-1}$ vor $a_i$ und diesem folgt immer $a_{i+1}$ -- sofern $a_{i\pm 1}$ existieren. Eine Folge wird in dieser Art dargestellt:
\[
(a_1, a_2, \dots)
\]
\end{definition}

Folgen sind keine Mengen, in Folgen können Elemente mehrfach vorkommen, an unterschiedlichen Positionen in der Abfolge der Elemente! 

Folgen haben unendlich viele Elemente. Sie sind mit den natürlichen Zahlen abzählbar unendlich (siehe Definition \ref{abzaehlbar}).

\subsection{Charakterisierung}

\begin{definition}
Eine Folge heißt \textsl{monoton steigend}, wenn für alle $i\in \mathbb{N}$ gilt $a_i\le a_{i+1}$. Sie heißt \textsl{streng monoton steigend}, wenn $a_i < a_{i+1}$ gilt. Äquivalent heißt sie \textsl{monoton fallend}, wenn $a_i\ge a_{i+1}$ gilt, bzw. \textsl{streng monoton fallend}, wenn $a_i > a_{i+1}$ gilt.
\end{definition}

Monotonie kann auch erst ab einem gewissen $n\in \mathbb{N}, n>1$ nachweisbar sein. Die Folge wird dann als monoton ab dem Folgenglied $n$ bezeichnet.

\begin{definition}
Eine reelle Folge ($a_i \in \mathbb{R}$) heißt \textsl{beschränkt}, wenn es ein $S\in \mathbb{R}$ gibt, sodass für alle $i\in \mathbb{N}$, $a_i \le S$ gilt. $S$ ist eine \textsl{obere Schranke} der Folge. Gibt es eine kleinste obere Schranke $S'$, so wird diese \textsl{Supremum} genannt. 

Gibt es ein $I\in \mathbb{R}$ mit $a_i \ge I$, so wird $I$ \textsl{untere Schranke} genannt. Die größte untere Schranke wird \textsl{Infimum} genannt.
\end{definition}

\begin{definition}
Eine Folge, deren Elemente alle identisch sind, wird \textsl{konstante Folge} genannt. Eine Folge, deren Elemente abwechselnd positiv und negativ sind, wird \textsl{alternierend} genannt. Eine Folge deren Elemente sich immer weiter der 0 annähern, wird \textsl{Nullfolge} genannt.
\end{definition}

\subsection{Konvergenz}

\begin{definition}\label{def:lim}
Es sei $(a_i)$ eine Folge in den reellen Zahlen und $a\in \mathbb{R}$ eine \textsl{Grenzwert} genannte Zahl. Weiterhin sei $\epsilon>0$ eine beliebige reelle Zahl. Wenn es für jeden solchen Wert $\epsilon$ einen Folgenindex $n_\epsilon \in \mathbb{N}$ gibt, sodass $\vert a_i -a\vert <\epsilon$ für alle $i\ge n_\epsilon$ gilt, so heißt die Folge \textsl{kovergent}. Des Weiteren konvergiert diese Folge gegen $a$. 
\end{definition}

Man sollte sich das $\epsilon$ als eine sehr kleine Zahl vorstellen. Für große Zahlen sind die Bedingungen der Konvergenz recht einfach zu erfüllen. Schwierig wird es erst, wenn $\epsilon\ll 1$ ist. ($\ll$ bedeutet "`wesentlich kleiner als"').

Die oben dargestellte Definition für Konvergenz beruht darauf, dass der Punkt $a$ bereits bekannt ist. Was ist aber, wenn wir $a$ nicht kennen? Gibt es dann auch ein Kriterium für die Konvergenz?

\begin{definition}
Es sei $(a_i)$ eine Folge. Es sei $\epsilon \ll 1$. Wenn es zu jedem $\epsilon$ ein $N\in \mathbb{N}$ gibt, sodass für $n,m > N$ gilt $\vert a_n -a_m \vert <\epsilon$, so ist auch diese Folge konvergent. Bemerkenswert dabei ist, dass wir nur die Abstände der Folgenelemente dafür betrachten und keinen Grenzwert. Eine solche Folge wird \textsl{Cauchy-Folge}\footnote{benannt nach dem französischen Mathematiker \textbf{Augustin-Louis Cauchy}, *21. August 1789, \ding{61}23. Mai 1857 in Sceaux} genannt.
\end{definition}

Cauchy-Folgen haben die positive Eigenschaft, dass man ihre Konvergenz nachweisen kann, ohne den Grenzwert zu kennen. Z.B. in den rationalen Zahlen $\mathbb{Q}$ gibt es Cauchy-Folgen, die keinen Grenzwert besitzen, weil dieser in den irrationalen Zahlen liegt.


\begin{definition}
Zu einer Folge $a_i$ heißt die Menge $\lbrace a_i \rbrace$ die \textsl{Menge der Folgenwerte}. Es wird der Einfachheit halber eine Folge auch oft mit der Menge ihrer Folgenwerte gleichgesetzt. Dies ist aber nicht vollständig korrekt, da eine Folge auch Informationen über die Reihenfolge ihrer Elemente enthält, die Menge der Folgenwerte aber nicht. 
\end{definition}

\section{Grenzwert}

In Definition \ref{def:lim} wurde bereits der Begriff \textsl{Grenzwert} verwendet. Hier soll er noch einmal formal dargestellt werden. 

\begin{definition}
Sei $(a_i)$ eine Folge in $A\subseteq \mathbb{R}$. Wenn die Folge gegen einen Wert $a\in A$ konvergiert, so heißt $a$ \textsl{Grenzwert}, oder auch \textsl{Limes}\footnote{Limes lat. "`Grenzweg"'} der Folge $(a_i)$. Dass $(a_i)$ den Grenzwert $a$ besitzt, wird wie folgt ausgedrückt:
\[
\lim_{n\rightarrow \infty} a_n = a
\]
\end{definition}

\begin{definition}\label{def:voll}
Wenn jede Cauchy-Folge mit $(a_n)\in A\subseteq \mathbb{R}$ konvergiert und ihr Grenzwert 
\[\lim_{n\rightarrow \infty} a_n = a\in A\] 
existiert, so heißt $A$ \textsl{vollständig}. 

Die Vollständigkeit der reellen Zahlen in der Form, dass alle Cauchy-Folgen konvergieren und ihren Grenzwert annehmen, ist eine Konsequenz aus dem sogenannten \textsl{Vollständigkeitsaxiom}. Die Vollständigkeit folgert also nicht aus der Konvergenz der Cauchy-Folgen, sondern genau umgekehrt folgert die Konvergenz der Cauchy-Folgen aus der Vollständigkeit. 
\end{definition}


\section{Reihen}

\begin{definition}
Sei $(a_i)$ eine Folge, so ist 
\[
s_n = \sum_{i=1}^{n} a_i
\]
die $n$-te \textsl{Partialsumme}. Die Partialsummen bilden wiederum eine Folge $(s_i)$. Falls der Grenzwert 
\[
R(a) = \lim_{n\rightarrow \infty} s_n = \lim_{n\rightarrow \infty} \sum_{i=1}^{n} a_i
\]
existiert, so heißt $R(a)$ eine \textsl{Reihe}.
\end{definition}

Ob eine Reihe existiert, bestimmt die Partialsummen-Folge. Konvergiert diese, so existiert der Grenzwert, und damit auch die Reihe. Aus der Definition der Partialsummen folgt, dass wenn die Partialsummen-Folge konvergiert, $(a_n)$ eine Null-Folge sein muss. Falls $(a_n)$ keine Null-Folge ist, divergieren die Partialsummen und der Grenzwert  existiert nicht. Allerdings reicht es nicht aus $(a_n)$ als Null-Folge zu wählen, es gibt Null-Folgen, deren Partialsummen trotzdem divergieren. 

Ein Beispiel für eine nicht konvergente Summe aus einer Null-Folge ist die Harmonische Reihe:
\[
H_n = \sum_{i=1}^{n} \frac{1}{i}
\]
Die Folge $\lbrace 1, \frac{1}{2}, \frac{1}{3}, \dots \rbrace$ ist eine Null-Folge, aber die $H_n$ sind divergent.

\begin{definition}
Eine Reihe heißt \textsl{absolut konvergent}, wenn 
\[
R'(a) = \lim_{n\rightarrow \infty} \sum_{i=1}^{n} \vert a_i \vert
\]
konvergiert.
\end{definition}

\begin{lemma}
Falls zu jedem $\epsilon \ll 1$ ein Index $N$ existiert, sodass für alle $n>N$ gilt
\[
\left\vert \frac{a_{n+1}}{a_n} \right\vert \le \epsilon
\]
dann konvergiert die Reihe 
\[
\lim\limits_{n\rightarrow \infty} \sum_{i=1}^{n} \vert a_i \vert
\]
Sie ist also absolut konvergent. Dies wird als \textsl{Quotienten-Kriterium} bezeichnet.
\end{lemma}
\begin{proof}
TODO
\end{proof}


\section{Stetigkeit}

Es sei $A\subseteq \mathbb{R}$. Wir betrachten alle Folgen $(a_i)\in A$, die einen Grenzwert $a\in A$ besitzen, sowie eine Funktion $f: A\longrightarrow \mathbb{R}$. 

\begin{definition}
Die Funktion $f$ heißt \textsl{stetig}\index{stetig}, wenn für jede Folge $(a_i)$ mit Grenzwert $a$ auch die Folge $(f(a_i))$ gegen $f(a)$ konvergiert. Formal definiert
\[
\lim_{n\rightarrow \infty} f(a_n) = f(\lim_{n\rightarrow \infty} a_n) = f(a)
\]
\end{definition}

Diese Definition der Stetigkeit wird auch \textsl{Folgenstetigkeit}\index{Folgenstetigkeit} genannt. Eine Definition der Stetigkeit unabhängig von Folgen ist die folgende: 

\begin{definition}
$f$ ist genau dann in $a$ \textsl{stetig}, wenn es zu jedem $\epsilon>0$ ein $\delta_\epsilon > 0$ gibt, so dass für alle $b\in A$ mit $\vert b-a\vert < \delta_\epsilon $ immer $\vert f(b)-f(a)\vert < \epsilon$ ist. Diese Definition wird meist $\delta$-$\epsilon$-Stetigkeit genannt. \index{$\delta$-$\epsilon$-stetig}
\end{definition}

Wir hatten bisher versucht, mehrere Definitionen für einen Begriff zu vermeiden. Allerdings passiert es, dass die Stetigkeitseigenschaft in unterschiedlichen Zusammenhängen ausgenutzt werden muss. Daher ist es zuweilen auch notwendig, unterschiedliche Definitionen heranzuziehen, um in einem Beweis die Stetigkeit prägnant auszunutzen. Stetigkeit ist noch auf weitere Arten definierbar: Z.B. im topologischen Sinne, dass unter stetigen Abbildungen die Urbilder offener Mengen wieder offene Mengen sind. Aber das würde an dieser Stelle zu weit führen.







\chapter{Vektor, Matrix, Tensor}



\section{Vektorraum}\label{vectorspace}
\index{Vektorraum}

Sei $(K,+,\cdot )$ ein Körper und $(V,+ ,\cdot)$ eine Menge mit zwei Verknüpfungen:
\begin{eqnarray*}
+ : V\times V &\longrightarrow& V \\
\cdot : K \times V &\longrightarrow& V 
\end{eqnarray*}
Hierbei ist zu beachten, dass die Verknüpfungen einmal auf $K$ und einmal auf $V$ definiert wurden. Man könnte hier verschiedene Zeichen zur Unterscheidung der Additionen einführen. Dies ist im allgemeinen aber unüblich, sodass hier vom Lernenden verlangt wird, den Unterschied selbst zu erkennen. Auf $V$ wird die Addition als \textsl{Vektoraddition} bezeichnet, sowie die Multiplikation als \textsl{Skalare Multiplikation}, da sie nicht zwischen zwei Vektoren, sondern einem Element des zugrunde liegenden Körpers und einem Vektor definiert ist. 

Man nennt $(V,+ ,\cdot)$ einen \textsl{Vektorraum}\index{Vektorraum} über dem Körper $K$, oder auch \textsl{K-Vektorraum}, wenn folgende Eigenschaften erfüllt sind:

\noindent Es seien $u,v,w \in V$ und $a,b \in K$:

Die Anforderungen an den Vektorraum werden in solche, die an die Addition (A*) und solche, die an die Multiplikation (M*) gestellt werden unterschieden:

\begin{description}
\item[(A1)] Assoziativgesetz: $u+(v+w) = (u+v)+w$
\item[(A2)] $0$ Element: $0\in V$ mit $v+0=0+v=v$
\item[(A3)] Inverses Element zur Addition: Zu jedem $v\in V$ gibt es ein $-v\in V$ mit $v+(-v) = -v+v = 0$
\item[(A4)] Kommutativgesetz: $u+v = v+u$
\end{description}

\begin{description}
\item[(M1)] Assoziativgesetz: $(a \cdot b)\cdot v = a\cdot (b\cdot v)$
\item[(M2)] 1 Element: $1\cdot v = v$
\item[(M3)] Distributivgesetz: $(a+b)\cdot v = a\cdot v + b\cdot v$
\item[(M4)] Distributivgesetz: $a\cdot(v+w) = a\cdot v + a\cdot w$
\end{description}

Es ist wichtig zu verstehen, wo die Unterschiede zwischen (M3) und (M4) liegen. (M3) besagt, dass die Summe zweier Zahlen in $K$ auf den Vektor $v$ verteilt (distribuiert) werden kann, während (M4) besagt, dass die Summe zweier Elemente aus $V$ auf $a$ verteilt werden kann. Das heißt: (M3) ist eine Eigenschaft der Addition in $K$ während (M4) eine Eigenschaft der Addition aus $V$ ist! 

\begin{svgraybox}
Zur Vertiefung der sei an dieser Stelle auf das grundlegende Werk von Egbert Brieskorn \cite{Brieskorn1} hingewiesen. 
\end{svgraybox}

\begin{definition}
Ein Element eines Vektorraums wird \textsl{Vektor} genannt.
\end{definition}

\subsection{Linearkombination}

\begin{definition}
Eine beliebige Summe von $p$ Vektoren $v_i \in V$ mit Faktoren $a_i \in K$ wird als \textsl{Linearkombination} bezeichnet.
\[
l = \sum_{i=1}^{p} a_i \cdot v_i = a_1\cdot v_1 + a_2 \cdot v_2 + \dots + a_p \cdot v_p
\]
Die Summe ist wiederum ein Vektor in $V$.

\end{definition}

\subsection{Basis}

Der Begriff der Basis ist an dieser Stelle noch nicht einführbar, ohne weitere Begriffe definiert zu haben, die aktuell von wenig Nutzen sind und den Lernenden eher verwirren. Daher beschränken wir uns aktuell nur auf eine Basis:

\begin{definition}
Als die \textsl{Standard-Basis} des $\mathbb{R}^n$ werden die Vektoren $e_i$ mit $i=1,\dots,n$ bezeichnet, deren Einträge überall null sind, bis auf den Eintrag $i$ und dieser ist 1. 
\begin{equation}
e_1 = \begin{pmatrix}
1\\
0\\
\vdots \\
0
\end{pmatrix}, e_2 = \begin{pmatrix}
0\\
1\\
\vdots \\
0
\end{pmatrix}, \dots, e_n = \begin{pmatrix}
0\\
0\\
\vdots \\
1
\end{pmatrix}
\end{equation}
\end{definition}

Jeder Vektor in $v\in V$ kann als Linearkombination dieser Basis geschrieben werden. Im Besonderen ist -- aufgrund der Wahl der Basisvektoren $e_i$ -- jeder Faktor dieser Linearkombination identisch mit dem $i$-ten Eintrag des Vektors. 

\[
v = \begin{pmatrix}
v_1 \\
v_2 \\
\vdots \\
v_n
\end{pmatrix} = \sum_{i=1}^{n} v_i \cdot e_i
\]

\subsection{Lineare Unabhängigkeit}

\begin{definition}
Die \textsl{linare Unabhängigkeit} \index{Unabhängigkeit, linear} von Vektoren wird definiert durch eine Linearkombination, d.h. eine Summe von Vektoren mit konstanten Faktoren. Wenn diese Linearkombination nur dadurch zu einem Null Vektor gemacht werden kann, dass alle Faktoren zu 0 werden, dann sind die Vektoren linear unabhängig. Seien $x_i$ aus einem K-Vektorraum und $\lambda_i $ aus dem zugrunde liegenden Körper. 

\begin{equation}\label{eq:linunabh}
\sum_{i=1}^{n} \lambda_i x_i = 0 \quad \text{dann, und nur dann, wenn} \quad \lambda_i = 0
\end{equation}
\end{definition}

Dies sollte näher erklärt werden: Nehmen wir ein einfaches Beispiel und sehen uns zwei Vektoren in der Ebene an. 

\bigskip

\begin{center}
\begin{tikzpicture}[>=stealth]
\coordinate (A) at (0,0);
\coordinate (B) at (3,1);
\coordinate (C) at (1,3);
\coordinate (D) at (-2,2);
\tikzset{-}
\draw[lightgray, step=1cm] (-3,-1) grid (4,4);
\tikzset{-}
\draw[lightgray, dashed] (D) -- (C);
\draw[lightgray, dashed] (B) -- (C);
\tikzset{->}
\draw (A) -- (B) node[midway, sloped, above] {$\lambda_1 \cdot x_1$};
\draw (A) -- (D) node[midway, sloped, above] {$\lambda_2 \cdot x_2$};
\draw[line width=2] (A) -- (C) node[midway, sloped, above] {$\lambda_1 \cdot x_1+\lambda_2 \cdot x_2$};
\end{tikzpicture}
\end{center}

\bigskip

Per Definition der linearen Unabhängigkeit kann der dicke Pfeil nur dann die Länge 0 haben, wenn $\lambda_1$ und $\lambda_2$ beide 0 sind. Salopp gesprochen könnte man sagen: Der Durchmesser eines Parallelograms\footnote{Ein Viereck dessen gegenüberliegenden Seiten parallel zueinander sind, nennt man Parallelogramm.} -- wie das, welches die Vektoren aufspannen -- kann nur dann 0 sein, wenn die Seitenlängen 0 sind.


\section{Lineare Gleichungssysteme}

Lineare Gleichungssysteme treten häufig in der Physik und den Ingenieurswissenschaften auf. Dabei handelt es sich um Probleme in mehreren Unbekannten, die durch eine Reihe von Randbedingungen -- in Form von Gleichungen -- spezifiziert werden.

Bei diesen Problemen ist es notwendig, dass alle Unbekannte die Randbedingungen gleichzeitig erfüllen:

\begin{equation}\label{eq:syseq}
\begin{split}
a_{1,1}x_1 + a_{1,2}x_2 + \dots + a_{1,n}x_n &= b_1 \\
a_{2,1}x_1 + a_{2,2}x_2 + \dots + a_{2,n}x_n &= b_2 \\
a_{3,1}x_1 + a_{3,2}x_2 + \dots + a_{3,n}x_n &= b_3 \\
\vdots &= \vdots \\
a_{m,1}x_1 + a_{m,2}x_2 + \dots + a_{m,n}x_n &= b_m 
\end{split}
\end{equation}
Die Werte für $a_{i,j}$ und $b_{i}$ sind vorgegeben. Gesucht werden die $x_j$ Werte. 

Gleichungssysteme werden nach der Anzahl ihrer Gleichungen in folgende Kategorien eingeteilt:

\begin{description}
\item[$m<n$] Solche Systeme bezeichnet man als \textsl{unterbestimmt}, es existieren weniger Gleichungen als Unbekannte. Die Lösung solcher Systeme ist im allgemeinen nicht eindeutig. Solche Gleichungssysteme treten in der linearen Optimierung auf. 
\item[$m>n$] Solche Systeme bezeichnet man als \textsl{überbestimmt}, es existieren mehr Gleichungen als Unbekannte. Es ist im allgemeinen nicht möglich eine Lösung für solche Systeme anzugeben, da es möglich ist, sich widersprechende Randbedingungen in den Gleichungen zu formulieren. 
\item[$m=n$] Auch als quadratische Gleichungssysteme bezeichnet. Sofern bestimmte Bedingungen von den $a_{i,j}$ Werten erfüllt werden, gibt es genau eine Lösung. Dies sind die Systeme, mit denen wir uns hier beschäftigen werden.
\end{description}

\subsection{Kompakte Notation}

Mathematiker machen sich gerne das Leben einfacher, in dem sie ihre Formeln abkürzen. Das Gleichungssystem (\ref{eq:syseq}) ist sehr unhandlich. Daher führen wir folgende Schreibweisen ein:

Wir schreiben die $x_j$ und $b_i$ Werte in einer Spaltenform: 

\[
x = \begin{pmatrix}
x_1 \\
x_2 \\
\vdots \\
x_n
\end{pmatrix}, \quad b = \begin{pmatrix}
b_1 \\
b_2 \\
\vdots \\
b_m
\end{pmatrix}
\]
Sowie für die $a_{i,j}$ Werte eine rechteckige Tabellenform:

\[
A = \begin{pmatrix}
a_{1,1} & a_{1,2} & \cdots & a_{1,n} \\
a_{2,1} & a_{2,2} & \cdots & a_{2,n} \\
\vdots & \vdots & \ddots & \vdots \\
a_{m,1} & a_{m,2} & \cdots & a_{m,n}
\end{pmatrix}
\]

Definieren wir nun eine Multiplikation zwischen der rechteckigen Tabellenform und einer Spaltenform auf diese Weise:
\begin{equation*}
a_{i,1}x_1 + a_{i,2}x_2 + \dots + a_{1,n}x_n = \sum_{j=1}^{n} a_{i,j}x_j
\end{equation*}
für jede Zeile $i$, so lässt sich das Gleichungssystem (\ref{eq:syseq}) schreiben als:

\begin{equation*}
\begin{pmatrix}
a_{1,1} & a_{1,2} & \cdots & a_{1,n} \\
a_{2,1} & a_{2,2} & \cdots & a_{2,n} \\
\vdots & \vdots & \ddots & \vdots \\
a_{m,1} & a_{m,2} & \cdots & a_{m,n}
\end{pmatrix} \cdot \begin{pmatrix}
x_1 \\
x_2 \\
\vdots \\
x_n
\end{pmatrix} = \begin{pmatrix}
b_1 \\
b_2 \\
\vdots \\
b_m
\end{pmatrix}
\end{equation*}
oder
\begin{equation*}
Ax=b
\end{equation*}

Hier haben wir uns noch keine Gedanken darüber gemacht, was diese Spalten und Tabellen eigentlich sind. Sie stellen für uns lediglich eine Vereinfachung der Schreibweise dar. Später werden wir erkennen, dass die Spalten Vektoren sind und die Tabelle eine Matrix.


\subsection{Matrix}

\begin{definition}
Eine \textsl{Matrix} \index{Matrix} ist eine rechteckige Struktur von Elementen des Körpers, über dem sie gebildet werden. Hier werden im allgemeinen nur reelle Matrizen betrachtet. Somit sind diese Matrizen aus dem $\mathbb{R}^{m\times n}$:
\end{definition}

\begin{equation*}
(A)_{i,j} = a_{i,j} \in \mathbb{R}
\end{equation*}

Mit der Matrix Addition und der Skalaren Multiplikation
\begin{eqnarray*}
A+B &=& (a_{i,j} + b_{i,j})_{i,j} \\
\alpha A &=& (\alpha a_{i,j})_{i,j}
\end{eqnarray*}
wird der $\mathbb{R}^{m\times n}$ zu einem Vektorraum. Das Nachrechnen der A1-A4 und M1-M4 Eigenschaften aus Kapitel \ref{vectorspace} ist Teil der Aufgaben.

Dabei ist zu beachten, dass $0 \in \mathbb{R}^{m\times n} = (0)_{i,j}$ die Null Matrix ist, und das neutrale Element der Multiplikation die Einsmatrix $\mathbf{1}$, oder auch Identität $I$ genannt:

\[
\mathbf{1} = I =
\begin{pmatrix}
1 & 0 & 0 & \cdots & 0 \\
0 & 1 & 0 & \cdots & 0 \\
0 & 0 & 1 & \cdots & 0 \\
\vdots & \vdots & \vdots & \ddots & \vdots \\
0 & 0 & 0 & \cdots & 1
\end{pmatrix}
\]

\subsubsection{Rechenregeln}

\begin{definition}
Die Zahlen in den einzelnen Zellen der Tabelle nennt man \textsl{Einträge} oder auch \textsl{Komponenten}. Das selbe gilt für einen Vektor, der als einspaltige Matrix interpretiert werden kann.
\end{definition}

Die Addition zweier Matrizen wird auf die einzelnen Einträge abgebildet. Als Abkürzung wird hier die Notation $(\dots)_{i,j}$ verwendet. Dies ist eine Darstellung des Eintrags an der $i$-ten Zeile und $j$-ten Spalte. Wenn keine Einschränkung an $i$ und $j$ gemacht werden, so gilt diese Abkürzung für jeden Eintrag der Matrix. Und somit bestimmt diese Abkürzung auch die gesamte Matrix.

\begin{equation*}
A + B = \left( a_{i,j} + b_{i,j} \right)_{i,j} = \begin{pmatrix}
a_{1,1}+b_{1,1} & a_{1,2}+b_{1,2} & \cdots & a_{1,n}+b_{1,n} \\
a_{2,1}+b_{2,1} & a_{2,2}+b_{2,2} & \cdots & a_{2,n}+b_{2,n} \\
\vdots & \vdots & \ddots & \vdots \\
a_{m,1}+b_{m,1} & a_{m,2}+b_{m,2} & \cdots & a_{m,n}+b_{m,n}
\end{pmatrix}
\end{equation*}
In gleicher Weise wird die skalare Multiplikation realisiert

\begin{equation*}
\alpha \cdot A = \left( \alpha \cdot a_{i,j} \right)_{i,j} = \begin{pmatrix}
\alpha \cdot a_{1,1} & \alpha \cdot a_{1,2} & \cdots & \alpha \cdot a_{1,n} \\
\alpha \cdot a_{2,1} & \alpha \cdot a_{2,2} & \cdots & \alpha \cdot a_{2,n} \\
\vdots & \vdots & \ddots & \vdots \\
\alpha \cdot a_{m,1} & \alpha \cdot a_{m,2} & \cdots & \alpha \cdot a_{m,n}
\end{pmatrix}
\end{equation*}

Die Multiplikation von Matrizen $A,B$ ist in der Form definiert, dass der Eintrag an der Stelle $i,j$ definiert ist durch die Summe der Produkte der Einträge der $i$-ten Zeile von A mit den Einträgen der $j$-ten Spalte von B. Sei $A\in \mathbb{R}^{m\times n}$ und $B\in \mathbb{R}^{n\times o}$

\begin{equation*}
A \cdot B = \left( \sum_{k=1}^{n} a_{i,k} \cdot b_{k,j} \right)_{i,j} = \begin{pmatrix}
\left( \sum_{k=1}^{n} a_{1,k} \cdot b_{k,1} \right) & \left( \sum_{k=1}^{n} a_{1,k} \cdot b_{k,2} \right) & \cdots & \left( \sum_{k=1}^{n} a_{1,k} \cdot b_{k,o} \right) \\
\left( \sum_{k=1}^{n} a_{2,k} \cdot b_{k,1} \right) & \left( \sum_{k=1}^{n} a_{2,k} \cdot b_{k,2} \right) & \cdots & \left( \sum_{k=1}^{n} a_{2,k} \cdot b_{k,o} \right) \\
\vdots & \vdots & \ddots & \vdots \\
\left( \sum_{k=1}^{n} a_{m,k} \cdot b_{k,1} \right) & \left( \sum_{k=1}^{n} a_{m,k} \cdot b_{k,2} \right) & \cdots & \left( \sum_{k=1}^{n} a_{m,k} \cdot b_{k,o} \right) \\
\end{pmatrix}
\end{equation*}

Aus dieser Definition folgt, dass die Anzahl der Spalten der Matrizen links vom Multiplikationszeichen und die Anzahl der Zeilen der Matrix rechts vom Multiplikationszeichen identisch sein müssen. Sind sie nicht identisch, ist die Mulitplikation \textbf{nicht} definiert und die Matrizen können nicht miteinander multipliziert werden. Des Weiteren besitzt die resultierende Matrix soviel Zeilen, wie Matrix $A$ aber soviel Spalten wie Matrix $B$.

\begin{svgraybox}
Hat Matrix $B$ nur eine Spalte -- ist also ein Vektor --, dann ist dadurch auch gleichzeitig die Matrix-Vektor Multiplikation definiert. 
\end{svgraybox}

Seien $A\in \mathbb{R}^{m\times n}$ und $b\in \mathbb{R}^n$, dann ist die Matrix-Vektor Multiplikation

\begin{equation*}
A \cdot b = \left( \sum_{k=1}^{n} a_{i,k} \cdot b_{k} \right)_{i} = \begin{pmatrix}
\left( \sum_{k=1}^{n} a_{1,k} \cdot b_{k} \right) \\
\left( \sum_{k=1}^{n} a_{2,k} \cdot b_{k} \right) \\
\vdots \\
\left( \sum_{k=1}^{n} a_{m,k} \cdot b_{k} \right) \\
\end{pmatrix}
\end{equation*}

\subsubsection{Transposition}

Die Transponierte einer Matrix erhält man durch vertauschen der Indizes. Sei $A \in \mathbb{R}^{m\times n} $ eine Matrix. 

\begin{definition}
Die Transponierte $A^T \in \mathbb{R}^{n\times m}$ ist definiert durch:
\[
	A^T = (a_{j,i})_{i=1,\dots, n; j=1,\dots, m}
\]
\end{definition}
Das Transponieren gilt auch für Vektoren genauso, da diese ja einspaltige Matrizen sind. Sei $v\in \mathbb{R}^{m\times 1}$ ein Vektor. Der transponierte Vektor $v^T$ ist dann im aus dem Raum $\mathbb{R}^{1\times m}$

\[
v^T = \begin{pmatrix}
v_1\\
v_2\\
\vdots \\
v_m
\end{pmatrix}^T = (v_1, v_2, \dots , v_n)
\]

Aufgrund der Regel, dass Matrizen nur dann miteinander multipliziert werden dürfen, wenn die Spalten-Anzahl der linken Matrix mit der Zeilen-Anzahl der rechten Matrix übereinstimmen muss, konnten Vektoren bisher nicht an Matrizen von links multipliziert werden. Mit dem transponierten Vektor geht dies, da er eine Spalten-Anzahl hat, die mit der Matrix übereinstimmen kann. Seien wieder $v\in \mathbb{R}^m$ und $A\in \mathbb{R}^{m\times n}$.

\[
v^T \cdot A = (\sum_{i=1}^{m}v_i\cdot a_{i,j})_{j} = \begin{pmatrix}
\sum_{i=1}^{m}v_i\cdot a_{i,1} \\
\sum_{i=1}^{m}v_i\cdot a_{i,2} \\
\vdots \\
\sum_{i=1}^{m}v_i\cdot a_{i,n}
\end{pmatrix} \in \mathbb{R}^n
\]

\subsubsection{Dimensionalität, Rang und Determinante}

TODO

\section{Algebraische Strukturen}

TODO

\subsection{Homomorphismen}

\begin{definition}

Ein \textsl{Vektorraumhomomorphismus}\index{Homomorphismus}\index{Vektorraumhomomorphismus} -- auch verkürzt einfach Homomorphismus genannt -- von $V$ nach $W$ ist eine Abbildung $\phi : V \longrightarrow W$ mit folgenden Eigenschaften:
\begin{enumerate}
\item Für alle $v,w \in V$ gilt: $\phi(v+w) = \phi(v)+\phi(w)$
\item Für alle $a\in K$ und $v\in V$ ist $\phi(av)=a\phi(v)$
\end{enumerate}
\end{definition}

\subsection{Linearformen}

\begin{definition}
Eine \textsl{Linearform} \index{Linearform} ist eine lineare Abbildung von einem K-Vektorraum in seinen zugrundeliegenden Körper
\[
l : V \longrightarrow K
\]
mit der Eigenschaft
\[l(\alpha v + \beta w) = \alpha l(v) + \beta l(w) \]
für $v,w\in V$ und $\alpha, \beta \in K$
\end{definition}

\subsubsection{Bilinearformen}

\begin{definition}
Eine \textsl{Bilinearform} $b$ ist eine lineare Funktion
\[
	b : V \times V \longrightarrow K
\]
mit den Eigenschaften:
\begin{eqnarray*}
b(\alpha v + \beta w,x) &=& \alpha b(v,x) + \beta b(w,x) \\
b(v,\alpha w + \beta x) &=& \alpha b(v,w) + \beta b(v,x) 
\end{eqnarray*}
für $v,w\in V$ und $\alpha, \beta \in K$. Um genau zu sein ist eine Bilinearform eine zwei Parametrige lineare Funktion, die sich in beiden Parametern wie eine Linearform verhält.

Jede Bilinearform hat eine Matrix-Darstellung, denn zu jeder Bilinearform $\beta(.,.)$ kann folgende Matrix definiert werden:
\[
B := (b_{i,j})_{i,j} = (\beta(e_i, e_j))_{i,j}
\]
mit den Basisvektoren $e_i$.

\subsubsection{Multilinearformen}



\end{definition}

\subsection{Dual-Raum}

\begin{definition}
Die Menge aller Linearformen eines K-Vektorraums werden \textsl{Dualraum} von V genannt. Der Dualraum wird meist mit einem Stern gekennzeichnet: $V^*$
\end{definition}

Aufgrund der Eigenschaften der Linearformen bilden diese einen eigenen Vektorraum. 

\subsection{Skalarprodukt-Räume}

TODO


\section{Tensor}



\section{Aufgaben}

\begin{prob}
\label{matrix.1}

Weise nach, dass der $\mathbb{R}^{m\times n}$ ein Vektorraum ist. 

\end{prob}





\chapter{Exponentialrechnung und Polynome}


\section{Exponentialrechnung}

\subsection{Binomische Formeln}

\begin{eqnarray}
(a+b)^2 &=& a^2 +2ab +b^2 \label{eq:binom1} \\
(a-b)^2 &=& a^2 -2ab +b^2 \label{eq:binom2} \\
(a+b)(a-b) &=& a^2 -b^2 \label{eq:binom3}
\end{eqnarray}


\section{Polynome}

\begin{definition}
Eine reelle oder komplexe Funktion
\[ f_{a,n}(x) = a\cdot x^n  \]
wird als \textsl{Monom} bezeichnet. Summen verschiedener Monome als \textsl{Polynom}
\end{definition}
% TODO: decide where to put the proof chapter!

\chapter{Beweisführung}

Dieses Kapitel ist nicht speziell für die Lineare Algebra gedacht, aber da wir im Folgenden die ersten ernsthaften Beweise führen werden, sollte hier auf die Grundlagen der Beweisführung in der Mathematik eingegangen werden.

\section{Aussage-Typen}

Die folgenden Definitionen von Begriffen sind -- soweit es im Wissensbereich des Autors liegt -- nicht im strengen Sinne definiert worden. Daher werden sie meist so eingesetzt, wie der Autor eines Textes sie interpretiert. Jedoch hat sich ein allgemeines Verständnis für diese Begriffe gebildet, das hier in kurzer Übersichtsform dargestellt werden soll.

\subsection{Axiom}

Ein \textsl{Axiom} ist eine Grundlage für eine Theorie. Innerhalb dieser Theorie wird das Axiom als richtig vorausgesetzt und bedarf keines Beweises. Ein Axiom kann also als eine Art Rand- oder Anfangsbedingung für eine Argumentationskette interpretiert werden.

\subsection{Definition}

Eine \textsl{Definition} ist eine Aussage, die einen bestimmten Sachverhalt darlegt. Eine Definition kann im allgemeinen nicht aus anderen Aussagen abgeleitet werden. Sie führt die Beschreibung des Sachverhalts auch meist durch Einführung eines Namens zu einem feststehenden Begriff zusammen. Durch die Definition bekommt also der Begriff seine Bedeutung.

\subsection{Theorem oder Satz}

Als \textsl{Theorem} oder \textsl{Satz} wird eine Aussage verstanden, deren Inhalt logisch aus den Axiomen einer Theorie abgeleitet werden kann. Diese Form von Aussagen wird am häufigsten da eingesetzt, wo die Fundamente einer Theorie erarbeitet werden. Mithilfe der Theoreme wird aus einer Idee eine mathematische Realität. Die Beweise von Theoremen innerhalb einer Theorie stellt den Grundstein der Arbeit eines Mathematikers dar. In den Beweisen findet die Mathematik statt.

\subsection{Korollar}

Ein \textsl{Korollar} bezeichnet eine Aussage, die sich aus einem bewiesenen Theorem, dem Beweis des Theorems oder aus einer Definition ohne Aufwand ergibt. Im allgemeinen müssen die Aussagen eines Korollars nicht bewiesen werden, weil der Beweis trivial wäre oder so eng mit dem Beweis des Theorems verbunden ist, dass die Beweisführung redundant wäre. In seltenen Einzelfällen kann es notwendig sein, die Aussage eines Korollars zu beweisen. Sollte dies notwendig sein, sollte der Mathematiker sich aber überlegen, ob er aus dem Korollar nicht einen Hilfssatz macht.

\subsection{Lemma oder Hilfssatz}

Als \textsl{Lemma} oder \textsl{Hilfssatz} wird eine zu beweisende Aussage bezeichnet, deren Inhalt für den Beweis eines Theorems notwendig ist. Daraus folgt, dass ein Lemma bewiesen werden muss und im allgemeinen im Kontext einer größeren Beweisführung auftaucht. 

Die Unterscheidung, ob eine Aussage ein Lemma oder ein Theorem darstellt, ist oft nicht leicht zu treffen. Es wird passieren, dass eine Aussage in dem einen Buch über Mathematik als Theorem, in einem anderen "`nur"' als Lemma dargestellt wird -- als Beispiel sei hier der Fundamentalsatz der Algebra erwähnt, der eigentlich ein Satz der Analysis ist und dort nur eine eher geringe Wertschätzung erfährt. 

Dies sollte für den Lernenden nicht entscheidend sein. Der Inhalt von Theorem und Lemma ist in aller Regel interessant und der Beweis von beiden sollte verstanden und nachvollziehbar sein. 

\section{Beweisarten}

Es gibt einige Standard Schemata, mit deren Hilfe Beweise zu führen sind. Einige der prominentesten Vertreter wollen wir hier kennen lernen. Zunächst aber noch eine kurze Definition:
\begin{definition}
Jede Beweisführung in der Mathematik wird durch die lateinischen Worte "`quod erat demonstrandum"' (lat. für "`was zu beweisen war"') abgeschlossen. Es ist üblich, dafür entweder die Abkürzung "`q.e.d."' zu verwenden, oder ein kleines, rechtsbündiges Quadrat, wie am Ende dieser Definition.\qed
\end{definition}

\subsection{Vollständige Induktion}

Die \textsl{Vollständige Induktion} bezieht sich auf Aussagen über natürliche Zahlen. Die Induktion vollzieht sich in zwei Schritten. Man versucht zunächst die Aussage für eine natürliche Zahl -- wenn möglich eine kleine -- zu beweisen. Im zweiten Schritt versucht man die Aussage für die natürliche Zahl $n+1$ zu beweisen, indem man voraussetzt, dass die Aussage für $n$ richtig ist. Gelingt dies, ist der volle Beweis erbracht, denn man hat den Anfang der Kette ja im ersten Schritt nachgewiesen und für jede weitere natürliche Zahl folgt die Richtigkeit aus der Richtigkeit der um eins kleineren Zahl.

Ein Beispiel soll dies verdeutlichen: Carl Friedrich Gauss soll als neunjähriger die folgende Formel hergeleitet haben:

\begin{theorem}
Summenformel von Carl Friedrich Gauss
\begin{claim}
Für alle $n\in \mathbb{N}$ gilt:
\[ 1+2+3+ \dots +n = \sum_{k=1}^{n}k = {n(n+1)\over 2} \]
\end{claim}
\begin{proof}
Für $n=2$ ist

\[ {n(n+1)\over 2} = {2(2+1)\over 2} = {6\over 2} = 3 \]

Sei die Behauptung für $n$ bewiesen, so prüfe für $n+1$:
\begin{eqnarray*}
\sum_{k=1}^{n+1} k &=& \sum_{k=1}^{n} k  + (n+1) = {n(n+1)\over 2} +(n+1) \\
 &=& {n(n+1)\over 2} +{2(n+1)\over 2} = {n(n+1)+2(n+1) \over 2} \\
 &=& {(n+1)(n+2) \over 2}
\end{eqnarray*}
\end{proof}

\end{theorem}


\subsection{Beweis durch Widerspruch}

\subsection{Dirichlet'sches Schubfach Prinzip}




\chapter{Eigenwerte}

Kommen wir auf die reellen $n\times n$ Matrizen zurück, wobei $n<\infty$ ist. Sie stellen lineare Abbildungen vom $\mathbb{R}^n$ in sich selbst dar. 
\begin{definition}
Die Menge aller invertierbaren $n\times n$-Matrizen bildet, zusammen mit der Matrixmultiplikation als Verknüpfung, eine Gruppe. Sie wird $\mathfrak{gl}_n(\mathbb{R})$ genannt. Die Abkürzung $\mathfrak{gl}$ kommt von der "`general linear group"', der allgemeinen linearen Gruppe, moderner auch als $GL(n,\mathbb{R})$ bezeichnet. Dass diese Matrizen eine Gruppe bezüglich der Matrixmultiplikation bilden, wird in den Aufgaben nachgewiesen.
\end{definition}


\begin{definition}
Es seien $A\in \mathfrak{gl}_n(\mathbb{R})$ und $v\in \mathbb{R}^n$ ein Vektor. Falls weiter gilt
\[
Av = \lambda v
\]
mit $\lambda\in \mathbb{R}$, so heißt $v$ ein \emph{Eigenvektor} von $A$. Sowie $\lambda$ ein \emph{Eigenwert} zum Eigenvektor $v$.
\end{definition}

Diese Definition mag überraschend aussehen, denn schließlich ist $A$ ein vermeintlich "`kompliziertes"' und hochdimensionales Objekt, während auf der rechten Seite nur die Multiplikation einer Zahl mit einem Vektor steht, also eine Streckung des Vektors. Doch letztlich ist $A$ eine lineare Abbildung. Das bedeutet, $A$ bildet Vektoren im $\mathbb{R}^n$ wieder auf Vektoren ab. Dies kann im Grunde nur durch Drehungen, Verzerrungen und Spiegelungen geschehen. Erst nicht-lineare Abbildungen haben die Möglichkeit komplexeres Verhalten an den Tag zu legen. 

\begin{definition}
Seien $A,v,\lambda$ wie oben, des Weiteren sei $x\in \mathbb{R}$ ein Parameter und $I_n$ das neutrale Element der Multiplikation des $\mathfrak{gl}_n(\mathbb{R})$. Dann ist 
\[
p(x) = \det(A-x\cdot I_n) : \mathbb{R} \longrightarrow \mathbb{R}
\]
ein normiertes Polynom vom Grade $n$. Es wird \emph{charakteristisches Polynom} genannt.
\end{definition}

\begin{theorem}
Die Nullstellen des charakteristischen Polynoms zur Matrix $A\in \mathfrak{gl}_n(\mathbb{R})$ sind die Eigenwerte von $A$.
\end{theorem}
\begin{proof}
\begin{TODO}

\end{TODO}
\end{proof}

\section{Aufgaben}
\begin{TODO}

\end{TODO}

\chapter{Numerische Lösungsverfahren}
\begin{TODO}

\end{TODO}

\section{Gauß Elemination}
\begin{TODO}

\end{TODO}

\section{QR Zerlegung}
\begin{TODO}

\end{TODO}

\section{Iterative Lösungsmethoden}
\begin{TODO}
cg-Verfahren. Vielleicht GMRES? Auf jeden Fall allgemeine Abstiegsverfahren. Als Einstieg das naive Residuums-verfahren.
\end{TODO}







% Removed due to changes
%\include{./geometry/geometrie}
%
\chapter{Trigonometrie}

\section{Grundlegende Begriffe}

Die Trigonometrie beschäftigt sich in erster Linie mit Dreiecken in der Ebene. Wie in Zeichnung \ref{fig:triangle} dargestellt, bezeichnen wir die Eckpunkte eines Dreiecks zunächst mit großen Buchstaben $A,B,C$. Die einem Punkt gegenüber befindliche Gerade bezeichnen wir analog dem gegenüberliegenden Punkt mit den Kleinbuchstaben $a,b,c$. 

\begin{figure}\label{fig:triangle}
\begin{center}
\begin{tikzpicture}[line cap=round,line join=round,x=2.0cm,y=2.0cm]
    \clip(-1,-0.25) rectangle (5.25,3);
    \draw (0,0) coordinate (A);
    \draw (5,0) coordinate (B);
    \draw (3.2,2.4) coordinate (C);
    \draw (A)--(B)--(C)--(A);
    \filldraw  (A) circle (1.5pt) node[left] {$A$};
    \filldraw  (B) circle (1.5pt) node[right] {$B$};
    \filldraw  (C) circle (1.5pt) node[above] {$C$};
    \draw [shift={(A)},color=colWin,fill=colWin,fill opacity=0.1] (0,0) -- (0:0.5)
        arc (0:36.8699:0.5) -- cycle;
    \draw [shift={(B)},color=colWin,fill=colWin,fill opacity=0.1] (0,0) -- (126.8699:0.5)
        arc (126.8699:180:0.5) -- cycle;
%    \draw [shift={(C)},color=colWin,fill=colWin,fill opacity=0.1] (0,0) -- (-143.1301:0.5)
%        arc (-143.1301:-90:0.5) -- cycle;
    \draw [shift={(C)},color=colWin,fill=colWin,fill opacity=0.1] (0,0) -- (-143.1301:0.5)
        arc (-143.1301:-53.1301:0.5) -- cycle;
    \draw [color=black] (0.35,0.1) node {$\alpha$};
    \draw [color=black] (4.64,0.15) node {$\beta$};
    \draw [color=black] (3.15,2.05) node {$\gamma$};
    \draw (4.1,1.2) node[rotate=-53.1301, above] {$a$};
    \draw (1.6,1.2) node[rotate=36.8698,above] {$b$};
    \draw (2.5,0) node[below] {$c$};
\end{tikzpicture}
\caption[Dreieck]{Dreieck}
\end{center}
\end{figure}

Den Punkten anliegende Winkel bezeichnen wir mit den Griechischen Buchstaben $\alpha, \beta, \gamma$ in den Ecken des Dreiecks $A,B,C$.


\section{Kreis}

Der Zusammenhang mit dieser Begriffe wird in einem Kreisdiagram\footnote{Diese Zeichnung basiert auf einem Beispiel aus Till Tantau's ausgezeichneter Dokumentation zu seiner \LaTeX \ Bibliothek Ti$kZ$ } in Abbildung \ref{fig:circle} näher dargestellt. 

\begin{figure}\label{fig:circle}
\begin{center}
\begin{tikzpicture}
[scale=3,line cap=round,
% Styles
axes/.style=,
important line/.style={very thick},
information text/.style={rounded corners,fill=red!10,inner sep=1ex}]
% Local definitions
\def\costhirty{0.8660256}
\draw (0,0) circle (1cm);
\begin{scope}[axes]
\draw[->] (-1.5,0) -- (1.5,0) node[right] {$x$} coordinate(x axis);
\draw[->] (0,-1.5) -- (0,1.5) node[above] {$y$} coordinate(y axis);
\foreach \x/\xtext in {-1, -.5/-\frac{1}{2}, 1}
\draw[xshift=\x cm] (0pt,1pt) -- (0pt,-1pt) node[below,fill=white] {$\xtext$};
\foreach \y/\ytext in {-1, -.5/-\frac{1}{2}, .5/\frac{1}{2}, 1}
\draw[yshift=\y cm] (1pt,0pt) -- (-1pt,0pt) node[left,fill=white] {$\ytext$};
\end{scope}
\filldraw[fill=black!20] (0,0) -- (3mm,0pt) arc(0:30:3mm);
\draw (15:2mm) node[black] {$\alpha$};
\draw[important line,black]
(30:1cm) -- node[left=1pt,fill=white] {$\sin \alpha$} (30:1cm |- x axis);
\draw[important line,black]
(30:1cm |- x axis) -- node[below=2pt,fill=white] {$\cos \alpha$} (0,0);
\path [name path=upward line] (1,0) -- (1,1);
\path [name path=sloped line] (0,0) -- (30:1.5cm);
\draw [name intersections={of=upward line and sloped line, by=t}]
[very thick,black] (1,0) -- node [right=1pt,fill=white]
{$\displaystyle \tan \alpha \color{black}=
\frac{{\sin \alpha}}{\cos \alpha}$} (t);
\draw (0,0) -- (t);
\end{tikzpicture}
\caption[Zusammenhang trigonometrischer Begriffe]{Darstellung des Zusammenhangs trigonometrischer Begriffe}
\end{center}
\end{figure}



\part{Analysis}


\chapter{Grundlagen}

Der Begriff einer Abbildung ist uns schon öfter untergekommen. Eine Funktion ist eine Abbildung, die zwei Mengen (hier $X$ und $Y$) in Beziehung setzt:

\[ f: X \longrightarrow Y \]
$f$ ordnet jedem Element $x\in X$ ein Element $y\in Y$ zu, indem
\[f(x) =y\]
gilt. Die Umkehrung gibt es allgemein nicht, aber wenn sie existiert, so wird sie als $f^{-1}$ bezeichnet und es gilt:
\[ f^{-1}(y) =x\]


\section{Polynome}

Wir hatten in Kapitel \ref{chap:poly} Polynome bereits kennengelernt. 

\subsection{Ableitung von Polynomen}



\section{Kurvendiskussion}



\chapter{Folgen}


\section{Cauchy Folgen}


\section{Vollständigkeit}



\chapter{Differentiation}



\chapter{Integration}




\chapter{Grundlagen}

Die Analysis, auch Infinitesimalrechnung genannt, wurde von Gottfried Wilhelm Leibnitz\footnote{\textbf{Gottfried Wilhelm Leibnitz}, *1. Juli 1646 in Leibzig; \ding{61}14. November 1716 in Hannover} und Isaac Newton\footnote{\textbf{Sir Isaac Newton}, *4. Januar 1643 in Woolsthorpe-by-Colsterworth in Lincolnshire; \ding{61}31. März 1727 in Kensington} unabhängig voneinander entwickelt. Sie behandelt Grenzwerte von Folgen und Reihen, Funktionen reeller Zahlen, deren Stetigkeit, Differenzierbarkeit und Integrierbarkeit. 

\section{Folgen}

Folgen sind ein zentraler Begriff in der Mathematik, im Besonderen der Analysis. Die Betrachtung von Folgen und ihrem Verhalten, hat viele Bereiche der Mathematik inspiriert, so wäre die Funktionentheorie ohne die Banach\footnote{\textbf{Stefan Banach}, polnischer Mathematiker,  *30. März 1892 in Krakau, \ding{61}31. August 1945 in Lemberg}-Räume kaum vorstellbar. \index{Banach Raum}\index{Banach, Stefan} Banachräume sind bestimmte Vektorräume, in denen jede Cauchy-Folge gegen einen im Raum enthaltenen Wert konvergiert.

\begin{definition}
Eine \textsl{Folge} ist eine -- wie der Name suggeriert -- Abfolge von Mathematischen Objekten. Dies können zum Beispiel Zahlen, Vektoren oder Funktionen sein, oder jedes andere Mathematische Objekt, solange sich diese in eine Reihenfolge bringen lassen. Bei Folgen ist eine Reihenfolge der Elemente festgelegt. Sind $a_i$ die Elemente einer Folge, so ist immer $a_{i-1}$ vor $a_i$ und diesem folgt immer $a_{i+1}$ -- sofern $a_{i\pm 1}$ existieren. Eine Folge wird in dieser Art dargestellt:
\[
(a_1, a_2, \dots)
\]
\end{definition}

Folgen sind keine Mengen, in Folgen können Elemente mehrfach vorkommen, an unterschiedlichen Positionen in der Abfolge der Elemente! 

Folgen haben unendlich viele Elemente. Sie sind mit den natürlichen Zahlen abzählbar unendlich (siehe Definition \ref{abzaehlbar}).

\subsection{Charakterisierung}

\begin{definition}
Eine Folge heißt \textsl{monoton steigend}, wenn für alle $i\in \mathbb{N}$ gilt $a_i\le a_{i+1}$. Sie heißt \textsl{streng monoton steigend}, wenn $a_i < a_{i+1}$ gilt. Äquivalent heißt sie \textsl{monoton fallend}, wenn $a_i\ge a_{i+1}$ gilt, bzw. \textsl{streng monoton fallend}, wenn $a_i > a_{i+1}$ gilt.
\end{definition}

Monotonie kann auch erst ab einem gewissen $n\in \mathbb{N}, n>1$ nachweisbar sein. Die Folge wird dann als monoton ab dem Folgenglied $n$ bezeichnet.

\begin{definition}
Eine reelle Folge ($a_i \in \mathbb{R}$) heißt \textsl{beschränkt}, wenn es ein $S\in \mathbb{R}$ gibt, sodass für alle $i\in \mathbb{N}$, $a_i \le S$ gilt. $S$ ist eine \textsl{obere Schranke} der Folge. Gibt es eine kleinste obere Schranke $S'$, so wird diese \textsl{Supremum} genannt. 

Gibt es ein $I\in \mathbb{R}$ mit $a_i \ge I$, so wird $I$ \textsl{untere Schranke} genannt. Die größte untere Schranke wird \textsl{Infimum} genannt.
\end{definition}

\begin{definition}
Eine Folge, deren Elemente alle identisch sind, wird \textsl{konstante Folge} genannt. Eine Folge, deren Elemente abwechselnd positiv und negativ sind, wird \textsl{alternierend} genannt. Eine Folge deren Elemente sich immer weiter der 0 annähern, wird \textsl{Nullfolge} genannt.
\end{definition}

\subsection{Konvergenz}

\begin{definition}\label{def:lim}
Es sei $(a_i)$ eine Folge in den reellen Zahlen und $a\in \mathbb{R}$ eine \textsl{Grenzwert} genannte Zahl. Weiterhin sei $\epsilon>0$ eine beliebige reelle Zahl. Wenn es für jeden solchen Wert $\epsilon$ einen Folgenindex $n_\epsilon \in \mathbb{N}$ gibt, sodass $\vert a_i -a\vert <\epsilon$ für alle $i\ge n_\epsilon$ gilt, so heißt die Folge \textsl{kovergent}. Des Weiteren konvergiert diese Folge gegen $a$. 
\end{definition}

Man sollte sich das $\epsilon$ als eine sehr kleine Zahl vorstellen. Für große Zahlen sind die Bedingungen der Konvergenz recht einfach zu erfüllen. Schwierig wird es erst, wenn $\epsilon\ll 1$ ist. ($\ll$ bedeutet "`wesentlich kleiner als"').

Die oben dargestellte Definition für Konvergenz beruht darauf, dass der Punkt $a$ bereits bekannt ist. Was ist aber, wenn wir $a$ nicht kennen? Gibt es dann auch ein Kriterium für die Konvergenz?

\begin{definition}
Es sei $(a_i)$ eine Folge. Es sei $\epsilon \ll 1$. Wenn es zu jedem $\epsilon$ ein $N\in \mathbb{N}$ gibt, sodass für $n,m > N$ gilt $\vert a_n -a_m \vert <\epsilon$, so ist auch diese Folge konvergent. Bemerkenswert dabei ist, dass wir nur die Abstände der Folgenelemente dafür betrachten und keinen Grenzwert. Eine solche Folge wird \textsl{Cauchy-Folge}\footnote{benannt nach dem französischen Mathematiker \textbf{Augustin-Louis Cauchy}, *21. August 1789, \ding{61}23. Mai 1857 in Sceaux} genannt.
\end{definition}

Cauchy-Folgen haben die positive Eigenschaft, dass man ihre Konvergenz nachweisen kann, ohne den Grenzwert zu kennen. Z.B. in den rationalen Zahlen $\mathbb{Q}$ gibt es Cauchy-Folgen, die keinen Grenzwert besitzen, weil dieser in den irrationalen Zahlen liegt.


\begin{definition}
Zu einer Folge $a_i$ heißt die Menge $\lbrace a_i \rbrace$ die \textsl{Menge der Folgenwerte}. Es wird der Einfachheit halber eine Folge auch oft mit der Menge ihrer Folgenwerte gleichgesetzt. Dies ist aber nicht vollständig korrekt, da eine Folge auch Informationen über die Reihenfolge ihrer Elemente enthält, die Menge der Folgenwerte aber nicht. 
\end{definition}

\section{Grenzwert}

In Definition \ref{def:lim} wurde bereits der Begriff \textsl{Grenzwert} verwendet. Hier soll er noch einmal formal dargestellt werden. 

\begin{definition}
Sei $(a_i)$ eine Folge in $A\subseteq \mathbb{R}$. Wenn die Folge gegen einen Wert $a\in A$ konvergiert, so heißt $a$ \textsl{Grenzwert}, oder auch \textsl{Limes}\footnote{Limes lat. "`Grenzweg"'} der Folge $(a_i)$. Dass $(a_i)$ den Grenzwert $a$ besitzt, wird wie folgt ausgedrückt:
\[
\lim_{n\rightarrow \infty} a_n = a
\]
\end{definition}

\begin{definition}\label{def:voll}
Wenn jede Cauchy-Folge mit $(a_n)\in A\subseteq \mathbb{R}$ konvergiert und ihr Grenzwert 
\[\lim_{n\rightarrow \infty} a_n = a\in A\] 
existiert, so heißt $A$ \textsl{vollständig}. 

Die Vollständigkeit der reellen Zahlen in der Form, dass alle Cauchy-Folgen konvergieren und ihren Grenzwert annehmen, ist eine Konsequenz aus dem sogenannten \textsl{Vollständigkeitsaxiom}. Die Vollständigkeit folgert also nicht aus der Konvergenz der Cauchy-Folgen, sondern genau umgekehrt folgert die Konvergenz der Cauchy-Folgen aus der Vollständigkeit. 
\end{definition}


\section{Reihen}

\begin{definition}
Sei $(a_i)$ eine Folge, so ist 
\[
s_n = \sum_{i=1}^{n} a_i
\]
die $n$-te \textsl{Partialsumme}. Die Partialsummen bilden wiederum eine Folge $(s_i)$. Falls der Grenzwert 
\[
R(a) = \lim_{n\rightarrow \infty} s_n = \lim_{n\rightarrow \infty} \sum_{i=1}^{n} a_i
\]
existiert, so heißt $R(a)$ eine \textsl{Reihe}.
\end{definition}

Ob eine Reihe existiert, bestimmt die Partialsummen-Folge. Konvergiert diese, so existiert der Grenzwert, und damit auch die Reihe. Aus der Definition der Partialsummen folgt, dass wenn die Partialsummen-Folge konvergiert, $(a_n)$ eine Null-Folge sein muss. Falls $(a_n)$ keine Null-Folge ist, divergieren die Partialsummen und der Grenzwert  existiert nicht. Allerdings reicht es nicht aus $(a_n)$ als Null-Folge zu wählen, es gibt Null-Folgen, deren Partialsummen trotzdem divergieren. 

Ein Beispiel für eine nicht konvergente Summe aus einer Null-Folge ist die Harmonische Reihe:
\[
H_n = \sum_{i=1}^{n} \frac{1}{i}
\]
Die Folge $\lbrace 1, \frac{1}{2}, \frac{1}{3}, \dots \rbrace$ ist eine Null-Folge, aber die $H_n$ sind divergent.

\begin{definition}
Eine Reihe heißt \textsl{absolut konvergent}, wenn 
\[
R'(a) = \lim_{n\rightarrow \infty} \sum_{i=1}^{n} \vert a_i \vert
\]
konvergiert.
\end{definition}

\begin{lemma}
Falls zu jedem $\epsilon \ll 1$ ein Index $N$ existiert, sodass für alle $n>N$ gilt
\[
\left\vert \frac{a_{n+1}}{a_n} \right\vert \le \epsilon
\]
dann konvergiert die Reihe 
\[
\lim\limits_{n\rightarrow \infty} \sum_{i=1}^{n} \vert a_i \vert
\]
Sie ist also absolut konvergent. Dies wird als \textsl{Quotienten-Kriterium} bezeichnet.
\end{lemma}
\begin{proof}
TODO
\end{proof}


\section{Stetigkeit}

Es sei $A\subseteq \mathbb{R}$. Wir betrachten alle Folgen $(a_i)\in A$, die einen Grenzwert $a\in A$ besitzen, sowie eine Funktion $f: A\longrightarrow \mathbb{R}$. 

\begin{definition}
Die Funktion $f$ heißt \textsl{stetig}\index{stetig}, wenn für jede Folge $(a_i)$ mit Grenzwert $a$ auch die Folge $(f(a_i))$ gegen $f(a)$ konvergiert. Formal definiert
\[
\lim_{n\rightarrow \infty} f(a_n) = f(\lim_{n\rightarrow \infty} a_n) = f(a)
\]
\end{definition}

Diese Definition der Stetigkeit wird auch \textsl{Folgenstetigkeit}\index{Folgenstetigkeit} genannt. Eine Definition der Stetigkeit unabhängig von Folgen ist die folgende: 

\begin{definition}
$f$ ist genau dann in $a$ \textsl{stetig}, wenn es zu jedem $\epsilon>0$ ein $\delta_\epsilon > 0$ gibt, so dass für alle $b\in A$ mit $\vert b-a\vert < \delta_\epsilon $ immer $\vert f(b)-f(a)\vert < \epsilon$ ist. Diese Definition wird meist $\delta$-$\epsilon$-Stetigkeit genannt. \index{$\delta$-$\epsilon$-stetig}
\end{definition}

Wir hatten bisher versucht, mehrere Definitionen für einen Begriff zu vermeiden. Allerdings passiert es, dass die Stetigkeitseigenschaft in unterschiedlichen Zusammenhängen ausgenutzt werden muss. Daher ist es zuweilen auch notwendig, unterschiedliche Definitionen heranzuziehen, um in einem Beweis die Stetigkeit prägnant auszunutzen. Stetigkeit ist noch auf weitere Arten definierbar: Z.B. im topologischen Sinne, dass unter stetigen Abbildungen die Urbilder offener Mengen wieder offene Mengen sind. Aber das würde an dieser Stelle zu weit führen.








\chapter{Differentialrechnung}\label{chap:diff}

TODO


\section{Überlegungen}

Das langfristige Ziel dieses Kapitels ist es, zu einer Funktion $f$, die noch zu bestimmenden Anforderungen genügt, eine Funktion $f'$ zu bestimmen, die an jeder Stelle den Wert der Steigung von $f$ annimmt. 

Es sei im folgenden $f : \mathbb{R} \longrightarrow \mathbb{R}$ immer eine Funktion. Wir betrachten eine Stelle $a\in \mathbb{R}$ und geben ein $0<\epsilon < 1$ vor. Die Sehne von $f$ zwischen Punkt $a$ und $a+\epsilon$ ist
\[
S = \frac{f(a+\epsilon)-f(a)}{\epsilon}
\]
Wir suchen nun eine Funktion, die zumindest an der Stelle $a$ den Funktionswert $S$ annimmt. Die einfachsten Funktionen, die wir kennen, sind die linearen Funktionen.



\section{Stetigkeit}



\chapter{Integralrechnung}




\chapter{Trigonometrie}

\section{Grundlegende Begriffe}

Die Trigonometrie beschäftigt sich in erster Linie mit Dreiecken in der Ebene. Wie in Zeichnung \ref{fig:triangle} dargestellt, bezeichnen wir die Eckpunkte eines Dreiecks zunächst mit großen Buchstaben $A,B,C$. Die einem Punkt gegenüber befindliche Gerade bezeichnen wir analog dem gegenüberliegenden Punkt mit den Kleinbuchstaben $a,b,c$. 

\begin{figure}\label{fig:triangle}
\begin{center}
\begin{tikzpicture}[line cap=round,line join=round,x=2.0cm,y=2.0cm]
    \clip(-1,-0.25) rectangle (5.25,3);
    \draw (0,0) coordinate (A);
    \draw (5,0) coordinate (B);
    \draw (3.2,2.4) coordinate (C);
    \draw (A)--(B)--(C)--(A);
    \filldraw  (A) circle (1.5pt) node[left] {$A$};
    \filldraw  (B) circle (1.5pt) node[right] {$B$};
    \filldraw  (C) circle (1.5pt) node[above] {$C$};
    \draw [shift={(A)},color=colWin,fill=colWin,fill opacity=0.1] (0,0) -- (0:0.5)
        arc (0:36.8699:0.5) -- cycle;
    \draw [shift={(B)},color=colWin,fill=colWin,fill opacity=0.1] (0,0) -- (126.8699:0.5)
        arc (126.8699:180:0.5) -- cycle;
%    \draw [shift={(C)},color=colWin,fill=colWin,fill opacity=0.1] (0,0) -- (-143.1301:0.5)
%        arc (-143.1301:-90:0.5) -- cycle;
    \draw [shift={(C)},color=colWin,fill=colWin,fill opacity=0.1] (0,0) -- (-143.1301:0.5)
        arc (-143.1301:-53.1301:0.5) -- cycle;
    \draw [color=black] (0.35,0.1) node {$\alpha$};
    \draw [color=black] (4.64,0.15) node {$\beta$};
    \draw [color=black] (3.15,2.05) node {$\gamma$};
    \draw (4.1,1.2) node[rotate=-53.1301, above] {$a$};
    \draw (1.6,1.2) node[rotate=36.8698,above] {$b$};
    \draw (2.5,0) node[below] {$c$};
\end{tikzpicture}
\caption[Dreieck]{Dreieck}
\end{center}
\end{figure}

Den Punkten anliegende Winkel bezeichnen wir mit den Griechischen Buchstaben $\alpha, \beta, \gamma$ in den Ecken des Dreiecks $A,B,C$.


\section{Kreis}

Der Zusammenhang mit dieser Begriffe wird in einem Kreisdiagram\footnote{Diese Zeichnung basiert auf einem Beispiel aus Till Tantau's ausgezeichneter Dokumentation zu seiner \LaTeX \ Bibliothek Ti$kZ$ } in Abbildung \ref{fig:circle} näher dargestellt. 

\begin{figure}\label{fig:circle}
\begin{center}
\begin{tikzpicture}
[scale=3,line cap=round,
% Styles
axes/.style=,
important line/.style={very thick},
information text/.style={rounded corners,fill=red!10,inner sep=1ex}]
% Local definitions
\def\costhirty{0.8660256}
\draw (0,0) circle (1cm);
\begin{scope}[axes]
\draw[->] (-1.5,0) -- (1.5,0) node[right] {$x$} coordinate(x axis);
\draw[->] (0,-1.5) -- (0,1.5) node[above] {$y$} coordinate(y axis);
\foreach \x/\xtext in {-1, -.5/-\frac{1}{2}, 1}
\draw[xshift=\x cm] (0pt,1pt) -- (0pt,-1pt) node[below,fill=white] {$\xtext$};
\foreach \y/\ytext in {-1, -.5/-\frac{1}{2}, .5/\frac{1}{2}, 1}
\draw[yshift=\y cm] (1pt,0pt) -- (-1pt,0pt) node[left,fill=white] {$\ytext$};
\end{scope}
\filldraw[fill=black!20] (0,0) -- (3mm,0pt) arc(0:30:3mm);
\draw (15:2mm) node[black] {$\alpha$};
\draw[important line,black]
(30:1cm) -- node[left=1pt,fill=white] {$\sin \alpha$} (30:1cm |- x axis);
\draw[important line,black]
(30:1cm |- x axis) -- node[below=2pt,fill=white] {$\cos \alpha$} (0,0);
\path [name path=upward line] (1,0) -- (1,1);
\path [name path=sloped line] (0,0) -- (30:1.5cm);
\draw [name intersections={of=upward line and sloped line, by=t}]
[very thick,black] (1,0) -- node [right=1pt,fill=white]
{$\displaystyle \tan \alpha \color{black}=
\frac{{\sin \alpha}}{\cos \alpha}$} (t);
\draw (0,0) -- (t);
\end{tikzpicture}
\caption[Zusammenhang trigonometrischer Begriffe]{Darstellung des Zusammenhangs trigonometrischer Begriffe}
\end{center}
\end{figure}


\chapter{Grundlagen}

Die Algebra konzentriert sich -- im Gegensatz zur Arithmetik -- auf die Verallgemeinerung der Begriffe der Analyse bestimmter Sachverhalte. Der Arithmetik Teil dieses Buches begann mit Berechnungen auf Basis von Äpfeln. Diese geben einen unmittelbaren Zugang zu den Begriffen des Rechnens. In der Algebra wird es solches nicht geben. Die Ansätze hier sind rein abstrakt zu verstehen. Auch wenn direkte, anschauliche Beispiele zu bestimmten algebraischen Sachverhalten existieren, so sollte der Lernende versuchen, nicht anhand dieser sein Verständnis auszubilden, sondern rein in der Sache selbst. Das führt letztlich dazu, dass er vollständig ohne konkrete Anschauung Sachverhalte analysieren und Probleme lösen kann. Vollständig unabhängig davon, ob diese Probleme einen realen, abstrakten oder mit dem Verstand nicht nachvollziehbaren Hintergrund haben. Wenn z.B. im Raum der Polynome zwei Funktionen die Bedingung erfüllen
\[f^2(x) + g^2(x) = 1 \]
so stellt dies genauso den Satz des Pytagoras dar, als wenn ein Dreieck betrachtet würde. 
\[a^2 +b^2 = c^2\]

\section{Zeichen}

In der Algebra verwendet man anstelle von Zahlen im allgemeinen Buchstaben. Es immer vom Kontext abhängig, welcher Buchstabe was bedeutet. Aber es haben sich bestimmte Dinge eingebürgert. So sind Konstanten meistens mit den Buchstaben
\[a, b, c, \dots \]
bezeichnet. Unbekannte in Gleichungen meist mit 
\[x, y, z, p, q, \dots \]
Sowie Indizes mit 
\[i, j, k, \dots \]
Es können aber auch griechische Buchstaben auftauchen. Hier eine Übersicht

\bigskip

\begin{tabular}{c|c|l}
\hline
\textbf{Kleiner Buchstabe} & \textbf{Großbuchstabe} & \textbf{Bezeichnung} \\
\hline
$\alpha $ & $A $ & Alpha \\
$\beta $ & $B $ & Beta \\
$\gamma $ & $\Gamma $ & Gamma \\
$\delta $ & $\Delta $ & Delta \\
$\epsilon $ & $E $ & Epsilon \\
$\zeta $ & $Z $ & Zeta \\
$\eta $ & $H $ &  Eta\\
$\theta $ & $\Theta $ & Theta \\
$\iota $ & $I $ & Iota \\
$\kappa $ & $K $ & Kappa \\
$\lambda $ & $\Lambda $ & Lambda \\
$\mu $ & $M $ & Mu \\
$\nu $ & $N $ & Nu \\
$\xi $ & $\Xi $ &  Xi \\
$\omicron $ & $O $ & Omicron \\
$\pi $ & $\Pi $ & Pi \\
$\rho $ & $P $ & Rho \\
$\sigma $ & $\Sigma $ & Sigma  \\
$\tau $ & $T $ & Tau \\
$\upsilon $ & $\Upsilon $ & Ypsilon \\
$\phi $ & $\Phi $ & Phi \\
$\chi $ & $X $ & Chi \\
$\psi $ & $\Psi $ & Psi \\
$\omega $ & $\Omega $ & Omega \\
\hline
\end{tabular}

\section{Gesetze}

Da diese im Arithmetikteil nicht abstrakt definiert wurden, seien sie hier noch einmal wiederholt:

\bigskip

\noindent \textsl{Kommutativgesetz}
\begin{eqnarray*}
a+b &=& b+a \\
a\cdot b &=& b\cdot a
\end{eqnarray*}

\noindent \textsl{Assoziativgesetz}
\begin{eqnarray*}
a+(b+c) &=& (a+b)+c \\
a\cdot (b\cdot c) &=& (a\cdot b)\cdot c
\end{eqnarray*}

\noindent \textsl{Distributivgesetz}
\begin{eqnarray*}
a\cdot (b+c) &=& a\cdot b+ a\cdot c
\end{eqnarray*}


\chapter{Mengen, Gruppen, Ringe, Körper}

Kategorisierung spielt in der Algebra eine sehr wichtige Rolle. Kennt man die Eigenschaften eines Dinges, so kann man damit umgehen.



\appendix
\part{Anhang}


\chapter{Schlussbemerkungen}

\section{Beweise}

Wir haben in den diversen Kapiteln des Buches den einen oder anderen Beweis unterschlagen, weil seine Beweisführung an der Stelle, wo er aufgetaucht war, noch zu unverständlich gewesen wäre. Dies soll hier nun nachgeholt werden.


\subsection{Es gibt unendlich viele Primzahlen}\label{chap:proofprime}
Wir hatten in Abschnitt \ref{sec:infty} beweisen wollen, dass es unendlich viele Primzahlen gibt. Der Beweis brauchte noch einige Vorkenntnisse, die wir hier nun voraussetzen können und den Beweis angehen. Zuvor brauchen wir noch ein

\begin{lemma}{(Lemma von Bézout\footnote{\textbf{Étienne Bézout}, französischer Mathematiker. *31. März 1730; \ding{61}27. September 1783 in Basses-Loges})}\index{Lemma von Bézout}
Der größte gemeinsame Teiler $\text{ggT}(a,b) $ zweier ganzer Zahlen $a$ und $b$, von denen mindestens eine ungleich 0 ist, kann als Linearkombination von $a$ und $b$ mit ganzzahligen Koeffizienten $s,t\in \mathbb{Z}$ dargestellt werden. 
\[
\text{ggT}(a,b) = s\cdot a+t\cdot b
\]
Weiter gilt, dass wenn $a$ und $b$ teilerfremd sind, existieren $s,t \in \mathbb{Z}$, sodass
\begin{equation}\label{eq:1}
1 = s\cdot a+t\cdot b
\end{equation}
\end{lemma}
\begin{proof}
Es sei $d$ die kleinste aller Liniearkombinationen von $a$ und $b$, die größer als null ist:
\[
d = \min_x \left\lbrace x = s\cdot a+t\cdot b \mid x>0; s,t\in \mathbb{Z} \right\rbrace
\]
Dafür müssen nur die $s,t\in \mathbb{Z}$ entsprechend gewählt werden. Da $\text{ggT}(a,b)$ sowohl $a$ als auch $b$ teilt, teilt $\text{ggT}(a,b)$ auch $d$. Wäre $d=1$, wäre der Beweis fertig und Gleichung (\ref{eq:1}) gilt. Kümmern wir uns also um den Fall $d>1$. 

Die Division mit Rest liefert uns:
\[
a = q\cdot d+r
\]
wobei $0\le r < d$. Wir setzen für $d$ die Linearkombination ein
\[
a = q\cdot(s\cdot a+t\cdot b)+r
\]
und formen nach $r$ um
\[
r = (1-q\cdot s)\cdot a+(-q\cdot t)\cdot b
\]
Wegen der Minimalität von $d$ muss $r=0$ sein. Also ist $d$ ein Teiler von $a$, also auch von $b$. Daraus folgt, dass $d\le \text{ggT}(a,b)$. Vorher hatten wir gesehen, dass $\text{ggT}(a,b)$ ein Teiler von $d$ ist, also gilt $d=\text{ggT}(a,b)$.

\qed
\end{proof}

Das Lemma von Bézout stellt eine wichtige Erkenntnis in der Mathematik und im Besonderen der Zahlentheorie dar. Wir werden es hier verwenden, um das Lemma von Euklid zu beweisen:

\begin{lemma}{(Lemma von Euklid\footnote{\textbf{Euklid}, griechischer Mathematiker, hat vermutlich im 3. Jahrhundert v. Chr. in Alexandria gelebt.})}\index{Lemma von Euklid}
Teilt eine Primzahl $p$ ein Produkt $ab$, so auch einen (oder beide) der Faktoren.
\end{lemma}
\begin{proof}
Es seien $a,b \in \mathbb{Z}$ beliebig und $p$ eine Primzahl. Würde die Primzahl $p$ das Produkt $ab$ teilen, aber nicht den Faktor $a$, dann ist zu zeigen, dass $p$ ein Teiler von $b$ ist. Aus der Annahme folgt, dass $a$ und $p$ teilerfremd sind. Mit dem Lemma von Bézout existieren dann zwei ganze Zahlen, sodass $1=sp+ta$ gilt. Multipliziert man diese Gleichung mit $b$ erhält man
\[
p(sb)+(ab)t=b
\]
Da $p$ ein Teiler von $ab$ ist, existiert ein $c\in \mathbb{Z}$ mit $ab=cp$. Also ist
\[
p(sb+ct) = b
\]
Damit ist $p$ ein Teiler von $b$.

\qed
\end{proof}

\begin{lemma}\label{lem:prim}
Jede natürliche Zahl $>1$ hat eine eindeutige Zerlegung in Primfaktoren.
\end{lemma}
\begin{proof}
Der Beweis geschieht in zwei Schritten. Zunächst wird die Existenz der Zerlegung bewiesen und danach ihre Eindeutigkeit.

\noindent\textsl{Schritt 1:}
Für jede Primzahl $p$ ist die Behauptung trivialerweise richtig, denn sie ist ihre eigene Primfaktorzerlegung. 
Wir nehmen an, dass es Zahlen gibt, die sich nicht in Primfaktoren zerlegen lassen. Es gibt eine kleinste solche Zahl $n$. Da $n>1$ und keine Primzahl ist, gibt es Teiler von $n=ab$. Für beide gilt $1<a,b<n$. Da $n$ die kleinste Zahl war, die keine Primfaktorzerlegung hat, gibt es Primfaktorzerlegungen für $a$ und $b$. Das bedeutet, $a = \Pi p_i$ und $b=\Pi q_j$ mit $p_i,q_j$ Primzahlen. Damit ist aber auch $\Pi p_i \cdot \Pi q_j$ eine Primfaktorzerlegung von $n$, was der Behauptung widerspricht, dass es Zahlen gäbe, die sich nicht in Primfaktoren zerlegen lassen.

\noindent\textsl{Schritt 2:}
Wir nehmen an, es gibt Zahlen mit unterschiedlichen Primfaktorzerlegungen. Es sei $n=p_1 p_2 \dots p_n = q_1 q_2 \dots q_m$ die kleinste dieser Zahlen mit zwei Zerlegungen in Primfaktoren. Die $p_i$ und $q_j$ müssen alle verschieden sein, denn gäbe es ein $p_k=q_l$, so teilte diese Zahl $n$ und $n/p_k$ hätte wiederum zwei unterschiedliche Zerlegungen, was gegen die Annahme verstößt, dass $n$ die kleinste solche Zahl ist.
Es sei 
\[
n= p_i\cdot \prod_{k\ne i} p_k= q_j\cdot \prod_{k\ne j} q_k
\]
Das Lemma von Euklid besagt nun, dass wenn eine Primzahl ein Produkt teilt ($p_i$ teilt $n$ und somit auch $q_j\cdot \prod_{k\ne j} q_k$), dann auch einen seiner Faktoren. $q_j$ kann nicht durch $p_i$ geteilt werden, da sonst $q_j=p_i$ wäre, entgegen der Annahme. Also teilt $p_i$ den Rest $\prod_{k\ne j} q_k$. Das würde aber bedeuten, dass $p_i$ im $\prod_{k\ne j} q_k$ Produkt enthalten wäre, und das widerspricht wieder der Annahme, dass die Primfaktoren alle unterschiedlich sind. Daraus folgt, dass es keine unterschiedlichen Zerlegungen gibt.

Mit den Schritten 1 und 2 ist nun bewiesen, dass die Zerlegung einer natürlichen Zahl $>1$ in Primfaktoren existiert und eindeutig ist. 

\qed
\end{proof}

Nach diesen ganzen Vorbereitungen kommen wir nun endlich zu dem, was wir eigentlich beweisen wollten, nämlich:

\begin{theorem}
Es gibt unendlich viele Primzahlen.
\end{theorem}
\begin{proof}
Der Beweis wird durch einen Widerspruch geführt. Wir behaupten, dass es nur endlich viele Primzahlen gibt. Wenn dem so wäre, dann gäbe es eine größte Primzahl $N$. Sei

\[ M = 2\cdot 3\cdot 5\dots \cdot N +1 \]
eine Zahl gebildet aus dem Produkt aller Primzahlen (es gibt ja nur endlich viele) addiert mit 1.

Wir wissen, dass $M$ eine eindeutige Zerlegung in Primfaktoren besitzt, nach Lemma \ref{lem:prim}. Da aber $M$ bereits aus dem Produkt aller bekannten Primzahlen erzeugt wurde, und somit nicht durch eine davon teilbar ist (weil M um 1 größer ist, als das Produkt aller Primzahlen), bleiben nur zwei Möglichkeiten: 

\begin{description}
\item[a)] $M$ ist selbst eine Primzahl, es ist aber $M>N$, was gegen die Voraussetzung verstößt, dass $N$ die größte Primzahl ist. Oder
\item[b)] die Zerlegung von $M$ in Primfaktoren enthält eine Primzahl, die größer ist, als $N$, was wiederum gegen die Voraussetzung verstößt.
\end{description}

Also ist die Behauptung, dass es nur endlich viele Primzahlen gibt, falsch und somit stimmt die eigentliche Aussage, die wir beweisen wollten, nämlich dass es unendlich viele Primzahlen gibt. 

\qed
\end{proof}

\section{Nachreichung}

\subsection{Reelle Zahlen}\label{chap:realfinal}

In Arithmetik Teil, Kapitel \ref{chap:realbegin}, wurden die reellen Zahlen nicht wirklich definiert, was hier nachgeholt werden soll. 

TODO



\section{Historische Anmerkung}

Während eines Mathematik-Studiums trifft man immer wieder auf eine Reihe von Namen in verschiedenen Zusammenhängen. Nach einer gewissen Anzahl von Nennungen wird man ganz automatisch interessiert und versucht etwas über die Person hinter einem benannten Theorem oder Lemma herauszufinden. Das was mich am meisten an den Lebensgeschichten beeindruckte war, dass viele dieser Mathematiker sich mit komplizierten mathematischen Sachverhalten zu Zeiten beschäftigten, in denen geschichtliche oder politische Umwälzungen im Gang waren, sowie Weltanschauungen vertreten wurden, die heute so abenteuerlich erscheinen, dass sie (hoffentlich) niemals wieder Eingang in unsere Gesellschaft finden. Jedoch die mathematischen Erkenntnisse aus diesen Zeiten nahezu unverändert heute noch Bestand haben. 

Ganz besonders gilt dies für die Mathematiker jüdischer Abstammung, die mit Hitlers Machtübernahme am 30. Januar 1933 und den darauf folgenden Repressionen gezwungen waren, Deutschland zu verlassen, Hab und Gut zurück zu lassen und ihre wissenschaftliche Karriere abzubrechen. In den meisten Fällen lag einer Suspendierung oder Entlassung das, am 7. April 1933 erlassene, "`Gesetz zur Wiederherstellung des Berufsbeamtentums"' zugrunde. Ein Gesetz das unter anderem zum Ziel hatte, die rassenpolitischen Ideale der NSDAP zu verwirklichen. 

Im ehrenvollen Gedenken an diese Menschen sei hier eine -- gewiss nicht vollständige -- Aufzählung von Mathematikern jüdischer Abstammung dargelegt, deren Arbeit von den Nazis beendet wurde, und deren Schicksal in Einzelfällen auch sehr tragisch endete. Für die nicht aufgelisteten entschuldige ich mich im Vorfeld und bitte, mir diese zur Kenntnis zu bringen, dass ich sie hier mit aufnehmen kann.


\begin{description}
\item[\textbf{Felix Bernstein}] *14. Februar 1878 in Halle (Saale); \ding{61}3. Dezember 1956 in Zürich. Studierte in Göttingen bei Felix Klein und David Hilbert, bei dem er über die Mengenlehre promovierte. 1934 emigrierte er in die USA um dem Nazi Regime zu entkommen.
\item[\textbf{Ludwig Otto Blumenthal}] *20. Juli 1876 in Frankfurt am Main; \ding{61}12. November 1944 im Konzentrationslager Theresienstadt. Blumenthal war der erste Doktorand von David Hilbert. Im April 1943 wurde das Ehepaar Blumenthal ins KZ Herzogenbusch deportiert. Im Durchgangslager Westerbork verstarb seine Frau Mali, Mai 1943. Schließlich wurde er ins KZ Theresienstadt deportiert, wo er dann im November 1943 an einer Lungenentzündung starb.
\item[\textbf{Max Born}] *11. Dezember 1882 in Breslau; \ding{61}5. Januar 1970 in Göttingen. Born wurde aufgrund des Berufsbeamtengesetztes 1933 zwangsbeurlaubt und 1936 entzog man ihm die deutsche Staatsbürgerschaft. Er emigrierte nach England, wo er bis 1953 blieb. Er kehrte dann nach Deutschland zurück. 1954 erhielt er den Nobelpreis für Physik aufgrund seiner Beiträge zur Quantenmechanik.
\item[\textbf{Richard Courant}] *8. Januar 1888 in Lublinitz; \ding{61}27. Januar 1972 in New York. Nach der Machtergreifung verließ er im Sommer 1933 Deutschland. Nach einem Jahr in Cambridge ging er schließlich nach New York. Dort baute er das Institut für Angewandte Mathematik auf, welches zu seinen Ehren seit 1964 "`Courant Institute for Mathematical Sciences"' heißt.
\item[\textbf{Max Wilhelm Dehn}] *13. November 1878 in Hamburg; \ding{61}27. Juni 1952 in Black Mountain, North Carolina. 1935 verlor er seine Stelle in Frankfurt und verließ 1939 Deutschland zunächst in Richtung Kopenhagen, später nach Trondheim. Schließlich ging er in die USA, wo er aufgrund der Vielzahl von emigrierten Wissenschaftlern Schwierigkeiten hatte eine Stelle zu bekommen. Letztlich wurde er vom Black Mountain College eingestellt, wo er der einzige Mathematiker war. 
\item[\textbf{Adolf Abraham Halevi Fraenkel}] *17. Februar 1891 in München; \ding{61}15. Oktober 1965 in Jerusalem. Im Oktober 1933 schaffte er es mit seiner Familie nach Jerusalem auszuwandern, wo er 1938 Rektor der Universität wurde. 
\item[\textbf{Felix Hausdorff}] *8. November 1868 in Breslau; \ding{61}26. Januar 1942 in Bonn. Trotz seiner im letzten Moment noch geglückten Emeritierung 1935, war Hausdorff im weiteren Anfeindungen ausgesetzt. Er versuchte über Courant in die USA zu emigrieren, doch dies schlug fehl. 1942 wurde Hausdorff mit seiner Frau Charlotte und deren Schwester Edith Pappenheim in ein Kloster in Bonn-Endenich deportiert, als Vorbereitung zur Überführung in ein KZ. Dort nahmen sie sich gemeinsam am 26. Januar 1942 das Leben.
\item[\textbf{Ernst David Hellinger}] *30. September 1883 in Striegau; \ding{61}28. März 1950 in Chicago. 1936 von den Nationalsozialisten in den Zwangsruhestand versetzt, weigerte sich Hellinger bis 1939 aus Deutschland zu fliehen. 13. November 1938 wurde er ins Konzentrationslager Dachau deportiert. Aufgrund der Fürsprache von Freunden und unter der Bedingung, dass er emigrieren würde, entließ man Hellinger aus dem KZ. Im Februar 1939 reiste er in die USA und blieb dort bis zu seinem Tode.
\item[\textbf{Edmund Landau}] *14. Februar 1877 in Berlin; \ding{61}19. Februar 1938 in Berlin. 1934 wurde Landau in den Zwangsruhestand versetzt und konnte bis zu seinem Tod nur noch sporadisch in Brüssel und Cambridge lehren. 
\item[\textbf{Emmy Noether}] *23. März 1882 in Erlangen; \ding{61}14. April 1935 in Bryn Mawr, Pennsylvania. Emmy Noether behauptete sich als eine der ganz wenigen Frauen in der Mathematik. Ihr wurde 1917 die Habilitation verweigert aufgrund eines Erlasses, der dies für Frauen unzulässig machte. David Hilbert nahm sie als Assistentin, für den sie Vorlesungen unter dessen Namen hielt. Ab 1933 wurde ihr aber die Lehrerlaubnis entzogen, sie schaffte es glücklicherweise in die USA, wo sie für kurze Zeit am Women's College Bryn Mawr lehrte. Sie verstarb allerdings 1935 an den Folgen einer Operation.
\item[\textbf{Hugo Dionizy Steinhaus}] *4. Januar 1887 in Jas\l o; \ding{61}25. Februar 1972 in Breslau. Steinhaus und seine Frau lebten ab Juli 1941 im Untergrund, so entgingen sie einer Deportation. Sie schafften es, sich bis 1945 durchzuschlagen und zogen dann nach Breslau. Dort konnte er unter seinem Namen wieder arbeiten und nahm eine Professur an der Universität Breslau an.
\item[\textbf{Otto Toeplitz}] *1. August 1881 in Breslau; \ding{61}15. Februar 1940 in Jerusalem. Toeplitz konnte noch bis 1935 in Bonn lehren. Er war unter anderem mit Hausdorff befreundet. Anfang 1939 wurde der Druck auf ihn so groß, dass er nach Palästina emigrierte. Dort verstarb er nur ein Jahr später. 

\end{description}






\chapter{Lösungen}


\section{Ganze Zahlen}
\begin{sol}{arith.1.1}
\begin{center}
\begin{tabular}{C{2cm}C{2cm}C{2cm}C{2cm}}
falsch & richtig & richtig & richtig \\
richtig & richtig & falsch & richtig \\
richtig & richtig & richtig & falsch
\end{tabular}
\end{center}
\end{sol}


\begin{sol}{arith.1.2}

\begin{eqnarray*}
2+3 &=& 5 \hskip 1cm | +2 \\
2+3+2 &=& 5+2 \hskip 1cm | \cdot 3 \\
(2+3+2) \cdot 3 &=& (5+2)\cdot 3 \\
2\cdot 3 + 3\cdot 3 + 2\cdot 3 &=& 5\cdot 3 +2 \cdot 3 \\
6+9+6 &=& 15 + 6 \\
21 &=& 21
\end{eqnarray*}

\end{sol}

\begin{sol}{arith.1.3}
\begin{center}
\begin{tabular}{C{2cm}C{2cm}C{2cm}C{2cm}}
$8+9=17$ & $8-9 =-1$ & $3\cdot 4=12$ & $8/2=4$ \\
$13+5=18$ & $13-5=8$ & $13\cdot 5=65$ & $27/3=9$ \\
$17+3=20$ & $17-3=14$ & $17\cdot 3=51$ & $25/5=5$
\end{tabular}
\end{center}
\end{sol}

\begin{sol}{arith.1.4}
\[-(2-3) = (-1)\cdot (2-3) = (-1)\cdot 2 + (-1)\cdot (-1) \cdot 3 = -2+3\]
Weil $(-1)\cdot (-1)=1$ und $1\cdot 3 = 3$.
\end{sol}

\section{Vektor, Matrix, Tensor}

\begin{sol}{matrix.1}

Seien $A,B,C\in \mathbb{R}^{m\times n}$ und $\alpha , \beta \in \mathbb{R}$. Der Nachweis wird ausschließlich auf den einzelnen Komponenten der Matrix geführt, da Addition und skalare Multiplikation ausschließlich auf den Komponenten der Matrix ausgeführt werden. Daraus folgt praktisch schon die Behauptung, dass $\mathbb{R}^{m\times n}$ ein Vektorraum ist. Aber hier sei der Nachweis noch vorgerechnet:

\begin{description}
\item[(A1)] $A+(B+C) = (a_{i,j}+(b_{i,j}+c_{i,j}))_{i,j} = ((a_{i,j}+b_{i,j})+c_{i,j})_{i,j} = (A+B)+C$. Hierbei wird die Assoziativität der reellen Zahlen verwendet.
\item[(A2)] $ (a_{i,j}+0)_{i,j} = (0+a_{i,j})_{i,j} = (a_{i,j})_{i,j}$
\item[(A3)] Da $\mathbb{R}$ ein Körper ist, gibt es zu jedem $a_{i,j}\in \mathbb{R}$ ein $ -a_{i,j}\in \mathbb{R}$, sodass $a_{i,j} +(-a_{i,j}) = -a_{i,j} + a_{i,j} = 0$ daraus folgt, dass $A+(-A) = -A +A = 0$.
\item[(A4)] $ A+B = (a_{i,j}+b_{i,j})_{i,j} = (b_{i,j}+a_{i,j})_{i,j} = B+A$
\item[(M1)] $ (\alpha \beta) A = ((\alpha \beta)a_{i,j})_{i,j} = (\alpha (\beta a_{i,j})_{i,j} = \alpha (\beta A) $
\item[(M2)] $ I\cdot A = ( \sum_{k=1}^{n} I_{i,k}a_{k,j})_{i,j} $, da $I_{i,k}=0$ für alle $i\ne k$, bleibt als Ergebnis der Summe nur $I_{i,i}a_{i,j} = a_{i,j} $ und damit ist $I\cdot A = A$
\item[(M3)] $ (\alpha +\beta )A = ((\alpha + \beta)a_{i,j})_{i,j} = (\alpha a_{i,j} + \beta a_{i,j})_{i,j} = \alpha A + \beta A $
\item[(M4)] $ \alpha (A+B) = (\alpha(a_{i,j} + b_{i,j})_{i,j} = (\alpha a_{i,j} + \alpha b_{i,j})_{i,j} = \alpha A + \alpha B $
\end{description}

\end{sol}


\backmatter
% bibliography, glossary and index would go here.

\begin{thebibliography}{123}

\bibitem{Brieskorn1}
  Egbert Brieskorn,
  \emph{Lineare Algebra und analytische Geometrie I}.
  Vieweg, Wiebaden; Braunschweig,
  1. Auflage, Nachdruck,
  1983/1985.

\end{thebibliography}

\printindex

\end{document}