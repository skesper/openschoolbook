\documentclass[graybox,envcountchap,sectrefs,deutsch]{openschoolbook}

\usepackage[utf8]{inputenc}
\usepackage{mathptmx}
\usepackage{eurosym}
\usepackage{type1cm}         
\usepackage{booktabs}
\usepackage{array}
\usepackage{needspace}
\usepackage{makeidx}
\usepackage{graphicx}
\usepackage{multicol}
\usepackage[bottom]{footmisc}

\usepackage[T1]{fontenc}
\usepackage[ngerman]{babel}

\usepackage{stmaryrd}

\usepackage{pifont} % \ding{61} = Death Symbol, cross

\usepackage{bbding} % Hands and Point Symbols

\usepackage{color}
\usepackage{tikz}
\usetikzlibrary{calc}


% Farben für Konstruktionen
\definecolor{colPkt}{rgb}{0,0,1} % für gegebene Punkte
\definecolor{colPktKon}{rgb}{0.49,0.49,1} % für konstruierte Punkte
\definecolor{colWin}{rgb}{1,0,0} % für Winkel

\newcolumntype{L}[1]{>{\raggedright\let\newline\\\arraybackslash\hspace{0pt}}m{#1}}
\newcolumntype{C}[1]{>{\centering\let\newline\\\arraybackslash\hspace{0pt}}m{#1}}
\newcolumntype{R}[1]{>{\raggedleft\let\newline\\\arraybackslash\hspace{0pt}}m{#1}}

\usepackage{amssymb}
\usepackage{amsmath}

\def\currency{\officialeuro} % de
%\def\currency{\$} % us
% ...

\makeindex

\author{Stephan Kesper}
\title{Einführung in die Mathematik}
\subtitle{Zum Selbststudium oder als Grundlage für den Unterricht}


\smartqed

\begin{document}

\maketitle

\frontmatter

\preface

Dies ist ein sogenanntes "`offenes"' Schulbuch, das bedeutet, dass dieses Buch kostenlos jedem ohne Vorbehalt zur Verfügung gestellt wird. Es enthält Wissen, das von Freiwilligen ohne finanzielle Interessen zusammengetragen wurde. 

Es wird unter einer Commons-Creative-Lizenz veröffentlicht, wie unten angegeben. Lehrer sind herzlich willkommen Inhalte dieses Buches im Unterricht zu verwenden, auch in Teilen und Auszugsweise. Verlage können das Buch als solches Drucken und in gedruckter Form vertreiben, solange die Creative-Commons-Lizenz dadurch nicht beeinträchtigt wird.

Die Mathematik stellt zusammen mit der Sprache, eine Basis allen Wissens dar. Jedoch ist die Sprache grundlegender in dem Sinne, das sie Voraussetzung für die Erklärungen zum Verständnis der Mathematik ist. Folgerichtig ist dieses Buch erst dann zu verwenden, wenn der Lernende bereits ein gewisses Grundverständnis von Sprache besitzt. 

\bigskip

Mathematik ist eine der wenigen Wissenschaften, deren Elemente vollständig aus dem Geist von Menschen entstanden sind. Es gibt kein "`natürliches"' Vorbild für die mathematischen Elemente. Daher ist sie eine der Wissenschaften, deren Inhalte zumindest von einem Menschen bereits verstanden worden sein müssen, nämlich jenem, der die Inhalte aufschrieb und gegebenenfalls bewies.

Das bedeutet, dass die Mathematik vollständig verstehbar ist, sofern der Lernende genug Zeit und Willen aufbringt, sich mit ihr auseinander zusetzen. Diese Erkenntnis steht in direktem Widerspruch zur landläufigen Meinung, dass Mathematik schwer verständlich und unzugänglich ist. Warum dies so ist, wird von Fall zu Fall unterschiedlich sein und es gibt wohl kaum eine allgemeine Begründung. Doch viele Menschen lassen sich speziell von den Formeln abschrecken. So kompliziert diese Formeln auch im Einzelnen aussehen möchten, so sind sie doch unabdingbar in dem Sinne, dass sie vollkommen unzweideutig Sachverhalte darstellen können. Und zwar auf einem Niveau dem natürliche Sprachen weit hinterher hinken. 

Die mathematischen Formalismen\footnote{Im Sinne von Formel und nicht von einer Form-gerechten Vorgehensweise.} werden in Schulen meist auf eine Weise den Schülern beigebracht, dass sie sich davon abgeschreckt fühlen. Auf der anderen Seite sind mathematische Kenntnisse in vielen Bereichen des täglichen Lebens durchaus wichtig, und sei es nur um die Zahlenspielereien der Banken und Versicherungen auf Kredit- und Versicherungsverträgen zu durchschauen. 

Dieses Buch versucht in erster Linie Wissen zu vermitteln. Die Didaktik, wie es vermittelt werden soll, bleibt dabei -- schon aufgrund fehlender didaktischer Fachkenntnis -- auf der Strecke. Es ist also im Sinne einer Referenz zu verstehen, sodass sehr viel tieferes Wissen zur Verfügung steht, als dies in Schulen benötigt wird. Nichts desto trotz erscheint es dem Autor aber sinnvoll das Buch auf diese Weise zu schreiben, da es dann interessierten Lernenden die Möglichkeit gibt, sich auch weit über das Schulwissen hinaus zu bilden. In der Hoffnung, dass es zumindest für einige Menschen von Nutzen sein wird.


\vspace{\baselineskip}
\begin{flushright}\noindent
Koblenz, \today \hfill {\it Stephan Kesper}
\end{flushright}

\vfill

\noindent Dieser Inhalt ist unter der Creative-Commons-Lizenz vom Typ Namensnennung - Nicht-kommerziell - Weitergabe unter gleichen Bedingungen 3.0 Unported lizenziert. Um eine Kopie dieser Lizenz einzusehen, besuchen Sie

\bigskip
\begin{center}
\texttt{http://creativecommons.org/licenses/by-nc-sa/3.0/}
\end{center}

\bigskip

\noindent oder schreiben Sie einen Brief an Creative Commons, 444 Castro Street, Suite 900, Mountain View, California, 94041, USA.




\tableofcontents

\listoffigures


\mainmatter

\chapter{Vorbemerkungen}

\section{Zur Schulmathematik}

In der Schule eingesetzte Mathematik-Bücher haben (in einem gewissen Sinne völlig zurecht) die Eigenschaft, die Vermittlung des Wissens vor das Wissen selbst zu stellen. Am Anfang der Entwicklung ist es wichtig, den Lernenden nicht zu demotivieren und ihn nicht vor einen Berg Wissen zu stellen, um ihm dann das Gefühl zu geben, diesen alleine besteigen zu müssen. Das ist ein gutes Vorhaben und es soll hier nicht daran gerüttelt werden. 

Dem gegenüber steht die Erfahrung, die ich während meiner Schulzeit machte, dass die Informationen, die man als Schüler bekam -- im Besonderen aus den Mathematik-Büchern -- nicht vollständig erschienen. Dass elementare Informationen fehlten, und dass das Abarbeiten von Textaufgaben nicht der zentrale Dreh- und Angelpunkt einer Mathematikerkarriere sein konnte (nicht, dass ich eine solche je anstrebte).

Im ersten Semester Analysis sowie Lineare Algebra an der Hochschule wurde von den Professoren dann genau dies angeprangert: Die Schulmathematik kümmere sich um viele Teilbereiche der Mathematik, aber um keinen davon auf korrekte Art und Weise. Eine ernüchternde Frage des Analysis-Professors brachte seine Ansichten auf den Punkt: Er fragte die anwesenden Studenten (gute 500), ob sie in der Schule Differentialrechnung hatten. Alle, soweit ich sie sehen konnte, hoben die Hand. Nächste Frage: Welche hatten von der Vollständigkeit der reellen Zahlen gehört? Zwei oder drei Arme erhoben sich. Seine Reaktion war entsprechend. 

Mathematik ist eine reiche und lebendige Wissenschaft. Ihre Theorien mögen für Nicht-Mathematiker kompliziert und undurchdringlich erscheinen. Es wird oft vom viel beschworenen Elfenbein-Turm geredet. Aber letztlich ist es die Mathematik, die es Wissenschaftlern anderer Disziplinen erlaubt zu tun, was sie tun, indem sie die formalen Grundlagen und Werkzeuge zur Verfügung stellt, die ebendiese brauchen zur Formulierung ihrer Ideen.

Mathematik erschöpft sich nicht im Überprüfen des Kassenbons im Supermarkt. Ebenso wenig machen sich Mathematiker ständig Gedanken darüber, wie schnell Arbeiter einen Graben ausheben können und wie man dies beschleunigen könnte. 

Das führt natürlich direkt zur Frage: Was ist Mathematik? Eine umfassende Antwort auf diese Frage kann hier nicht gegeben werden. Jedoch soviel zumindest: Im Gegensatz zu dem, was viele glauben, besteht Mathematik zu einem großen Teil aus Argumentieren und dem Beweisen von Behauptungen. Da geschieht Mathematik! Nicht beim Ausrechnen, wie viele Quadratmeter eine Wohnung hat, nicht beim Bestimmen der Mehrwertsteuer oder beim Aufsummieren von gekauften Artikeln. Dies würde man mit dem Begriff \textsl{Rechnen} verbinden und das Fach Mathematik in den Schulen sollte entweder in "`Rechnen"' umbenannt werden, oder -- was meiner Vorstellung eher entspräche -- die Schüler sollten näher an das herangebracht werden, was Mathematik ausmacht: Verstehen, abstrahieren, erweitern, beweisen. Dann würden sie im ersten Semester an der Universität wissen, warum man die Vollständigkeit zum Verständnis des Differentialbegriffs benötigt. 

In diesem Sinne ist dieses Buch zu verstehen. Es konzentriert sich nicht auf die Art der Wissensvermittlung. Es konzentriert sich darauf konsistent zu sein. Nicht vollständig, denn das wäre gar nicht möglich. Jeder Teil dieses Buches könnte gleich mehrere Bücher füllen. Aber wenn etwas in dieses Buch aufgenommen wurde, dann sind alle zum Verständnis notwendigen Teile ebenfalls vorhanden.


\section{Definitionssprint!}

Das Verständnis von bestimmten Sachverhalten in der Mathematik wird meist dadurch verbessert, in dem alle notwendigen Definitionen, die für den Sachverhalt bestimmend sind, an einer Stelle zusammengefasst sind. So können zum Beispiel Ringe nur dann verstanden werden, wenn der Lernende weiß, was Gruppen und im Besonderen abelsche Gruppen sind. Sowie es notwendig ist zu verstehen, was eine Menge zu einer Gruppe macht. 

Die zu einem bestimmten Thema gehörenden Definitionen werden in sogenannten "`Definitionssprints"' zusammengefasst und können dort schnell und übersichtlich nachvollzogen werden. In diesen Sprints stehen alle neuen und zum Thema gehörigen Definitionen. Solche aus vorherigen Kapiteln werden vorausgesetzt, sodass die Definitionssprints aufeinander aufbauen.

Definitionssprints können an den Überschriften erkannt werden die -- offensichtlicherweise -- "`Definitionssprint!"' genannt wurden.

\section{Hinweise}

Es gibt zwei verschiedene Arten von Hinweisen: 

\begin{svgraybox}
Die grau hinterlegte Box beinhaltet wichtige Hinweise und sollten unbedingt beachtet werden. Falls die Hinweise für eine eingeschränkte Empfängergruppe bestimmt sind, ist dies am Anfang der Box bemerkt. Also z.B. "`Hinweis für Lehrer"'. 
\end{svgraybox}

\bigskip

\HandRight \qquad Die Hinweise mit "`Hand"' sind als Tipps zu verstehen. An diesen Stellen werden praktische Hinweise gegeben, die das Leben vereinfachen, oder auf einen Interessanten Sachverhalt aufmerksam machen wollen. Sie sind für das Verständnis nicht notwendig. 

\bigskip

Namen von historischen Personen, wie z.B. Blaise Pascal\footnote{\textbf{Blaise Pascal}, französischen Mathematiker *19. Juni 1623 in Clermont-Ferrand, \ding{61}19. August 1662 in Paris.}, werden im allgemeinen als Fußnote dargestellt. 


\section{Unbekannte in Gleichungen}

Im ersten Teil (Arithmetik) werden Unbekannte Werte in (Un-)Gleichungen immer mit einem "`?"' Fragezeichen dargestellt. Das soll unterstreichen, dass wir den Wert nicht kennen. Unabhängig davon, dass das Fragezeichen sehr selten in mathematischen Gleichungen auftaucht, sollte es dem Lernenden egal sein, ob an dieser Stelle ein Fragezeichen oder eine Unbekannte "`x"' auftaucht, die in solchen Momenten sehr viel öfter gewählt wird. Ein Zeichen ist so gut wie jedes andere um eine Unbekannte darzustellen. Es hätte auch ein exotisches Zeichen sein können, wie \ding{65}, oder \ding{68}. Da aber das Fragezeichen von seiner alltäglichen Verwendung schon auf etwas unbekanntes hinweist, wurde dies gewählt. 





\chapter{Natürliche Zahlen}

\section{Was sind natürliche Zahlen?}

Der Mengenbegriff (im Sinne von Anzahl) birgt ein natürliches Verständnis dafür, was eine Zahl ist. Im Verständnis der meisten Menschen ist eine Zahl unmittelbar mit dem Begriff der Anzahl von Dingen verbunden --- zum Beispiel drei Äpfel oder fünf Orangen.

Um einem Obsthändler zu erklären, wie viele Äpfel man haben möchte, ist die Nutzung von Zahlen durchaus praktisch. Genauso wie beim Metzger Zahlen etwas abstrakter verwendet werden. So möchte man 500 Gramm Hackfleisch. Das Hackfleisch besteht nicht aus 500 Einzelteilen, sondern die Zahl bezeichnet das Gewicht dessen, was man bestellt.

Diese Zahlen werden in der Mathematik als \textbf{natürliche} Zahlen bezeichnet. Ihre Gesamtheit, d.h. alle natürlichen Zahlen inklusive etwas, das als "`unendlich"' bezeichnet wird -- darauf kommen wir später zurück -- wird mit dem Zeichen $\mathbb{N}$ abgekürzt.

Die folgenden Symbole bezeichnen die ersten neun natürlichen Zahlen. 

\[ 
1, 2, 3, 4, 5, 6, 7, 8, 9
\]

Sie bilden den Grundstock aller folgenden Zahlen, die aus diesen zusammengesetzt sind. Wie diese zusammen gesetzt werden, erfahren wir gleich.

\section{Operationen}

\subsection{Gleichheit, Ungleichheit, Vergleiche}

Mit dem Symbol "`="' wird die Anforderung beschrieben, dass alles, was auf der linken Seite des Symbols steht den selben Wert hat wie das, was auf der rechten Seite des Symbols steht. Folgendes ist demnach korrekt:

\begin{eqnarray*}
1 &=& 1 \\
5 &=& 5 \\
7 &=& 7
\end{eqnarray*}
Während das folgende falsch ist:
\begin{eqnarray*}
1 &=& 7 \\
5 &=& 3 \\
7 &=& 1
\end{eqnarray*}

Es liegt in der Verantwortung desjenigen, der die Gleichung aufstellt, dafür zu garantieren, dass die Gleichheit auch wirklich erfüllt ist. Papier ist geduldig! Man kann hinschreiben, was man will, ob ein hingeschriebenes Gleichheitszeichen auch wirklich Gleichheit bedeutet, kann nur der Hinschreibende wissen.

Möchte man ausdrücken, dass die linke und rechte Seite nicht übereinstimmt, so verwendet man das Symbol "`$\neq $"'. So werden oben angegebene Falschaussagen wieder korrekt indem man das Ungleichzeichen verwendet:
\begin{eqnarray*}
1 &\neq & 7 \\
5 &\neq & 3 \\
7 &\neq & 1
\end{eqnarray*}

Der Ungleichheit stehen noch qualifizierende Ungleichzeichen zur Seite. Dass 1 nicht gleich 7 ist stimmt zwar, ist aber weniger interessant, als die Aussage, dass 1 kleiner als 7 ist. Wenn Hans einen Apfel besitzt und Peter zwei, dann hat Hans weniger Äpfel als Peter, wie Peter mehr Äpfel hat als Hans. Die drückt man durch die folgenden Symbole aus:

\begin{eqnarray*}
1 & < & 7 \\
5 & > & 3 \\
7 & > & 1
\end{eqnarray*}

Hat man eine Ungleichung aufzustellen, bei der es akzeptabel ist, dass beide Seite auch den gleichen Wert haben können, so verwendet man die selben Symbole mit einem Unterstrich:
\begin{eqnarray*}
1 & \le & 7 \\
5 & \ge & 3 \\
7 & \ge & 1
\end{eqnarray*}

Das Symbol "`$\le$"' wird "`kleiner oder gleich"' ausgesprochen und das Symbol "`$\ge$"' "`größer oder gleich"'. Es ist zu beachten, dass folgende Ungleichungen ebenfalls \textbf{alle} korrekt sind:
\begin{eqnarray*}
4 & \le & 5 \\
5 & \le & 5 \\
6 & \ge & 5 \\
5 & \ge & 5
\end{eqnarray*}

Wenn für beiden Seiten einer Ungleichung sowohl das $\le$ als auch das $\ge$ korrekt ist, so gilt $=$ Gleichheit. Dies sollte im Kopf behalten werden, denn es gibt einige Beweise in der Mathematik, die genau dies ausnutzen.


\subsection{Addition und Multiplikation}

Um mit natürlichen Zahlen rechnen zu können, definieren wir zwei Operationen. Die Multiplikation, dargestellt durch das Zeichen "`$\cdot$"' und die Addition, dargestellt durch das Zeichen "`$+$"'.

Mit der Addition fügen wir einzelne Mengen zu größeren Mengen zusammen. Hat Hans $2$ Äpfel und Peter $3$ Äpfel, so haben sie zusammen $5$ Äpfel, oder einfacher formuliert:
\[
2+3=5
\]

\noindent Hätte Peter keinen Apfel
\[2+?=2\]
dann wäre eine Zahl hinzuzuzählen, die keinen Wert besitzt. Eine solche Zahl wird mit dem Symbol "`0"' bezeichnet. Sie ist eine Invariante bezüglich der Addition -- d.h. Additionen mit dieser Zahl ändern nicht den Wert der ursprünglichen Zahl. Man nennt sie "`Null"'. Demnach erweitern sich die Symbole der ersten zehn natürlichen Zahlen auf diese Weise:

\[0,1,2,3,4,5,6,7,8,9\]

Die Null ist per Definition nicht Teil der natürlichen Zahlen. Das hat historische Gründe -- so dachte man im Mittelalter, dass die Zahl Null vom Teufel erdacht worden sei\footnote{Historischer Beleg?}. Für unsere Betrachtung sollte sie aber Teil der natürlichen Zahlen sein. Daher werden wir von nun an mit einer Zahlenmenge umgehen, die mit $\mathbb{N}_0$ bezeichnet wird, sie besteht aus allen natürlichen Zahlen, inklusive der 0.

\textsl{Was kommt nach der 9?}

Die Zahl, die nach der 9 kommt, ist definiert durch die Summe $9+1$. Es wurde in frühen Zeiten einmal festgelegt\footnote{Historischer Beleg?}, dass wir ein Zahlensystem bestehend aus zehn Ziffern verwenden. Die zehn Ziffern beinhalten die Null, daher ist für die zehn kein eigenes Symbol mehr übrig. Daher wurden die zusammengesetzten Zahlen erfunden. So ist die Symbolfolge
\[10\]
diejenige Zahl, die nach der 9 kommt, also
\[9+1=10\]
Das ist eine Definition die bedeutet, dass wenn einem die Symbole ausgehen, kann man eine zusätzliche Stelle verwenden um diese darzustellen. Zusätzliche -- im Sinne von höherwertigen -- Stellen werden links an die Zahl angehängt. Das funktioniert auch mit drei, vier und allen weiteren Stellen:
\begin{eqnarray*}
99+1 &=& 100 \\
999+1 &=& 1000 \\
\cdots
\end{eqnarray*}

\noindent \textsl{Wie werden aber dann Zahlen zusammengesetzt?}

\noindent Sehen wir uns ein Beispiel an:
\[9+5 = ?\]
Wir wissen, dass 
\[9+1 = 10\]
ist, und dass
\[1+4 = 5\]
So können wir schreiben:
\[9+1+4 = 10 + 4 = 14\]
An dieser Stelle haben wir eine Eigenschaft verwendet, die das \textbf{Assoziativgesetz} genannt wird. Wir haben 5 durch die Summe von zwei Zahlen ersetzt und die gesamte Summe anders kombiniert. Sehen wir uns das mit Klammern an:
\[9+5 = 9+(1+4) = (9+1)+4 = 10+4 = 14\]
Klammern bevorzugen eine Operation. So ist
\[9+(1+4)=9+5\]
dass wir die 1 von $1+4$ wegnehmen und zu $9+1$ hinzufügen dürfen ist in der Anschauung vollkommen klar, ob nun Hans 9 Äpfel und Peter 5 Äpfel besitzen, oder Hans 10 und Peter 4, macht für die Gesamtanzahl keinen Unterschied. Aber dieser Operation liegt das obengenannte Assoziativgesetz zugrunde, das wir im restlichen Buch immer wieder verwenden werden. Dass wir es verwenden dürfen, liegt an der "`Harmlosigkeit"' der natürlichen Zahlen $\mathbb{N}$. Wir werden später noch Konstrukte kennen lernen, die das Assoziativ- und im Besonderen das Kommutativgesetz nicht erfüllen. Trotzdem kann man mit diesen Konstrukten genauso rechnen, wie wir das mit den natürlichen Zahlen tun, man darf mit ihnen nur nicht alles machen, was man mit Zahlen tun kann.

Die nächste Operation, die wir kennenlernen, ist die Multiplikation:
\[3\cdot 4 = ?\]
Diese entspricht ebenfalls der Anschauung. Lägen in drei Körben jeweils vier Äpfel, so hätte man insgesamt
\[\underbrace{4+4+4}_{3 \, \mathrm{mal}} = 12\]
Äpfel. Also sind 
\[3\cdot 4 = 12\]
Das Ergebnis bliebe gleich, wenn man vier Körbe mit jeweils drei Äpfeln hätte:
\[3\cdot 4 = 4\cdot 3 = 12\]
Das ist das \textbf{Kommutativgesetz}.

Gehen wir jetzt mal davon aus, wir hätten drei Körbe mit jeweils 2 grünen und 4 roten Äpfeln. Wie viele grüne Äpfel, wie viele rote und wie viele Äpfel insgesamt hätten wir dann?

Die einzelnen Summen können wir leicht bestimmen: 
\[3\cdot 2 = 6\]
grüne Äpfel,
\[3\cdot 4 = 12 \]
rote Äpfel und somit
\[6+12 = 18\]
Äpfel insgesamt. Dabei haben wir zunächst nur die Anzahl der grünen, dann die der roten Äpfel berechnet. Wir haben also folgendes gemacht:
\[3\cdot \underbrace{(2+4)}_{\mathrm{Inhalt\ eines\ Korbes}} = 3\cdot 2 + 3\cdot 4 = 18\]
Zur Berechnung der Gesamtsumme hätten wir aber auch gleich die roten und grünen Äpfel pro Korb summieren können:
\[3\cdot (2+4) = 3\cdot (6) = 3\cdot 6 = 18 \]

Dass wir zuerst die Anzahlen aller grünen Äpfel und dann die aller roten Äpfel berechnen konnten, wird als  \textbf{Distributivgesetz} bezeichnet.

\subsection{Subtraktion und Division}

Wenn Hans drei Äpfel besitzt und zwei davon isst, bleibt ihm nur einer übrig. Die Frage, die sich der Lernende stellen sollte ist:

\textsl{Welche Zahl erfüllt folgende Gleichung?}

\[3 + ? = 1\]







Der Grund warum Subtraktion und Division ein eigenes Unter-Kapitel bilden liegt darin begründet, dass sie nicht das Kommutativgesetz beachten. 

\subsection{Bruchrechnung}

\subsection{Potenzrechnung}


\chapter{Ganze Zahlen}

Der Mengenbegriff (im Sinne von Anzahl) birgt ein natürliches Verständnis dafür, was eine Zahl ist. Im Verständnis der meisten Menschen ist eine Zahl unmittelbar mit dem Begriff der Anzahl von Dingen verbunden --- zum Beispiel drei Äpfel oder fünf Orangen.

Um einem Obsthändler zu erklären, wie viele Äpfel man haben möchte, ist die Nutzung von Zahlen durchaus praktisch. Genauso wie beim Metzger Zahlen etwas abstrakter verwendet werden. So möchte man 500 Gramm Hackfleisch. Das Hackfleisch besteht nicht aus 500 Einzelteilen, sondern die Zahl bezeichnet das Gewicht dessen, was man bestellt.

\section{Natürliche Zahlen}

Dem Ziel, zu erklären, was ganze Zahlen sind, nähern wir uns über den Umweg der natürlichen Zahlen, die einen wesentlichen Bestandteil der ganzen Zahlen bilden. 


\begin{definition}
Zahlen -- im Sinne von Anzahl -- werden in der Mathematik als \emph{natürliche} Zahlen bezeichnet. Ihre Gesamtheit, d.h. alle natürlichen Zahlen inklusive etwas, das als "`unendlich"' bezeichnet wird -- darauf kommen wir später zurück -- wird mit dem Zeichen $\mathbb{N}$ \index{Natürliche Zahlen} abgekürzt.
\end{definition}

Die folgenden Symbole bezeichnen die ersten neun natürlichen Zahlen. 

\[ 
1, 2, 3, 4, 5, 6, 7, 8, 9
\]

Sie bilden den Grundstock aller folgenden Zahlen, die aus diesen zusammengesetzt sind. Wie diese zusammen gesetzt werden, erfahren wir gleich.

\section{Grundlagen}

\subsection{Gleichheit, Ungleichheit, Vergleiche}

\begin{definition}
Mit dem Symbol "`="' wird die Anforderung beschrieben, dass alles, was auf der linken Seite des Symbols steht den selben Wert hat wie das, was auf der rechten Seite des Symbols steht. \index{Gleichheit =}
\end{definition}
Folgendes ist demnach korrekt:

\begin{eqnarray*}
1 &=& 1 \\
5 &=& 5 \\
7 &=& 7
\end{eqnarray*}
Während das folgende falsch ist:
\begin{eqnarray*}
1 &=& 7 \\
5 &=& 3 \\
7 &=& 1
\end{eqnarray*}

Es liegt in der Verantwortung desjenigen, der die Gleichung aufstellt, dafür zu garantieren, dass die Gleichheit auch wirklich erfüllt ist. Papier ist geduldig! Man kann hinschreiben, was man will, ob ein hingeschriebenes Gleichheitszeichen auch wirklich Gleichheit bedeutet, kann nur der Hinschreibende wissen.

Für Gleichungen gilt im allgemeinen, dass Gleichheit erhalten bleibt, wenn auf beiden Seiten der Gleichung die selben Operationen ausgeführt werden. Multipliziert man beide Seiten mit der selben Zahl, oder addiert auf beiden Seiten die selbe Zahl, so bleibt die Gleichheit erhalten.

\begin{eqnarray*}
5 &=& 5  \hspace{1cm}| +1\\
5+1 &=& 5+1 \\
6 &=& 6
\end{eqnarray*}

Das, was man auf beiden Seiten einer Gleichung macht, kann durch einen abgesetzten senkrechten Strich dargestellt werden. Alles, was auf der rechten Seite des Strichs steht, wird auf beiden Seiten der Gleichung angewendet. Hier also jeweils eine 1 addiert.

\begin{definition}
Möchte man ausdrücken, dass die linke und rechte Seite nicht übereinstimmt, so verwendet man das Symbol "`$\neq $"'. 
\end{definition}
So werden oben angegebene Falschaussagen wieder korrekt indem man das Ungleichzeichen verwendet: \index{Ungleichheit $\ne$}

\begin{eqnarray*}
1 &\neq & 7 \\
5 &\neq & 3 \\
7 &\neq & 1
\end{eqnarray*}

Der Ungleichheit stehen noch qualifizierende Ungleichzeichen zur Seite. Dass 1 nicht gleich 7 ist stimmt zwar, ist aber weniger interessant, als die Aussage, dass 1 kleiner als 7 ist. Wenn Hans einen Apfel besitzt und Peter zwei, dann hat Hans weniger Äpfel als Peter, wie Peter mehr Äpfel hat als Hans. Dies drückt man durch die folgenden Symbole aus: \index{Kleiner als $<$} \index{Größer als $>$}

\begin{eqnarray*}
1 & < & 7 \\
1 & < & 2 \\
2 & > & 1
\end{eqnarray*}

Hat man eine Ungleichung aufzustellen, bei der es akzeptabel ist, dass beide Seite auch den gleichen Wert haben können, so verwendet man die selben Symbole mit einem Unterstrich:
\begin{eqnarray*}
1 & \le & 7 \\
5 & \ge & 3 \\
7 & \ge & 1
\end{eqnarray*}

Das Symbol "`$\le$"' wird "`kleiner oder gleich"' \index{Kleiner oder gleich $\le$} ausgesprochen und das Symbol "`$\ge$"' "`größer oder gleich"'\index{Größer oder gleich $\ge$}. Es ist zu beachten, dass folgende Ungleichungen ebenfalls \textbf{alle} korrekt sind:
\begin{eqnarray*}
4 & \le & 5 \\
5 & \le & 5 \\
6 & \ge & 5 \\
5 & \ge & 5
\end{eqnarray*}

Wenn für beiden Seiten einer Ungleichung sowohl das $\le$ als auch das $\ge$ korrekt ist, so gilt $=$ Gleichheit. Dies sollte im Kopf behalten werden, denn es gibt einige Beweise in der Mathematik, die genau dies ausnutzen.

Auch für Ungleichungen gilt, dass Operationen, die auf beiden Seiten ausgeführt werden, die Ungleichung erhalten. Das gilt bei Ungleichungen allerdings nur in etwas eingeschränkter Form, nämlich die Multiplikation mit einer negativen Zahl ($<0$) führt dazu, dass sich das Ungleichzeichen umdreht. Aber dazu später mehr.

\subsection{Addition und Multiplikation}

\begin{definition}
Um mit natürlichen Zahlen rechnen zu können, definieren wir zwei Operationen. Die Multiplikation\index{Multiplikation}, dargestellt durch das Zeichen "`$\cdot$"' und die Addition\index{Addition}, dargestellt durch das Zeichen "`$+$"'. Diese Operationen werden als \emph{Verknüpfung} bezeichnet, weil sie zwei Zahlen miteinander "`verknüpfen"' zu einer neuen Zahl.
\end{definition}

Mit der Addition fügen wir einzelne Mengen zu größeren Mengen zusammen. Hat Hans $2$ Äpfel und Peter $3$ Äpfel, so haben sie zusammen $5$ Äpfel, oder einfacher formuliert:
\[
2+3=5
\]

\noindent Hätte Peter keinen Apfel
\[2+?=2\]
dann wäre eine Zahl hinzuzuzählen, die keinen Wert besitzt. 
\begin{definition}
Eine Zahl ohne Wert wird mit dem Symbol "`0"' dargestellt.\index{Null "`0"'}
\end{definition}

Sie ist eine Invariante bezüglich der Addition -- d.h. Additionen mit dieser Zahl ändern nicht den Wert der ursprünglichen Zahl. Man nennt sie "`Null"'. 

\begin{definition}
Etwas wird als \emph{Invariante}\index{Invariante} einer Verknüpfung bezeichnet, wenn sie den Wert einer Zahl mit der sie Verknüpft wird, nicht verändert.
\end{definition}

Demnach erweitern sich die Symbole der ersten zehn natürlichen Zahlen auf diese Weise:

\[0,1,2,3,4,5,6,7,8,9\]

Die Null ist per Definition nicht Teil der natürlichen Zahlen. Das hat historische Gründe -- so dachte man im Mittelalter, dass die Zahl Null vom Teufel erdacht worden sei\footnote{Historischer Beleg?}. Für unsere Betrachtung sollte sie aber Teil der natürlichen Zahlen sein. Daher werden wir von nun an mit einer Zahlenmenge umgehen, die mit $\mathbb{N}_0$ bezeichnet wird, sie besteht aus allen natürlichen Zahlen, inklusive der 0.

\textsl{Was kommt nach der 9?}

Die Zahl, die nach der 9 kommt, ist definiert durch die Summe $9+1$. Es wurde in frühen Zeiten einmal festgelegt\footnote{Historischer Beleg?}, dass wir ein Zahlensystem bestehend aus zehn Ziffern verwenden. Die zehn Ziffern beinhalten die Null, daher ist für die zehn kein eigenes Symbol mehr übrig. Daher wurden die zusammengesetzten Zahlen erfunden. So ist die Symbolfolge
\[10\]
diejenige Zahl, die nach der 9 kommt, also
\[9+1=10\]
Das ist eine Definition die bedeutet, dass wenn einem die Symbole ausgehen, kann man eine zusätzliche Stelle verwenden um diese darzustellen. Zusätzliche -- im Sinne von höherwertigen -- Stellen werden links an die Zahl angehängt. Das funktioniert auch mit drei, vier und allen weiteren Stellen:
\begin{eqnarray*}
99+1 &=& 100 \\
999+1 &=& 1000 \\
\cdots
\end{eqnarray*}

Interessant an der Zehn ist, dass sie aus einer 1 sowie der 0 zusammengesetzt ist. Also hatten die Menschen früher kein Problem mit der 0, solange sie als Teil einer anderen Zahl auftrat. Nur wenn sie alleine stand, wurde sie als verwirrend angesehen.

Aufgrund dessen, dass unser Zahlensystem auf der Zahl 10 basiert, sind gerade die Multiplikationen mit 10 besonders einfach: 
\begin{eqnarray*}
3\cdot 10 &=& 30 \\
30\cdot 10 &=& 300 \\
300\cdot 10 &=& 3000\\
\cdots
\end{eqnarray*}
Man hängt einfach nur eine 0 an die Zahl an.

\bigskip

\noindent \textsl{Wie werden aber dann Zahlen zusammengesetzt?}

\noindent Sehen wir uns ein Beispiel an:
\[9+5 = ?\]
Wir wissen, dass 
\[9+1 = 10\]
ist, und dass
\[1+4 = 5\]
So können wir schreiben:
\[9+1+4 = 10 + 4 = 14\]
An dieser Stelle haben wir eine Eigenschaft verwendet, die das \emph{Assoziativgesetz} genannt wird. Wir haben 5 durch die Summe von zwei Zahlen ersetzt und die gesamte Summe anders kombiniert. Sehen wir uns das mit Klammern an:
\[9+5 = 9+(1+4) = (9+1)+4 = 10+4 = 14\]
Klammern bevorzugen eine Operation. So ist
\[9+(1+4)=9+5\]
dass wir die 1 von $1+4$ wegnehmen und zu $9+1$ hinzufügen dürfen ist in der Anschauung vollkommen klar, ob nun Hans 9 Äpfel und Peter 5 Äpfel besitzen, oder Hans 10 und Peter 4, macht für die Gesamtanzahl keinen Unterschied. Aber dieser Operation liegt das obengenannte Assoziativgesetz\index{Assoziativgesetz} zugrunde, das wir im restlichen Buch immer wieder verwenden werden. Dass wir es verwenden dürfen, liegt an der "`Harmlosigkeit"' der natürlichen Zahlen $\mathbb{N}$. Wir werden später noch Konstrukte kennen lernen, die das Assoziativ- und im Besonderen das Kommutativgesetz nicht erfüllen. Trotzdem kann man mit diesen Konstrukten genauso rechnen, wie wir das mit den natürlichen Zahlen tun, man darf mit ihnen nur nicht alles machen, was man mit Zahlen tun kann.

Die nächste Operation, die wir kennenlernen, ist die Multiplikation:
\[3\cdot 4 = ?\]
Diese entspricht ebenfalls der Anschauung. Lägen in drei Körben jeweils vier Äpfel, so hätte man insgesamt
\[\underbrace{4+4+4}_{3 \, \textnormal{mal}} = 12\]
Äpfel. Also sind 
\[3\cdot 4 = 12\]
Das Ergebnis bliebe gleich, wenn man vier Körbe mit jeweils drei Äpfeln hätte:
\[3\cdot 4 = 4\cdot 3 = 12\]
Das ist das \emph{Kommutativgesetz}\index{Kommutativgesetz}.

Gehen wir jetzt mal davon aus, wir hätten drei Körbe mit jeweils 2 grünen und 4 roten Äpfeln. Wie viele grüne Äpfel, wie viele rote und wie viele Äpfel insgesamt hätten wir dann?

Die einzelnen Summen können wir leicht bestimmen: 
\[3\cdot 2 = 6\]
grüne Äpfel,
\[3\cdot 4 = 12 \]
rote Äpfel und somit
\[6+12 = 18\]
Äpfel insgesamt. Dabei haben wir zunächst nur die Anzahl der grünen, dann die der roten Äpfel berechnet. Wir haben also folgendes gemacht:
\[3\cdot \underbrace{(2+4)}_{\textnormal{Inhalt eines Korbes}} = 3\cdot 2 + 3\cdot 4 = 18\]
Zur Berechnung der Gesamtsumme hätten wir aber auch gleich die roten und grünen Äpfel pro Korb summieren können:
\[3\cdot (2+4) = 3\cdot (6) = 3\cdot 6 = 18 \]

Dass wir zuerst die Anzahlen aller grünen Äpfel und dann die aller roten Äpfel berechnen konnten, wird als  \emph{Distributivgesetz}\index{Distributivgesetz} bezeichnet.

\subsection{Subtraktion und Division}

Der Grund warum Subtraktion und Division ein eigenes Unter-Kapitel bilden liegt darin begründet, dass sie nicht das Kommutativgesetz beachten, wie wir gleich sehen werden.

Wenn Hans drei Äpfel besitzt und zwei davon isst, bleibt ihm nur einer übrig. Die Frage, die sich der Lernende stellen sollte ist:

\textsl{Welche Zahl erfüllt folgende Gleichung?}

\[3 + ? = 1\]

Die Zahl, die diese Gleichung erfüllt, muss kleiner als 0 sein, denn $3+0 > 1$. Was also ist kleiner als 0? Solche Zahlen werden als negative Zahlen bezeichnet. Addiert man eine natürliche Zahl mit einer negativen Zahl, so ist das Ergebnis kleiner, als die natürliche Zahl selbst. 

\begin{definition}
Eine negative Zahl ist eine natürliche Zahl, die mit einem "`$-$"'\index{Subtraktion} Zeichen als negativ gekennzeichnet wird. Sie ist identisch zu einer Zahl, die mit $-1$ multipliziert wird:
\[ -4 = (-1) \cdot 4 \]
Des Weiteren gilt
\begin{eqnarray*}
(-1)\cdot 1 &=& 1\cdot (-1) = -1 \\
(-1)\cdot (-1) &=& 1
\end{eqnarray*}

\end{definition}

Zur Vereinfachung der Darstellung gilt:

\begin{eqnarray*}
3 + -2 &=& 1 \hspace{1cm}\textnormal{falsch}\\
3 + (-2) &=& 1 \hspace{1cm}\textnormal{richtig, aber umständlich}\\
3 - 2  &=& 1 \hspace{1cm}\textnormal{ok}
\end{eqnarray*}

Die Menge aller negativen Zahlen wird mit dem Zeichen $-\mathbb{N} $ dargestellt. Die Vereinigung der Mengen $-\mathbb{N} $ und $\mathbb{N}_0$ wird als die Menge der \emph{ganzen Zahlen} bezeichnet und $\mathbb{Z}$ genannt. Also ist 

\[
\mathbb{Z} = -\mathbb{N} \cup \mathbb{N}_0 =  \{  \dots, -3, -2, -1, 0, 1, 2, 3, \dots \}
\]

Bei der Multiplikation mit negativen Zahlen, muss man das Assoziativgesetz anwenden:

\[ (-3)\cdot 6 = ((-1)\cdot 3) \cdot 6 = (-1)\cdot (3\cdot 6) = (-1)\cdot (18) = -18 \]

In dem gleichen Sinn, wie sich die Subtraktion umgekehrt zur Addition verhält, versuchen wir uns nun vorzustellen, dass es auch zur Multiplikation eine Zahl gibt, die sich umgekehrt zu dieser verhält. Kommen wir auf das vorher erwähnte Beispiel zurück, dass wir 3 Körbe mit jeweils 4 Äpfeln haben. Insgesamt haben wir also 12 Äpfel, wie wir festgestellt hatten. Ständen wir vor dem Problem, 12 Äpfel auf drei Körbe gleichmäßig zu verteilen, wüssten wir also sofort, dass wir 4 Äpfel in jeden Korb legen müssten. Wir haben also die 12 Äpfel in drei Körbe aufgeteilt und etwas zu teilen ist Gegenstand der Division. Sie wird in der folgenden Form dargestellt:

\[ \frac{12}{3} = 4 \] 
Manchmal (z.B. wenn nicht so viel Platz ist) auch
\[ 12 / 3 = 4 \]
oder
\[ 12 \div 3 = 4 \]
Diese Gleichungen bedeuten alle das selbe, auch wenn für die Division hier drei verschiedene Zeichen (---, /, $\div$) verwendet wurden. Leider hat sich keins davon als Standard durchgesetzt, es werden in verschiedenen Situationen immer wieder diese Zeichen auftauchen. Das sollte den Lernenden nicht verwirren, denn alle bedeuten dasselbe. Wir werden hier in der Regel nur den waagerechten Stich verwenden, bis auf wenige Ausnahmen. Das $\div$ Zeichen nur beim Umformen von Gleichungen um anzuzeigen, dass durch eine Zahl geteilt werden soll.

Division\index{Division} ist in diesem Kapitel nur unzureichend zu erklären, da es nur in bestimmten Sonderfällen möglich ist, eine Zahl durch eine andere Zahl zu dividieren, um dann wieder eine ganze Zahl herauszubekommen. In solchen Fällen spricht man davon, dass eine Zahl ein \emph{Teiler} einer anderen Zahl ist. In obigem Beispiel ist 3 ein Teiler von 12, da 3 die Anzahl der Körbe ist, auf die wir unsere Äpfel aufteilen. Genauso wie 4 ein Teiler von 12 ist, da dies die Anzahl der Äpfel ist, die wir in die Körbe aufteilen. Wir werden im nächsten Kapitel darauf eingehen, was passiert, wenn wir z.B. nur 11 Äpfel hätten, aber trotzdem diese auf 3 Körbe aufteilen möchten. 

Wir hatten vorher gesehen, dass es für die Multiplikation keine Rolle spielt, ob in vier Körben jeweils drei Äpfel liegen, oder in drei Körben jeweils vier Äpfel. In beiden Fällen ist die Gesamtanzahl der Äpfel zwölf. 

Bei der Division ist dem nicht so. Wenden wir das Kommutativgesetz auf eine Division an, passiert das Folgende:
\[ 12 / 3 = 3/12 \]
Diese Gleichung würde bedeuten, dass wenn man zwölf Äpfel auf drei Körbe aufteilt, in jedem Korb genauso viele Äpfel lägen, als würde man drei Äpfel auf zwölf Körbe aufteilen, was offensichtlich falsch ist. Richtig ist:
\[ 12 / 3 \ne 3/12 \]
Also gilt das Kommutativgesetz \textbf{\underline{nicht}} für die Division. Bei der Subtraktion gilt es "`fast"':

\[ 3-2 \ne 2-3 \]
aber
\[ 3-2 = 3+(-1)\cdot 2 = \underbrace{(-1)\cdot ((-1)\cdot 3}_{\textsl{weil $(-1)\cdot (-1)=1$}} + 2) = (-1)\cdot(2-3) = -(2-3)\]
also
\[ 3-2 = -(2-3) \]

Bei der Subtraktion gilt das Kommutativgesetz, bis auf das sogenannte Vorzeichen.
\begin{definition}
Als \emph{Vorzeichen} wird entweder das $+$ oder das $-$ bezeichnet, das vor eine Zahl steht. Es bestimmt, ob die Zahl positiv ($+$) oder negativ ($-$) ist.
\end{definition}

Spätestens hier sollte es dem Lernenden klar sein, dass die Subtraktion identisch ist mit der Addition bei umgekehrtem Vorzeichen.
\[ 3-2 = 3+(-1)\cdot 2 = 3+(-2) \]

\begin{fancyquotes}
Eine Randbemerkung: Im Grunde wird hier das Kommutativgesetz der Addition angewendet, denn
\[ 3-2 = 3+(-1)\cdot 2 = (-1)\cdot 2+3 = -2+3 \]
Demnach ist $-(2-3) = -2+3$, dies wird in den Aufgaben nachgewiesen.
\end{fancyquotes}

Wie wir schon bei der Multiplikation gesehen haben, ist die Division mit 10 ebenfalls besonders einfach. Im Gegensatz zur Multiplikation werden anhängende Nullen einfach weggenommen:

\begin{eqnarray*}
3000 / 10 &=& 300 \\
300 / 10 &=& 30 \\
30 / 10 &=& 3
\end{eqnarray*}

Die 3 ist in den ganzen Zahlen nicht weiter durch 10 teilbar. Das Ergebnis von $3/10$ können Sie berechnen, nachdem Sie die Rationalen Zahlen kennen gelernt haben. 

\section{Reihenfolge bei Operationen}

Sind in einer Berechnung sowohl Multiplikationen, Divisionen, Additionen und Subtraktionen durchzuführen, so gibt es eine vorrangige Reihenfolge, in der diese durchgeführt werden müssen. Es gilt der Satz

\begin{quote}
Punktrechnung geht vor Strichrechnung
\end{quote}

\noindent Als Punktrechnung werden Multiplikation und Division bezeichnet aufgrund der Zeichen $\cdot$ und $\div$, als Strichrechnung Addition und Subtraktion aufgrund der Zeichen + und --.

Folgende Berechnung soll dies veranschaulichen:
\begin{eqnarray*}
3+5\cdot 2-4\div 2 &=&  11 \\
&\ne & 6
\end{eqnarray*}
Das Ergebnis 6 bekommt man, wenn man keine Reihenfolge beachtet und einfach von links nach rechts alle Berechnungen durchführt, also 3+5=8, multipliziert mit 2 ergibt 16, minus 4 ergibt 12, geteilt durch 2 ergibt 6. Die korrekte Berechnung ist die folgende: Zuerst berechnet man die multiplikativen Teile, $5\cdot 2=10$ und $4\div 2=2$, also bleibt 3+10-2=11.

Die einzige Möglichkeit diese Reihenfolge aufzuheben ist es Klammern zu verwenden. Klammern gehen vor Punktrechnung und somit auch vor Strichrechnung. Wäre die Aufgabe also gewesen
\[
(3+5)\cdot 2-4\div 2
\]
so ist zuerst die Klammer auszurechnen 3+5=8, multipliziert mit 2 ergibt 16. Nun geht aber trotzdem die Division vor die Subtraktion, also muss zuerst $4\div 2=2$ berechnet werden und von 16 abgezogen werden, also ergibt dies 14.

\section{Primzahlen}

Betrachten wir die Zahl 12. Sie ist -- das hatten wir bereits gesehen -- durch 4 und durch 3 teilbar, denn wir wissen, dass $3\cdot 4 = 12$ ist. Und wir wissen, dass die 2 die 4 teilt. Aber wir kennen keine Zahl, die die 3 teilt. Wie sieht es mit der 5 aus? Und wie mit der 11?

Offensichtlich gibt es Zahlen, die Teiler haben und Zahlen, die nicht teilbar sind. Nun könnte man sich fragen, ob dies unter Umständen daran liegt, dass wir sehr kleine Zahlen betrachten? Sind größere Zahlen leichter durch andere Zahlen teilbar? 

Diese Frage ist nur sehr schwer zu beantworten. Und für sehr große Zahlen gibt es immer noch Geheimnisse zu entdecken. Fakt ist aber: Egal wie groß die Zahlen werden, es gibt immer noch dazwischen welche, die durch keine der vorherigen Zahlen teilbar sind. Daher die folgende Definition:

\begin{definition}
Zahlen, die nur durch 1 oder sich selbst teilbar sind, nennen wir \emph{Primzahlen}\index{Primzahl}. Tabelle \ref{tab:primes} zeigt die Primzahlen $p$ mit $1 < p < 1000$.
\end{definition}


\begin{table}
%\centering
\begin{tabular}{ccccccccccccc}
2 & 3 & 5 & 7 & 11 & 13 & 17 & 19 & 23 & 29 & 31 & 37 &  \\
41 & 43 & 47 & 53 & 59 & 61 & 67 & 71 & 73 & 79 & 83 & 89 &  \\
97 & 101 & 103 & 107 & 109 & 113 & 127 & 131 & 137 & 139 & 149 & 151 &  \\
157 & 163 & 167 & 173 & 179 & 181 & 191 & 193 & 197 & 199 & 211 & 223 &  \\
227 & 229 & 233 & 239 & 241 & 251 & 257 & 263 & 269 & 271 & 277 & 281 &  \\
283 & 293 & 307 & 311 & 313 & 317 & 331 & 337 & 347 & 349 & 353 & 359 &  \\
367 & 373 & 379 & 383 & 389 & 397 & 401 & 409 & 419 & 421 & 431 & 433 &  \\
439 & 443 & 449 & 457 & 461 & 463 & 467 & 479 & 487 & 491 & 499 & 503 &  \\
509 & 521 & 523 & 541 & 547 & 557 & 563 & 569 & 571 & 577 & 587 & 593 &  \\
599 & 601 & 607 & 613 & 617 & 619 & 631 & 641 & 643 & 647 & 653 & 659 &  \\
661 & 673 & 677 & 683 & 691 & 701 & 709 & 719 & 727 & 733 & 739 & 743 &  \\
751 & 757 & 761 & 769 & 773 & 787 & 797 & 809 & 811 & 821 & 823 & 827 &  \\
829 & 839 & 853 & 857 & 859 & 863 & 877 & 881 & 883 & 887 & 907 & 911 &  \\
919 & 929 & 937 & 941 & 947 & 953 & 967 & 971 & 977 & 983 & 991 & 997 
\end{tabular}
\caption{Die Primzahlen zwischen 1 und 1000}\label{tab:primes}
\end{table}

Besonders zu beachten ist, dass die zwei eine Sonderrolle hat: Sie ist die einzige gerade Primzahl. Das liegt daran, dass alle geraden Zahlen durch zwei teilbar sind, aber alleine die zwei nur noch einen einzigen weiteren Teiler hat, nämlich die 1, als einzige kleinere Zahl.

Die Menge der Primzahlen liegt in den natürlichen Zahlen $\mathbb{N}$. Die Frage, die wir uns stellen sollten ist: Gibt es eine größte Primzahl? Oder anders formuliert: Gibt es unendlich viele Primzahlen? Diese Frage beantworten wir im Anhang in Kapitel \ref{chap:proofprime}.

Das besondere an den Primzahlen ist, dass wir diejenigen Zahlen, die keine Primzahlen sind, durch das Produkt von Primzahlen darstellen können. Und das wiederum ist so besonders, weil dieses Produkt eindeutig ist. Das heißt, jede natürliche Zahl kann eindeutig in ein Produkt von Primzahlen zerlegt werden.

\begin{definition}
Die Zerlegung einer Zahl in das Produkt von Primzahlen heißt \emph{Primzahlzerlegung}\index{Primzahlzerlegung}.
\end{definition}

Ebenfalls im Anhang werden wir mit Lemma \ref{lem:prim} den Beweis kennenlernen, dass Primzahlzerlegungen für jede natürliche Zahl $>1$ existieren und eindeutig sind. Solche Primzahlzerlegungen sind in erster Linie für Mathematiker interessant, auf der anderen Seite hat jeder schon einmal einen Internet Browser mit dem \texttt{https}-Protokoll Benutzt. Also zum Beispiel \texttt{https://www.google.de}. Die Basis für dieses verschlüsselte Protokoll ist der sogenannte RSA-Algorithmus. Benannt nach seinen Erfindern Rivest, Shamir und Adleman\footnote{\textbf{Ronald Linn Rivest}, *1947 in Schenectady, New York. \textbf{Adi Shamir}, *6. Juli 1952 in Tel-Aviv. \textbf{Leonard Adleman}, *31. Dezember 1945 in San Francisco}. Die Sicherheit dieses Algorithmus wird dadurch garantiert, dass es selbst mit den größten und leistungsfähigsten Computern sehr, sehr lange dauert, die Primzahlzerlegung einer Zahl zu berechnen, die größer ist als eine 1 mit 300 Nullen im Falle eines 1024 Bit Schlüssels, bzw. eine 1 mit über 600 Nullen im Falle eines 2048 Bit Schlüssels.

Sehen wir uns einige Beispiele an: 

\begin{align*}
2\cdot 2\cdot 3 &= 12 & 3\cdot 7 &= 21 & 2\cdot 3\cdot 5\cdot 7 &= 210 \\
11\cdot 13\cdot 17\cdot 19 &= 46189 & 2\cdot 2\cdot 2 &= 8 & 7\cdot 7\cdot 7 &= 343
\end{align*}
Wie wir sehen, kommen in Primzahlzerlegungen die Primzahlen nicht notwendigerweise einfach vor. Primzahlen können beliebig oft in Zerlegungen vorkommen.

\section{Aufgaben}

\begin{prob}
\label{arith.1.1}

Welche der folgenden Gleichungen und Ungleichungen sind richtig oder falsch?
\begin{center}
\begin{tabular}{C{2cm}C{2cm}C{2cm}C{2cm}}
$1=2$ & $3<5$ & $9<10$ & $10>9$ \\
$8=8$ & $3<6$ & $10<10$ & $11>10$ \\
$5=5$ & $10<10$ & $10<11$ & $999 \ge 1000$ 
\end{tabular}
\end{center}
\end{prob}

\begin{prob}
\label{arith.1.2}
Ausgehend von der Gleichung $2+3=5$ führe folgende Operationen auf beiden Seiten der Gleichung durch, ohne zu vereinfachen, bis es ausdrücklich gewünscht ist (letzter Punkt):
\begin{enumerate}
\item addiere auf beiden Seiten eine 2
\item multipliziere mit 3
\item wende das Distributiv Gesetz auf beiden Seiten der Gleichung an
\item vereinfache soweit, bis auf beiden Seiten nur noch eine Zahl steht.
\end{enumerate}
\end{prob}

\begin{prob}
\label{arith.1.3}

Bestimme die rechten Seiten der folgenden Aufgaben:
\begin{center}
\begin{tabular}{C{2cm}C{2cm}C{2cm}C{2cm}}
$8+9=$ & $8-9=$ & $3\cdot 4=$ & ${8/ 2}=$ \\
$13+5=$ & $13-5=$ & $13\cdot 5=$ & ${27/ 3}=$ \\
$17+3=$ & $17-3=$ & $17 \cdot 3=$ & ${25/ 5}=$ 
\end{tabular}
\end{center}
\end{prob}

\begin{prob}
\label{arith.1.4}
Warum ist folgende Gleichung richtig?
\[ -(2-3) = -2+3 \]
\end{prob}



\chapter{Rationale und Reelle Zahlen}

In diesem Kapitel wird der Begriff der Zahl erweitert. Im letzten Kapitel haben wir die ganzen Zahlen kennengelernt. Diese sind aber nur ein Teil der Zahlen, mit denen wir täglich umgehen. Beim Einkaufen zum Beispiel besorgen wir zwar Äpfel und Milchtüten in ganzen Einheiten, aber schon beim Brot lassen wir uns oft nur die "`Hälfte"' geben. Die Preise der Dinge, die wir im Supermarkt kaufen, haben -- im allgemeinen absichtlich -- nie einen ganzzahligen Preis, da \currency 1,99 vermeintlich "`viel billiger"' klingt, denn \currency 2,00.

Wir werden uns im Besonderen mit der Teilbarkeit von Zahlen auseinander setzten, wie auch der Unteilbarkeit (Primzahlen), als auch mit den irrationalen und den komplexen Zahlen.

\section{Rationale Zahlen}

Wir hatten im vorhergehenden Kapitel immer wieder das Beispiel der drei Körbe mit den vier Äpfeln erwähnt und auch gleich die Frage gestellt, was passiert, wenn man nicht zwölf, sondern nur elf Äpfel hätte.
\[ \frac{11}{3} = ? \]
Versuchen wir uns zunächst dem Problem zu nähern, indem wir solange Äpfel in Körbe legen, bis wir zu wenig Äpfel haben, um dies gleichmäßig zu tun. Es ist klar, dass wir in jeden Korb drei Äpfel legen können und danach zwei übrig haben. Wenn wir die Äpfel nicht zerschneiden wollten, würden wir sagen, dass wir drei Äpfel in die Körbe legen konnten und zwei übrig behalten haben. 

\begin{definition}
Dies bezeichnen wir als \textsl{Division mit Rest}. \index{Division mit Rest}
\end{definition}
In unserem Beispiel würden man sagen:
\[ \frac{11}{3} = 3 \textsl{ Rest } 2 \]

Aber ist das wirklich einfacher? Sicher nicht. Das Ergebnis $11/3$ ist, vom mathematischen Standpunkt aus betrachtet, ein vollkommen korrektes Ergebnis und benötigt keine andere Darstellung. Daher tendiert man dazu, solche Ergebnisse in der Teilerform zu behalten und nicht weiter umzuformen. Division mit Rest wurde in früheren Zeiten verwendet und entspricht sehr der Anschauung. Heute würden schwierigere Rechenaufgaben grundsätzlich mit einem Taschenrechner oder Computer durchgeführt. Diese rechnen nie mit Brüchen und schon gar nicht mit einer Division mit Rest. Sie verwenden Dezimalzahlen, die später erklärt werden. So hat die Division mit Rest -- im Rechnen mit Zahlen -- ausgedient und soll hier nur der Vollständigkeit halber erwähnt sein. 

Es sei aber trotzdem noch gesagt, dass in der linearen Algebra endliche Zahlenkörper eine Rolle spielen und dort Divisionen mit Rest sehr wohl eine Bedeutung haben und intensiv genutzt werden. Diese endlichen Zahlenkörper treten u.a. in der Kryptographie auf und sind daher auch z.B. beim Online Banking von Bedeutung, wenn ein Internet Browser eine verschlüsselte Verbindung zum Bankserver auf nimmt (Stichwort https-Protokoll).

\begin{definition}
Eine Teilerform nennt man einen \textsl{Bruch}\index{Bruch}, sowie das Rechnen mit Brüchen \textsl{Bruchrechnung}.\index{Bruchrechnung}
\end{definition}

\subsection{Bruchrechnung}

\begin{definition}
Ein Bruch besteht aus drei Teilen. Oben steht eine Zahl, die als \textsl{Zähler} bezeichnet wird, unten eine Zahl, die \textsl{Nenner} genannt wird und dazwischen ist der \textsl{Bruchstrich}:
\[ \frac{\textsl{Zähler}}{\textsl{Nenner}} \]
\end{definition}

Brüche können, wie andere Zahlen auch, addiert, subtrahiert, multipliziert und dividiert werden. Dies geschieht nach folgenden Regeln:

\begin{definition}
Brüche können nur dann addiert oder subtrahiert werden, wenn ihr Nenner übereinstimmt. 
\end{definition}

Das liegt daran, dass man eine Gleichung einfach mit dem Nenner multiplizieren kann, um so die bekannten Operationen der ganzen Zahlen zu verwenden. Beispiel:

\begin{eqnarray*}
\frac{3}{8}+\frac{5}{8} &=& \frac{3}{8}+\frac{5}{8} \hskip 1cm | \cdot 8 \\
3 +5 &=& 3+5 \\
3+5 &=& 8 \hskip 1cm | \div 8\\
\frac{3}{8}+\frac{5}{8} &=& \frac{8}{8} \\
\frac{3}{8}+\frac{5}{8} &=& 1
\end{eqnarray*}
Auf der rechten Seite haben wir uns in der vorletzten Zeile zunutze gemacht, dass ein Bruch seinen Wert nicht ändert, falls Nenner und Zähler mit der selben Zahl multipliziert werden. So ist 

\[ 1 = \frac{a}{a} \textsl{ für alle } a\in \mathbb{Z} \]
wie auch 
\begin{equation}
\label{relation}
\frac{x}{y} = \frac{a\cdot x}{a\cdot y} \textsl{ für alle } a\in \mathbb{Z}
\end{equation}

\begin{definition}
Wir bezeichnen Zahlen der Form 

\[ \frac{a}{b} \textsl{ für alle } a,b \in \mathbb{Z}, b\ne 0\]
als \textsl{Rationale Zahlen} und stellen sie mit dem Symbol $\mathbb{Q}$ dar. 
\end{definition}
Also ist 

\[ \mathbb{Q} = \left\{ \frac{a}{b} \middle\vert a,b \in \mathbb{Z}, b\ne 0 \right\} \]
Die rationalen Zahlen beinhalten die ganzen Zahlen auf eine natürliche Weise, da für $n,a\in \mathbb{Z}$ gilt
\[ n = \frac{n}{1} = \frac{n \cdot a}{1\cdot a}\]
In diesem Fall ist $a$ ein Teiler der Zahl $n\cdot a$. 

\begin{quote}
\textsl{Hinweis für Lehrer:}

Es gibt Mathematiker, die bei der Definition der rationalen Zahlen einen axiomatischen Ansatz wählen, der die rationalen Zahlen auf Basis von Äquivalenzrelationen definieren. So wäre dann eine rationale Zahl immer durch genau eine Äquivalenzklasse dargestellt. Da Äquivalenzklassen im allgemeinen durch ihr Erzeugendes Element dargestellt werden, würden nach diesem Ansatz nur diejenigen Brüche zu den rationalen Zahlen gehören, die die Erzeugenden Elemente ihrer Äquivalenzklasse wären. Der Autor hält dies für sehr verwirrend und kaum vermittelbar, warum der Bruch 
\[ \frac{3}{11}\]
zu den rationalen Zahlen zählen soll, der Bruch
\[ \frac{9}{33}\]
aber nicht, sondern nur zur Äquivalenzklasse von $\frac{3}{11}$.
\end{quote}

\subsection{Weitere Begriffe}

\begin{definition}
Es seien Zähler $Z$ und Nenner $N$ eines Bruchs durch die selbe Zahl $a$ teilbar. Das bedeutet, $Z=a\cdot z$ sowie $N=a\cdot n$ mit $a,z,n\in \mathbb{Z}$. Also gilt:
\[
\frac{Z}{N} = \frac{a\cdot z}{a\cdot n} = \frac{a}{a}\cdot \frac{z}{n} = 1\cdot \frac{z}{n} = \frac{z}{n}
\]
In diesem Fall sagt man, dass man den Bruch um $a$ \textsl{kürzen}\index{kürzen} kann.
\end{definition}

Des Weiteren ist mit $Z$ und $N$ wie oben
\[
\frac{Z}{N} = \frac{\frac{Z}{a}}{\frac{N}{a}} = \frac{Z}{a\cdot \frac{N}{a}} = \frac{Z \cdot a}{a\cdot N}
\]
Das bedeutet, ist im Zähler ein Bruch, so kann dessen Nenner zum Nenner multipliziert werden. Steht im Nenner ein Bruch, so wird dessen Nenner zum Zähler multipliziert. 

\subsection{Dezimalzahlen}

\begin{definition}
Eine \textsl{Dezimalzahl} ist eine Zahl der folgenden Form:\index{Dezimalzahl}
\[ z_m z_{m-1} \dots z_1 z_0, z_{-1} z_{-2} \dots z_{-n} \]
\end{definition}
Die $z_i$ sind Ziffern und somit Zahlen aus der Menge $\{0,1,2,3,4,5,6,7,8,9\}$.

Zur Erläuterung betrachten wir zunächst die Zahlen zwischen 0 und 1. Dies sind die Zahlen 
\[ \frac{1}{n}\] für alle $n\in \mathbb{N}$. 

\begin{eqnarray*}
1/2 &=& 0,5 \\
1/4 &=& 0,25 \\
1/5 &=& 0,2 \\
1/8 &=& 0,125 \\
\dots
\end{eqnarray*}

Die Brüche $1/3$, $1/6$ und $1/7$ haben eine besondere Eigenschaft, sie haben nämlich keine endliche Dezimaldarstellung:

\begin{eqnarray*}
1/3 &=& 0,3333333333\dots \\
1/6 &=& 0,1666666666\dots \\
1/7 &=& 0,142857142857142857142857142857\dots
\end{eqnarray*}
Die sich wiederholenden Teile der Nachkommastellen werden mit einem Strich darüber gekennzeichnet:

\begin{eqnarray*}
1/3 &=& 0,\overline{3} \\
1/6 &=& 0,1\overline{6} \\
1/7 &=& 0,\overline{142857}
\end{eqnarray*}
Natürlich gibt es auch Dezimalzahlen, die größer sind als 1:

\begin{eqnarray*}
97/3 &=& 32,\overline{3} \\
21/5 &=& 4,2 \\
177/7 &=& 25,\overline{285714}
\end{eqnarray*}
Dezimalzahlen treten immer wieder als Ergebnisse von Berechnungen von Taschenrechnern und Computern auf, daher sind sie sehr wichtig. 

Größere Dezimalzahlen werden der Übersichtlichkeit halber mit Trennzeichen gegliedert. So werden 
\[100000\]
auch als
\[100.000\]
geschrieben. Sind Kommastellen zu berücksichtigen, ist zwischen Punkt und Komma strikt zu unterscheiden:
\[100.000.00 \ne 100.000,00\]
Als Trennzeichen zu den Nach-"'Komma"'-Stellen, ist immer das Komma zu verwenden. Vermutlich aufgrund zunehmender Amerikanisierung und dem Gebrauch von Taschenrechnern hat sich auch der Punkt als Trennzeichen für die Nachkommastellen eingebürgert. Aber dies ist falsch im deutschen Zahlengebrauch. Die hier vorgestellte Darstellung gilt in dieser Form nur in Deutschland. 

\bigskip

\begin{tabular}{L{4cm}R{4cm}}
Deutschland & 1.000.000,00 \\
Amerika/England & 1,000,000.00 \\
Schweiz & 1'000'000,00 \\
ISO 31-0 & 1\,000\,000,00
\end{tabular}

\bigskip

Die Internationale Organisation für Normung (ISO) hat im ISO 31-0 Standard festgelegt, dass als Tausender-Trennzeichen ein schmales Leerzeichen zu verwenden sei. Man sieht dies vereinzelt in Veröffentlichungen aber ansonsten hat sich diese Schreibweise nur wenig durchgesetzt. 

\section{Potenzrechnung}

Einleitend sei hier bemerkt, dass sich das Potenzieren mit natürlichen Zahlen zur Multiplikation verhält, wie die Multiplikation zur Addition:

\[ 4\cdot 3 = \underbrace{4+4+4}_{\text{3 mal}} = 12 \]

\[ 4^3 = \underbrace{4\cdot 4\cdot 4}_{\text{3 mal}} = 64 \]

\begin{definition}
Die Zahl, die potenziert werden soll, nennt man \textsl{Grundzahl}\index{Grundzahl}, und die Zahl mit der potenziert werden soll \textsl{Hochzahl}\index{Hochzahl}, oder \textsl{Exponent}\index{Exponent}. Den Wert der Potenz einer Grundzahl ($a$) mit einer natürlichen Zahl ($n \in \mathbb{N}$) als Exponent erhält man durch:
\[ a^n = 1\cdot \underbrace{a\cdot a \cdot \dots \cdot a}_{\text{n mal}} \]
Den Wert einer Potenz mit einer negativen natürlichen Zahl durch:
\[ a^{-n} = 1 \div \underbrace{a\div a \div \dots \div a}_{\text{n mal}} = \frac{1}{a^n} \]
Des Weiteren gilt immer
\[ a^0 = 1,\ a^1=a \]
\end{definition}
Das Kommutativgesetz gilt nicht:
\[ 64 = 4^3 \ne 3^4 = 81 \]


\section{Reelle und irrationale Zahlen}\label{chap:realbegin}

Die reellen Zahlen sind -- zusammen mit den komplexen Zahlen -- mit die gebräuchlichsten Zahlenkörper in der Mathematik. Die über diesen Körpern gebildeten, dreidimensionalen Vektorräume entsprechen in direkter Weise unserer räumlichen Vorstellung (Euklidischer Raum). 

Auch wenn es seit Euklid (etwa 300 v. Chr.) ein Verständnis für die reellen Zahlen im Sinne eines "`Kontinuums"' gibt, wurde eine erste, formale Konstruktion der reellen Zahlen zum ersten Mal durch Karl Weierstraß\footnote{\textbf{Karl Theodor Wilhelm Weierstraß}, *31. Oktober 1815 in Ostenfelde; \ding{61}19. Februar 1897 in Berlin} dargelegt. Die verschiedenen Konstruktionsarten der reellen Zahlen darzulegen, würde weit über das Ziel hinaus gehen, ein Buch für Schüler zu schreiben. Daher können wir nur versuchen, argumentativ ein Gefühl für die reellen Zahlen zu entwickeln. 
\begin{quote}
Zu je zwei reellen Zahlen, egal wie nahe sie beieinander liegen, gibt es immer eine dritte dazwischen.
\end{quote}
Dieser Satz ist leicht daher gesagt, stellt aber eine sehr zentrale Erkenntnis im Zusammenhang mit den reellen Zahlen dar, denn dies gilt für keine andere Zahlenmenge, die wir bisher kennen gelernt haben.

Die rationalen Zahlen waren schon "`ziemlich"' vollständig in dem Sinne, dass wir immer zu je zwei Zahlen aus $\mathbb{Q}$ folgendes tun können: Seien $a=\frac{a_z}{a_n}$ und $b=\frac{b_z}{b_n}$ rationale Zahlen. Dann kann immer die Zahl $c$ gefunden werden mit
\[
c = \frac{a_z\cdot b_n + a_n\cdot b_z}{2\cdot a_n\cdot b_n}
\]
$c$ liegt damit immer zwischen den Zahlen $a$ und $b$. Da $a_z,a_n,b_z$ und $b_n$ natürliche Zahlen sind, ist der Zähler und Nenner von $c$ jeweils wieder eine natürliche Zahl und deshalb ist $c\in \mathbb{Q}$. 

Es gibt allerdings Zahlen, die nicht durch einen Bruch dargestellt werden können, egal wie groß die Zahlen in Zähler und Nenner sind. Trotzdem bleibt immer ein Unterschied. Beispiele sind das Verhältnis eines Kreises zu seinem Durchmesser. Diese Zahl wird mit $\pi$ abgekürzt. Genauso wie der Durchmesser eines Quadrates mit Kantenlänge 1, dies ist die $\sqrt{2}$. Keine solche Zahl kann mit Brüchen "`genau genug"' angenähert werden und diese sind nicht in den rationalen Zahlen enthalten. Solche Zahlen werden irrational genannt

Die Vereinigung aller irrationalen Zahlen mit den rationalen Zahlen gelten als die reellen Zahlen und werden mit dem Symbol $\mathbb{R}$ bezeichnet.

Im Anhang \ref{chap:realfinal} werden die reellen Zahlen genauer eingeführt. 

\section{Intervalle}

\begin{definition}
Teilmengen der reellen Zahlen, die einen zusammenhängenden Bereich abdecken, werden auch als \textsl{Intervalle} bezeichnet. Sie haben eine obere und eine untere Grenze. Alles, was dazwischen liegt, gehört zu dieser Teilmenge. Sei $a<b$, dann wird mit
\[
\lbrack a,b \rbrack
\]
die Menge aller Zahlen $x$ bezeichnet, für die gilt $a\le x\le b$
\end{definition}

\begin{definition}
Gehört die obere und untere Zahle nicht zum Intervall, wenn also gilt:
\[
a<x<b
\]
dann heißt das Intervall \textsl{offen}. In diesem Fall werden runde Klammern verwendet
\[
x\in (a,b)
\]
\end{definition}

\begin{definition}
Natürlicherweise gilt dies auch für nur einseitig offene Intervalle, 
\[
\lbrack a,b) \text{ oder } (a,b\rbrack
\]
Solche Intervalle werden auch als \textsl{halboffen} bezeichnet und es gilt
\[
a\le x <b \text{ bzw. } a<x \le b
\]
\end{definition}

\section{Betrag und Vorzeichen}

Jede Zahl in $r\in \mathbb{R}_+$ hat eine Entsprechung im Negativen $-r$. Das führt uns zu der Überlegung, dass der Wert einer Zahl unabhängig von ihrem Vorzeichen von Interesse sein könnte.

\begin{definition}
Der \textsl{Betrag}, oder \textsl{Absolutbetrag} ist eine Funktion $\vert . \vert : \mathbb{R} \longrightarrow \mathbb{R}_+$
\begin{equation}
\vert r \vert = \begin{cases}
r & \text{falls } r\ge 0 \\
-r & \text{falls } r<0
\end{cases}
\end{equation}
Da $\mathbb{Z} \subset \mathbb{Q} \subset \mathbb{R}$ sind, gilt diese Definition auch für die ganzen und rationalen Zahlen. Für die natürlichen Zahlen ist sie sinnlos, da es keine negativen natürlichen Zahlen gibt. Aber da $\mathbb{N} \subset \mathbb{Z}$, ist trivialer Weise $\vert n \vert = n$ für alle $n\in \mathbb{N}$.
\end{definition}

Das meist weggelassene $+$ vor den Zahlen, sowie das $-$ wird als \textsl{Vorzeichen} bezeichnet. Die folgende Definition ist daher naheliegend, wenn auch zu diesem Stadium noch nicht weiter von Interesse:

\begin{definition}
Die Funktion 
\[
\text{sign} : \mathbb{R} \longrightarrow \lbrace -1,1\rbrace
\]
bestimmt das Vorzeichen für jede reelle Zahl. Sei $x \in \mathbb{R}$
\begin{equation}
\text{sign}(x) = \begin{cases}
1 & \text{falls } x\ge 0 \\
-1 & \text{falls } x < 0
\end{cases}
\end{equation}
\end{definition}
Seien $a,b\in \mathbb{R}$, dann gilt:
\begin{equation}
\begin{split}
|a\cdot b| &= |a|\cdot |b| \\
|a\div b| &= |a|\div |b| \\
|a+b| &\le |a|+|b| \\
|a-b| &\le |a|+|b| \\
|a|-|b| &\le |a-b| \\
\end{split}
\end{equation}


\section{Aufgaben}
TODO


\chapter{Komplexe Zahlen}

Um die komplexen Zahlen zu erklären, wollen wir uns zunächst eine Gleichung ansehen. 

\[ ?^2 +1 = 0 \]
Welche Zahl kann diese Gleichung erfüllen? Wir formen um:
\begin{eqnarray*}
?^2 +1 &=& 0 \hskip 1cm | -1 \\
?^2 +1 -1 &=& -1 \hskip 1cm | \sqrt{.} \\
\sqrt{?^2} &=& \sqrt{-1} \\
? &=& \sqrt{-1}
\end{eqnarray*}
Stimmt das? Wir setzen ein:
\begin{eqnarray*}
\sqrt{-1}^2 +1 &=& 0 \\
(-1)^{\frac{1}{2}\cdot 2} +1 &=& 0 \\
(-1)^{\frac{2}{2}} +1 &=& 0 \\
(-1)^{1} +1 &=& 0 \\
-1+1 &=& 0 \\
0 &=& 0
\end{eqnarray*}
Die Gleichung stimmt offensichtlich, doch hatten wir gelernt, dass Wurzeln aus negativen Zahlen nicht definiert sind. Kann man also einfach mit dem Quadrieren einer negativen Zahl rechnen und erwarten, dass etwas sinnvolles dabei heraus kommt? Die einfache Antwort ist: "`Ja"', man kann. Denn Fakt ist: Als wir die Gleichung mit einem $?$ umgeformt hatten, wussten wir noch nicht, dass das Fragezeichen für die Wurzel aus $-1$ steht. Wendet man also die Rechenschritte korrekt auf eine Zahl (oder Variable, wie wir im späteren sehen werden) an, so bleibt das Resultat korrekt. 

Die Definition der komplexen Zahlen fußt auf dieser Erkenntnis. Sie verwendet die negativen Wurzeln indem ein Buchstabe $i$ \index{$i = \sqrt{-1}$} den Wert $\sqrt{-1}$ annimmt und so zum Rechenelement wird.
\begin{eqnarray*}
\sqrt{-17} &=& \sqrt{-1\cdot 17}\\
&=& \sqrt{-1}\cdot \sqrt{17}\\
&=& i \cdot \sqrt{17}
\end{eqnarray*}
oder allgemein:

\begin{eqnarray*}
\sqrt{-x^2} &=& \sqrt{-1\cdot x^2}\\
&=& \sqrt{-1}\cdot \sqrt{x^2}\\
&=& i \cdot x
\end{eqnarray*}
wobei $x$ jede reelle Zahl sein kann, also $x\in \mathbb{R}$. Allgemein werden die komplexen Zahlen definiert als 

\[ \mathbb{C} = \{ a+i\cdot b \ |\ a,b \in \mathbb{R} \} = \mathbb{R}+i\mathbb{R} \]
Das bedeutet, dass komplexe Zahlen aus zwei reellen Zahlen bestehen. Sie können nicht miteinander addiert werden, da das $i$ dies verhindert. Weil die komplexen Zahlen dadurch aus zwei reellen Komponenten gebildet werden, nennt man sie auch oft zweidimensionale Zahlen. Als Ausblick sei erwähnt, dass in der linearen Algebra die komplexen Zahlen mit einem zwei dimensionalen reellen Vektorraum identifiziert werden können.

\section{Die Grundrechenarten der komplexen Zahlen}

Der Einfachheit halber wird im Folgenden $i\cdot b$ durch $ib$ ersetzt. Der Multiplikationspunkt kann weggelassen werden. Und alle $a,b,c,d \in \mathbb{R}$ sowie alle $p,q \in \mathbb{C}$ mit $p = a+ib$ und $q = c+id$. Beachte, dass $i^2=\sqrt{-1}^2 = -1$.

\begin{definition} Addition
\[ p+q = (a+c)+i(b+d)\]
\end{definition}

\begin{definition} Subtraktion
\[ p-q = (a-c)+i(b-d)\]
\end{definition}

\begin{definition} Multiplikation
\[ p\cdot q = a(c+id)+ib(c+id) = (ac-bd)+i(ad+bc)  \]
\end{definition}

\begin{definition} Division
\[ \frac{p}{q} = \frac{a+ib}{c+id} = \frac{(a+ib)(c-id)}{(c+id)(c-id)} = \frac{ac+bd}{c^2+d^2} + i\frac{bc-ad}{c^2+d^2}\]
\end{definition}

\section{Betrag}

Wie die reellen Zahlen, haben auch die komplexen Zahlen einen Betrags-Begriff.
\[
\vert z \vert = \sqrt{a^2 +b^2}
\]
für $z = a+ib \in \mathbb{C}$. Wir werden später sehen, dass diese Definition einer sogenannten euklidischen Norm entspricht. Doch dazu später mehr.


\section{Historische Bemerkung} 

Gerolamo Cardano\footnote{\textbf{Gerolamo Cardano}, italienischer Arzt, Philosoph und Mathematiker, *24. September 1501 in Pavia, \ding{61}21. September 1576 in Rom.}\index{Cardano, Gerolamo} behandelte in seinem 1545 erschienen Buch \textit{Artis magnae sive de regulis algebraicis liber unus} die Aufgabe zwei Zahlen zu finden, deren Produkt 40 und deren Summe 10 sei. Er setzte dafür die Gleichung an:

\[ x^2-10x+40=0 \]

Er erkannte, dass diese Gleichung keine Lösung hat, fügte aber die Bemerkung hinzu, dass falls die entsprechenden Umformungen sinnvoll und erlaubt wären, dass dann $5+\sqrt{-15}$ sowie $5-\sqrt{-15} $ in der Tat Lösungen der Gleichung wären. 

In diesem Sinne hatte Cardano bereits den ersten Schritt in die richtige Richtung getan. Doch dauerte es noch eine ganze Weile, bis komplexe Zahlen sich durchsetzten.


\section{Unendlich $\infty$}\label{sec:infty}

\index{Unendlich $\infty$}
Der Begriff \textsl{Unendlich} bezeichnet keine Zahl, sondern im Grunde eher einen Zustand. Unendlich ist etwas -- wie der Begriff nahe legt --, wenn es kein Ende besitzt. So haben zum Beispiel die natürlichen Zahlen kein oberes Ende:

\[
\mathbb{N} = \left\lbrace 1,2,3, \dots \right\rbrace
\]
So unscheinbar die "`$\dots$"' auch sind, so bezeichnen sie den allergrößten Teil der Menge $\mathbb{N}$. Denn auch die Zahl 
\[9.287.375.864.825.551.256.365.751.255\]
ist eine natürliche Zahl, genauso wie 
\[2^{9.287.375.864.825.551.256.365.751.255}\]
oder
\begin{equation} \label{eq:huge}
10^{9.287.375.864.825.551.256.365.751.255}
\end{equation}
oder sogar
\begin{equation}\label{eq:reallyhuge}
9.287.375.864.825.551.256.365.751.255^{9.287.375.864.825.551.256.365.751.255}
\end{equation}

Es gibt Zahlen in der Mathematik, die so groß sind, dass selbst wenn man alle Atome im Universum in Papier und Tinte umwandelte, sie nicht ausreichen würden, um die Zahl aufzuschreiben. Die letzte oben angegebene Zahl ist ein gutes Beispiel dafür. Astronomen schätzen, dass es etwa $10^{77}$ Atome im Weltall gibt. Und das ist wesentlich weniger, als die in (\ref{eq:huge}) darstellte Zahl, geschweige denn (\ref{eq:reallyhuge}).

Der Punkt ist: Es gibt keine größte natürliche Zahl. Immer wenn man glaubt, eine gefunden zu haben, gibt es eine noch größere. Dieser Zustand, nämlich dass es keine größte Zahl gibt, sondern immer weitere, wird mit dem Zeichen $\infty$ abgekürzt. 

Dementsprechend ist es korrekter, wenn man die natürlichen Zahlen in dieser Form schreibt:
\[
\mathbb{N} = \left\lbrace 1,2,3, \dots, \infty \right\rbrace
\]

Aufgrund dessen, dass $\infty$ keine Zahl ist, gelten auch sämtliche Rechenregeln, die wir kennen gelernt haben, nicht für $\infty$. So ist zum Beispiel $\infty +1$ unsinnig, genauso wie $\infty-1$. Das selbe gilt für die Multiplikation und Division. Es gibt allerdings einige Regeln, die es zu wissen gilt. Denn es passiert, dass man beim Rechnen oder umformen auf $\infty$ stößt und dann damit umgehen können muss. 

Für $a\in \mathbb{R}$ (beachte, dass $\mathbb{N} \subset \mathbb{Z} \subset \mathbb{R}$) mit $-\infty < a < \infty$ gilt

\begin{eqnarray*}
\infty + a &=& \infty \\
\infty \cdot a &=& \infty \\
\frac{a}{\infty} &=& 0
\end{eqnarray*}

Für $a=\infty$ oder $a=-\infty$ sind alle oben angegebenen Regeln hinfällig. Alle Operationen mit diesen Werten sind unzulässig und nicht definiert!

Eine der erstaunlichsten und gleichzeitig verwirrendsten Eigenschaften von unendlich großen Mengen ist, dass sie echte Teilmengen besitzen, die ebenfalls wieder unendlich groß sind. Also "`gleich viele"'\footnote{Hier "`gleich viele"' zu sagen ist natürlich falsch. Die Mathematiker verwenden in diesem Zusammenhang den Begriff der Mächtigkeit. Die natürlichen Zahlen und die Primzahlen sind gleichmächtig. Aber selbstverständlich nicht gleich viele, da fast alle geraden Zahlen nicht zu den natürlichen Zahlen gehören. Da wir den Begriff der Mächtigkeit aber nicht weiter verwenden, sei dieser hier nur als Fußnote erwähnt.} Elemente haben, nämlich unendlich viele. Betrachten wir die Primzahlen in $\mathbb{N}$:

\begin{equation*}
\lbrace 2,3,5,7,11, \dots \rbrace
\end{equation*}
Dass es unendlich viele Primzahlen gibt, muss bewiesen werden, nur leider haben wir noch nicht genug Kenntnisse für den Beweis. Daher wird er nachgeliefert in Abschnitt \ref{chap:proofprime}. 

Somit hat die Aufzählung der Primzahlen kein Ende. Sie ist also in unserem vorher beschriebenen Sinne "`unendlich"'. Wenn also die Aufzählung unendlich ist, dann kann jeder natürlichen Zahl eine Primzahl zugewiesen werden. Daraus folgt, dass es genauso viele natürliche Zahlen gibt, wie es Primzahlen gibt. 

Unserer Anschauung nach, ist dies natürlich falsch, denn wir denken immer in endlichen Mengen. Und für endliche Mengen, wie z.B. die Zahlen zwischen 1 und 100 oder zwischen 1000 und 2000, stimmt es sicher, dass sich in diesen weniger Primzahlen befinden, als natürliche Zahlen. Aber im Unendlichen wird es richtig, weil es keinen Mengenbegriff im Unendlichen mehr gibt. Unendlich ist Unendlich.

\begin{definition}
Eine unendliche Menge, deren Elemente durchgezählt werden können, in dem jedem Element eine natürliche Zahl zugeordnet wird, nennt man \textsl{abzählbar unendlich}.\label{abzaehlbar} \index{abzählbar} \index{abzählbar unendlich}
\end{definition}

\section{Polarkoordinaten der Komplexen Zahlen}

Wenn man die Komplexen Zahlen als eine Ebene auffasst, so kann eine Zahl in der Ebene nicht nur durch ihre Koordinaten in reeller und imaginärer Richtung dargestellt werden, sondern auch durch eine Richtung zusammen mit einer Entfernung vom Ursprung ($=0+i0$). Hierfür kann man sich an den Nordpol versetzt vorstellen und die Linien der Längengerade stellen die Richtung dar, in der ein Punkt auf der Erde fixiert werden soll. Eine zusätzliche Angabe in Kilometern ergibt einen eindeutig fixierten Punkt auf der Erdoberfläche. Das heißt: Es reicht einen Winkel und eine Entfernung anzugeben um einen Punkt eindeutig zu benennen. 

\begin{definition}
Eine solche Darstellung mit Winkel (=Richtung) und Entfernungsangabe, wird \textsl{Polarkoordinate} genannt und durch 
\[(a,\phi)\]
das heißt eine Entfernungsangabe $a$ mit einer Winkelangabe $\phi$ dargestellt.
\end{definition}

Die Rechenregeln für die Multiplikation und Division in Polarkoordinaten sind einfacher als die in der Standarddarstellung, während die Addition und Subtraktion komplizierter sind. Im Folgenden seien $a,b \in \mathbb{R}$ und $\phi,\psi \in \left[-\pi, \pi\right]$.

\begin{definition} \textsl{Multiplikation in Polarkoordinaten}
\[
(a,\phi) \cdot (b,\psi) = (a\cdot b, \phi+\psi)
\]
\end{definition}

\begin{definition} \textsl{Division in Polarkoordinaten}
\[
(a,\phi) \div (b,\psi) = \left(\frac{a}{b}, \phi-\psi\right)
\]
\end{definition}

Die Polarkoordinaten spielen besonders in der Physik und Technik eine große Rolle. Die Umrechnung von $z=a+ib$ nach $z=(c,\phi)$ ist aber bereits mit den bisher kennengelernten Mitteln nicht zu leisten. Daher sind sie hier nur kurz erwähnt und werden im Analysis Teil des Buches erneut behandelt und dann mit der notwendigen Tiefe. 


\section{Aufgaben}
TODO


\chapter{Angewandte Mathematik}

In diesem Kapitel beschäftigen wir uns ausschließlich mit den Anwendungen der Mathematik im Alltag. 


\section{Prozentrechnung}

\subsection{Beispiel Mehrwertsteuer}

Ein Problem mit dem wir es täglich zu tun bekommen, ist die Mehrwertsteuer. Sie ist auf alle Handelswaren zu erheben und an das Finanzamt abzuführen. Sie beträgt allgemein 19\%, auf Lebensmittel 7\%. 

Das Zeichen "`\%"' wird "`Prozentzeichen"' genannt. Der Begriff kommt aus dem lateinisch-italienischen "`per cento"', "`vom Hundert"', und erklärt bereits, worum es dabei geht. Man teilt eine beliebige Menge in 100 Teile und nimmt sich (das Finanzamt) 19 Teile davon. 

Der aktuelle VW Golf\footnote{Preis von Dezember 2013} kostet 14.264,70 \officialeuro\  ohne Mehrwertsteuer. Dieser Preis in 100 Teile geteilt ergibt 142,647 \officialeuro, mit 19 multipliziert ergibt 2710,293 \officialeuro. Daher müssen Kunden, die den VW Golf kaufen möchten, 16.975,- \officialeuro\  bezahlen. Der Händler führt dann 2710,29 \officialeuro\  an das Finanzamt als Mehrwertsteuer ab.

\subsection{Allgemeine Prozentrechnung}

Sei im Folgenden $p$ der Prozentsatz, also z.B. $p=19\%$, und $w$ der Prozentwert, im obigen Beispiel die 2710,29 \officialeuro. Des Weiteren sei $K$ die Menge, von der uns die Prozent interessieren. Dann gilt:
\[
w = K\cdot \frac{p}{100}
\]
Interessiert nur der Gesamtwert $K' = K+w$, dann kann folgendes berechnet werden:
\[
K' = K+w = K+K\cdot \frac{p}{100} = K\cdot \left(1+\frac{p}{100}\right)
\]
Wenn $p=19\%$ weiterhin die Mehrwertsteuer ist, dann würde der letzte Teil der Gleichung ausgerechnet folgendes ergeben:
\[
K' = K\cdot 1,19
\]
Die 19\% tauchen also hinter dem Komma der Zahl auf. Dies ist die einfachste Möglichkeit, zu einer beliebigen Zahl einen beliebigen Prozentwert hinzu zurechnen.

Betrachten wir den Fall eines Rabatt-Angebots. Hier werden vom Preis Prozente abgezogen. Das heißt, der Prozentwert $w$ ändert sich nicht. Nur wird er
\[
K' = K-w
\]
von der Menge abgezogen.
\[
K' = K-w = K-K\cdot \frac{p}{100} = K\cdot \left(1-\frac{p}{100}\right) 
\]
Gäbe ein Geschäft einen Rabatt von 30\%, so wäre 
\[
K' = K\cdot \left(1-\frac{30}{100}\right) = K\cdot (1-0,3) = K\cdot 0,7
\]

Um es noch etwas komplizierter zu machen gibt das Geschäft 30\% Rabatt auf den sogenannten Netto-Preis, d.h. der Preis ohne Mehrwertsteuer. Es sei nun $K$ der Preis der Ware ohne Mehrwertsteuer. Das bedeutet, wir ziehen zunächst 30\% ab und rechnen dann 19\% Mehrwertsteuer dazu, damit wir wissen, was wir endlich bezahlen müssen:
\begin{eqnarray*}
K' &=& (K-30\% ) + 19\% = \left( K\cdot \left( 1-\frac{30}{100} \right) \right)\cdot \left( 1+\frac{19}{100} \right)\\
&=& K\cdot \left( 0,7 \cdot 1,19 \right)\\
&=& K\cdot 0,833
\end{eqnarray*}
Das bedeutet, dass der Unterschied zwischen dem ursprünglichen Netto-Preis und dem, was wir bezahlen müssen, 16,7\% beträgt.

\subsection{Prozentrechnung bei Krediten}

Wenn wir einen Kredit nehmen, verschulden wir uns bei unserer Bank. Das bedeutet, die Bank gibt uns Geld, das wir in Raten wieder zurückzahlen müssen. Natürlich macht die Bank dies nicht kostenlos. Sie verlangt dafür Kreditzinsen in Form von Prozenten. Aktuell übliche Kreditzinsen liegen bei 8\% - 11\%. Aber worauf werden die Zinsen verlangt?

Sehen wir uns das genauer an: Stellen wir uns vor, wir möchten einen Studien-Urlaub zu den verwunschenen Orten der Maya nach Süd-Amerika machen. Ein Reiseveranstalter verlangt für eine solche Reise \currency 5.000,-. Wir haben nicht genug Geld auf unserem Konto, möchten diese Reise aber unbedingt machen. Also fragen wir die Bank nach einem Kredit. Nach einer Prüfung willigt diese ein und leiht uns das Geld zu einem Zinssatz von 8\%. Das bedeutet, dass die Bank jeden Monat ein zwölftel dieses Zinssatzes auf das Geld aufschlägt, das wir zum Anfang des Monats der Bank noch schulden. Das bedeutet, die Bank erhöht den Betrag, den wir ihr schulden jeden Monat um $\frac{8}{12}$\%.

Gleichzeitig tilgen\footnote{Tilgung = Rückzahlung} wir den Gesamtbetrag mit unseren Monatsraten um einen bestimmten Teilbetrag. Damit wir den Kredit überhaupt irgendwann zurückzahlen können, muss demnach der Teilbetrag, den wir jeden Monat zurückzahlen, höher sein, als der Betrag, der durch die Verzinsung dazu kommt.

Die Berechnung dazu sieht folgendermaßen aus, es sei $p=\frac{8}{12}\% $, $q = 1+\frac{p}{100}$ und $t$ unsere Tilgung. Dann entspricht der Restbetrag unseres Kredits $K_i$ in Monat $i$ nach beginn der Rückzahlung der folgenden Berechnung:
\[
K_i = \dots (((K-t)\cdot q-t)\cdot q-t)\cdot q \dots
\]
Die Punkte bedeuten, dass wir die Operation $(. -t)\cdot q$ genau $i$-Mal durchführen müssen.

Die Bank gewährt uns diesen Kredit bei einer monatlichen Rate von \currency 220,-. Die Bank hat die Rate so gewählt, dass wir in 24 Monaten den Kredit abbezahlt haben. Der ersten zwei Monate berechnen sich wie folgt:

\begin{eqnarray*}
K_1 &=& (5000-t)\cdot q = (5000 -220)\cdot 1,00\bar{6} = 4.811,87 \\
K_2 &=& (4.811,87 -t) \cdot q = (4.811,87 - 220)\cdot 1,00\bar{6} = 4.622,48
\end{eqnarray*}
und so weiter. 

Der sogenannte Tilgungsplan, also die Aufstellung aller Restbeträge für jeden der 24 Monate, ist in der folgenden Tabelle dargestellt:

\begin{center}
\begin{tabular}{C{4cm}C{4cm}}
\hline
\textbf{Monat} & \textbf{Restbetrag} \\
\hline
1& 4.811,87 \currency  \\
2&	 4.622,48 \currency  \\
3&	 4.431,83 \currency  \\
4&	 4.239,91 \currency  \\
5&	 4.046,71 \currency  \\
6&	 3.852,22 \currency  \\
7&	 3.656,43 \currency  \\
8&	 3.459,34 \currency  \\
9&	 3.260,94 \currency  \\
10&	 3.061,21 \currency  \\
11&	 2.860,15 \currency  \\
12&	 2.657,75 \currency  \\
13&	 2.454,01 \currency  \\
14&	 2.248,90 \currency  \\
15&	 2.042,42 \currency  \\
16&	 1.834,57 \currency  \\
17&	 1.625,34 \currency  \\
18&	 1.414,71 \currency  \\
19&	 1.202,67 \currency  \\
20&	 989,22 \currency  \\
21&	 774,35 \currency  \\
22&	 558,05 \currency  \\
23&	 340,30 \currency  \\
24&	 121,10 \currency  \\
\hline
\end{tabular}
\end{center}

\bigskip

\noindent Die Restrate beträgt nicht mehr die vollen \currency 220,-, sondern nur noch \currency 121,10. 

Weil der monatliche Betrag, den wir abbezahlen immer konstant ist -- bis auf den letzten Monat --, die Zinsen aber abnehmen, da der verzinste Betrag abnimmt, erhöht sich der Betrag, den wir zurückzahlen in jedem Monat. Es ist üblich, die Rückzahlung konstant zu halten. Es wäre unpraktisch, die Rate monatlich anzupassen, da der Kunde dann nie genau wüsste, wie viel er zu überweisen hätte (ungeachtet der Tatsache, dass Kreditraten im allgemeinen automatisch abgebucht werden).



\section{Dreisatz}

Der Dreisatz ist kein Mathematischer Satz, wie wir sie später kennenlernen werden. Er beschreibt eine Vorgehensweise zum Lösen von Verhältnisaufgaben. Diese Verhältnisse treten in zwei Variationen auf:

\begin{enumerate}
\item $a\div b = c\div x$, mit $a,b,c$ gegeben und $x$ gesucht.
\item $a\cdot b = c\cdot x$ mit $a,b,c$ gegeben und $x$ gesucht.
\end{enumerate}

Als eindrucksvolles Beispiel sei hier das \textsl{Kartoffel-Paradoxon}\index{Kartoffel-Paradoxon} erwähnt, welches den Dreisatz mit der Prozentrechnung verbindet. 

\begin{quote}
Ein Bauer erntet 100kg Kartoffeln und lagert sie über den Winter ein. Zum Zeitpunkt der Ernte bestehen sie zu 99\% aus Wasser. Im Frühjahr möchte er sie verkaufen und bringt sie zum Markt. Zu diesem Zeitpunkt bestehen sie nur noch zu 98\% aus Wasser, während sich der Trockenanteil nicht verändert hat. Wieviel kg Kartoffeln bringt er zum Markt?
\end{quote}

Die meisten Menschen schätzen, wenn sie die Aufgabe zum ersten Mal hören, dass sich das Gesamtgewicht der Kartoffeln nicht wesentlich verändert hat. Also auf knapp 99kg. Sehen wir uns das genauer an und berechnen die Trockenmasse, also den Wert, der sich nicht verändert:

\begin{equation}\label{eq:3satz1}
\text{Trockenmasse: } 1\text{kg} = 100\text{kg} \cdot 1\% = 100\text{kg} \cdot \frac{1}{100}
\end{equation}
Die Trockenmasse der Kartoffeln kann sich über den Winter nicht ändern, also bleibt diese erhalten und wir suchen das Gesamtgewicht von dem 2\% gerade 1kg sind:
\begin{equation}
\text{Trockenmasse: } 1\text{kg} = ?\text{kg} \cdot 2\% = ?\text{kg} \cdot \frac{2}{100}
\end{equation}
Wir formen um zu 
\begin{equation}\label{eq:3satz3}
\frac{1\text{kg}\cdot 100}{2} = 50\text{kg}
\end{equation}
das bedeutet, dass sich das Gesamtgewicht der Kartoffeln halbiert hat, obwohl sich der Wasseranteil nur um ein Prozent verringert hat. Auf der anderen Seite wird es einsichtiger, wenn man bedenkt, dass sich der Anteil der Trockenmasse verdoppelt hat, obwohl das Gewicht der Trockenmasse gleich geblieben ist.

Die Interpretation, warum der Dreisatz "`Drei"'-Satz heißt, sind uneinheitlich. Favorisierte Ideen sind, a) dass es drei Werte und eine Unbekannte gibt, und b) dass drei Schritte zur Lösung notwendig sind. Wir schließen uns hier der Interpretation b) an. Die Gleichungen (\ref{eq:3satz1}) bis (\ref{eq:3satz3}) stellen somit die drei Schritte des Dreisatzes dar. Hier ein einfacheres Beispiel zu einem Dreisatz erster Form:

\begin{quote}
Wir kaufen ein Netz mit 25 Orangen, es wiegt 5kg. Wie viele Orangen sind in einem Netz mit 3kg?
\end{quote}
Um die Anzahl der Orangen in einem 3kg Netz herauszubekommen, müssen wir wissen, wie viel eine Orange wiegt, denn wir haben nur das Gewicht 3kg als Anhaltspunkt. Deshalb müssen wir die 5kg durch die 25 Orangen teilen um das Gewicht einer Orange zu bestimmen. Haben wir dies, können wir die Gleichung aufstellen, dass 3kg geteilt durch eine uns unbekannte Anzahl von Orangen das gleiche ergeben muss, wie die Division von 5kg durch die 25 Orangen. Somit wird die Berechnung wieder in drei Schritten dar gestellt:
\begin{equation*}
\begin{split}
\frac{5kg}{25} &=  200g \\
\frac{3kg}{?} &= 200g \\
\frac{3000g}{200g} &= 15
\end{split}
\end{equation*}

Und noch ein Beispiel für einen Dreisatz zweiter Form:
\begin{quote}
Die Arbeiter in einer Fabrik für Autozubehör nehmen sich aus einem Regal Kisten mit Einzelteilen, die zu Armlehnen zusammengebaut werden. Die Zufuhr von Kisten ist nicht beschränkt, das bedeutet, es sind immer genug Kisten für alle Arbeiter vorhanden. 

Zehn Arbeiter können eine vorgegebene Menge an Armlehnen innerhalb eines Arbeitstages (acht Stunden) zusammenbauen. Eines Tages meldet sich morgens bei Schichtbeginn der Kunde und möchte noch am selben Tag eine Tagesration an Armlehnen haben. Um die Armlehnen zum Kunden zu bringen, braucht ein LKW sechs Stunden. Wie viele Arbeiter muss der Zulieferer zusätzlich einsetzen, um den Kunden zufrieden zu stellen?
\end{quote}
Die Arbeitsleistung der Arbeiter ist nicht bekannt, wir können also nicht wie mit den Orangen vorgehen und zunächst die Arbeitsleistung eines einzelnen Arbeiters ausrechnen. Wir müssen die Gleichung sofort ansetzen. Erschwerend kommt hinzu, dass wir nicht wissen, wie lange Zeit den Arbeiten bleibt. Daher müssen wir uns überlegen, was die Länge des Tages beeinflusst. Es bleibt nur die Lieferung per LKW, die mit sechs Stunden zu Buche schlägt. Demnach bleiben nur zwei Stunden für die Herstellung von soviel Armlehnen, wie 10 Arbeiter sonst in acht Stunden herstellen.
\begin{equation*}
\begin{split}
10\cdot 8 &= ? \cdot 2 \\
\frac{10\cdot 8}{2} &= ? = 40
\end{split}
\end{equation*}
In diesem Fall gibt es auch drei Gleichungen, nämlich die linke Seite und die rechte Seite der ersten Gleichung entsprechen den Schritten 1 und 2 des Dreisatzes. Und die zweite Gleichung dem Schritt 3. Das ergibt die Lösung 40, sprich der Zulieferer muss 40 Arbeiter einsetzen, damit diese in zwei Stunden vollbringen, was sonst 10 Arbeiter in acht Stunden erbringen.

Allgemein geht man beim Dreisatz wie folgt vor:
\begin{enumerate}
\item Aufstellung der Gleichung mit den bekannten Werten.
\item Aufstellung der Gleichung mit einem unbekannten Wert.
\item Umformung und Lösung.
\end{enumerate}
Bei der Aufstellung der Gleichung gibt es keine allgemeine Vorgehensweise. Der Lernende muss erkennen können, ob der Dreisatz von erster oder zweiter Form ist. 

\section{Textaufgaben}

Textaufgaben spielen in der Schul-Mathematik eine sehr große Rolle. Anhand von textuell dargelegten Problemen soll Schülern ein intuitiver Einstieg in die Mathematik gegeben werden. Im Gegensatz dazu erwartet Studierende in der universitären Mathematik im allgemeinen keine Textaufgaben. Sie gelten somit lediglich als didaktisches Mittel zum Zweck der Motivation von Schülern. 

Wie im vorhergehenden Abschnitt bereits gesehen, sind Dreisatzaufgaben oft in Form von Textaufgaben vorgegeben, sodass der Lernende sich mit Alltagsproblemen konfrontiert sieht. 

Aufgrund der Tatsache, dass Textaufgaben keinen einheitlichen Lösungsweg haben und sie direkt von der Formulierung abhängen, ist in diesem Abschnitt auch nicht viel Weiteres darüber zu sagen, als dass sie existieren und Schüler sich damit auseinander setzen müssen. Beispielaufgaben sind im folgenden Abschnitt zu finden.

\section{Aufgaben}

TODO

% % Insert some special declarations 

\part{Lineare Algebra}


\chapter{Grundlagen}

Die Algebra konzentriert sich -- im Gegensatz zur Arithmetik -- auf die Verallgemeinerung der Begriffe zur Analyse bestimmter Sachverhalte. Der Arithmetik Teil dieses Buches begann mit Berechnungen auf Basis von Äpfeln. Diese geben einen unmittelbaren Zugang zu den Begriffen des Rechnens. In der (Linearen-) Algebra wird es solches nicht geben. Die Ansätze hier sind rein abstrakt zu verstehen. Auch wenn direkte, anschauliche Beispiele zu bestimmten algebraischen Sachverhalten existieren, so sollte der Lernende versuchen, nicht anhand dieser sein Verständnis auszubilden, sondern rein in der Sache selbst. Das führt letztlich dazu, dass er ohne konkrete Anschauung Sachverhalte analysieren und Probleme lösen kann. Vollständig unabhängig davon, ob diese Probleme einen realen, abstrakten oder mit dem Verstand nicht nachvollziehbaren Hintergrund haben. 

So gilt z.B. der Satz des Pythagoras \index{Pythagoras}
\[\lVert a\rVert ^2 +\lVert b\rVert ^2 = \lVert c\rVert^2\]
in jedem Vektorraum der ein Skalarprodukt $\langle .,.\rangle $ besitzt, mit einer induzierten Norm $\lVert a \rVert = \sqrt{\langle a,a\rangle} $, sowie $a+b=c$ ($a,b,c$ ein Dreieck bilden) gilt und $\langle a,b\rangle = 0 $ (der Winkel zwischen $a,b$ ein rechter Winkel ist, also 90 Grad). Unabhängig davon, ob dieser ein euklidischer Raum ist, ein Raum der Polynome oder der Raum der unendlich oft differenzierbaren Funktionen.

Die folgenden Abschnitte sind oft Zusammenstellungen von Definitionen. Leider sind diese notwendiges Handwerkszeug für alles was danach kommt. Die Definitionen scheinen manchmal langweilig und unmotiviert. Ich hoffe, die Lernenden nicht dadurch abzuschrecken!

\section{Zeichen}

In der Algebra verwendet man anstelle von Zahlen im allgemeinen Buchstaben. Es immer vom Kontext abhängig, welcher Buchstabe was bedeutet. Aber es haben sich bestimmte Dinge eingebürgert. So sind Konstanten meistens mit den Buchstaben
\[a, b, c, \dots \]
bezeichnet. Unbekannte in Gleichungen meist mit 
\[x, y, z, p, q, \dots \]
Sowie Indizes mit 
\[i, j, k, \dots \]
Es können aber auch griechische Buchstaben auftauchen. Hier eine Übersicht

\bigskip

\begin{center}
\begin{tabular}{c|c|l}
\hline
\textbf{Kleiner Buchstabe} & \textbf{Großbuchstabe} & \textbf{Bezeichnung} \\
\hline
$\alpha $ & $A $ & Alpha \\
$\beta $ & $B $ & Beta \\
$\gamma $ & $\Gamma $ & Gamma \\
$\delta $ & $\Delta $ & Delta \\
$\epsilon $ & $E $ & Epsilon \\
$\zeta $ & $Z $ & Zeta \\
$\eta $ & $H $ &  Eta\\
$\theta $ & $\Theta $ & Theta \\
$\iota $ & $I $ & Iota \\
$\kappa $ & $K $ & Kappa \\
$\lambda $ & $\Lambda $ & Lambda \\
$\mu $ & $M $ & Mu \\
$\nu $ & $N $ & Nu \\
$\xi $ & $\Xi $ &  Xi \\
$\omicron $ & $O $ & Omicron \\
$\pi $ & $\Pi $ & Pi \\
$\rho $ & $P $ & Rho \\
$\sigma $ & $\Sigma $ & Sigma  \\
$\tau $ & $T $ & Tau \\
$\upsilon $ & $\Upsilon $ & Ypsilon \\
$\phi $ & $\Phi $ & Phi \\
$\chi $ & $X $ & Chi \\
$\psi $ & $\Psi $ & Psi \\
$\omega $ & $\Omega $ & Omega \\
\hline
\end{tabular}
\end{center}


\section{Summen und Produkte}

In den folgenden Kapiteln werden wir öfter auf Summen- und Produktreihen von Funktionen treffen. Um diese möglichst einfach und gut verständlich darzustellen hat man sich in der Mathematik auf die Summen- und Produktnotation geeinigt. Summen werden mit einem griechischen, großen "`Sigma"' dargestellt $\Sigma$ \index{Summe $\Sigma$} und Produkte durch ein großes "`Pi"' $\Pi$. \index{Produkt $\Pi$}

Will man z.B. die Zahlen 1 bis 10 summieren, so kann man dies so darstellen:
\[ \text{Summe } = \sum_{n=1}^{10} n = 1+2+3+4+5+6+7+8+9+10 \]
Durch die Grenzen oberhalb und unterhalb weiß man, welche Werte $n$ annimmt. Alternativ gibt es auch folgende Möglichkeit. Sei $N = \{ 1,2, \dots ,10 \}$, dann wird die Summe so dargestellt:
\[ \text{Summe} = \sum_{n\in N} n \]
Beide Möglichkeiten werden verwendet. 
Für die Multiplikation gilt das selbe:
\[ \text{Produkt } = \prod_{n=1}^{10} n = 1\cdot 2\cdot 3\cdot 4\cdot 5\cdot 6\cdot 7\cdot 8\cdot 9\cdot 10 \]
Sowie
\[ \text{Produkt} = \prod_{n\in N} n  \]
Es werden auch andere Operationen, wie z.B. die Vereinigung oder der Schnitt von Mengen auf diese Weise dargestellt. 
\begin{definition}
Die Schreibweise ein großes Symbol zu verwenden mit darunter, wie darüber befindlichen Einschränkungen wird \emph{Operatorschreibweise} genannt und folgt dem Schema:
 \[ \text{Wert } = \bigbox_{\dots}^{\dots}\left( \dots \right) \]
\end{definition}
wobei $\bigbox$ jeder beliebige Operator sein kann und die Punkte darüber und darunter die Grenzen angeben, über die eben dieser Operator angewendet wird. Und schließlich das Argument in Klammern stellt ebenjenes dar, auf das der Operator angewandt wird. 

\section{Aussagelogik}

\subsection{Quantor}

Im Folgenden werden wir zwei sogenannte \emph{Quantoren} \index{Quantor} kennen lernen. Sie sind wichtig um Aussagen unzweideutig und in kurzer Art zu formulieren. Der erste ist $\forall$ und bedeutet "`für alle"'\index{Für alle} oder "`für jedes"'. Man benutzt ihn zum Beispiel, um Aussagen "`für alle"' Elemente einer Menge zu definieren.

\[ \forall x\in \mathbb{N} \]
bedeutet für alle $x$ in den natürlichen Zahlen.

Der zweite ist $\exists$, dieser bedeutet "`es gibt"' \index{Es gibt, es existiert} oder "`gibt es"'. So wäre 
\[ \forall x\in \mathbb{N}\ \exists y\in \mathbb{Q},\text{ mit }  y={x\over 1} \]
gleichbedeutend mit dem Satz: "`\textbf{Für alle} $x$ in den natürlichen Zahlen \textbf{gibt es} ein $y$ in den rationalen Zahlen mit $y={x \over 1}$"'.

Der $\exists$ Quantor kann mit einem Ausrufezeichen (!) verbunden werden um auszudrücken, dass es "`genau eins gibt"', also bedeutet
\[ \forall x\in \mathbb{N}\ \exists! y\in \mathbb{Q},\text{ mit }  y={x\over 1} \]
folgendes: "`\textbf{Für alle} $x$ in den natürlichen Zahlen \textbf{gibt es genau ein} $y$ in den rationalen Zahlen mit $y={x \over 1}$"'.

\subsection{Junktor}

Junktoren verknüpfen Aussagen zu einer Gesamtaussage.\index{Junktor}

\begin{enumerate}
\item $\wedge $ ist der "`und"' Junktor. Z.B. seien $a\in \mathbb{N} \wedge b\in \mathbb{Q} $
\item $\vee$ ist der "`oder"' Junktor. Z.B. sei $x\in \mathbb{R} \vee x\in \mathbb{Q}$.
\item $\Rightarrow$ ist der Implikations-Junktor. Z.B. $A\Rightarrow B$ bedeutet, aus $A$ folgt $B$ oder $x\in \mathbb{N} \Rightarrow -x \in -\mathbb{N}$
\item $\Leftrightarrow$ ist der Äquivalenz-Junktor. $A\Leftrightarrow B$ bedeutet $A$ gilt genau dann, wenn $B$ gilt.
\item $\neg$ negiert eine Aussage, z.B. bedeutet $\neg(x\in \mathbb{R})$, dass $x$ eben nicht aus den reellen Zahlen ist, oder $\neg \exists y\in \mathbb{Q}$ bedeutet, dass es kein $y$ in den rationalen Zahlen gibt.
\end{enumerate}


\section{Mengen}

\index{Menge}
Kategorisierung spielt in der (Linearen) Algebra eine sehr wichtige Rolle. Kennt man die Eigenschaften eines Dinges, so kann man damit umgehen. Daher ist es nicht verwunderlich, dass Mathematiker versuchen, Dinge, die identische Eigenschaften besitzen zu Mengen zusammenzustellen. Und -- soweit möglich -- Operationen und Eigenschaften mit dem Namen dieser Zusammenstellung zu verbinden. Die bereits im Arithmetik Teil vorgekommene Menge der natürlichen Zahlen $\mathbb{N}$ bildet da ein gutes Beispiel für den Einstieg. 

Im Folgenden werden wir immer mehr Vorgehensweisen kennenlernen, mit denen man Mengen untersuchen kann. Kann man bestimmte Eigenschaften an einer Menge feststellen, so wird sie zur Gruppe. Hat man weitere Eigenschaften an der Gruppe, so wird sie zum Ring. Und eine letzte Eigenschaft macht einen Ring zu einem Körper. Der Körper ist das, was unserer üblichen Vorstellung von Zahlen am nächsten kommt. Für viele Objekte der Mathematik kann man aber z.B. nur Gruppenstruktur nachweisen. 

\begin{quote}
HINWEIS: Der hier verwendete Begriff "`Menge"' hat nichts mit der Anzahl zu tun, wie er im Arithmetik Teil verwendet wurde. Hier bezeichnet "`Menge"' eine Ansammlung von gleichartigen Objekten.
\end{quote}

Mengen werden in der Mathematik in geschweiften Klammern $\{\dots \}$ dargestellt. In diesen Klammern stehen entweder die Elemente der Menge in Form einer Liste oder Aufzählung (diese kann unvollständig sein, wie wir dies gleich verwenden um die natürlichen Zahlen zu beschreiben), oder in Form einer Regel. Solche Regeln werden in Form von Aussage-Prädikaten formuliert, also z.B. $\forall$ entspricht "`für alle"' und $\exists$ entspricht "`existiert"', aber dies werden wir noch ausführlicher kennen lernen, wenn wir solche Konstrukte einsetzen. 

Die natürlichen Zahlen werden wie folgt definiert:
\[\mathbb{N} := \{ 1,2,3,4,5, \dots \} \]
Die Punkte "`$\dots$"' symbolisieren dabei, dass die Reihe nicht enden soll. Dementsprechend gehört $\infty$ "`Unendlich"' zu den natürlichen Zahlen.
\[\mathbb{N} := \{ 1,2,3,4,5, \dots, \infty \} \]

Zusammengefasst sind Mengen einfache Ansammlungen von im allgemeinen gleichartigen Dingen. Die folgende Zeichenfolge bedeutet, dass $x$ ein Element der Menge $M$ ist:
\[x \in M \]
So ist zum Beispiel
\[7 \in \mathbb{N} \]

\subsection{Definitionssprint!}
Im Folgenden seien $A$ und $B$ immer Mengen.

\begin{definition}\index{Teilmenge}
Wenn für alle $a\in A$ gilt $a\in B$, dann ist $A$ eine \emph{Teilmenge} von $B$. Dies wird so dargestellt:
\[
A \subset B
\]
\end{definition}

\begin{definition}\index{Leere Menge}
Eine leere Menge wird mit $\emptyset$ oder $\{\}$ dargestellt.
\end{definition}

\begin{definition}\index{Vereinigungsmenge}
Die Vereinigung von Mengen (auch \emph{Vereinigungsmenge} genannt) wird definiert als
\[A \cup B = \{ x | x\in A \vee x\in B \}\]
\end{definition}

\begin{definition}\index{Schnittmenge}
Die \emph{Schnittmenge} wird definiert als 
\[A \cap B = \{x| x\in A \wedge x\in B \} \]
\end{definition}

\begin{definition}
Subtraktion von Mengen
\[ A \backslash B = \{ x \vert x\in A, \  \neg (x\in B) \} \]
\end{definition}

\section{Abbildung}
\index{Abbildung}

Zur Darstellung einer Abbildung benötigt man im allgemeinen drei Informationen. 

\begin{enumerate}
\item Den Namen der Abbildung
\item Den Definitionsbereich
\item Den Bildbereich
\end{enumerate}

Definitions- und Bildbereich sind, sofern sie nicht weiter spezifiziert werden, Mengen. Die Abbildung wird somit als eine Relation (Beziehung) von zwei Mengen gesehen. Diese Beziehung wird durch einen Pfeil dargestellt. Sei $D$ der Definitions- und $B$ der Bildbereich, dann sieht diese Beziehung so aus:

\begin{equation}
D \longrightarrow B
\end{equation}
Um dieser Beziehung einen Namen zu geben, stellt man diesen vor diese Beziehung und trennt die Informationen durch einen Doppelpunkt. Sei zum Beispiel $a$ der Name der Abbildung, dann ist die vollständige Darstellung
\begin{equation}
a : D \longrightarrow B
\end{equation}
Eine Abbildung bildet Elemente aus einer Menge auf Elemente einer anderen Menge ab. Seien $X,Y$ Mengen. Dann ordnet die Abbildung
\[ f : X \longrightarrow Y \]
das Element $x\in X$ auf $y\in Y$ ab. Das wird so ausgedrückt:
\[ f(x) = y\]

Die Addition ist ebenfalls eine Abbildung. Sie bildet zwei Elemente einer Menge auf Elemente der selben Menge ab:
\[ + : X \times X \longrightarrow X \]
Also kann man die Summe 
\[ x+y = z\]
auch in Funktionenform schreiben:
\[ +(x,y) = z\]

\section{Gruppen, Ringe, Körper}
Im Folgenden bezeichen $X,Y$ und $Z$ immer Mengen.

\subsection{Definitionssprint!}

\begin{definition}
Eine Funktion \index{Verknüpfung}
\[ \circ : X \times X \longrightarrow X \]
wird als \emph{Verknüpfung} auf $X$ bezeichnet. Zum Beispiel sind die Operationen $+$ und $\cdot$ Verknüpfungen. 
\end{definition}

\begin{definition} \index{assoziativ}
Eine Verknüpfung wird \emph{assoziativ} genannt, wenn gilt:
\[ a \circ (b \circ c) = (a \circ b) \circ c \]
\end{definition}
\begin{definition}\index{kommutativ}
Sie wird \emph{kommutativ} genannt, wenn gilt: 
\[ a \circ b = b \circ a \]
\end{definition}
\noindent Für alle $a,b,c \in X$

\begin{definition}\index{Neutrales Element}
Ein Element $e\in X$ wird als \emph{neutrales Element} bezüglich einer Verknüpfung bezeichnet, wenn 
\[ e\circ a = a \circ e = a \]
für alle $a\in X$ gilt.
\end{definition}

\begin{claim}
Falls es in $X$ ein neutrales Element gibt, so ist es eindeutig. 
\end{claim}
\begin{proof}
Seien $e_1$ und $e_2$ neutrale Element bezüglich $\circ$, dann gilt: 
\[ e_1 = e_1\circ e_2 = e_2 \]
\end{proof}

\begin{definition}\index{Monoid}
Eine Menge $X$ mit assoziativer Verknüpfung und zugehörigem neutralem Element heißt \emph{Monoid}.
\end{definition}

\begin{definition}
Sei $X$ ein Monoid mit der Verknüpfung $\circ$ und zugehörigem neutralen Element $e$. Wenn gilt
\[ a\circ b = e \]
für $a,b \in X$, so wird $a$ das \emph{Linksinverse}\index{Linksinverse} von $b$ genannt, sowie $b$ das \emph{Rechtsinverse}\index{Rechtsinverse} von a. Gilt zudem
\[ a\circ b = b \circ a = e \]
Dann sind $a$ und $b$ jeweils ihre \emph{Inversen}. \index{Inverse}
\end{definition}

\begin{definition}\index{Gruppe}
Ist in einem Monoid jedes Element invertierbar, so nennt man dies eine \emph{Gruppe}.
\end{definition}

\begin{definition}\index{Gruppe, abelsch}
Eine Gruppe, deren Verknüpfung kommutativ ist, nennt man eine \emph{abelsche Gruppe}\footnote{Benannt nach dem norwegischen Mathematiker \textbf{Niels Henrik Abel}, * 5. August 1802 auf der Insel Finn\o y, Ryfylke, Norwegen; \ding{61} 6. April 1829 in Froland, Aust-Agder, Norwegen} .
\end{definition}

\begin{definition}\index{Ring}
Ein \emph{Ring} $R$ ist eine Menge mit zwei Verknüpfungen "`$+$"' und "`$\cdot$"'. Wobei $R$ bezüglich der Addition eine abelsche Gruppe ist und bezüglich der Multiplikation ein Monoid. Des Weiteren muss das Distributivgesetz gelten
\[ a\cdot (b+c) = a\cdot b + a\cdot c \]
für alle $a,b,c \in R$. Ist $R$ bezüglich "`$\cdot$"' kommutativ, so ist $R$ ein \emph{kommutativer Ring}.\index{Ring, kommutativer}
\end{definition}

\begin{definition}
Sei $K$ ein kommutativer Ring, dessen neutrales Element bezüglich der Addition $0$ ist. Falls alle Elemente von $K\backslash \{0\}$ invertierbar sind, heißt $K$ ein \emph{Körper}.\index{Körper}
\end{definition}

\section{Aufgaben}

TODO



\chapter{Vektor, Matrix, Tensor}



\section{Vektorraum}\label{vectorspace}
\index{Vektorraum}

Sei $(K,+,\cdot )$ ein Körper und $(V,+ ,\cdot)$ eine Menge mit zwei Verknüpfungen:
\begin{eqnarray*}
+ : V\times V &\longrightarrow& V \\
\cdot : K \times V &\longrightarrow& V 
\end{eqnarray*}
Hierbei ist zu beachten, dass die Verknüpfungen einmal auf $K$ und einmal auf $V$ definiert wurden. Man könnte hier verschiedene Zeichen zur Unterscheidung der Additionen einführen. Dies ist im allgemeinen aber unüblich, sodass hier vom Lernenden verlangt wird, den Unterschied selbst zu erkennen. Auf $V$ wird die Addition als \textsl{Vektoraddition} bezeichnet, sowie die Multiplikation als \textsl{Skalare Multiplikation}, da sie nicht zwischen zwei Vektoren, sondern einem Element des zugrunde liegenden Körpers und einem Vektor definiert ist. 

Man nennt $(V,+ ,\cdot)$ einen \textsl{Vektorraum}\index{Vektorraum} über dem Körper $K$, oder auch \textsl{K-Vektorraum}, wenn folgende Eigenschaften erfüllt sind:

\noindent Es seien $u,v,w \in V$ und $a,b \in K$:

Die Anforderungen an den Vektorraum werden in solche, die an die Addition (A*) und solche, die an die Multiplikation (M*) gestellt werden unterschieden:

\begin{description}
\item[(A1)] Assoziativgesetz: $u+(v+w) = (u+v)+w$
\item[(A2)] $0$ Element: $0\in V$ mit $v+0=0+v=v$
\item[(A3)] Inverses Element zur Addition: Zu jedem $v\in V$ gibt es ein $-v\in V$ mit $v+(-v) = -v+v = 0$
\item[(A4)] Kommutativgesetz: $u+v = v+u$
\end{description}

\begin{description}
\item[(M1)] Assoziativgesetz: $(a \cdot b)\cdot v = a\cdot (b\cdot v)$
\item[(M2)] 1 Element: $1\cdot v = v$
\item[(M3)] Distributivgesetz: $(a+b)\cdot v = a\cdot v + b\cdot v$
\item[(M4)] Distributivgesetz: $a\cdot(v+w) = a\cdot v + a\cdot w$
\end{description}

Es ist wichtig zu verstehen, wo die Unterschiede zwischen (M3) und (M4) liegen. (M3) besagt, dass die Summe zweier Zahlen in $K$ auf den Vektor $v$ verteilt (distribuiert) werden kann, während (M4) besagt, dass die Summe zweier Elemente aus $V$ auf $a$ verteilt werden kann. Das heißt: (M3) ist eine Eigenschaft der Addition in $K$ während (M4) eine Eigenschaft der Addition aus $V$ ist! 

\begin{svgraybox}
Zur Vertiefung der sei an dieser Stelle auf das grundlegende Werk von Egbert Brieskorn \cite{Brieskorn1} hingewiesen. 
\end{svgraybox}

\begin{definition}
Ein Element eines Vektorraums wird \textsl{Vektor} genannt.
\end{definition}

\subsection{Linearkombination}

\begin{definition}
Eine beliebige Summe von $p$ Vektoren $v_i \in V$ mit Faktoren $a_i \in K$ wird als \textsl{Linearkombination} bezeichnet.
\[
l = \sum_{i=1}^{p} a_i \cdot v_i = a_1\cdot v_1 + a_2 \cdot v_2 + \dots + a_p \cdot v_p
\]
Die Summe ist wiederum ein Vektor in $V$.

\end{definition}

\subsection{Basis}

Der Begriff der Basis ist an dieser Stelle noch nicht einführbar, ohne weitere Begriffe definiert zu haben, die aktuell von wenig Nutzen sind und den Lernenden eher verwirren. Daher beschränken wir uns aktuell nur auf eine Basis:

\begin{definition}
Als die \textsl{Standard-Basis} des $\mathbb{R}^n$ werden die Vektoren $e_i$ mit $i=1,\dots,n$ bezeichnet, deren Einträge überall null sind, bis auf den Eintrag $i$ und dieser ist 1. 
\begin{equation}
e_1 = \begin{pmatrix}
1\\
0\\
0\\
\vdots \\
0
\end{pmatrix}, e_2 = \begin{pmatrix}
0\\
1\\
0\\
\vdots \\
0
\end{pmatrix}, \dots, e_n = \begin{pmatrix}
0\\
0\\
\vdots \\
0\\
1
\end{pmatrix}
\end{equation}
\end{definition}

Jeder Vektor in $v\in V$ kann als Linearkombination dieser Basis geschrieben werden. Im Besonderen ist -- aufgrund der Wahl der Basisvektoren $e_i$ -- jeder Faktor dieser Linearkombination identisch mit dem $i$-ten Eintrag des Vektors. 

\[
v = \begin{pmatrix}
v_1 \\
v_2 \\
\vdots \\
v_n
\end{pmatrix} = \sum_{i=1}^{n} v_i \cdot e_i
\]

\subsection{Lineare Unabhängigkeit}

\begin{definition}
Die \textsl{linare Unabhängigkeit} \index{Unabhängigkeit, linear} von Vektoren wird definiert durch eine Linearkombination, d.h. eine Summe von Vektoren mit konstanten Faktoren. Wenn diese Linearkombination nur dadurch zu einem Null Vektor gemacht werden kann, indem alle Faktoren zu 0 werden, dann sind die Vektoren linear unabhängig. Seien $x_i$ aus einem K-Vektorraum und $\lambda_i $ aus dem zugrunde liegenden Körper. 

\begin{equation}\label{eq:linunabh}
\sum_{i=1}^{n} \lambda_i x_i = 0 \quad \text{dann, und nur dann, wenn} \quad \lambda_i = 0
\end{equation}
\end{definition}

Dies sollte näher erklärt werden: Nehmen wir ein einfaches Beispiel und sehen uns zwei Vektoren in der Ebene an. 

\bigskip

\begin{figure}
\begin{center}
\begin{tikzpicture}[>=stealth]
\coordinate (A) at (0,0);
\coordinate (B) at (3,1);
\coordinate (C) at (1,3);
\coordinate (D) at (-2,2);
\tikzset{-}
\draw[lightgray, step=1cm] (-3,-1) grid (4,4);
\tikzset{-}
\draw[lightgray, dashed] (D) -- (C);
\draw[lightgray, dashed] (B) -- (C);
\tikzset{->}
\draw (A) -- (B) node[midway, sloped, above] {$\lambda_1 \cdot x_1$};
\draw (A) -- (D) node[midway, sloped, above] {$\lambda_2 \cdot x_2$};
\draw[line width=2] (A) -- (C) node[midway, sloped, above] {$\lambda_1 \cdot x_1+\lambda_2 \cdot x_2$};
\end{tikzpicture}
\caption[Lineare Unabhängigkeit von Vektoren]{Darstellung der linearen Unabhängigkeit von Vektoren}
\end{center}
\end{figure}

\bigskip

Per Definition der linearen Unabhängigkeit kann der dicke Pfeil nur dann die Länge 0 haben, wenn $\lambda_1$ und $\lambda_2$ beide 0 sind. Salopp gesprochen könnte man sagen: Der Durchmesser eines Parallelograms\footnote{Ein Viereck dessen gegenüberliegenden Seiten parallel zueinander sind, nennt man Parallelogramm.} -- wie das, welches die Vektoren aufspannen -- kann nur dann 0 sein, wenn die Seitenlängen 0 sind.

\subsection{Dimension}

\begin{definition}
Die maximale Anzahl von linear unabhängigen Basisvektoren eines Vektorraums wird \textsl{Dimension}\index{Dimension} genannt. Der $\mathbb{R}^n$ hat die Dimension $n$, da es genau $n$ Möglichkeiten gibt, Basisvektoren zu bilden, die aus $n-1$ Nullen und einer 1 bestehen.
\end{definition}


\section{Lineare Gleichungssysteme}

Lineare Gleichungssysteme treten häufig in der Physik und den Ingenieurswissenschaften auf. Dabei handelt es sich um Probleme in mehreren Unbekannten, die durch eine Reihe von Randbedingungen -- in Form von Gleichungen -- spezifiziert werden.

Bei diesen Problemen ist es notwendig, dass alle Unbekannte die Randbedingungen gleichzeitig erfüllen:

\begin{equation}\label{eq:syseq}
\begin{split}
a_{1,1}x_1 + a_{1,2}x_2 + \dots + a_{1,n}x_n &= b_1 \\
a_{2,1}x_1 + a_{2,2}x_2 + \dots + a_{2,n}x_n &= b_2 \\
a_{3,1}x_1 + a_{3,2}x_2 + \dots + a_{3,n}x_n &= b_3 \\
\vdots &= \vdots \\
a_{m,1}x_1 + a_{m,2}x_2 + \dots + a_{m,n}x_n &= b_m 
\end{split}
\end{equation}
Die Werte für $a_{i,j}$ und $b_{i}$ sind vorgegeben. Gesucht werden die $x_j$ Werte. 

Gleichungssysteme werden nach der Anzahl ihrer Gleichungen in folgende Kategorien eingeteilt:

\begin{description}
\item[$m<n$] Solche Systeme bezeichnet man als \textsl{unterbestimmt}, es existieren weniger Gleichungen als Unbekannte. Die Lösung solcher Systeme ist im allgemeinen nicht eindeutig. Solche Gleichungssysteme treten in der linearen Optimierung auf. 
\item[$m>n$] Solche Systeme bezeichnet man als \textsl{überbestimmt}, es existieren mehr Gleichungen als Unbekannte. Es ist im allgemeinen nicht möglich eine Lösung für solche Systeme anzugeben, da es möglich ist, sich widersprechende Randbedingungen in den Gleichungen zu formulieren. 
\item[$m=n$] Auch als quadratische Gleichungssysteme bezeichnet. Sofern bestimmte Bedingungen von den $a_{i,j}$ Werten erfüllt werden, gibt es genau eine Lösung. Dies sind die Systeme, mit denen wir uns hier beschäftigen werden.
\end{description}

\subsection{Kompakte Notation}

Mathematiker machen sich gerne das Leben einfacher, in dem sie ihre Formeln abkürzen. Das Gleichungssystem (\ref{eq:syseq}) ist sehr unhandlich. Daher führen wir folgende Schreibweisen ein:

Wir schreiben die $x_j$ und $b_i$ Werte in einer Spaltenform: 

\[
x = \begin{pmatrix}
x_1 \\
x_2 \\
\vdots \\
x_n
\end{pmatrix}, \quad b = \begin{pmatrix}
b_1 \\
b_2 \\
\vdots \\
b_m
\end{pmatrix}
\]
Sowie für die $a_{i,j}$ Werte eine rechteckige Tabellenform:

\[
A = \begin{pmatrix}
a_{1,1} & a_{1,2} & \cdots & a_{1,n} \\
a_{2,1} & a_{2,2} & \cdots & a_{2,n} \\
\vdots & \vdots & \ddots & \vdots \\
a_{m,1} & a_{m,2} & \cdots & a_{m,n}
\end{pmatrix}
\]

Definieren wir nun eine Multiplikation zwischen der rechteckigen Tabellenform und einer Spaltenform auf diese Weise:
\begin{equation*}
a_{i,1}x_1 + a_{i,2}x_2 + \dots + a_{1,n}x_n = \sum_{j=1}^{n} a_{i,j}x_j
\end{equation*}
für jede Zeile $i$, so lässt sich das Gleichungssystem (\ref{eq:syseq}) schreiben als:

\begin{equation*}
\begin{pmatrix}
a_{1,1} & a_{1,2} & \cdots & a_{1,n} \\
a_{2,1} & a_{2,2} & \cdots & a_{2,n} \\
\vdots & \vdots & \ddots & \vdots \\
a_{m,1} & a_{m,2} & \cdots & a_{m,n}
\end{pmatrix} \cdot \begin{pmatrix}
x_1 \\
x_2 \\
\vdots \\
x_n
\end{pmatrix} = \begin{pmatrix}
b_1 \\
b_2 \\
\vdots \\
b_m
\end{pmatrix}
\end{equation*}
oder
\begin{equation*}
Ax=b
\end{equation*}

Hier haben wir uns noch keine Gedanken darüber gemacht, was diese Spalten und Tabellen eigentlich sind. Sie stellen für uns lediglich eine Vereinfachung der Schreibweise dar. Später werden wir erkennen, dass die Spalten Vektoren sind und die Tabelle eine Matrix.


\subsection{Matrix}

\begin{definition}
Eine \textsl{Matrix} \index{Matrix} ist eine rechteckige Struktur von Elementen des Körpers, über dem sie gebildet werden. Hier werden im allgemeinen nur reelle Matrizen betrachtet. Somit sind diese Matrizen aus dem $\mathbb{R}^{m\times n}$:
\end{definition}

\begin{equation*}
(A)_{i,j} = a_{i,j} \in \mathbb{R}
\end{equation*}

Mit der Matrix Addition und der Skalaren Multiplikation
\begin{eqnarray*}
A+B &=& (a_{i,j} + b_{i,j})_{i,j} \\
\alpha A &=& (\alpha a_{i,j})_{i,j}
\end{eqnarray*}
wird der $\mathbb{R}^{m\times n}$ zu einem Vektorraum. Das Nachrechnen der A1-A4 und M1-M4 Eigenschaften aus Kapitel \ref{vectorspace} ist Teil der Aufgaben.

Dabei ist zu beachten, dass $0 \in \mathbb{R}^{m\times n} = (0)_{i,j}$ die Null Matrix ist, und das neutrale Element der Multiplikation die Einsmatrix $\mathbf{1}$, oder auch Identität $I$ genannt:

\[
\mathbf{1} = I =
\begin{pmatrix}
1 & 0 & 0 & \cdots & 0 \\
0 & 1 & 0 & \cdots & 0 \\
0 & 0 & 1 & \cdots & 0 \\
\vdots & \vdots & \vdots & \ddots & \vdots \\
0 & 0 & 0 & \cdots & 1
\end{pmatrix}
\]

\subsubsection{Rechenregeln}

\begin{definition}
Die Zahlen in den einzelnen Zellen der Tabelle nennt man \textsl{Einträge} oder auch \textsl{Komponenten}. Das selbe gilt für einen Vektor, der als einspaltige Matrix interpretiert werden kann.
\end{definition}

Die Addition zweier Matrizen wird auf die einzelnen Einträge abgebildet. Als Abkürzung wird hier die Notation $(\dots)_{i,j}$ verwendet. Dies ist eine Darstellung des Eintrags an der $i$-ten Zeile und $j$-ten Spalte. Wenn keine Einschränkung an $i$ und $j$ gemacht werden, so gilt diese Abkürzung für jeden Eintrag der Matrix. Und somit bestimmt diese Abkürzung auch die gesamte Matrix.

\begin{equation*}
A + B = \left( a_{i,j} + b_{i,j} \right)_{i,j} = \begin{pmatrix}
a_{1,1}+b_{1,1} & a_{1,2}+b_{1,2} & \cdots & a_{1,n}+b_{1,n} \\
a_{2,1}+b_{2,1} & a_{2,2}+b_{2,2} & \cdots & a_{2,n}+b_{2,n} \\
\vdots & \vdots & \ddots & \vdots \\
a_{m,1}+b_{m,1} & a_{m,2}+b_{m,2} & \cdots & a_{m,n}+b_{m,n}
\end{pmatrix}
\end{equation*}
In gleicher Weise wird die skalare Multiplikation realisiert

\begin{equation*}
\alpha \cdot A = \left( \alpha \cdot a_{i,j} \right)_{i,j} = \begin{pmatrix}
\alpha \cdot a_{1,1} & \alpha \cdot a_{1,2} & \cdots & \alpha \cdot a_{1,n} \\
\alpha \cdot a_{2,1} & \alpha \cdot a_{2,2} & \cdots & \alpha \cdot a_{2,n} \\
\vdots & \vdots & \ddots & \vdots \\
\alpha \cdot a_{m,1} & \alpha \cdot a_{m,2} & \cdots & \alpha \cdot a_{m,n}
\end{pmatrix}
\end{equation*}

Die Multiplikation von Matrizen $A,B$ ist in der Form definiert, dass der Eintrag an der Stelle $i,j$ definiert ist durch die Summe der Produkte der Einträge der $i$-ten Zeile von A mit den Einträgen der $j$-ten Spalte von B. Sei $A\in \mathbb{R}^{m\times n}$ und $B\in \mathbb{R}^{n\times o}$

\begin{equation*}
\begin{split}
A \cdot B & = \left( \sum_{k=1}^{n} a_{i,k} \cdot b_{k,j} \right)_{i,j} \\
 &= \begin{pmatrix}
\left( \sum_{k=1}^{n} a_{1,k} \cdot b_{k,1} \right) & \left( \sum_{k=1}^{n} a_{1,k} \cdot b_{k,2} \right) & \cdots & \left( \sum_{k=1}^{n} a_{1,k} \cdot b_{k,o} \right) \\
\left( \sum_{k=1}^{n} a_{2,k} \cdot b_{k,1} \right) & \left( \sum_{k=1}^{n} a_{2,k} \cdot b_{k,2} \right) & \cdots & \left( \sum_{k=1}^{n} a_{2,k} \cdot b_{k,o} \right) \\
\vdots & \vdots & \ddots & \vdots \\
\left( \sum_{k=1}^{n} a_{m,k} \cdot b_{k,1} \right) & \left( \sum_{k=1}^{n} a_{m,k} \cdot b_{k,2} \right) & \cdots & \left( \sum_{k=1}^{n} a_{m,k} \cdot b_{k,o} \right) \\
\end{pmatrix}
\end{split}
\end{equation*}

Aus dieser Definition folgt, dass die Anzahl der Spalten der Matrizen links vom Multiplikationszeichen und die Anzahl der Zeilen der Matrix rechts vom Multiplikationszeichen identisch sein müssen. Sind sie nicht identisch, ist die Mulitplikation \textbf{nicht} definiert und die Matrizen können nicht miteinander multipliziert werden. Des Weiteren besitzt die resultierende Matrix soviel Zeilen, wie Matrix $A$ aber soviel Spalten wie Matrix $B$.

\begin{svgraybox}
Hat Matrix $B$ nur eine Spalte ($B\in \mathbb{R}^{n\times 1}$) -- ist also ein Vektor --, dann ist dadurch auch gleichzeitig die Matrix-Vektor Multiplikation definiert. 
\end{svgraybox}

Seien $A\in \mathbb{R}^{m\times n}$ und $b\in \mathbb{R}^{n\times 1}$, dann ist die Matrix-Vektor Multiplikation

\begin{equation*}
A \cdot b = \left( \sum_{k=1}^{n} a_{i,k} \cdot b_{k} \right)_{i} = \begin{pmatrix}
\left( \sum_{k=1}^{n} a_{1,k} \cdot b_{k} \right) \\
\left( \sum_{k=1}^{n} a_{2,k} \cdot b_{k} \right) \\
\vdots \\
\left( \sum_{k=1}^{n} a_{m,k} \cdot b_{k} \right) \\
\end{pmatrix}
\end{equation*}
Im Weiteren ist der $\mathbb{R}^{n\times 1} = \mathbb{R}^n$.


\subsubsection{Transposition}

Die Transponierte einer Matrix erhält man durch vertauschen der Indizes. Sei $A \in \mathbb{R}^{m\times n} $ eine Matrix. 

\begin{definition}
Die Transponierte $A^T \in \mathbb{R}^{n\times m}$ ist definiert durch:
\[
	A^T = (a_{j,i})_{i=1,\dots, n; j=1,\dots, m}
\]
\end{definition}
Das Transponieren gilt auch für Vektoren genauso, da diese ja einspaltige Matrizen sind. Sei $v\in \mathbb{R}^{m\times 1}$ ein Vektor. Der transponierte Vektor $v^T$ ist dann im aus dem Raum $\mathbb{R}^{1\times m}$

\[
v^T = \begin{pmatrix}
v_1\\
v_2\\
\vdots \\
v_m
\end{pmatrix}^T = (v_1, v_2, \dots , v_n)
\]

Aufgrund der Regel, dass Matrizen nur dann miteinander multipliziert werden dürfen, wenn die Spalten-Anzahl der linken Matrix mit der Zeilen-Anzahl der rechten Matrix übereinstimmen muss, konnten Vektoren bisher nicht an Matrizen von links multipliziert werden. Mit dem transponierten Vektor geht dies, da er eine Spalten-Anzahl hat, die mit der Matrix übereinstimmen kann. Seien wieder $v\in \mathbb{R}^m$ und $A\in \mathbb{R}^{m\times n}$.

\[
v^T \cdot A = (\sum_{i=1}^{m}v_i\cdot a_{i,j})_{j} = \begin{pmatrix}
\sum_{i=1}^{m}v_i\cdot a_{i,1} \\
\sum_{i=1}^{m}v_i\cdot a_{i,2} \\
\vdots \\
\sum_{i=1}^{m}v_i\cdot a_{i,n}
\end{pmatrix} \in \mathbb{R}^n
\]

\subsubsection{Rang und Determinante}

TODO

\section{Algebraische Strukturen}

TODO

\subsection{Eigenschaften von Abbildungen}


\begin{definition}
Eine Abbildung $\phi : V \longrightarrow W$ wird als \textsl{injektiv}\index{injektiv} bezeichnet, wenn für $u,v \in V$ 
\[ u\ne v \Rightarrow \phi(u) \ne \phi(v) \]
\end{definition}
Umgekehrt gilt: Aus $\phi(u)=\phi(v)$ folgt, dass $u=v$ ist. 

\begin{definition}
Eine Abbildung $\phi : V \longrightarrow W$ wird als \textsl{surjektiv}\index{surjektiv} bezeichnet, wenn für alle $w\in W$ mindestens ein $v\in V$ existiert mit 
\[\phi(v) = w\]
\end{definition}

\begin{definition}
Eine Abbildung $\phi : V \longrightarrow W$ wird als \textsl{bijektiv}\index{bijektiv} bezeichnet, wenn sie sowohl injektiv als auch surjektiv ist. 
\end{definition}

Aufgrund der Surjektivität gibt es zu $w\in W$ immer ein $v\in V$  mit $\phi(v)=w$ und dieses $v$ ist eindeutig aufgrund der Injektivität. Daher gilt eine bijektive Abbildung $\phi$ als \textsl{umkehrbar eindeutig}\index{umkehrbar eindeutig}. D.h. es gibt ein $\phi^{-1} : W \longrightarrow V$ mit $v = \phi^{-1}(w)$.

\subsection{Morphologie}

\begin{definition}

Ein \textsl{Vektorraumhomomorphismus}\index{Homomorphismus}\index{Vektorraumhomomorphismus} -- auch verkürzt einfach Homomorphismus genannt -- von $V$ nach $W$ ist eine Abbildung $\phi : V \longrightarrow W$ mit folgenden Eigenschaften:
\begin{enumerate}
\item Für alle $v,w \in V$ gilt: $\phi(v+w) = \phi(v)+\phi(w)$
\item Für alle $a\in K$ und $v\in V$ ist $\phi(av)=a\phi(v)$
\end{enumerate}
Ist $V=W$, so wird $\phi : V\longrightarrow V$ als \textsl{Endomorphismus}\index{Endomorphismus} bezeichnet.
\end{definition}

\begin{definition}
Ist $\phi : V\longrightarrow W$ ein bijektiver Homomorphismus, so wird $\phi$ als \textsl{Isomorphismus}\index{Isomorphismus} bezeichnet.
\end{definition}

\begin{definition}
Ist $\phi : V\longrightarrow V$ ein bijektiver Endomorphismus,  so wird $\phi$ als \textsl{Automorphismus}\index{Automorphismus} bezeichnet.
\end{definition}

\subsection{Linearformen}

\begin{definition}
Eine \textsl{Linearform} \index{Linearform} ist eine lineare Abbildung von einem K-Vektorraum in seinen zugrundeliegenden Körper
\[
l : V \longrightarrow K
\]
mit der Eigenschaft
\[l(\alpha v + \beta w) = \alpha l(v) + \beta l(w) \]
für $v,w\in V$ und $\alpha, \beta \in K$
\end{definition}

\subsubsection{Bilinearformen}

\begin{definition}
Seien $V,W$ K-Vektorräume. Eine \textsl{Bilinearform} $b$ ist eine lineare Funktion
\[
	b : V \times W \longrightarrow K
\]
mit den Eigenschaften:
\begin{eqnarray*}
b(\alpha v + \beta w,x) &=& \alpha b(v,x) + \beta b(w,x) \\
b(v,\alpha w + \beta x) &=& \alpha b(v,w) + \beta b(v,x) 
\end{eqnarray*}
für $v\in V$, $w\in W$ und $\alpha, \beta \in K$. Um genau zu sein ist eine Bilinearform eine zwei Parametrige lineare Funktion, die sich in beiden Parametern wie eine Linearform verhält.
\end{definition}

Jede Bilinearform hat eine Matrix-Darstellung, denn zu jeder Bilinearform $\beta(.,.)$ kann folgende Matrix definiert werden:
\[
B = (b_{i,j})_{i,j} := (\beta(e_i, e'_j))_{i,j}
\]
mit den Basisvektoren $e_i\in V$ und $e'_j\in W$.

\begin{definition}
Eine spezielle Bilinearform $\langle .,.\rangle : V\times V \longrightarrow K$ wird \textsl{Skalarprodukt} oder auch \textsl{inneres Produkt} genannt. (Man darf dies nicht mit der Skalarmultiplikation verwechseln, also dem Produkt aus einer Zahl aus dem Körper mit einem Vektor). In reellen Vektorräumen nimmt es die folgende Form an:
\[
\langle u,v \rangle = \sum_{i=1}^{n} u_i\cdot v_i = u^T \cdot v
\]
für $u,v \in V$
\end{definition}

\begin{definition}
Seien $u,v \in V$ und $\langle .,.\rangle$ ein Skalarprodukt auf $V$. Seien $u\ne 0$ und $v\ne 0$. Falls 
\[ \langle u,v \rangle = \langle v,u \rangle = 0\]
ist, so bezeichnet man $u$ und $v$ als zueinander \textsl{orthogonal}. \index{orthogonal}
\end{definition}


\subsubsection{Definitheit}


\begin{definition}
Sei $b(.,.)$ eine Bilinearform. $b(.,.)$ ist

\begin{enumerate}
\item \textsl{positiv definit}, falls $b(v,v)>0$\index{positiv definit}
\item \textsl{positiv semidefinit}, falls $b(v,v)\ge 0$\index{positiv semidefinit}
\item \textsl{negativ definit}, falls $b(v,v) <0$\index{negativ definit}
\item \textsl{negativ semidefinit}, falls $b(v,v)\le 0$\index{negativ semidefinit}
\end{enumerate}
für alle $v\in V $
\end{definition}



\subsubsection{Norm und Abstände}

\begin{definition}
Eine \textsl{Norm}\index{Norm} ist eine lineare Abbildung $\Vert . \Vert : V \longrightarrow \mathbb{R}_+$, vom Vektorraum in die positiven reellen Zahlen. Dabei gelten folgende Eigenschaften:
\begin{description}
\item[(1)] Aus $\Vert v\Vert = 0$ folgt $v=0$
\item[(2)] $\Vert a\cdot v\Vert = \vert a\vert \cdot \Vert v\Vert$
\item[(3)] $\Vert u+v\Vert \le \Vert u\Vert + \Vert v\Vert $ (Dreiecksungleichung)
\end{description}
Mit $u,v\in V$, $a\in K$ und $\vert a\vert$ dem Absolutbetrag des Skalars $a$.
\end{definition}
Zu beachten ist, dass die Norm nicht in den zugrundeliegenden Körper abbildet. Falls $K=\mathbb{C}$, dann ist trotzdem das Ergebnis der Norm in den positiven reellen Zahlen. Ein Vektorraum, in dem eine Norm definiert ist, nennt man einen \textsl{normierten Raum}\index{Raum, normiert}, oder \textsl{normierter Vektorraum}.

\begin{definition}
Sei $V$ ein reeller Vektorraum $\mathbb{R}^n$ mit einem Skalarprodukt $\langle .,.\rangle$. Die Linearform 
\[
\Vert u \Vert = \sqrt{\langle u,u \rangle}
\]
wird \textsl{euklidische Norm}\index{Norm, euklidische} oder vom Skalarprodukt \textsl{induzierte Norm}\index{Norm, induziert} genannt. Sie ist hier separat definiert, da sie in diesem Kapitel eine zentrale Rolle spielt. Die folgenden Definitionen sind ebenfalls korrekt:
\begin{equation}\label{eq:2norm}
\Vert u \Vert = \sqrt{u^T \cdot u} = \sqrt{\sum_{i=1}^{n} u_i^2}= \sqrt{\sum_{i=1}^{n} \vert u_i \vert^2}
\end{equation}
\end{definition}

Dass die Euklidische Norm die Eigenschaften (1)-(3) erfüllt, soll in den Aufgaben nachgerechnet werden. Normen stellen ein wichtiges Werkzeug dar, denn mit ihnen können Längen von Vektoren gemessen werden. Einen Vektorraum, in dem ein Skalarprodukt definiert ist, nennt man einen \textsl{Skalarproduktraum}.\index{Skalarproduktraum}\index{Raum mit Skalarprodukt}

\begin{definition}
Eine Bilinearform $d : V\times V \longrightarrow \mathbb{R}_+$ heißt \textsl{Metrik}\index{Metrik}, falls folgende Eigenschaften erfüllt sind:
\begin{description}
\item[(1)] Aus $d(u,v) = 0$ folgt $u=v$
\item[(2)] $d(u,v) = d(v,u)$, diese Eigenschaft wird auch als Symmetrie bezeichnet.
\item[(3)] $d(u,w) \le d(u,v) + d(u,w)$ (Dreiecksungleichung)
\end{description}
\end{definition}

Eine Metrik wird dazu benutzt einen Abstand zwischen zwei Vektoren zu messen. Ein Vektorraum, in dem eine Metrik definiert ist, nennt man \textsl{metrischen Raum}\index{Raum, metrisch}. 

\begin{definition}
Jeder normierte Raum kann zu einem metrischen Raum gemacht werden, indem eine Metrik auf Basis der Norm definiert wird:
\[
d(u,v) = \Vert u-v \Vert
\]
\end{definition}

\begin{definition}
In Anlehnung an Formel (\ref{eq:2norm}) kann allgemein eine \textsl{p-Norm} definiert werden:
\[
\Vert u \Vert_p = \left( \sum_{i=1}^{n} \vert u_i \vert^p \right)^{\frac{1}{p}}
\]
für $p\in \mathbb{N}$. Nach dieser Definition ist die euklidische Norm eine 2-Norm. 
\end{definition}

Für $p \longrightarrow \infty$ nähert sich die Norm immer weiter an den größten Wert der $\vert u_i \vert$ an, da dieser mit dem immer größer werdenden Exponenten die Summe immer weiter beherrscht. Es gilt also für $p=\infty$
\[
\Vert u \Vert_\infty = \max_{i=1,\dots, n}  \vert u_i\vert
\]
(Der Beweis, dass die $\infty -Norm$ gleich dem absoluten Maximum ist, wird nachgeliefert. Dafür ist eine Grenzwertbetrachtung notwendig, die bisher noch nicht erklärt wurde).

\subsection{Dualraum}

\begin{definition}
Die Menge aller Linearformen eines K-Vektorraums werden \textsl{Dualraum} von V genannt. Der Dualraum wird meist mit einem Stern gekennzeichnet
\[V^* \]
\end{definition}

Aufgrund der Eigenschaften der Linearformen bilden diese einen eigenen Vektorraum. 

\subsection{Skalarproduktraum}

\begin{definition}
Ein \textsl{Skalarproduktraum} ist ein K-Vektorraum $(V,+,\cdot)$ zusammen mit einem Skalarprodukt $\langle .,. \rangle$.
\end{definition}

Jeder Skalarproduktraum wird mit 
\[ \Vert u \Vert = \sqrt{\langle u,u \rangle} \]
zu einem normierten Raum \index{Raum, normiert}. Sowie mit 
\[ d(u,v) = \Vert u-v \Vert \]
zu einem metrischen Raum. \index{Raum, metrisch} In diesem Sinne ist die Definition eines Skalarproduktes die erste Wahl zur Erweiterung eines Vektorraums. 



\section{Tensor}

TODO

\section{Aufgaben}

\begin{prob}
\label{matrix.1}

Weise nach, dass der $\mathbb{R}^{m\times n}$ ein Vektorraum ist. 

\end{prob}





\chapter{Exponentialrechnung und Polynome}


\section{Exponentialrechnung}

\subsection{Binomische Formeln}

Die Binomischen Formeln sind elementare Umformungen zweier Summanden zum Quadrat. Dabei gibt es die folgenden Varianten:

\begin{eqnarray}
(a+b)^2 &=& a^2 +2ab +b^2 \label{eq:binom1} \\
(a-b)^2 &=& a^2 -2ab +b^2 \label{eq:binom2} \\
(a+b)(a-b) &=& a^2 -b^2 \label{eq:binom3}
\end{eqnarray}

Interessant ist es, die Summe $(a+b)$ mit natürlichen Exponenten zu untersuchen:

\begin{eqnarray*}
(a+b)^0 &=& 1 \\
(a+b)^1 &=& a+b \\
(a+b)^2 &=& a^2 +2ab +b^2 \\
(a+b)^3 &=& a^3 + 3a^2b + 3ab^2 +b^3 \\
(a+b)^4 &=& a^4 + 4a^3b + 6a^2b^2 +4ab^3 + b^4
\end{eqnarray*}


\begin{definition}
Schreiben wir die Faktoren der einzelnen Summanden in einer dreieckigen Struktur auf, wobei $n$ den Exponent der Summe darstellt:

\begin{center}
\begin{tabular}{rccccccccc} 
$n=0$:& & & & & 1\\
\noalign{\smallskip} $n=1$:& & & & 1 & & 1\\
\noalign{\smallskip} $n=2$:& & & 1 & & 2 & & 1\\
\noalign{\smallskip} $n=3$:& & 1 & & 3 & & 3 & & 1\\
\noalign{\smallskip} $n=4$:& 1 & & 4 & & 6 & & 4 & & 1
\end{tabular}
\end{center}
Wie wir sehen, besteht jede Zeile (abgesehen von der ersten) dieser Tabelle aus zwei Einsen an den Rändern und die Werte dazwischen berechnen sich aus der Summe der beiden darüber befindlichen Zahlen. Also $2=1+1$, $3=1+2$, $6=3+3$, usw.
\end{definition}

Dieses Dreieck kann für beliebige $n\in \mathbb{N}_0$ berechnet werden und stellt immer die Faktoren der Binomischen Formel $(a+b)^n$ dar. Es wird als Pascal'sches Dreieck\index{Dreieck, Pascal}, Tartaglia-Dreieck\index{Dreieck, Tartaglia}, Yang-Hui-Dreieck\index{Dreieck, Yang-Hui}, oder auch Chayy\={a}m-Dreieck\index{Dreieck, Chayy\={a}m} genannt. In Deutschland ist die Bezeichnung nach Pascal\index{Pascal, Blaise}\footnote{\textbf{Blaise Pascal}, französischen Mathematiker *19. Juni 1623 in Clermont-Ferrand, \ding{61}19. August 1662 in Paris.} üblich.

\begin{definition}
Sei der \textsl{Binomialkoeffizient} definiert als 
\[
	\binom{n}{k} = \frac{n!}{k!\cdot (n-k)!}
\]
\end{definition}
mit $n! = 1\cdot 2\cdot ... \cdot n$ der \textsl{Fakultät}. Dann kann die allgemeine Binomische Formel geschrieben werden als:

\begin{equation}
(a+b)^n = \sum_{k=0}^{n} \binom{n}{k} \cdot a^{n-k}\cdot b^k
\end{equation}

\begin{claim}
Dass ein Koeffizient im Pascal'schen Dreieck durch die Summe der darüber befindlichen Koeffizienten gebildet wird, ist konsistent mit der Definition des Binomialkoeffizienten und es gilt:
\[
\binom{n+1}{k+1} = \binom{n}{k} + \binom{n}{k+1}
\]
\end{claim}

\begin{proof}
\begin{eqnarray*}
\binom{n+1}{k+1} &=& \frac{(n+1)!}{(k+1)!(n-k)!} = \frac{n!(n+1)}{(k+1)!(n-k)!} = \frac{n!(n+1+k-k)}{(k+1)!(n-k)!} \\
&=& \frac{n!(k+1) +n!(n-k)}{(k+1)!(n-k)!} = \frac{n!(k+1)}{(k+1)!(n-k)!}+\frac{n!(n-k)}{(k+1)!(n-k)!} \\
&=& \frac{n!}{k!(n-k)!} + \frac{n!}{(k+1)!(n-k-1)!} = \binom{n}{k}+\binom{n}{k+1}
\end{eqnarray*}
\qed
\end{proof}

\HandRight \qquad Im letzten Term der ersten Zeile wurde eine null der Form $0=+k-k$ hinzugefügt. Durch diesen Trick ist es möglich, die notwendigen Umformungen zu machen. Geschicktes hinzu addieren und gleichzeitiges subtrahieren ist ein vielseitig einsetzbares Mittel um notwendige Umformungen zu realisieren und sollte im Repertoire keines Mathematikers fehlen. 

\bigskip

Mit Hilfe der Binomialkoeffizienten ist die Darstellung des Pascal'schen Dreiecks auch auf folgende Weise möglich:

\begin{center}
\begin{tabular}{rccccccccc} 
$n=0$:& & & & & $\binom{0}{0}$\\
\noalign{\smallskip} $n=1$:& & & & $\binom{1}{0}$ & & $\binom{1}{1}$\\
\noalign{\smallskip} $n=2$:& & & $\binom{2}{0}$ & & $\binom{2}{1}$ & & $\binom{2}{2}$\\
\noalign{\smallskip} $n=3$:& & $\binom{3}{0}$ & & $\binom{3}{1}$ & & $\binom{3}{2}$ & & $\binom{3}{3}$\\
\noalign{\smallskip} $n=4$:& $\binom{4}{0}$ & & $\binom{4}{1}$ & & $\binom{4}{2}$ & & $\binom{4}{3}$ & & $\binom{4}{4}$
\end{tabular}
\end{center}


\section{Polynome}\label{chap:poly}

\begin{definition}
Eine reelle oder komplexe Funktion
\[ m_n(x) = a\cdot x^n  \]
wird als \textsl{Monom}\index{Monom} bezeichnet. Die Summe verschiedener Monome als \textsl{Polynom}\index{Polynom}:
\[
p_n(x) = \sum_{i=0}^{n} a_i \cdot x^i
\]
Beachte, dass $x^0 = 1$ ist, somit gibt es in einem Polynom einen konstanten Wert $a_0$. Sei $a_n\ne 0$, dann wird $a_n$ als \textsl{Leitkoeffizient}\index{Leitkoeffizient} bezeichnet. Als \textsl{Rang}\index{Rang, Polynom} eines Polynoms wird der höchste Exponent $q$ bezeichnet, dessen zugehöriger Monom-Faktor $a_q \ne 0$ ist.

Ist $a_n = 1$, so wird das Polynom als \textsl{normiert}\index{normiert} oder manchmal auch \textsl{monisch}\index{monisch} bezeichnet.
\end{definition}



% TODO: decide where to put the proof chapter!

\chapter{Beweisführung}

Dieses Kapitel ist nicht speziell für die Lineare Algebra gedacht, aber da wir im Folgenden die ersten ernsthaften Beweise führen werden, sollte hier auf die Grundlagen der Beweisführung in der Mathematik eingegangen werden.

\section{Aussage-Typen}

Die folgenden Definitionen von Begriffen sind -- soweit es im Wissensbereich des Autors liegt -- nicht im strengen Sinne definiert worden. Daher werden sie meist so eingesetzt, wie der Autor eines Textes sie interpretiert. Jedoch hat sich ein allgemeines Verständnis für diese Begriffe gebildet, das hier in kurzer Übersichtsform dargestellt werden soll.

\subsection{Axiom}

Ein \emph{Axiom} ist eine Grundlage für eine Theorie. Innerhalb dieser Theorie wird das Axiom als richtig vorausgesetzt und bedarf keines Beweises. Ein Axiom kann also als eine Art Rand- oder Anfangsbedingung für eine Argumentationskette interpretiert werden.

\subsection{Definition}

Eine \emph{Definition} ist eine Aussage, die einen bestimmten Sachverhalt darlegt. Eine Definition kann im allgemeinen nicht aus anderen Aussagen abgeleitet werden. Sie führt die Beschreibung des Sachverhalts auch meist durch Einführung eines Namens zu einem feststehenden Begriff zusammen. Durch die Definition bekommt also der Begriff seine Bedeutung.

\subsection{Theorem oder Satz}

Als \emph{Theorem} oder \emph{Satz} wird eine Aussage verstanden, deren Inhalt logisch aus den Axiomen einer Theorie abgeleitet werden kann. Diese Form von Aussagen wird am häufigsten da eingesetzt, wo die Fundamente einer Theorie erarbeitet werden. Mithilfe der Theoreme wird aus einer Idee eine mathematische Realität. Die Beweise von Theoremen innerhalb einer Theorie stellt den Grundstein der Arbeit eines Mathematikers dar. In den Beweisen findet die Mathematik statt.

\subsection{Korollar}

Ein \emph{Korollar} bezeichnet eine Aussage, die sich aus einem bewiesenen Theorem, dem Beweis des Theorems oder aus einer Definition ohne Aufwand ergibt. Im allgemeinen müssen die Aussagen eines Korollars nicht bewiesen werden, weil der Beweis trivial wäre oder so eng mit dem Beweis des Theorems verbunden ist, dass die Beweisführung redundant wäre. In seltenen Einzelfällen kann es notwendig sein, die Aussage eines Korollars zu beweisen. Sollte dies notwendig sein, sollte der Mathematiker sich aber überlegen, ob er aus dem Korollar nicht einen Hilfssatz macht.

\subsection{Lemma oder Hilfssatz}

Als \emph{Lemma} oder \emph{Hilfssatz} wird eine zu beweisende Aussage bezeichnet, deren Inhalt für den Beweis eines Theorems notwendig ist. Daraus folgt, dass ein Lemma bewiesen werden muss und im allgemeinen im Kontext einer größeren Beweisführung auftaucht. 

Die Unterscheidung, ob eine Aussage ein Lemma oder ein Theorem darstellt, ist oft nicht leicht zu treffen. Es wird passieren, dass eine Aussage in dem einen Buch über Mathematik als Theorem, in einem anderen "`nur"' als Lemma dargestellt wird -- als Beispiel sei hier der Fundamentalsatz der Algebra erwähnt, der eigentlich ein Satz der Analysis ist und dort nur eine eher geringe Wertschätzung erfährt. 

Dies sollte für den Lernenden nicht entscheidend sein. Der Inhalt von Theorem und Lemma ist in aller Regel interessant und der Beweis von beiden sollte verstanden und nachvollziehbar sein. 

\section{Beweisarten}

Es gibt einige Standard Schemata, mit deren Hilfe Beweise zu führen sind. Einige der prominentesten Vertreter wollen wir hier kennen lernen. Zunächst aber noch eine kurze Definition:
\begin{definition}
Jede Beweisführung in der Mathematik wird durch die lateinischen Worte "`quod erat demonstrandum"' (lat. für "`was zu beweisen war"') abgeschlossen. Es ist üblich, dafür entweder die Abkürzung "`q.e.d."' zu verwenden, oder ein kleines, rechtsbündiges Quadrat, wie am Ende dieser Definition.\qed
\end{definition}

\subsection{Vollständige Induktion}

Die \emph{Vollständige Induktion} bezieht sich auf Aussagen über natürliche Zahlen. Die Induktion vollzieht sich in zwei Schritten. Man versucht zunächst die Aussage für eine natürliche Zahl -- wenn möglich eine kleine -- zu beweisen. Im zweiten Schritt versucht man die Aussage für die natürliche Zahl $n+1$ zu beweisen, indem man voraussetzt, dass die Aussage für $n$ richtig ist. Gelingt dies, ist der volle Beweis erbracht, denn man hat den Anfang der Kette ja im ersten Schritt nachgewiesen und für jede weitere natürliche Zahl folgt die Richtigkeit aus der Richtigkeit der um eins kleineren Zahl.

Ein Beispiel soll dies verdeutlichen: Carl Friedrich Gauss soll als neunjähriger die folgende Formel hergeleitet haben:

\begin{theorem}
Summenformel von Carl Friedrich Gauss
\begin{claim}
Für alle $n\in \mathbb{N}$ gilt:
\[ 1+2+3+ \dots +n = \sum_{k=1}^{n}k = {n(n+1)\over 2} \]
\end{claim}
\begin{proof}
Für $n=2$ ist

\[ {n(n+1)\over 2} = {2(2+1)\over 2} = {6\over 2} = 3 \]

Sei die Behauptung für $n$ bewiesen, so prüfe für $n+1$:
\begin{eqnarray*}
\sum_{k=1}^{n+1} k &=& \sum_{k=1}^{n} k  + (n+1) = {n(n+1)\over 2} +(n+1) \\
 &=& {n(n+1)\over 2} +{2(n+1)\over 2} = {n(n+1)+2(n+1) \over 2} \\
 &=& {(n+1)(n+2) \over 2}
\end{eqnarray*}
\end{proof}

\end{theorem}


\subsection{Beweis durch Widerspruch}
\begin{TODO}

\end{TODO}


\subsection{Dirichlet'sches Schubfach Prinzip}
\begin{TODO}
Beispiel mit den Haaren auf dem Kopf. 
\end{TODO}




\chapter{Eigenwerte}

Kommen wir auf die reellen $n\times n$ Matrizen zurück, wobei $n<\infty$ ist. Sie stellen lineare Abbildungen vom $\mathbb{R}^n$ in sich selbst dar. 
\begin{definition}
Die Menge aller invertierbaren $n\times n$-Matrizen bildet, zusammen mit der Matrixmultiplikation als Verknüpfung, eine Gruppe. Sie wird $\mathfrak{gl}_n(\mathbb{R})$ genannt. Die Abkürzung $\mathfrak{gl}$ kommt von der "`general linear group"', der allgemeinen linearen Gruppe, auch oft $GL(n,\mathbb{R})$ genannt. Dass diese Matrizen eine Gruppe bezüglich der Matrixmultiplikation bilden, wird in den Aufgaben nachgewiesen.
\end{definition}


\begin{definition}
Es seien $A\in \mathfrak{gl}_n(\mathbb{R})$ und $v\in \mathbb{R}^n$ ein Vektor. Falls weiter gilt
\[
Av = \lambda v
\]
mit $\lambda\in \mathbb{R}$, so heißt $v$ ein \textsl{Eigenvektor} von $A$. Sowie $\lambda$ ein \textsl{Eigenwert} zum Eigenvektor $v$.
\end{definition}

Diese Definition mag überraschend aussehen, denn schließlich ist $A$ ein vermeintlich "`kompliziertes"' und hochdimensionales Objekt, während auf der rechten Seite nur die Multiplikation einer Zahl mit einem Vektor steht, also eine Streckung des Vektors. Doch letztlich ist $A$ eine lineare Abbildung. Das bedeutet, $A$ bildet Vektoren im $\mathbb{R}^n$ wieder auf Vektoren ab. Dies kann im Grunde nur durch Drehungen, Verzerrungen und Spiegelungen geschehen. Erst nicht-lineare Abbildungen haben die Möglichkeit komplexeres Verhalten an den Tag zu legen. 

\begin{definition}
Seien $A,v,\lambda$ wie oben, des Weiteren sei $x\in \mathbb{R}$ ein Parameter und $I_n$ das neutrale Element der Multiplikation des $\mathfrak{gl}_n(\mathbb{R})$. Dann ist 
\[
p(x) = \det(A-x\cdot I_n) : \mathbb{R} \longrightarrow \mathbb{R}
\]
ein normiertes Polynom vom Grade $n$. Es wird \textsl{charakteristisches Polynom} genannt.
\end{definition}

\begin{theorem}
Die Nullstellen des charakteristischen Polynoms zur Matrix $A\in \mathfrak{gl}_n(\mathbb{R})$ sind die Eigenwerte von $A$.
\end{theorem}
\begin{proof}
TODO
\qed
\end{proof}

\section{Aufgaben}
TODO

\chapter{Numerische Lösungsverfahren}

\section{Gauß Elemination}

\section{QR Zerlegung}

\section{Iterative Lösungsmethoden}







\part{Geometrie}



\part{Analysis}


\chapter{Funktion}

Der Begriff einer Abbildung ist uns schon öfter untergekommen. Eine Funktion ist eine Abbildung, die zwei Mengen (hier $X$ und $Y$) in Beziehung setzt:

\[ f: X \longrightarrow Y \]
$f$ ordnet jedem Element $x\in X$ ein Element $y\in Y$ zu, indem
\[f(x) =y\]
gilt. Die Umkehrung gibt es allgemein nicht, aber wenn sie existiert, so wird sie als $f^{-1}$ bezeichnet und es gilt:
\[ f^{-1}(y) =x\]




\chapter{Grundlagen}

Die Algebra konzentriert sich -- im Gegensatz zur Arithmetik -- auf die Verallgemeinerung der Begriffe zur Analyse bestimmter Sachverhalte. Der Arithmetik Teil dieses Buches begann mit Berechnungen auf Basis von Äpfeln. Diese geben einen unmittelbaren Zugang zu den Begriffen des Rechnens. In der (Linearen-) Algebra wird es solches nicht geben. Die Ansätze hier sind rein abstrakt zu verstehen. Auch wenn direkte, anschauliche Beispiele zu bestimmten algebraischen Sachverhalten existieren, so sollte der Lernende versuchen, nicht anhand dieser sein Verständnis auszubilden, sondern rein in der Sache selbst. Das führt letztlich dazu, dass er ohne konkrete Anschauung Sachverhalte analysieren und Probleme lösen kann. Vollständig unabhängig davon, ob diese Probleme einen realen, abstrakten oder mit dem Verstand nicht nachvollziehbaren Hintergrund haben. 

So gilt z.B. der Satz des Pythagoras \index{Pythagoras}
\[\lVert a\rVert ^2 +\lVert b\rVert ^2 = \lVert c\rVert^2\]
in jedem Vektorraum der ein Skalarprodukt $\langle .,.\rangle $ besitzt, mit einer induzierten Norm $\lVert a \rVert = \sqrt{\langle a,a\rangle} $, sowie $a+b=c$ ($a,b,c$ ein Dreieck bilden) gilt und $\langle a,b\rangle = 0 $ (der Winkel zwischen $a,b$ ein rechter Winkel ist, also 90 Grad). Unabhängig davon, ob dieser ein euklidischer Raum ist, ein Raum der Polynome oder der Raum der unendlich oft differenzierbaren Funktionen.

Die folgenden Abschnitte sind oft Zusammenstellungen von Definitionen. Leider sind diese notwendiges Handwerkszeug für alles was danach kommt. Die Definitionen scheinen manchmal langweilig und unmotiviert. Ich hoffe, die Lernenden nicht dadurch abzuschrecken!

\section{Zeichen}

In der Algebra verwendet man anstelle von Zahlen im allgemeinen Buchstaben. Es immer vom Kontext abhängig, welcher Buchstabe was bedeutet. Aber es haben sich bestimmte Dinge eingebürgert. So sind Konstanten meistens mit den Buchstaben
\[a, b, c, \dots \]
bezeichnet. Unbekannte in Gleichungen meist mit 
\[x, y, z, p, q, \dots \]
Sowie Indizes mit 
\[i, j, k, \dots \]
Es können aber auch griechische Buchstaben auftauchen. Hier eine Übersicht

\bigskip

\begin{center}
\begin{tabular}{c|c|l}
\hline
\textbf{Kleiner Buchstabe} & \textbf{Großbuchstabe} & \textbf{Bezeichnung} \\
\hline
$\alpha $ & $A $ & Alpha \\
$\beta $ & $B $ & Beta \\
$\gamma $ & $\Gamma $ & Gamma \\
$\delta $ & $\Delta $ & Delta \\
$\epsilon $ & $E $ & Epsilon \\
$\zeta $ & $Z $ & Zeta \\
$\eta $ & $H $ &  Eta\\
$\theta $ & $\Theta $ & Theta \\
$\iota $ & $I $ & Iota \\
$\kappa $ & $K $ & Kappa \\
$\lambda $ & $\Lambda $ & Lambda \\
$\mu $ & $M $ & Mu \\
$\nu $ & $N $ & Nu \\
$\xi $ & $\Xi $ &  Xi \\
$\omicron $ & $O $ & Omicron \\
$\pi $ & $\Pi $ & Pi \\
$\rho $ & $P $ & Rho \\
$\sigma $ & $\Sigma $ & Sigma  \\
$\tau $ & $T $ & Tau \\
$\upsilon $ & $\Upsilon $ & Ypsilon \\
$\phi $ & $\Phi $ & Phi \\
$\chi $ & $X $ & Chi \\
$\psi $ & $\Psi $ & Psi \\
$\omega $ & $\Omega $ & Omega \\
\hline
\end{tabular}
\end{center}


\section{Summen und Produkte}

In den folgenden Kapiteln werden wir öfter auf Summen- und Produktreihen von Funktionen treffen. Um diese möglichst einfach und gut verständlich darzustellen hat man sich in der Mathematik auf die Summen- und Produktnotation geeinigt. Summen werden mit einem griechischen, großen "`Sigma"' dargestellt $\Sigma$ \index{Summe $\Sigma$} und Produkte durch ein großes "`Pi"' $\Pi$. \index{Produkt $\Pi$}

Will man z.B. die Zahlen 1 bis 10 summieren, so kann man dies so darstellen:
\[ \text{Summe } = \sum_{n=1}^{10} n = 1+2+3+4+5+6+7+8+9+10 \]
Durch die Grenzen oberhalb und unterhalb weiß man, welche Werte $n$ annimmt. Alternativ gibt es auch folgende Möglichkeit. Sei $N = \{ 1,2, \dots ,10 \}$, dann wird die Summe so dargestellt:
\[ \text{Summe} = \sum_{n\in N} n \]
Beide Möglichkeiten werden verwendet. 
Für die Multiplikation gilt das selbe:
\[ \text{Produkt } = \prod_{n=1}^{10} n = 1\cdot 2\cdot 3\cdot 4\cdot 5\cdot 6\cdot 7\cdot 8\cdot 9\cdot 10 \]
Sowie
\[ \text{Produkt} = \prod_{n\in N} n  \]
Es werden auch andere Operationen, wie z.B. die Vereinigung oder der Schnitt von Mengen auf diese Weise dargestellt. 
\begin{definition}
Die Schreibweise ein großes Symbol zu verwenden mit darunter, wie darüber befindlichen Einschränkungen wird \emph{Operatorschreibweise} genannt und folgt dem Schema:
 \[ \text{Wert } = \bigbox_{\dots}^{\dots}\left( \dots \right) \]
\end{definition}
wobei $\bigbox$ jeder beliebige Operator sein kann und die Punkte darüber und darunter die Grenzen angeben, über die eben dieser Operator angewendet wird. Und schließlich das Argument in Klammern stellt ebenjenes dar, auf das der Operator angewandt wird. 

\section{Aussagelogik}

\subsection{Quantor}

Im Folgenden werden wir zwei sogenannte \emph{Quantoren} \index{Quantor} kennen lernen. Sie sind wichtig um Aussagen unzweideutig und in kurzer Art zu formulieren. Der erste ist $\forall$ und bedeutet "`für alle"'\index{Für alle} oder "`für jedes"'. Man benutzt ihn zum Beispiel, um Aussagen "`für alle"' Elemente einer Menge zu definieren.

\[ \forall x\in \mathbb{N} \]
bedeutet für alle $x$ in den natürlichen Zahlen.

Der zweite ist $\exists$, dieser bedeutet "`es gibt"' \index{Es gibt, es existiert} oder "`gibt es"'. So wäre 
\[ \forall x\in \mathbb{N}\ \exists y\in \mathbb{Q},\text{ mit }  y={x\over 1} \]
gleichbedeutend mit dem Satz: "`\textbf{Für alle} $x$ in den natürlichen Zahlen \textbf{gibt es} ein $y$ in den rationalen Zahlen mit $y={x \over 1}$"'.

Der $\exists$ Quantor kann mit einem Ausrufezeichen (!) verbunden werden um auszudrücken, dass es "`genau eins gibt"', also bedeutet
\[ \forall x\in \mathbb{N}\ \exists! y\in \mathbb{Q},\text{ mit }  y={x\over 1} \]
folgendes: "`\textbf{Für alle} $x$ in den natürlichen Zahlen \textbf{gibt es genau ein} $y$ in den rationalen Zahlen mit $y={x \over 1}$"'.

\subsection{Junktor}

Junktoren verknüpfen Aussagen zu einer Gesamtaussage.\index{Junktor}

\begin{enumerate}
\item $\wedge $ ist der "`und"' Junktor. Z.B. seien $a\in \mathbb{N} \wedge b\in \mathbb{Q} $
\item $\vee$ ist der "`oder"' Junktor. Z.B. sei $x\in \mathbb{R} \vee x\in \mathbb{Q}$.
\item $\Rightarrow$ ist der Implikations-Junktor. Z.B. $A\Rightarrow B$ bedeutet, aus $A$ folgt $B$ oder $x\in \mathbb{N} \Rightarrow -x \in -\mathbb{N}$
\item $\Leftrightarrow$ ist der Äquivalenz-Junktor. $A\Leftrightarrow B$ bedeutet $A$ gilt genau dann, wenn $B$ gilt.
\item $\neg$ negiert eine Aussage, z.B. bedeutet $\neg(x\in \mathbb{R})$, dass $x$ eben nicht aus den reellen Zahlen ist, oder $\neg \exists y\in \mathbb{Q}$ bedeutet, dass es kein $y$ in den rationalen Zahlen gibt.
\end{enumerate}


\section{Mengen}

\index{Menge}
Kategorisierung spielt in der (Linearen) Algebra eine sehr wichtige Rolle. Kennt man die Eigenschaften eines Dinges, so kann man damit umgehen. Daher ist es nicht verwunderlich, dass Mathematiker versuchen, Dinge, die identische Eigenschaften besitzen zu Mengen zusammenzustellen. Und -- soweit möglich -- Operationen und Eigenschaften mit dem Namen dieser Zusammenstellung zu verbinden. Die bereits im Arithmetik Teil vorgekommene Menge der natürlichen Zahlen $\mathbb{N}$ bildet da ein gutes Beispiel für den Einstieg. 

Im Folgenden werden wir immer mehr Vorgehensweisen kennenlernen, mit denen man Mengen untersuchen kann. Kann man bestimmte Eigenschaften an einer Menge feststellen, so wird sie zur Gruppe. Hat man weitere Eigenschaften an der Gruppe, so wird sie zum Ring. Und eine letzte Eigenschaft macht einen Ring zu einem Körper. Der Körper ist das, was unserer üblichen Vorstellung von Zahlen am nächsten kommt. Für viele Objekte der Mathematik kann man aber z.B. nur Gruppenstruktur nachweisen. 

\begin{quote}
HINWEIS: Der hier verwendete Begriff "`Menge"' hat nichts mit der Anzahl zu tun, wie er im Arithmetik Teil verwendet wurde. Hier bezeichnet "`Menge"' eine Ansammlung von gleichartigen Objekten.
\end{quote}

Mengen werden in der Mathematik in geschweiften Klammern $\{\dots \}$ dargestellt. In diesen Klammern stehen entweder die Elemente der Menge in Form einer Liste oder Aufzählung (diese kann unvollständig sein, wie wir dies gleich verwenden um die natürlichen Zahlen zu beschreiben), oder in Form einer Regel. Solche Regeln werden in Form von Aussage-Prädikaten formuliert, also z.B. $\forall$ entspricht "`für alle"' und $\exists$ entspricht "`existiert"', aber dies werden wir noch ausführlicher kennen lernen, wenn wir solche Konstrukte einsetzen. 

Die natürlichen Zahlen werden wie folgt definiert:
\[\mathbb{N} := \{ 1,2,3,4,5, \dots \} \]
Die Punkte "`$\dots$"' symbolisieren dabei, dass die Reihe nicht enden soll. Dementsprechend gehört $\infty$ "`Unendlich"' zu den natürlichen Zahlen.
\[\mathbb{N} := \{ 1,2,3,4,5, \dots, \infty \} \]

Zusammengefasst sind Mengen einfache Ansammlungen von im allgemeinen gleichartigen Dingen. Die folgende Zeichenfolge bedeutet, dass $x$ ein Element der Menge $M$ ist:
\[x \in M \]
So ist zum Beispiel
\[7 \in \mathbb{N} \]

\subsection{Definitionssprint!}
Im Folgenden seien $A$ und $B$ immer Mengen.

\begin{definition}\index{Teilmenge}
Wenn für alle $a\in A$ gilt $a\in B$, dann ist $A$ eine \emph{Teilmenge} von $B$. Dies wird so dargestellt:
\[
A \subset B
\]
\end{definition}

\begin{definition}\index{Leere Menge}
Eine leere Menge wird mit $\emptyset$ oder $\{\}$ dargestellt.
\end{definition}

\begin{definition}\index{Vereinigungsmenge}
Die Vereinigung von Mengen (auch \emph{Vereinigungsmenge} genannt) wird definiert als
\[A \cup B = \{ x | x\in A \vee x\in B \}\]
\end{definition}

\begin{definition}\index{Schnittmenge}
Die \emph{Schnittmenge} wird definiert als 
\[A \cap B = \{x| x\in A \wedge x\in B \} \]
\end{definition}

\begin{definition}
Subtraktion von Mengen
\[ A \backslash B = \{ x \vert x\in A, \  \neg (x\in B) \} \]
\end{definition}

\section{Abbildung}
\index{Abbildung}

Zur Darstellung einer Abbildung benötigt man im allgemeinen drei Informationen. 

\begin{enumerate}
\item Den Namen der Abbildung
\item Den Definitionsbereich
\item Den Bildbereich
\end{enumerate}

Definitions- und Bildbereich sind, sofern sie nicht weiter spezifiziert werden, Mengen. Die Abbildung wird somit als eine Relation (Beziehung) von zwei Mengen gesehen. Diese Beziehung wird durch einen Pfeil dargestellt. Sei $D$ der Definitions- und $B$ der Bildbereich, dann sieht diese Beziehung so aus:

\begin{equation}
D \longrightarrow B
\end{equation}
Um dieser Beziehung einen Namen zu geben, stellt man diesen vor diese Beziehung und trennt die Informationen durch einen Doppelpunkt. Sei zum Beispiel $a$ der Name der Abbildung, dann ist die vollständige Darstellung
\begin{equation}
a : D \longrightarrow B
\end{equation}
Eine Abbildung bildet Elemente aus einer Menge auf Elemente einer anderen Menge ab. Seien $X,Y$ Mengen. Dann ordnet die Abbildung
\[ f : X \longrightarrow Y \]
das Element $x\in X$ auf $y\in Y$ ab. Das wird so ausgedrückt:
\[ f(x) = y\]

Die Addition ist ebenfalls eine Abbildung. Sie bildet zwei Elemente einer Menge auf Elemente der selben Menge ab:
\[ + : X \times X \longrightarrow X \]
Also kann man die Summe 
\[ x+y = z\]
auch in Funktionenform schreiben:
\[ +(x,y) = z\]

\section{Gruppen, Ringe, Körper}
Im Folgenden bezeichen $X,Y$ und $Z$ immer Mengen.

\subsection{Definitionssprint!}

\begin{definition}
Eine Funktion \index{Verknüpfung}
\[ \circ : X \times X \longrightarrow X \]
wird als \emph{Verknüpfung} auf $X$ bezeichnet. Zum Beispiel sind die Operationen $+$ und $\cdot$ Verknüpfungen. 
\end{definition}

\begin{definition} \index{assoziativ}
Eine Verknüpfung wird \emph{assoziativ} genannt, wenn gilt:
\[ a \circ (b \circ c) = (a \circ b) \circ c \]
\end{definition}
\begin{definition}\index{kommutativ}
Sie wird \emph{kommutativ} genannt, wenn gilt: 
\[ a \circ b = b \circ a \]
\end{definition}
\noindent Für alle $a,b,c \in X$

\begin{definition}\index{Neutrales Element}
Ein Element $e\in X$ wird als \emph{neutrales Element} bezüglich einer Verknüpfung bezeichnet, wenn 
\[ e\circ a = a \circ e = a \]
für alle $a\in X$ gilt.
\end{definition}

\begin{claim}
Falls es in $X$ ein neutrales Element gibt, so ist es eindeutig. 
\end{claim}
\begin{proof}
Seien $e_1$ und $e_2$ neutrale Element bezüglich $\circ$, dann gilt: 
\[ e_1 = e_1\circ e_2 = e_2 \]
\end{proof}

\begin{definition}\index{Monoid}
Eine Menge $X$ mit assoziativer Verknüpfung und zugehörigem neutralem Element heißt \emph{Monoid}.
\end{definition}

\begin{definition}
Sei $X$ ein Monoid mit der Verknüpfung $\circ$ und zugehörigem neutralen Element $e$. Wenn gilt
\[ a\circ b = e \]
für $a,b \in X$, so wird $a$ das \emph{Linksinverse}\index{Linksinverse} von $b$ genannt, sowie $b$ das \emph{Rechtsinverse}\index{Rechtsinverse} von a. Gilt zudem
\[ a\circ b = b \circ a = e \]
Dann sind $a$ und $b$ jeweils ihre \emph{Inversen}. \index{Inverse}
\end{definition}

\begin{definition}\index{Gruppe}
Ist in einem Monoid jedes Element invertierbar, so nennt man dies eine \emph{Gruppe}.
\end{definition}

\begin{definition}\index{Gruppe, abelsch}
Eine Gruppe, deren Verknüpfung kommutativ ist, nennt man eine \emph{abelsche Gruppe}\footnote{Benannt nach dem norwegischen Mathematiker \textbf{Niels Henrik Abel}, * 5. August 1802 auf der Insel Finn\o y, Ryfylke, Norwegen; \ding{61} 6. April 1829 in Froland, Aust-Agder, Norwegen} .
\end{definition}

\begin{definition}\index{Ring}
Ein \emph{Ring} $R$ ist eine Menge mit zwei Verknüpfungen "`$+$"' und "`$\cdot$"'. Wobei $R$ bezüglich der Addition eine abelsche Gruppe ist und bezüglich der Multiplikation ein Monoid. Des Weiteren muss das Distributivgesetz gelten
\[ a\cdot (b+c) = a\cdot b + a\cdot c \]
für alle $a,b,c \in R$. Ist $R$ bezüglich "`$\cdot$"' kommutativ, so ist $R$ ein \emph{kommutativer Ring}.\index{Ring, kommutativer}
\end{definition}

\begin{definition}
Sei $K$ ein kommutativer Ring, dessen neutrales Element bezüglich der Addition $0$ ist. Falls alle Elemente von $K\backslash \{0\}$ invertierbar sind, heißt $K$ ein \emph{Körper}.\index{Körper}
\end{definition}

\section{Aufgaben}

TODO




\chapter{Differentialrechnung}\label{chap:diff}

Es gibt sehr viele -- in erster Linie mathematische -- Gründe, warum die Steigung einer Funktion eine interessante und wissenswerte Information darstellt. Doch sehen wir uns zuerst eine Situation an, die in unserem täglichen Leben vorkommt.

Der Tacho eines Autos ist etwas, dem wir ständig begegnen. Sei es als Autofahrer oder als Beifahrer, oder gar als Fahrradfahrer. Der Tacho ist ein Messinstrument und gibt die aktuelle Geschwindigkeit des Fahrzeugs an. Geschwindigkeit ist ein physikalischer Begriff. Physiker verwenden Mathematik zur Beschreibung ihrer Beobachtungen und zur Entwicklung ihrer Theorien. Sie haben zum Beispiel herausgefunden, dass wenn man den Weg, den ein Fahrzeug zurücklegt, durch die Zeit teilt, die es dafür brauchte, erhält man ein Maß für die Geschwindigkeit, bzw. ein Maß für die Änderung der zurückgelegten Strecke. Diese recht grobe Angabe stellt nur die Durchschnittsgeschwindigkeit dar. Aber man ist natürlich auch an der Geschwindigkeit interessiert, die das Fahrzeug zu jedem Zeitpunkt inne hat. Teilen wir die Strecke in $n$ gleich große Teile und messen die Zeit, die das Fahrzeug für jeden einzelnen dieser Abschnitte brauchte. Dann bekommen wir die Durchschnittsgeschwindigkeit auf jedem dieser Bereiche. Machen wir nun die Bereiche, und damit auch die Zeitabschnitte, immer kleiner ($n \rightarrow \infty$), dann nähert sich der Wert, den wir in den einzelnen Bereichen messen, immer mehr der zu diesem Zeitpunkt gefahrenen Geschwindigkeit. 

Daraus folgt ein uns völlig bekannter Zusammenhang: Nämlich, dass der Fahrer über die Geschwindigkeit, direkte Kontrolle über die zurückgelegten Strecke besitzt. Fährt er schneller, so legt er in gleicher Zeit mehr Strecke zurück. 

Eine ähnlich bekannte Situation betrifft die Beschleunigung. Werden wir in die Sitze des Autos gedrückt, so gibt der Fahrer Gas und das Fahrzeug fährt schneller. Werden wir in den Gurt gedrückt, bremst der Fahrer und das Fahrzeug wird langsamer. Die Beschleunigung ist also ein Maß für die Änderung der Geschwindigkeit -- sprich, die Beschleunigung ist die erste Ableitung der Geschwindigkeit, genauso wie die zweite Ableitung der zurückgelegten Strecke.


\section{Überlegungen}


Die von uns im Folgenden betrachteten Funktionen müssen bestimmten Voraussetzungen genügen, damit sie für das Differenzieren und Integrieren in Betracht kommen. Sie müssen auf eine gewisse Weise "`harmlos"' sein. Die folgenden Definitionen sind als "`Harmlosigkeits"'-Eigenschaften zu verstehen.

\begin{definition}
Eine Funktion $f : A \subseteq \mathbb{R} \longrightarrow \mathbb{R}$ heißt \textsl{dehnungsbeschränkt}, wenn es zu beliebigen Werten $x,y\in A$ eine Konstante $K$ gibt, so dass
\[
\left\vert f(x)-f(y)  \right\vert \le K\cdot \vert x-y \vert
\]
oder auch in dieser Form dargestellt:
\[
\left\vert \frac{f(x)-f(y)}{x-y}  \right\vert \le K
\]
\end{definition}




\chapter{Integralrechnung}




\chapter{Grundlagen}

Die Algebra konzentriert sich -- im Gegensatz zur Arithmetik -- auf die Verallgemeinerung der Begriffe der Analyse bestimmter Sachverhalte. Der Arithmetik Teil dieses Buches begann mit Berechnungen auf Basis von Äpfeln. Diese geben einen unmittelbaren Zugang zu den Begriffen des Rechnens. In der Algebra wird es solches nicht geben. Die Ansätze hier sind rein abstrakt zu verstehen. Auch wenn direkte, anschauliche Beispiele zu bestimmten algebraischen Sachverhalten existieren, so sollte der Lernende versuchen, nicht anhand dieser sein Verständnis auszubilden, sondern rein in der Sache selbst. Das führt letztlich dazu, dass er vollständig ohne konkrete Anschauung Sachverhalte analysieren und Probleme lösen kann. Vollständig unabhängig davon, ob diese Probleme einen realen, abstrakten oder mit dem Verstand nicht nachvollziehbaren Hintergrund haben. Wenn z.B. im Raum der Polynome zwei Funktionen die Bedingung erfüllen
\[f^2(x) + g^2(x) = 1 \]
so stellt dies genauso den Satz des Pytagoras dar, als wenn ein Dreieck betrachtet würde. 
\[a^2 +b^2 = c^2\]

\section{Zeichen}

In der Algebra verwendet man anstelle von Zahlen im allgemeinen Buchstaben. Es immer vom Kontext abhängig, welcher Buchstabe was bedeutet. Aber es haben sich bestimmte Dinge eingebürgert. So sind Konstanten meistens mit den Buchstaben
\[a, b, c, \dots \]
bezeichnet. Unbekannte in Gleichungen meist mit 
\[x, y, z, p, q, \dots \]
Sowie Indizes mit 
\[i, j, k, \dots \]
Es können aber auch griechische Buchstaben auftauchen. Hier eine Übersicht

\bigskip

\begin{tabular}{c|c|l}
\hline
\textbf{Kleiner Buchstabe} & \textbf{Großbuchstabe} & \textbf{Bezeichnung} \\
\hline
$\alpha $ & $A $ & Alpha \\
$\beta $ & $B $ & Beta \\
$\gamma $ & $\Gamma $ & Gamma \\
$\delta $ & $\Delta $ & Delta \\
$\epsilon $ & $E $ & Epsilon \\
$\zeta $ & $Z $ & Zeta \\
$\eta $ & $H $ &  Eta\\
$\theta $ & $\Theta $ & Theta \\
$\iota $ & $I $ & Iota \\
$\kappa $ & $K $ & Kappa \\
$\lambda $ & $\Lambda $ & Lambda \\
$\mu $ & $M $ & Mu \\
$\nu $ & $N $ & Nu \\
$\xi $ & $\Xi $ &  Xi \\
$\omicron $ & $O $ & Omicron \\
$\pi $ & $\Pi $ & Pi \\
$\rho $ & $P $ & Rho \\
$\sigma $ & $\Sigma $ & Sigma  \\
$\tau $ & $T $ & Tau \\
$\upsilon $ & $\Upsilon $ & Ypsilon \\
$\phi $ & $\Phi $ & Phi \\
$\chi $ & $X $ & Chi \\
$\psi $ & $\Psi $ & Psi \\
$\omega $ & $\Omega $ & Omega \\
\hline
\end{tabular}

\section{Gesetze}

Da diese im Arithmetikteil nicht abstrakt definiert wurden, seien sie hier noch einmal wiederholt:

\bigskip

\noindent \textsl{Kommutativgesetz}
\begin{eqnarray*}
a+b &=& b+a \\
a\cdot b &=& b\cdot a
\end{eqnarray*}

\noindent \textsl{Assoziativgesetz}
\begin{eqnarray*}
a+(b+c) &=& (a+b)+c \\
a\cdot (b\cdot c) &=& (a\cdot b)\cdot c
\end{eqnarray*}

\noindent \textsl{Distributivgesetz}
\begin{eqnarray*}
a\cdot (b+c) &=& a\cdot b+ a\cdot c
\end{eqnarray*}


\chapter{Mengen, Gruppen, Ringe, Körper}

Kategorisierung spielt in der Algebra eine sehr wichtige Rolle. Kennt man die Eigenschaften eines Dinges, so kann man damit umgehen.



\appendix
\part{Anhang}


\chapter{Lösungen}


\section{Ganze Zahlen}
\begin{sol}{arith.1.1}
\begin{center}
\begin{tabular}{C{2cm}C{2cm}C{2cm}C{2cm}}
falsch & richtig & richtig & richtig \\
richtig & richtig & falsch & richtig \\
richtig & richtig & richtig & falsch
\end{tabular}
\end{center}
\end{sol}


\begin{sol}{arith.1.2}

\begin{eqnarray*}
2+3 &=& 5 \hskip 1cm | +2 \\
2+3+2 &=& 5+2 \hskip 1cm | \cdot 3 \\
(2+3+2) \cdot 3 &=& (5+2)\cdot 3 \\
2\cdot 3 + 3\cdot 3 + 2\cdot 3 &=& 5\cdot 3 +2 \cdot 3 \\
6+9+6 &=& 15 + 6 \\
21 &=& 21
\end{eqnarray*}

\end{sol}

\begin{sol}{arith.1.3}
\begin{center}
\begin{tabular}{C{2cm}C{2cm}C{2cm}C{2cm}}
$8+9=17$ & $8-9 =-1$ & $3\cdot 4=12$ & $8/2=4$ \\
$13+5=18$ & $13-5=8$ & $13\cdot 5=65$ & $27/3=9$ \\
$17+3=20$ & $17-3=14$ & $17\cdot 3=51$ & $25/5=5$
\end{tabular}
\end{center}
\end{sol}

\begin{sol}{arith.1.4}
\[-(2-3) = (-1)\cdot (2-3) = (-1)\cdot 2 + (-1)\cdot (-1) \cdot 3 = -2+3\]
Weil $(-1)\cdot (-1)=1$ und $1\cdot 3 = 3$.
\end{sol}

\section{Vektor, Matrix, Tensor}

\begin{sol}{matrix.1}

Seien $A,B,C\in \mathbb{R}^{m\times n}$ und $\alpha , \beta \in \mathbb{R}$. Der Nachweis wird ausschließlich auf den einzelnen Komponenten der Matrix geführt, da Addition und skalare Multiplikation ausschließlich auf den Komponenten der Matrix ausgeführt werden. Daraus folgt praktisch schon die Behauptung, dass $\mathbb{R}^{m\times n}$ ein Vektorraum ist. Aber hier sei der Nachweis noch vorgerechnet:

\begin{description}
\item[(A1)] $A+(B+C) = (a_{i,j}+(b_{i,j}+c_{i,j}))_{i,j} = ((a_{i,j}+b_{i,j})+c_{i,j})_{i,j} = (A+B)+C$. Hierbei wird die Assoziativität der reellen Zahlen verwendet.
\item[(A2)] $ (a_{i,j}+0)_{i,j} = (0+a_{i,j})_{i,j} = (a_{i,j})_{i,j}$
\item[(A3)] Da $\mathbb{R}$ ein Körper ist, gibt es zu jedem $a_{i,j}\in \mathbb{R}$ ein $ -a_{i,j}\in \mathbb{R}$, sodass $a_{i,j} +(-a_{i,j}) = -a_{i,j} + a_{i,j} = 0$ daraus folgt, dass $A+(-A) = -A +A = 0$.
\item[(A4)] $ A+B = (a_{i,j}+b_{i,j})_{i,j} = (b_{i,j}+a_{i,j})_{i,j} = B+A$
\item[(M1)] $ (\alpha \beta) A = ((\alpha \beta)a_{i,j})_{i,j} = (\alpha (\beta a_{i,j})_{i,j} = \alpha (\beta A) $
\item[(M2)] $ I\cdot A = ( \sum_{k=1}^{n} I_{i,k}a_{k,j})_{i,j} $, da $I_{i,k}=0$ für alle $i\ne k$, bleibt als Ergebnis der Summe nur $I_{i,i}a_{i,j} = a_{i,j} $ und damit ist $I\cdot A = A$
\item[(M3)] $ (\alpha +\beta )A = ((\alpha + \beta)a_{i,j})_{i,j} = (\alpha a_{i,j} + \beta a_{i,j})_{i,j} = \alpha A + \beta A $
\item[(M4)] $ \alpha (A+B) = (\alpha(a_{i,j} + b_{i,j})_{i,j} = (\alpha a_{i,j} + \alpha b_{i,j})_{i,j} = \alpha A + \alpha B $
\end{description}

\end{sol}


\backmatter
% bibliography, glossary and index would go here.

\begin{thebibliography}{123}

\bibitem{Brieskorn1}
  Egbert Brieskorn,
  \emph{Lineare Algebra und analytische Geometrie I}.
  Vieweg, Wiebaden; Braunschweig,
  1. Auflage, Nachdruck,
  1983/1985.

\end{thebibliography}

\printindex

\end{document}