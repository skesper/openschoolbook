\part{Lineare Algebra}


\chapter{Grundlagen}

Die Algebra konzentriert sich -- im Gegensatz zur Arithmetik -- auf die Verallgemeinerung der Begriffe zur Analyse bestimmter Sachverhalte. Der Arithmetik Teil dieses Buches begann mit Berechnungen auf Basis von Äpfeln. Diese geben einen unmittelbaren Zugang zu den Begriffen des Rechnens. In der Algebra wird es solches nicht geben. Die Ansätze hier sind rein abstrakt zu verstehen. Auch wenn direkte, anschauliche Beispiele zu bestimmten algebraischen Sachverhalten existieren, so sollte der Lernende versuchen, nicht anhand dieser sein Verständnis auszubilden, sondern rein in der Sache selbst. Das führt letztlich dazu, dass er ohne konkrete Anschauung Sachverhalte analysieren und Probleme lösen kann. Vollständig unabhängig davon, ob diese Probleme einen realen, abstrakten oder mit dem Verstand nicht nachvollziehbaren Hintergrund haben. 

So gilt z.B. der Satz des Pythagoras 
\[\lVert a\rVert ^2 +\lVert b\rVert ^2 = \lVert c\rVert^2\]
in jedem Vektorraum der ein Skalarprodukt $\langle .,.\rangle $ besitzt, mit einer induzierten Norm $\lVert a \rVert = \sqrt{\langle a,a\rangle} $, sowie $a+b=c$ gilt und $\langle a,b\rangle = 0 $. Unabhängig davon, ob dieser ein euklidischer Raum ist, ein Raum der Polynome oder der Raum der unendlich oft differenzierbaren Funktionen.


\section{Zeichen}

In der Algebra verwendet man anstelle von Zahlen im allgemeinen Buchstaben. Es immer vom Kontext abhängig, welcher Buchstabe was bedeutet. Aber es haben sich bestimmte Dinge eingebürgert. So sind Konstanten meistens mit den Buchstaben
\[a, b, c, \dots \]
bezeichnet. Unbekannte in Gleichungen meist mit 
\[x, y, z, p, q, \dots \]
Sowie Indizes mit 
\[i, j, k, \dots \]
Es können aber auch griechische Buchstaben auftauchen. Hier eine Übersicht

\bigskip

\begin{center}
\begin{tabular}{c|c|l}
\hline
\textbf{Kleiner Buchstabe} & \textbf{Großbuchstabe} & \textbf{Bezeichnung} \\
\hline
$\alpha $ & $A $ & Alpha \\
$\beta $ & $B $ & Beta \\
$\gamma $ & $\Gamma $ & Gamma \\
$\delta $ & $\Delta $ & Delta \\
$\epsilon $ & $E $ & Epsilon \\
$\zeta $ & $Z $ & Zeta \\
$\eta $ & $H $ &  Eta\\
$\theta $ & $\Theta $ & Theta \\
$\iota $ & $I $ & Iota \\
$\kappa $ & $K $ & Kappa \\
$\lambda $ & $\Lambda $ & Lambda \\
$\mu $ & $M $ & Mu \\
$\nu $ & $N $ & Nu \\
$\xi $ & $\Xi $ &  Xi \\
$\omicron $ & $O $ & Omicron \\
$\pi $ & $\Pi $ & Pi \\
$\rho $ & $P $ & Rho \\
$\sigma $ & $\Sigma $ & Sigma  \\
$\tau $ & $T $ & Tau \\
$\upsilon $ & $\Upsilon $ & Ypsilon \\
$\phi $ & $\Phi $ & Phi \\
$\chi $ & $X $ & Chi \\
$\psi $ & $\Psi $ & Psi \\
$\omega $ & $\Omega $ & Omega \\
\hline
\end{tabular}
\end{center}


\section{Mengen}


Kategorisierung spielt in der (Linearen) Algebra eine sehr wichtige Rolle. Kennt man die Eigenschaften eines Dinges, so kann man damit umgehen. Daher ist es nicht verwunderlich, dass Mathematiker versuchen, Dinge, die identische Eigenschaften besitzen zu Mengen zusammenzustellen. Und -- soweit möglich -- Operationen und Eigenschaften mit dem Namen dieser Zusammenstellung zu verbinden. Die bereits im Arithmetik Teil vorgekommene Menge der natürlichen Zahlen $\mathbb{N}$ bildet da ein gutes Beispiel für den Einstieg. 

Im Folgenden werden wir immer mehr Vorgehensweisen kennenlernen, mit denen man Mengen untersuchen kann. Kann man bestimmte Eigenschaften an einer Menge feststellen, so wird sie zur Gruppe. Hat man weitere Eigenschaften an der Gruppe, so wird sie zum Ring. Und eine letzte Eigenschaft macht einen Ring zu einem Körper. Der Körper ist das, was unserer üblichen Vorstellung von Zahlen am nächsten kommt. Für viele Objekte der Mathematik kann man aber z.B. nur Gruppenstruktur nachweisen. 

\begin{svgraybox}
HINWEIS: Der hier verwendete Begriff "`Menge"' hat nichts mit der Anzahl zu tun, wie er im Arithmetik Teil verwendet wurde. Hier bezeichnet "`Menge"' eine Ansammlung von gleichartigen Objekten.
\end{svgraybox}

Mengen werden in der Mathematik in geschweiften Klammern $\{\dots \}$ dargestellt. In diesen Klammern stehen entweder die Elemente der Menge in Form einer Liste oder Aufzählung (diese kann unvollständig sein, wie wir dies gleich verwenden um die natürlichen Zahlen zu beschreiben), oder in Form einer Regel. Solche Regeln werden in Form von Aussage-Prädikaten formuliert, also z.B. $\forall$ entspricht "`für alle"' und $\exists$ entspricht "`existiert"', aber dies werden wir noch ausführlicher kennen lernen, wenn wir solche Konstrukte einsetzen. 

Die natürlichen Zahlen werden wie folgt definiert:
\[\mathbb{N} := \{ 1,2,3,4,5, \dots \} \]
Die Punkte "`$\dots$"' symbolisieren dabei, dass die Reihe nicht enden soll. Dementsprechend gehört $\infty$ "`Unendlich"' zu den natürlichen Zahlen.
\[\mathbb{N} := \{ 1,2,3,4,5, \dots, \infty \} \]

Zusammengefasst sind Mengen einfache Ansammlungen von im allgemeinen gleichartigen Dingen. Die folgende Zeichenfolge bedeutet, dass $x$ ein Element der Menge $M$ ist:
\[x \in M \]
So ist zum Beispiel
\[7 \in \mathbb{N} \]

\subsection{Abbildung}

Eine Abbildung bildet Elemente aus einer Menge auf Elemente einer anderen Menge ab. Seien $X,Y$ Mengen. Dann ordnet die Abbildung
\[ f : X \longrightarrow Y \]
das Element $x\in X$ auf $y\in Y$ ab. Das wird so ausgedrückt:
\[ f(x) = y\]

Die Addition ist ebenfalls eine Abbildung. Sie bildet zwei Elemente einer Menge auf Elemente der selben Menge ab:
\[ + : X \times X \longrightarrow X \]
Also kann man die Summe 
\[ x+y = z\]
auch in Funktionenform schreiben:
\[ +(x,y) = z\]

\subsection{Gruppen, Ringe, Körper}
Im Folgenden bezeichen $X,Y$ und $Z$ immer Mengen.

\begin{definition}
Eine Funktion
\[ \circ : X \times X \longrightarrow X \]
wird als Verknüpfung auf $X$ bezeichnet. Zum Beispiel sind die Operationen $+$ und $\cdot$ Verknüpfungen. 
\end{definition}

\begin{definition}
Eine Verknüpfung wird \textsl{assoziativ} genannt, wenn gilt:
\[ a \circ (b \circ c) = (a \circ b) \circ c \]
\end{definition}
\begin{definition}
Sie wird \textsl{kommutativ} genannt, wenn gilt: 
\[ a \circ b = b \circ a \]
\end{definition}
Für alle $a,b,c \in X$

\begin{definition}
Ein Element $e\in X$ wird als \textsl{neutrales Element} bezüglich einer Verknüpfung bezeichnet, wenn 
\[ e\circ a = a \circ e = a \]
für alle $a\in X$ gilt.
\end{definition}

\begin{claim}
Falls es in $X$ ein neutrales Element gibt, so ist es eindeutig. 
\end{claim}
\begin{proof}
\smartqed
Seien $e_1$ und $e_2$ neutrale Element bezüglich $\circ$, dann gilt: 
\[ e_1 = e_1\circ e_2 = e_2 \] \qed
\end{proof}

\begin{definition}
Eine Menge $X$ mit assoziativer Verknüpfung und zugehörigem neutralem Element heißt \textsl{Monoid}.
\end{definition}

\begin{definition}
Sei $X$ ein Monoid mit der Verknüpfung $\circ$ und zugehörigem neutralen Element $e$. Wenn gilt
\[ a\circ b = e \]
für $a,b \in X$, so wird $a$ das \textsl{Linksinverse} von $b$ genannt, sowie $b$ das \textsl{Rechtsinverse} von a. Gilt zudem
\[ a\circ b = b \circ a = e \]
Dann sind $a$ und $b$ jeweils ihre \textsl{Inversen}.
\end{definition}

\begin{definition}
Ist in einem Monoid jedes Element invertierbar, so nennt man dies eine \textsl{Gruppe}.
\end{definition}

\begin{definition}
Eine Gruppe, deren Verknüpfung kommutativ ist, nennt man eine \textsl{abelsche Gruppe}.
\end{definition}

\begin{definition}
Ein \textsl{Ring} $R$ ist eine Menge mit zwei Verknüpfungen $+$ und $\cdot$. Wobei $R$ bezüglich der Addition eine abelsche Gruppe ist und bezüglich der Multiplikation ein Monoid. Des Weiteren muss das Distributivgesetz gelten
\[ a\cdot (b+c) = a\cdot b + a\cdot c \]
für alle $a,b,c \in R$. Ist $R$ bezüglich $\cdot$ kommutativ, so ist $R$ ein \textsl{kommutativer Ring}.
\end{definition}

\begin{definition}
Sei $K$ ein kommutativer Ring, dessen neutrales Element bezüglich der Addition $0$ ist. Falls alle Elemente von $K\backslash \{0\}$ invertierbar sind, heißt $K$ ein \textsl{Körper}.
\end{definition}

\section{Vektorraum}

Sei $K$ ein Körper und $(V,+,\cdot)$ eine Menge mit zwei Verknüpfungen:
\begin{eqnarray*}
+: V\times V &\longrightarrow& V \\
\cdot: K \times V &\longrightarrow& V 
\end{eqnarray*}

\chapter{Beweisführung}

Dieses Kapitel ist nicht speziell für die Lineare Algebra gedacht, aber da wir im Folgenden die ersten ernsthaften Beweise führen werden, sollte hier auf die Grundlagen der Beweisführung in der Mathematik eingegangen werden.

\section{Aussage-Typen}

Die folgenden Definitionen von Begriffen sind -- soweit es im Wissensbereich des Autors liegt -- nicht im strengen Sinne definiert worden. Daher werden sie meist so eingesetzt, wie der Autor eines Textes sie interpretiert. Jedoch hat sich ein allgemeines Verständnis für diese Begriffe gebildet, das hier in kurzer Übersichtsform dargestellt werden soll.

\subsection{Axiom}

Ein \textsl{Axiom} ist eine Grundlage für eine Theorie. Innerhalb dieser Theorie wird das Axiom als richtig vorausgesetzt und bedarf keines Beweises. Ein Axiom kann also als eine Art Rand- oder Anfangsbedingung für eine Argumentationskette interpretiert werden.

\subsection{Definition}

Eine \textsl{Definition} ist eine Aussage, die einen bestimmten Sachverhalt darlegt. Eine Definition kann im allgemeinen nicht aus anderen Aussagen abgeleitet werden. Sie führt die Beschreibung des Sachverhalts auch meist durch Einführung eines Namens zu einem feststehenden Begriff zusammen. Durch die Definition bekommt also der Begriff seine Bedeutung.

\subsection{Theorem oder Satz}

Als \textsl{Theorem} oder \textsl{Satz} wird eine Aussage verstanden, deren Inhalt logisch aus den Axiomen einer Theorie abgeleitet werden kann. Diese Form von Aussagen wird am häufigsten da eingesetzt, wo die Fundamente einer Theorie erarbeitet werden. Mithilfe der Theoreme wird aus einer Idee eine mathematische Realität. Die Beweise von Theoremen innerhalb einer Theorie stellt den Grundstein der Arbeit eines Mathematikers dar. In den Beweisen findet die Mathematik statt.

\subsection{Korollar}

Ein \textsl{Korollar} bezeichnet eine Aussage, die sich aus einem bewiesenen Theorem, dem Beweis des Theorems oder aus einer Definition ohne Aufwand ergibt. Im allgemeinen müssen die Aussagen eines Korollars nicht bewiesen werden, weil der Beweis trivial wäre oder so eng mit dem Beweis des Theorems verbunden ist, dass die Beweisführung redundant wäre. In seltenen Einzelfällen kann es notwendig sein, die Aussage eines Korollars zu beweisen. Sollte dies notwendig sein, sollte der Mathematiker sich aber überlegen, ob er aus dem Korollar nicht einen Hilfssatz macht.

\subsection{Lemma oder Hilfssatz}

Als \textsl{Lemma} oder \textsl{Hilfssatz} wird eine zu beweisende Aussage bezeichnet, deren Inhalt für den Beweis eines Theorems notwendig ist. Daraus folgt, dass ein Lemma bewiesen werden muss und im allgemeinen im Kontext einer größeren Beweisführung auftaucht. 

Die Unterscheidung, ob eine Aussage ein Lemma oder ein Theorem darstellt, ist oft nicht leicht zu treffen. Es wird passieren, dass eine Aussage in dem einen Buch über Mathematik als Theorem, in einem anderen "`nur"' als Lemma dargestellt wird -- als Beispiel sei hier der Fundamentalsatz der Algebra erwähnt, der eigentlich ein Satz der Analysis ist und dort nur eine eher geringe Wertschätzung erfährt. 

Dies sollte für den Lernenden nicht entscheidend sein. Der Inhalt von Theorem und Lemma ist in aller Regel interessant und der Beweis von beiden sollte verstanden und nachvollziehbar sein. 

\section{Beweisarten}

\subsection{Vollständige Induktion}

\subsection{Beweis durch Widerspruch}



