
\part{Arithmetik}

\chapter{Ganze Zahlen}

Der Mengenbegriff (im Sinne von Anzahl) birgt ein natürliches Verständnis dafür, was eine Zahl ist. Im Verständnis der meisten Menschen ist eine Zahl unmittelbar mit dem Begriff der Anzahl von Dingen verbunden --- zum Beispiel drei Äpfel oder fünf Orangen.

Um einem Obsthändler zu erklären, wie viele Äpfel man haben möchte, ist die Nutzung von Zahlen durchaus praktisch. Genauso wie beim Metzger Zahlen etwas abstrakter verwendet werden. So möchte man 500 Gramm Hackfleisch. Das Hackfleisch besteht nicht aus 500 Einzelteilen, sondern die Zahl bezeichnet das Gewicht dessen, was man bestellt.

\section{Natürliche Zahlen}

Dem Ziel, zu erklären, was ganze Zahlen sind, nähern wir uns über den Umweg der natürlichen Zahlen, die einen wesentlichen Bestandteil der ganzen Zahlen bilden. 


\begin{definition}
Zahlen -- im Sinne von Anzahl -- werden in der Mathematik als \textsl{natürliche} Zahlen bezeichnet. Ihre Gesamtheit, d.h. alle natürlichen Zahlen inklusive etwas, das als "`unendlich"' bezeichnet wird -- darauf kommen wir später zurück -- wird mit dem Zeichen $\mathbb{N}$ abgekürzt.
\end{definition}

Die folgenden Symbole bezeichnen die ersten neun natürlichen Zahlen. 

\[ 
1, 2, 3, 4, 5, 6, 7, 8, 9
\]

Sie bilden den Grundstock aller folgenden Zahlen, die aus diesen zusammengesetzt sind. Wie diese zusammen gesetzt werden, erfahren wir gleich.

\section{Grundlagen}

\subsection{Gleichheit, Ungleichheit, Vergleiche}

\begin{definition}
Mit dem Symbol "`="' wird die Anforderung beschrieben, dass alles, was auf der linken Seite des Symbols steht den selben Wert hat wie das, was auf der rechten Seite des Symbols steht. 
\end{definition}
Folgendes ist demnach korrekt:

\begin{eqnarray*}
1 &=& 1 \\
5 &=& 5 \\
7 &=& 7
\end{eqnarray*}
Während das folgende falsch ist:
\begin{eqnarray*}
1 &=& 7 \\
5 &=& 3 \\
7 &=& 1
\end{eqnarray*}

Es liegt in der Verantwortung desjenigen, der die Gleichung aufstellt, dafür zu garantieren, dass die Gleichheit auch wirklich erfüllt ist. Papier ist geduldig! Man kann hinschreiben, was man will, ob ein hingeschriebenes Gleichheitszeichen auch wirklich Gleichheit bedeutet, kann nur der Hinschreibende wissen.

Für Gleichungen gilt im allgemeinen, dass Gleichheit erhalten bleibt, wenn auf beiden Seiten der Gleichung die selben Operationen ausgeführt werden. Multipliziert man beide Seiten mit der selben Zahl, oder addiert auf beiden Seiten die selbe Zahl, so bleibt die Gleichheit erhalten.

\begin{eqnarray*}
5 &=& 5  \hspace{1cm}| +1\\
5+1 &=& 5+1 \\
6 &=& 6
\end{eqnarray*}

Das, was man auf beiden Seiten einer Gleichung macht, kann durch einen abgesetzten senkrechten Strich dargestellt werden. Alles, was auf der rechten Seite des Strichs steht, wird auf beiden Seiten der Gleichung angewendet. Hier also jeweils eine 1 addiert.

\begin{definition}
Möchte man ausdrücken, dass die linke und rechte Seite nicht übereinstimmt, so verwendet man das Symbol "`$\neq $"'. 
\end{definition}
So werden oben angegebene Falschaussagen wieder korrekt indem man das Ungleichzeichen verwendet:
\begin{eqnarray*}
1 &\neq & 7 \\
5 &\neq & 3 \\
7 &\neq & 1
\end{eqnarray*}

Der Ungleichheit stehen noch qualifizierende Ungleichzeichen zur Seite. Dass 1 nicht gleich 7 ist stimmt zwar, ist aber weniger interessant, als die Aussage, dass 1 kleiner als 7 ist. Wenn Hans einen Apfel besitzt und Peter zwei, dann hat Hans weniger Äpfel als Peter, wie Peter mehr Äpfel hat als Hans. Dies drückt man durch die folgenden Symbole aus:

\begin{eqnarray*}
1 & < & 7 \\
1 & < & 2 \\
2 & > & 1
\end{eqnarray*}

Hat man eine Ungleichung aufzustellen, bei der es akzeptabel ist, dass beide Seite auch den gleichen Wert haben können, so verwendet man die selben Symbole mit einem Unterstrich:
\begin{eqnarray*}
1 & \le & 7 \\
5 & \ge & 3 \\
7 & \ge & 1
\end{eqnarray*}

Das Symbol "`$\le$"' wird "`kleiner oder gleich"' ausgesprochen und das Symbol "`$\ge$"' "`größer oder gleich"'. Es ist zu beachten, dass folgende Ungleichungen ebenfalls \textbf{alle} korrekt sind:
\begin{eqnarray*}
4 & \le & 5 \\
5 & \le & 5 \\
6 & \ge & 5 \\
5 & \ge & 5
\end{eqnarray*}

Wenn für beiden Seiten einer Ungleichung sowohl das $\le$ als auch das $\ge$ korrekt ist, so gilt $=$ Gleichheit. Dies sollte im Kopf behalten werden, denn es gibt einige Beweise in der Mathematik, die genau dies ausnutzen.

Auch für Ungleichungen gilt, dass Operationen, die auf beiden Seiten ausgeführt werden, die Ungleichung erhalten. Das gilt bei Ungleichungen allerdings nur in etwas eingeschränkter Form, nämlich die Multiplikation mit einer negativen Zahl ($<0$) führt dazu, dass sich das Ungleichzeichen umdreht. Aber dazu später mehr.

\subsection{Addition und Multiplikation}

\begin{definition}
Um mit natürlichen Zahlen rechnen zu können, definieren wir zwei Operationen. Die Multiplikation, dargestellt durch das Zeichen "`$\cdot$"' und die Addition, dargestellt durch das Zeichen "`$+$"'. Diese Operationen werden als \textsl{Verknüpfung} bezeichnet, weil sie zwei Zahlen miteinander "`verknüpfen"' zu einer neuen Zahl.
\end{definition}

Mit der Addition fügen wir einzelne Mengen zu größeren Mengen zusammen. Hat Hans $2$ Äpfel und Peter $3$ Äpfel, so haben sie zusammen $5$ Äpfel, oder einfacher formuliert:
\[
2+3=5
\]

\noindent Hätte Peter keinen Apfel
\[2+?=2\]
dann wäre eine Zahl hinzuzuzählen, die keinen Wert besitzt. 
\begin{definition}
Eine Zahl ohne Wert wird mit dem Symbol "`0"' dargestellt.
\end{definition}

Sie ist eine Invariante bezüglich der Addition -- d.h. Additionen mit dieser Zahl ändern nicht den Wert der ursprünglichen Zahl. Man nennt sie "`Null"'. 

\begin{definition}
Etwas wird als \textsl{Invariante} einer Verknüpfung bezeichnet, wenn sie den Wert einer Zahl mit der sie Verknüpft wird, nicht verändert.
\end{definition}

Demnach erweitern sich die Symbole der ersten zehn natürlichen Zahlen auf diese Weise:

\[0,1,2,3,4,5,6,7,8,9\]

Die Null ist per Definition nicht Teil der natürlichen Zahlen. Das hat historische Gründe -- so dachte man im Mittelalter, dass die Zahl Null vom Teufel erdacht worden sei\footnote{Historischer Beleg?}. Für unsere Betrachtung sollte sie aber Teil der natürlichen Zahlen sein. Daher werden wir von nun an mit einer Zahlenmenge umgehen, die mit $\mathbb{N}_0$ bezeichnet wird, sie besteht aus allen natürlichen Zahlen, inklusive der 0.

\textsl{Was kommt nach der 9?}

Die Zahl, die nach der 9 kommt, ist definiert durch die Summe $9+1$. Es wurde in frühen Zeiten einmal festgelegt\footnote{Historischer Beleg?}, dass wir ein Zahlensystem bestehend aus zehn Ziffern verwenden. Die zehn Ziffern beinhalten die Null, daher ist für die zehn kein eigenes Symbol mehr übrig. Daher wurden die zusammengesetzten Zahlen erfunden. So ist die Symbolfolge
\[10\]
diejenige Zahl, die nach der 9 kommt, also
\[9+1=10\]
Das ist eine Definition die bedeutet, dass wenn einem die Symbole ausgehen, kann man eine zusätzliche Stelle verwenden um diese darzustellen. Zusätzliche -- im Sinne von höherwertigen -- Stellen werden links an die Zahl angehängt. Das funktioniert auch mit drei, vier und allen weiteren Stellen:
\begin{eqnarray*}
99+1 &=& 100 \\
999+1 &=& 1000 \\
\cdots
\end{eqnarray*}

Interessant an der Zehn ist, dass sie aus einer 1 sowie der 0 zusammengesetzt ist. Also hatten die Menschen früher kein Problem mit der 0, solange sie als Teil einer anderen Zahl auftrat. Nur wenn sie alleine stand, wurde sie als verwirrend angesehen.

Aufgrund dessen, dass unser Zahlensystem auf der Zahl 10 basiert, sind gerade die Multiplikationen mit 10 besonders einfach: 
\begin{eqnarray*}
3\cdot 10 &=& 30 \\
30\cdot 10 &=& 300 \\
300\cdot 10 &=& 3000\\
\cdots
\end{eqnarray*}
Man hängt einfach nur eine 0 an die Zahl an.

\bigskip

\noindent \textsl{Wie werden aber dann Zahlen zusammengesetzt?}

\noindent Sehen wir uns ein Beispiel an:
\[9+5 = ?\]
Wir wissen, dass 
\[9+1 = 10\]
ist, und dass
\[1+4 = 5\]
So können wir schreiben:
\[9+1+4 = 10 + 4 = 14\]
An dieser Stelle haben wir eine Eigenschaft verwendet, die das \textsl{Assoziativgesetz} genannt wird. Wir haben 5 durch die Summe von zwei Zahlen ersetzt und die gesamte Summe anders kombiniert. Sehen wir uns das mit Klammern an:
\[9+5 = 9+(1+4) = (9+1)+4 = 10+4 = 14\]
Klammern bevorzugen eine Operation. So ist
\[9+(1+4)=9+5\]
dass wir die 1 von $1+4$ wegnehmen und zu $9+1$ hinzufügen dürfen ist in der Anschauung vollkommen klar, ob nun Hans 9 Äpfel und Peter 5 Äpfel besitzen, oder Hans 10 und Peter 4, macht für die Gesamtanzahl keinen Unterschied. Aber dieser Operation liegt das obengenannte Assoziativgesetz zugrunde, das wir im restlichen Buch immer wieder verwenden werden. Dass wir es verwenden dürfen, liegt an der "`Harmlosigkeit"' der natürlichen Zahlen $\mathbb{N}$. Wir werden später noch Konstrukte kennen lernen, die das Assoziativ- und im Besonderen das Kommutativgesetz nicht erfüllen. Trotzdem kann man mit diesen Konstrukten genauso rechnen, wie wir das mit den natürlichen Zahlen tun, man darf mit ihnen nur nicht alles machen, was man mit Zahlen tun kann.

Die nächste Operation, die wir kennenlernen, ist die Multiplikation:
\[3\cdot 4 = ?\]
Diese entspricht ebenfalls der Anschauung. Lägen in drei Körben jeweils vier Äpfel, so hätte man insgesamt
\[\underbrace{4+4+4}_{3 \, \textnormal{mal}} = 12\]
Äpfel. Also sind 
\[3\cdot 4 = 12\]
Das Ergebnis bliebe gleich, wenn man vier Körbe mit jeweils drei Äpfeln hätte:
\[3\cdot 4 = 4\cdot 3 = 12\]
Das ist das \textsl{Kommutativgesetz}.

Gehen wir jetzt mal davon aus, wir hätten drei Körbe mit jeweils 2 grünen und 4 roten Äpfeln. Wie viele grüne Äpfel, wie viele rote und wie viele Äpfel insgesamt hätten wir dann?

Die einzelnen Summen können wir leicht bestimmen: 
\[3\cdot 2 = 6\]
grüne Äpfel,
\[3\cdot 4 = 12 \]
rote Äpfel und somit
\[6+12 = 18\]
Äpfel insgesamt. Dabei haben wir zunächst nur die Anzahl der grünen, dann die der roten Äpfel berechnet. Wir haben also folgendes gemacht:
\[3\cdot \underbrace{(2+4)}_{\textnormal{Inhalt eines Korbes}} = 3\cdot 2 + 3\cdot 4 = 18\]
Zur Berechnung der Gesamtsumme hätten wir aber auch gleich die roten und grünen Äpfel pro Korb summieren können:
\[3\cdot (2+4) = 3\cdot (6) = 3\cdot 6 = 18 \]

Dass wir zuerst die Anzahlen aller grünen Äpfel und dann die aller roten Äpfel berechnen konnten, wird als  \textsl{Distributivgesetz} bezeichnet.

\subsection{Subtraktion und Division}

Der Grund warum Subtraktion und Division ein eigenes Unter-Kapitel bilden liegt darin begründet, dass sie nicht das Kommutativgesetz beachten, wie wir gleich sehen werden.

Wenn Hans drei Äpfel besitzt und zwei davon isst, bleibt ihm nur einer übrig. Die Frage, die sich der Lernende stellen sollte ist:

\textsl{Welche Zahl erfüllt folgende Gleichung?}

\[3 + ? = 1\]

Die Zahl, die diese Gleichung erfüllt, muss kleiner als 0 sein, denn $3+0 > 1$. Was also ist kleiner als 0? Solche Zahlen werden als negative Zahlen bezeichnet. Addiert man eine natürliche Zahl mit einer negativen Zahl, so ist das Ergebnis kleiner, als die natürliche Zahl selbst. 

\begin{definition}
Eine negative Zahl ist eine natürliche Zahl, die mit einem "`$-$"' Zeichen als negativ gekennzeichnet wird. Sie ist identisch zu einer Zahl, die mit $-1$ multipliziert wird:
\[ -4 = (-1) \cdot 4 \]
Des Weiteren gilt
\begin{eqnarray*}
(-1)\cdot 1 &=& 1\cdot (-1) = -1 \\
(-1)\cdot (-1) &=& 1
\end{eqnarray*}

\end{definition}

Zur Vereinfachung der Darstellung gilt:

\begin{eqnarray*}
3 + -2 &=& 1 \hspace{1cm}\textnormal{falsch}\\
3 + (-2) &=& 1 \hspace{1cm}\textnormal{richtig, aber umständlich}\\
3 - 2  &=& 1 \hspace{1cm}\textnormal{ok}
\end{eqnarray*}

Die Menge aller negativen Zahlen wird mit dem Zeichen $-\mathbb{N} $ dargestellt. Die Vereinigung der Mengen $-\mathbb{N} $ und $\mathbb{N}_0$ wird als die Menge der \textsl{ganzen Zahlen} bezeichnet und $\mathbb{Z}$ genannt. Also ist 

\[
\mathbb{Z} = -\mathbb{N} \cup \mathbb{N}_0 =  \{ -\infty, \dots, -3, -2, -1, 0, 1, 2, 3, \dots, \infty \}
\]

Bei der Multiplikation mit negativen Zahlen, muss man das Assoziativgesetz anwenden:

\[ (-3)\cdot 6 = ((-1)\cdot 3) \cdot 6 = (-1)\cdot (3\cdot 6) = (-1)\cdot (18) = -18 \]

In dem gleichen Sinn, wie sich die Subtraktion umgekehrt zur Addition verhält, versuchen wir uns nun vorzustellen, dass es auch zur Multiplikation eine Zahl gibt, die sich umgekehrt zu dieser verhält. Kommen wir auf das vorher erwähnte Beispiel zurück, dass wir 3 Körbe mit jeweils 4 Äpfeln haben. Insgesamt haben wir also 12 Äpfel, wie wir festgestellt hatten. Ständen wir vor dem Problem, 12 Äpfel auf drei Körbe gleichmäßig zu verteilen, wüssten wir also sofort, dass wir 4 Äpfel in jeden Korb legen müssten. Wir haben also die 12 Äpfel in drei Körbe aufgeteilt und etwas zu teilen ist Gegenstand der Division. Sie wird in der folgenden Form dargestellt:

\[ {12\over 3} = 4 \] 
Manchmal (z.B. wenn nicht so viel Platz ist) auch
\[ 12 / 3 = 4 \]
oder
\[ 12 \div 3 = 4 \]
Diese Gleichungen bedeuten alle das selbe, auch wenn für die Division hier drei verschiedene Zeichen (---, /, $\div$) verwendet wurden. Leider hat sich keins davon als Standard durchgesetzt, es werden in verschiedenen Situationen immer wieder diese Zeichen auftauchen. Das sollte den Lernenden nicht verwirren, denn alle bedeuten dasselbe. Wir werden hier in der Regel nur den waagerechten Stich verwenden, bis auf wenige Ausnahmen. Das $\div$ Zeichen nur beim Umformen von Gleichungen um anzuzeigen, dass durch eine Zahl geteilt werden soll.

Division ist in diesem Kapitel nur unzureichend zu erklären, da es nur in bestimmten Sonderfällen möglich ist, eine Zahl durch eine andere Zahl zu dividieren, um dann wieder eine ganze Zahl herauszubekommen. In solchen Fällen spricht man davon, dass eine Zahl ein \textsl{Teiler} einer anderen Zahl ist. In obigem Beispiel ist 3 ein Teiler von 12, da 3 die Anzahl der Körbe ist, auf die wir unsere Äpfel aufteilen. Genauso wie 4 ein Teiler von 12 ist, da dies die Anzahl der Äpfel ist, die wir in die Körbe aufteilen. Wir werden im nächsten Kapitel darauf eingehen, was passiert, wenn wir z.B. nur 11 Äpfel hätten, aber trotzdem diese auf 3 Körbe aufteilen möchten. 

Wir hatten vorher gesehen, dass es für die Multiplikation keine Rolle spielt, ob in vier Körben jeweils drei Äpfel liegen, oder in drei Körben jeweils vier Äpfel. In beiden Fällen ist die Gesamtanzahl der Äpfel zwölf. 

Bei der Division ist dem nicht so. Wenden wir das Kommutativgesetz auf eine Division an, passiert das Folgende:
\[ 12 / 3 = 3/12 \]
Diese Gleichung würde bedeuten, dass wenn man zwölf Äpfel auf drei Körbe aufteilt, in jedem Korb genauso viele Äpfel lägen, als würde man drei Äpfel auf zwölf Körbe aufteilen, was offensichtlich falsch ist. Richtig ist:
\[ 12 / 3 \ne 3/12 \]
Also gilt das Kommutativgesetz \textbf{\underline{nicht}} für die Division. Bei der Subtraktion gilt es "`fast"':

\[ 3-2 \ne 2-3 \]
aber
\[ 3-2 = 3+(-1)\cdot 2 = \underbrace{(-1)\cdot ((-1)\cdot 3}_{\textsl{weil $(-1)\cdot (-1)=1$}} + 2) = (-1)\cdot(2-3) = -(2-3)\]
also
\[ 3-2 = -(2-3) \]

Bei der Subtraktion gilt das Kommutativgesetz, bis auf das sogenannte Vorzeichen.
\begin{definition}
Als \textsl{Vorzeichen} wird entweder das $+$ oder das $-$ bezeichnet, das vor eine Zahl steht. Es bestimmt, ob die Zahl positiv ($+$) oder negativ ($-$) ist.
\end{definition}

Spätestens hier sollte es dem Lernenden klar sein, dass die Subtraktion identisch ist mit der Addition bei umgekehrtem Vorzeichen.
\[ 3-2 = 3+(-1)\cdot 2 = 3+(-2) \]

\begin{svgraybox}
Eine Randbemerkung: Im Grunde wird hier das Kommutativgesetz der Addition angewendet, denn
\[ 3-2 = 3+(-1)\cdot 2 = (-1)\cdot 2+3 = -2+3 \]
Demnach ist $-(2-3) = -2+3$, dies wird in den Aufgaben nachgewiesen.
\end{svgraybox}

Wie wir schon bei der Multiplikation gesehen haben, ist die Division mit 10 ebenfalls besonders einfach. Im Gegensatz zur Multiplikation werden anhängende Nullen einfach weggenommen:

\begin{eqnarray*}
3000 / 10 &=& 300 \\
300 / 10 &=& 30 \\
30 / 10 &=& 3
\end{eqnarray*}

Die 3 ist in den ganzen Zahlen nicht weiter durch 10 teilbar. Das Ergebnis von $3/10$ können Sie berechnen, nachdem Sie die Rationalen Zahlen kennen gelernt haben. 

\section{Aufgaben}

\begin{prob}
\label{arith.1.1}

Welche der folgenden Gleichungen und Ungleichungen sind richtig oder falsch?
\begin{center}
\begin{tabular}{C{2cm}C{2cm}C{2cm}C{2cm}}
$1=2$ & $3<5$ & $9<10$ & $10>9$ \\
$8=8$ & $3<6$ & $10<10$ & $11>10$ \\
$5=5$ & $10<10$ & $10<11$ & $999 \ge 1000$ 
\end{tabular}
\end{center}
\end{prob}

\begin{prob}
\label{arith.1.2}
Ausgehend von der Gleichung $2+3=5$ führe folgende Operationen auf beiden Seiten der Gleichung durch, ohne zu vereinfachen, bis es ausdrücklich gewünscht ist (letzter Punkt):
\begin{enumerate}
\item addiere auf beiden Seiten eine 2
\item multipliziere mit 3
\item wende das Distributiv Gesetz auf beiden Seiten der Gleichung an
\item vereinfache soweit, bis auf beiden Seiten nur noch eine Zahl steht.
\end{enumerate}
\end{prob}

\begin{prob}
\label{arith.1.3}

Bestimme die rechten Seiten der folgenden Aufgaben:
\begin{center}
\begin{tabular}{C{2cm}C{2cm}C{2cm}C{2cm}}
$8+9=$ & $8-9=$ & $3\cdot 4=$ & ${8/ 2}=$ \\
$13+5=$ & $13-5=$ & $13\cdot 5=$ & ${27/ 3}=$ \\
$17+3=$ & $17-3=$ & $17 \cdot 3=$ & ${25/ 5}=$ 
\end{tabular}
\end{center}
\end{prob}

\begin{prob}
\label{arith.1.4}
Warum ist folgende Gleichung richtig?
\[ -(2-3) = -2+3 \]
\end{prob}


\chapter{Rationale und Reelle Zahlen}

In diesem Kapitel wird der Begriff der Zahl erweitert. Im letzten Kapitel haben wir die ganzen Zahlen kennengelernt. Diese sind aber nur ein Teil der Zahlen, mit denen wir täglich umgehen. Beim Einkaufen zum Beispiel besorgen wir zwar Äpfel und Milchtüten in ganzen Einheiten, aber schon beim Brot lassen wir uns oft nur die "`Hälfte"' geben. Die Preise der Dinge, die wir im Supermarkt kaufen, haben -- im allgemeinen absichtlich -- nie einen ganzzahligen Preis, da \currency 1,99 vermeintlich "`viel billiger"' klingt, denn \currency 2,00.

Wir werden uns im Besonderen mit der Teilbarkeit von Zahlen auseinander setzten, wie auch der Unteilbarkeit (Primzahlen), als auch mit den irrationalen und den komplexen Zahlen.

\section{Rationale Zahlen}

Wir hatten im vorhergehenden Kapitel immer wieder das Beispiel der drei Körbe mit den vier Äpfeln erwähnt und auch gleich die Frage gestellt, was passiert, wenn man nicht zwölf, sondern nur elf Äpfel hätte.
\[ {11 \over 3} = ? \]
Versuchen wir uns zunächst dem Problem zu nähern, indem wir solange Äpfel in Körbe legen, bis wir zu wenig Äpfel haben, um dies gleichmäßig zu tun. Es ist klar, dass wir in jeden Korb drei Äpfel legen können und danach zwei übrig haben. Wenn wir die Äpfel nicht zerschneiden wollten, würden wir sagen, dass wir drei Äpfel in die Körbe legen konnten und zwei übrig behalten haben. 

\begin{definition}
Dies bezeichnen wir als \textsl{Division mit Rest}. 
\end{definition}
In unserem Beispiel würden man sagen:
\[ {11\over 3} = 3 \textsl{ Rest } 2 \]

Aber ist das wirklich einfacher? Sicher nicht. Das Ergebnis $11/3$ ist, vom mathematischen Standpunkt aus betrachtet, ein vollkommen korrektes Ergebnis und benötigt keine andere Darstellung. Daher tendiert man dazu, solche Ergebnisse in der Teilerform zu behalten und nicht weiter umzuformen. Division mit Rest wurde in früheren Zeiten verwendet und entspricht sehr der Anschauung. Heute würden schwierigere Rechenaufgaben grundsätzlich mit einem Taschenrechner oder Computer durchgeführt. Diese rechnen nie mit Brüchen und schon gar nicht mit einer Division mit Rest. Sie verwenden Dezimalzahlen, die später erklärt werden. So hat die Division mit Rest ausgedient und soll hier nur der Vollständigkeit halber erwähnt sein.

\begin{definition}
Eine Teilerform nennt man einen \textsl{Bruch}, sowie das Rechnen mit Brüchen \textsl{Bruchrechnung}.
\end{definition}

\subsection{Bruchrechnung}

\begin{definition}
Ein Bruch besteht aus drei Teilen. Oben steht eine Zahl, die als \textsl{Zähler} bezeichnet wird, unten eine Zahl, die \textsl{Nenner} genannt wird und dazwischen ist der \textsl{Bruchstrich}:
\[ \textsl{Zähler} \over \textsl{Nenner} \]
\end{definition}

Brüche können, wie andere Zahlen auch, addiert, subtrahiert, multipliziert und dividiert werden. Dies geschieht nach folgenden Regeln:

\begin{definition}
Brüche können nur dann addiert oder subtrahiert werden, wenn ihr Nenner übereinstimmt. 
\end{definition}

Das liegt daran, dass man eine Gleichung einfach mit dem Nenner multiplizieren kann, um so die bekannten Operationen der ganzen Zahlen zu verwenden. Beispiel:

\begin{eqnarray*}
{3\over 8}+{5\over 8} &=& {3\over 8}+{5\over 8} \hskip 1cm | \cdot 8 \\
3 +5 &=& 3+5 \\
3+5 &=& 8 \hskip 1cm | \div 8\\
{3\over 8}+{5\over 8} &=& {8\over 8} \\
{3\over 8}+{5\over 8} &=& 1
\end{eqnarray*}
Auf der rechten Seite haben wir uns in der vorletzten Zeile zunutze gemacht, dass ein Bruch seinen Wert nicht ändert, falls Nenner und Zähler mit der selben Zahl multipliziert werden. So ist 

\[ 1 = {a \over a} \textsl{ für alle } a\in \mathbb{Z} \]
wie auch 
\begin{equation}
\label{relation}
{x\over y} = {a\cdot x \over a\cdot y} \textsl{ für alle } a\in \mathbb{Z}
\end{equation}

\begin{definition}
Wir bezeichnen Zahlen der Form 

\[ {a\over b} \textsl{ für alle } a,b \in \mathbb{Z}, b\ne 0\]
als \textsl{Rationale Zahlen} und stellen sie mit dem Symbol $\mathbb{Q}$ dar. 
\end{definition}
Also ist 

\[ \mathbb{Q} = \left\{ {a\over b} \vert a,b \in \mathbb{Z}, b\ne 0 \right\} \]
Die rationalen Zahlen beinhalten die ganzen Zahlen auf eine natürliche Weise, da für $n,a\in \mathbb{Z}$ gilt
\[ n = {n \over 1} = {n \cdot a \over 1\cdot a}\]
In diesem Fall ist $a$ ein Teiler der Zahl $n\cdot a$. 

\begin{svgraybox}
\textsl{Hinweis für Lehrer:}

Es gibt Mathematiker, die bei der Definition der rationalen Zahlen einen axiomatischen Ansatz wählen, der die rationalen Zahlen auf Basis von Äquivalenzrelationen definieren. So wäre dann eine rationale Zahl immer durch genau eine Äquivalenzklasse dargestellt. Da Äquivalenzklassen im allgemeinen durch ihr Erzeugendes Element dargestellt werden, würden nach diesem Ansatz nur diejenigen Brüche zu den rationalen Zahlen gehören, die die Erzeugenden Elemente ihrer Äquivalenzklasse wären. Der Autor hält dies für sehr verwirrend und kaum vermittelbar, warum der Bruch 
\[ {3\over 11}\]
zu den rationalen Zahlen zählen soll, der Bruch
\[ 9\over 33\]
aber nicht, sondern nur zur Äquivalenzklasse von $3\over 11$.
\end{svgraybox}

\subsubsection{Dezimalzahlen}

\begin{definition}
Eine \textsl{Dezimalzahl} ist eine Zahl der folgenden Form:
\[ z_m z_{m-1} \dots z_1 z_0, z_{-1} z_{-2} \dots z_{-n} \]
\end{definition}
Die $z_i$ sind Ziffern und somit Zahlen aus der Menge $\{0,1,2,3,4,5,6,7,8,9\}$.

Zur Erläuterung betrachten wir zunächst die Zahlen zwischen 0 und 1. Dies sind die Zahlen 
\[ 1 \over n\] für alle $n\in \mathbb{N}$. 

\begin{eqnarray*}
1/2 &=& 0,5 \\
1/4 &=& 0,25 \\
1/5 &=& 0,2 \\
1/8 &=& 0,125 \\
\dots
\end{eqnarray*}

Die Brüche $1/3$, $1/6$ und $1/7$ haben eine besondere Eigenschaft, sie haben nämlich keine endliche Dezimaldarstellung:

\begin{eqnarray*}
1/3 &=& 0,3333333333\dots \\
1/6 &=& 0,1666666666\dots \\
1/7 &=& 0,142857142857142857142857142857\dots
\end{eqnarray*}
Die sich wiederholenden Teile der Nachkommastellen werden mit einem Strich darüber gekennzeichnet:

\begin{eqnarray*}
1/3 &=& 0,\overline{3} \\
1/6 &=& 0,1\overline{6} \\
1/7 &=& 0,\overline{142857}
\end{eqnarray*}
Natürlich gibt es auch Dezimalzahlen, die größer sind als 1:

\begin{eqnarray*}
97/3 &=& 32,\overline{3} \\
21/5 &=& 4,2 \\
177/7 &=& 25,\overline{285714}
\end{eqnarray*}
Dezimalzahlen treten immer wieder als Ergebnisse von Berechnungen von Taschenrechnern und Computern auf, daher sind sie sehr wichtig. 


\section{Potenzrechnung}

Einleitend sei hier bemerkt, dass sich das Potenzieren mit natürlichen Zahlen zur Multiplikation verhält, wie die Multiplikation zur Addition:

\[ 4\cdot 3 = \underbrace{4+4+4}_{\text{3 mal}} = 12 \]

\[ 4^3 = \underbrace{4\cdot 4\cdot 4}_{\text{3 mal}} = 64 \]

\begin{definition}
Die Zahl, die potenziert werden soll, nennt man \textsl{Grundzahl}, und die Zahl mit der potenziert werden soll \textsl{Hochzahl}, oder \textsl{Exponent}. Den Wert der Potenz einer Grundzahl ($a$) mit einer natürlichen Zahl ($n \in \mathbb{N}$) als Exponent erhält man durch:
\[ a^n = 1\cdot \underbrace{a\cdot a \cdot \dots \cdot a}_{\text{n mal}} \]
Den Wert einer Potenz mit einer negativen natürlichen Zahl durch:
\[ a^{-n} = 1 \div \underbrace{a\div a \div \dots \div a}_{\text{n mal}} = {1 \over a^n} \]
Des Weiteren gilt immer
\[ a^0 = 1\]
\end{definition}
Das Kommutativgesetz gilt nicht:
\[ 64 = 4^3 \ne 3^4 = 81 \]



\section{Irrationale Zahlen}



\subsection{Vollständigkeit}

$\mathbb{R}, e, \pi, \dots $

\section{Aufgaben}


\chapter{Komplexe Zahlen}

Um die komplexen Zahlen zu erklären, wollen wir uns zunächst eine Gleichung ansehen. 

\[ ?^2 +1 = 0 \]
Welche Zahl kann diese Gleichung erfüllen? Wir formen um:
\begin{eqnarray*}
?^2 +1 &=& 0 \hskip 1cm | -1 \\
?^2 +1 -1 &=& -1 \hskip 1cm | \sqrt{.} \\
\sqrt{?^2} &=& \sqrt{-1} \\
? &=& \sqrt{-1}
\end{eqnarray*}
Stimmt das? Wir setzen ein:
\begin{eqnarray*}
\sqrt{-1}^2 +1 &=& 0 \\
(-1)^{{1 \over 2}\cdot 2} +1 &=& 0 \\
(-1)^{{2 \over 2}} +1 &=& 0 \\
(-1)^{1} +1 &=& 0 \\
-1+1 &=& 0 \\
0 &=& 0
\end{eqnarray*}
Die Gleichung stimmt offensichtlich, doch hatten wir gelernt, dass Wurzeln aus negativen Zahlen nicht definiert sind. Kann man also einfach mit dem Quadrieren einer negativen Zahl rechnen und erwarten, dass etwas sinnvolles dabei heraus kommt? Die einfache Antwort ist: "`Ja"', man kann. Denn Fakt ist: Als wir die Gleichung mit einem $?$ umgeformt hatten, wussten wir noch nicht, dass das Fragezeichen für die Wurzel aus $-1$ steht. Wendet man also die Rechenschritte korrekt auf eine Zahl (oder Variable, wie wir im späteren sehen werden) an, so bleibt das Resultat korrekt. 

Die Definition der komplexen Zahlen fußt auf dieser Erkenntnis. Sie verwendet die negativen Wurzeln indem ein Buchstabe $i$ den Wert $\sqrt{-1}$ annimmt und so zum Rechenelement wird.
\begin{eqnarray*}
\sqrt{-17} &=& \sqrt{-1\cdot 17}\\
&=& \sqrt{-1}\cdot \sqrt{17}\\
&=& i \cdot \sqrt{17}
\end{eqnarray*}
oder allgemein:

\begin{eqnarray*}
\sqrt{-x^2} &=& \sqrt{-1\cdot x^2}\\
&=& \sqrt{-1}\cdot \sqrt{x^2}\\
&=& i \cdot x
\end{eqnarray*}
wobei $x$ jede reelle Zahl sein kann, also $x\in \mathbb{R}$. Allgemein werden die komplexen Zahlen definiert als 

\[ \mathbb{C} = \{ a+i\cdot b \ |\ a,b \in \mathbb{R} \} = \mathbb{R}+i\mathbb{R} \]
Das bedeutet, dass komplexe Zahlen aus zwei reellen Zahlen bestehen. Sie können nicht miteinander addiert werden, da das $i$ dies verhindert. Weil die komplexen Zahlen dadurch aus zwei reellen Komponenten gebildet werden, nennt man sie auch oft zweidimensionale Zahlen. Als Ausblick sei erwähnt, dass in der linearen Algebra die komplexen Zahlen mit einem zwei dimensionalen reellen Vektorraum identifiziert werden können.

\section{Die Grundrechenarten der komplexen Zahlen}

Der Einfachheit halber wird im Folgenden $i\cdot b$ durch $ib$ ersetzt. Der Multiplikationspunkt kann weggelassen werden. Und alle $a,b,c,d \in \mathbb{R}$ sowie alle $p,q \in \mathbb{C}$ mit $p = a+ib$ und $q = c+id$. Beachte, dass $i^2=\sqrt{-1}^2 = -1$.

\begin{definition} Addition
\[ p+q = (a+c)+i(b+d)\]
\end{definition}

\begin{definition} Subtraktion
\[ p-q = (a-c)+i(b-d)\]
\end{definition}

\begin{definition} Multiplikation
\[ p\cdot q = a(c+id)+ib(c+id) = (ac-bd)+i(ad+bc)  \]
\end{definition}

\begin{definition} Division
\[ {p\over q} = {a+ib \over c+id} = {(a+ib)(c-id) \over (c+id)(c-id)} = {ac+bd\over c^2+d^2} + i{bc-ad \over c^2+d^2}\]
\end{definition}

\section{Historische Bemerkung} 

Gerolamo Cardano (1501-1576) behandelte in seinem 1545 erschienen Buch \textit{Artis magnae sive de regulis algebraicis liber unus} die Aufgabe zwei Zahlen zu finden, deren Produkt 40 und deren Summe 10 sei. Er setzte dafür die Gleichung an:

\[ x^2-10x+40=0 \]

Er erkannte, dass diese Gleichung keine Lösung hat, fügte aber die Bemerkung hinzu, dass falls die entsprechenden Umformungen sinnvoll und erlaubt wären, dass dann $5+\sqrt{-15}$ sowie $5-\sqrt{-15} $ in der Tat Lösungen der Gleichung wären. 

In diesem Sinne hatte Cardano bereits den ersten Schritt in die richtige Richtung getan. Doch dauerte es noch eine ganze Weile, bis komplexe Zahlen sich durchsetzten.

\section{Polarkoordinaten der Komplexen Zahlen}

\section{Riemannsche Zahlenkugel}


\section{Aufgaben}

