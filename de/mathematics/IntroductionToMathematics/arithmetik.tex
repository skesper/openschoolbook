
\chapter{Natürliche Zahlen}

\section{Was sind natürliche Zahlen?}

Der Mengenbegriff (im Sinne von Anzahl) birgt ein natürliches Verständnis dafür, was eine Zahl ist. Im Verständnis der meisten Menschen ist eine Zahl unmittelbar mit dem Begriff der Anzahl von Dingen verbunden --- zum Beispiel drei Äpfel oder fünf Orangen.

Um einem Obsthändler zu erklären, wie viele Äpfel man haben möchte, ist die Nutzung von Zahlen durchaus praktisch. Genauso wie beim Metzger Zahlen etwas abstrakter verwendet werden. So möchte man 500 Gramm Hackfleisch. Das Hackfleisch besteht nicht aus 500 Einzelteilen, sondern die Zahl bezeichnet das Gewicht dessen, was man bestellt.

Diese Zahlen werden in der Mathematik als \textbf{natürliche} Zahlen bezeichnet. Ihre Gesamtheit, d.h. alle natürlichen Zahlen inklusive etwas, das als "`unendlich"' bezeichnet wird -- darauf kommen wir später zurück -- wird mit dem Zeichen $\mathbb{N}$ abgekürzt.

Die folgenden Symbole bezeichnen die ersten neun natürlichen Zahlen. 

\[ 
1, 2, 3, 4, 5, 6, 7, 8, 9
\]

Sie bilden den Grundstock aller folgenden Zahlen, die aus diesen zusammengesetzt sind. Wie diese zusammen gesetzt werden, erfahren wir gleich.

\section{Operationen}

\subsection{Gleichheit, Ungleichheit, Vergleiche}

Mit dem Symbol "`="' wird die Anforderung beschrieben, dass alles, was auf der linken Seite des Symbols steht den selben Wert hat wie das, was auf der rechten Seite des Symbols steht. Folgendes ist demnach korrekt:

\begin{eqnarray*}
1 &=& 1 \\
5 &=& 5 \\
7 &=& 7
\end{eqnarray*}
Während das folgende falsch ist:
\begin{eqnarray*}
1 &=& 7 \\
5 &=& 3 \\
7 &=& 1
\end{eqnarray*}

Es liegt in der Verantwortung desjenigen, der die Gleichung aufstellt, dafür zu garantieren, dass die Gleichheit auch wirklich erfüllt ist. Papier ist geduldig! Man kann hinschreiben, was man will, ob ein hingeschriebenes Gleichheitszeichen auch wirklich Gleichheit bedeutet, kann nur der Hinschreibende wissen.

Möchte man ausdrücken, dass die linke und rechte Seite nicht übereinstimmt, so verwendet man das Symbol "`$\neq $"'. So werden oben angegebene Falschaussagen wieder korrekt indem man das Ungleichzeichen verwendet:
\begin{eqnarray*}
1 &\neq & 7 \\
5 &\neq & 3 \\
7 &\neq & 1
\end{eqnarray*}

Der Ungleichheit stehen noch qualifizierende Ungleichzeichen zur Seite. Dass 1 nicht gleich 7 ist stimmt zwar, ist aber weniger interessant, als die Aussage, dass 1 kleiner als 7 ist. Wenn Hans einen Apfel besitzt und Peter zwei, dann hat Hans weniger Äpfel als Peter, wie Peter mehr Äpfel hat als Hans. Die drückt man durch die folgenden Symbole aus:

\begin{eqnarray*}
1 & < & 7 \\
5 & > & 3 \\
7 & > & 1
\end{eqnarray*}

Hat man eine Ungleichung aufzustellen, bei der es akzeptabel ist, dass beide Seite auch den gleichen Wert haben können, so verwendet man die selben Symbole mit einem Unterstrich:
\begin{eqnarray*}
1 & \le & 7 \\
5 & \ge & 3 \\
7 & \ge & 1
\end{eqnarray*}

Das Symbol "`$\le$"' wird "`kleiner oder gleich"' ausgesprochen und das Symbol "`$\ge$"' "`größer oder gleich"'. Es ist zu beachten, dass folgende Ungleichungen ebenfalls \textbf{alle} korrekt sind:
\begin{eqnarray*}
4 & \le & 5 \\
5 & \le & 5 \\
6 & \ge & 5 \\
5 & \ge & 5
\end{eqnarray*}

Wenn für beiden Seiten einer Ungleichung sowohl das $\le$ als auch das $\ge$ korrekt ist, so gilt $=$ Gleichheit. Dies sollte im Kopf behalten werden, denn es gibt einige Beweise in der Mathematik, die genau dies ausnutzen.


\subsection{Addition und Multiplikation}

Um mit natürlichen Zahlen rechnen zu können, definieren wir zwei Operationen. Die Multiplikation, dargestellt durch das Zeichen "`$\cdot$"' und die Addition, dargestellt durch das Zeichen "`$+$"'.

Mit der Addition fügen wir einzelne Mengen zu größeren Mengen zusammen. Hat Hans $2$ Äpfel und Peter $3$ Äpfel, so haben sie zusammen $5$ Äpfel, oder einfacher formuliert:
\[
2+3=5
\]

\noindent Hätte Peter keinen Apfel
\[2+?=2\]
dann wäre eine Zahl hinzuzuzählen, die keinen Wert besitzt. Eine solche Zahl wird mit dem Symbol "`0"' bezeichnet. Sie ist eine Invariante bezüglich der Addition -- d.h. Additionen mit dieser Zahl ändern nicht den Wert der ursprünglichen Zahl. Man nennt sie "`Null"'. Demnach erweitern sich die Symbole der ersten zehn natürlichen Zahlen auf diese Weise:

\[0,1,2,3,4,5,6,7,8,9\]

Die Null ist per Definition nicht Teil der natürlichen Zahlen. Das hat historische Gründe -- so dachte man im Mittelalter, dass die Zahl Null vom Teufel erdacht worden sei\footnote{Historischer Beleg?}. Für unsere Betrachtung sollte sie aber Teil der natürlichen Zahlen sein. Daher werden wir von nun an mit einer Zahlenmenge umgehen, die mit $\mathbb{N}_0$ bezeichnet wird, sie besteht aus allen natürlichen Zahlen, inklusive der 0.

\textsl{Was kommt nach der 9?}

Die Zahl, die nach der 9 kommt, ist definiert durch die Summe $9+1$. Es wurde in frühen Zeiten einmal festgelegt\footnote{Historischer Beleg?}, dass wir ein Zahlensystem bestehend aus zehn Ziffern verwenden. Die zehn Ziffern beinhalten die Null, daher ist für die zehn kein eigenes Symbol mehr übrig. Daher wurden die zusammengesetzten Zahlen erfunden. So ist die Symbolfolge
\[10\]
diejenige Zahl, die nach der 9 kommt, also
\[9+1=10\]
Das ist eine Definition die bedeutet, dass wenn einem die Symbole ausgehen, kann man eine zusätzliche Stelle verwenden um diese darzustellen. Zusätzliche -- im Sinne von höherwertigen -- Stellen werden links an die Zahl angehängt. Das funktioniert auch mit drei, vier und allen weiteren Stellen:
\begin{eqnarray*}
99+1 &=& 100 \\
999+1 &=& 1000 \\
\cdots
\end{eqnarray*}

\noindent \textsl{Wie werden aber dann Zahlen zusammengesetzt?}

\noindent Sehen wir uns ein Beispiel an:
\[9+5 = ?\]
Wir wissen, dass 
\[9+1 = 10\]
ist, und dass
\[1+4 = 5\]
So können wir schreiben:
\[9+1+4 = 10 + 4 = 14\]
An dieser Stelle haben wir eine Eigenschaft verwendet, die das \textbf{Assoziativgesetz} genannt wird. Wir haben 5 durch die Summe von zwei Zahlen ersetzt und die gesamte Summe anders kombiniert. Sehen wir uns das mit Klammern an:
\[9+5 = 9+(1+4) = (9+1)+4 = 10+4 = 14\]
Klammern bevorzugen eine Operation. So ist
\[9+(1+4)=9+5\]
dass wir die 1 von $1+4$ wegnehmen und zu $9+1$ hinzufügen dürfen ist in der Anschauung vollkommen klar, ob nun Hans 9 Äpfel und Peter 5 Äpfel besitzen, oder Hans 10 und Peter 4, macht für die Gesamtanzahl keinen Unterschied. Aber dieser Operation liegt das obengenannte Assoziativgesetz zugrunde, das wir im restlichen Buch immer wieder verwenden werden. Dass wir es verwenden dürfen, liegt an der "`Harmlosigkeit"' der natürlichen Zahlen $\mathbb{N}$. Wir werden später noch Konstrukte kennen lernen, die das Assoziativ- und im Besonderen das Kommutativgesetz nicht erfüllen. Trotzdem kann man mit diesen Konstrukten genauso rechnen, wie wir das mit den natürlichen Zahlen tun, man darf mit ihnen nur nicht alles machen, was man mit Zahlen tun kann.

Die nächste Operation, die wir kennenlernen, ist die Multiplikation:
\[3\cdot 4 = ?\]
Diese entspricht ebenfalls der Anschauung. Lägen in drei Körben jeweils vier Äpfel, so hätte man insgesamt
\[\underbrace{4+4+4}_{3 \, \mathrm{mal}} = 12\]
Äpfel. Also sind 
\[3\cdot 4 = 12\]
Das Ergebnis bliebe gleich, wenn man vier Körbe mit jeweils drei Äpfeln hätte:
\[3\cdot 4 = 4\cdot 3 = 12\]
Das ist das \textbf{Kommutativgesetz}.

Gehen wir jetzt mal davon aus, wir hätten drei Körbe mit jeweils 2 grünen und 4 roten Äpfeln. Wie viele grüne Äpfel, wie viele rote und wie viele Äpfel insgesamt hätten wir dann?

Die einzelnen Summen können wir leicht bestimmen: 
\[3\cdot 2 = 6\]
grüne Äpfel,
\[3\cdot 4 = 12 \]
rote Äpfel und somit
\[6+12 = 18\]
Äpfel insgesamt. Dabei haben wir zunächst nur die Anzahl der grünen, dann die der roten Äpfel berechnet. Wir haben also folgendes gemacht:
\[3\cdot \underbrace{(2+4)}_{\mathrm{Inhalt\ eines\ Korbes}} = 3\cdot 2 + 3\cdot 4 = 18\]
Zur Berechnung der Gesamtsumme hätten wir aber auch gleich die roten und grünen Äpfel pro Korb summieren können:
\[3\cdot (2+4) = 3\cdot (6) = 3\cdot 6 = 18 \]

Dass wir zuerst die Anzahlen aller grünen Äpfel und dann die aller roten Äpfel berechnen konnten, wird als  \textbf{Distributivgesetz} bezeichnet.

\subsection{Subtraktion und Division}

Wenn Hans drei Äpfel besitzt und zwei davon isst, bleibt ihm nur einer übrig. Die Frage, die sich der Lernende stellen sollte ist:

\textsl{Welche Zahl erfüllt folgende Gleichung?}

\[3 + ? = 1\]







Der Grund warum Subtraktion und Division ein eigenes Unter-Kapitel bilden liegt darin begründet, dass sie nicht das Kommutativgesetz beachten. 

\subsection{Bruchrechnung}

\subsection{Potenzrechnung}

