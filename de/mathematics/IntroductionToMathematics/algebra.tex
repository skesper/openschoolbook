\chapter{Grundlagen}

Die Algebra konzentriert sich -- im Gegensatz zur Arithmetik -- auf die Verallgemeinerung der Begriffe der Analyse bestimmter Sachverhalte. Der Arithmetik Teil dieses Buches begann mit Berechnungen auf Basis von Äpfeln. Diese geben einen unmittelbaren Zugang zu den Begriffen des Rechnens. In der Algebra wird es solches nicht geben. Die Ansätze hier sind rein abstrakt zu verstehen. Auch wenn direkte, anschauliche Beispiele zu bestimmten algebraischen Sachverhalten existieren, so sollte der Lernende versuchen, nicht anhand dieser sein Verständnis auszubilden, sondern rein in der Sache selbst. Das führt letztlich dazu, dass er vollständig ohne konkrete Anschauung Sachverhalte analysieren und Probleme lösen kann. Vollständig unabhängig davon, ob diese Probleme einen realen, abstrakten oder mit dem Verstand nicht nachvollziehbaren Hintergrund haben. Wenn z.B. im Raum der Polynome zwei Funktionen die Bedingung erfüllen
\[f^2(x) + g^2(x) = 1 \]
so stellt dies genauso den Satz des Pytagoras dar, als wenn ein Dreieck betrachtet würde. 
\[a^2 +b^2 = c^2\]

\section{Zeichen}

In der Algebra verwendet man anstelle von Zahlen im allgemeinen Buchstaben. Es immer vom Kontext abhängig, welcher Buchstabe was bedeutet. Aber es haben sich bestimmte Dinge eingebürgert. So sind Konstanten meistens mit den Buchstaben
\[a, b, c, \dots \]
bezeichnet. Unbekannte in Gleichungen meist mit 
\[x, y, z, p, q, \dots \]
Sowie Indizes mit 
\[i, j, k, \dots \]
Es können aber auch griechische Buchstaben auftauchen. Hier eine Übersicht

\bigskip

\begin{tabular}{c|c|l}
\hline
\textbf{Kleiner Buchstabe} & \textbf{Großbuchstabe} & \textbf{Bezeichnung} \\
\hline
$\alpha $ & $A $ & Alpha \\
$\beta $ & $B $ & Beta \\
$\gamma $ & $\Gamma $ & Gamma \\
$\delta $ & $\Delta $ & Delta \\
$\epsilon $ & $E $ & Epsilon \\
$\zeta $ & $Z $ & Zeta \\
$\eta $ & $H $ &  Eta\\
$\theta $ & $\Theta $ & Theta \\
$\iota $ & $I $ & Iota \\
$\kappa $ & $K $ & Kappa \\
$\lambda $ & $\Lambda $ & Lambda \\
$\mu $ & $M $ & Mu \\
$\nu $ & $N $ & Nu \\
$\xi $ & $\Xi $ &  Xi \\
$\omicron $ & $O $ & Omicron \\
$\pi $ & $\Pi $ & Pi \\
$\rho $ & $P $ & Rho \\
$\sigma $ & $\Sigma $ & Sigma  \\
$\tau $ & $T $ & Tau \\
$\upsilon $ & $\Upsilon $ & Ypsilon \\
$\phi $ & $\Phi $ & Phi \\
$\chi $ & $X $ & Chi \\
$\psi $ & $\Psi $ & Psi \\
$\omega $ & $\Omega $ & Omega \\
\hline
\end{tabular}

\section{Gesetze}

Da diese im Arithmetikteil nicht abstrakt definiert wurden, seien sie hier noch einmal wiederholt:

\bigskip

\noindent \textsl{Kommutativgesetz}
\begin{eqnarray*}
a+b &=& b+a \\
a\cdot b &=& b\cdot a
\end{eqnarray*}

\noindent \textsl{Assoziativgesetz}
\begin{eqnarray*}
a+(b+c) &=& (a+b)+c \\
a\cdot (b\cdot c) &=& (a\cdot b)\cdot c
\end{eqnarray*}

\noindent \textsl{Distributivgesetz}
\begin{eqnarray*}
a\cdot (b+c) &=& a\cdot b+ a\cdot c
\end{eqnarray*}


\chapter{Mengen, Gruppen, Ringe, Körper}

Kategorisierung spielt in der Algebra eine sehr wichtige Rolle. Kennt man die Eigenschaften eines Dinges, so kann man damit umgehen.

