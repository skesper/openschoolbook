
\chapter{Schlussbemerkungen}

\section{Beweise}

Wir haben in den diversen Kapiteln des Buches den einen oder anderen Beweis unterschlagen, weil seine Beweisführung an der Stelle, wo er aufgetaucht war, noch zu unverständlich gewesen wäre. Dies soll hier nun nachgeholt werden.


\subsection{Es gibt unendlich viele Primzahlen}\label{chap:proofprime}
Wir hatten in Abschnitt \ref{sec:infty} beweisen wollen, dass es unendlich viele Primzahlen gibt. Der Beweis brauchte noch einige Vorkenntnisse, die wir hier nun voraussetzen können und den Beweis angehen. Zuvor brauchen wir noch ein

\begin{lemma}{(Lemma von Bézout\footnote{\textbf{Étienne Bézout}, französischer Mathematiker. *31. März 1730; \ding{61}27. September 1783 in Basses-Loges})}\index{Lemma von Bézout}
Der größte gemeinsame Teiler $\text{ggT}(a,b) $ zweier ganzer Zahlen $a$ und $b$, von denen mindestens eine ungleich 0 ist, kann als Linearkombination von $a$ und $b$ mit ganzzahligen Koeffizienten $s,t\in \mathbb{Z}$ dargestellt werden. 
\[
\text{ggT}(a,b) = s\cdot a+t\cdot b
\]
Weiter gilt, dass wenn $a$ und $b$ teilerfremd sind, existieren $s,t \in \mathbb{Z}$, sodass
\begin{equation}\label{eq:1}
1 = s\cdot a+t\cdot b
\end{equation}
\end{lemma}
\begin{proof}
Es sei $d$ die kleinste aller Liniearkombinationen von $a$ und $b$, die größer als null ist:
\[
d = \min_x \left\lbrace x = s\cdot a+t\cdot b \mid x>0; s,t\in \mathbb{Z} \right\rbrace
\]
Dafür müssen nur die $s,t\in \mathbb{Z}$ entsprechend gewählt werden. Da $\text{ggT}(a,b)$ sowohl $a$ als auch $b$ teilt, teilt $\text{ggT}(a,b)$ auch $d$. Wäre $d=1$, wäre der Beweis fertig und Gleichung (\ref{eq:1}) gilt. Kümmern wir uns also um den Fall $d>1$. 

Die Division mit Rest liefert uns:
\[
a = q\cdot d+r
\]
wobei $0\le r < d$. Wir setzen für $d$ die Linearkombination ein
\[
a = q\cdot(s\cdot a+t\cdot b)+r
\]
und formen nach $r$ um
\[
r = (1-q\cdot s)\cdot a+(-q\cdot t)\cdot b
\]
Wegen der Minimalität von $d$ muss $r=0$ sein. Also ist $d$ ein Teiler von $a$, also auch von $b$. Daraus folgt, dass $d\le \text{ggT}(a,b)$. Vorher hatten wir gesehen, dass $\text{ggT}(a,b)$ ein Teiler von $d$ ist, also gilt $d=\text{ggT}(a,b)$.

\qed
\end{proof}

Das Lemma von Bézout stellt eine wichtige Erkenntnis in der Mathematik und im Besonderen der Zahlentheorie dar. Wir werden es hier verwenden, um das Lemma von Euklid zu beweisen:

\begin{lemma}{(Lemma von Euklid\footnote{\textbf{Euklid}, griechischer Mathematiker, hat vermutlich im 3. Jahrhundert v. Chr. in Alexandria gelebt.})}\index{Lemma von Euklid}
Teilt eine Primzahl $p$ ein Produkt $ab$, so auch einen (oder beide) der Faktoren.
\end{lemma}
\begin{proof}
Es seien $a,b \in \mathbb{Z}$ beliebig und $p$ eine Primzahl. Würde die Primzahl $p$ das Produkt $ab$ teilen, aber nicht den Faktor $a$, dann ist zu zeigen, dass $p$ ein Teiler von $b$ ist. Aus der Annahme folgt, dass $a$ und $p$ teilerfremd sind. Mit dem Lemma von Bézout existieren dann zwei ganze Zahlen, sodass $1=sp+ta$ gilt. Multipliziert man diese Gleichung mit $b$ erhält man
\[
p(sb)+(ab)t=b
\]
Da $p$ ein Teiler von $ab$ ist, existiert ein $c\in \mathbb{Z}$ mit $ab=cp$. Also ist
\[
p(sb+ct) = b
\]
Damit ist $p$ ein Teiler von $b$.

\qed
\end{proof}

\begin{lemma}\label{lem:prim}
Jede natürliche Zahl $>1$ hat eine eindeutige Zerlegung in Primfaktoren.
\end{lemma}
\begin{proof}
Der Beweis geschieht in zwei Schritten. Zunächst wird die Existenz der Zerlegung bewiesen und danach ihre Eindeutigkeit.

\noindent\textsl{Schritt 1:}
Für jede Primzahl $p$ ist die Behauptung trivialerweise richtig, denn sie ist ihre eigene Primfaktorzerlegung. 
Wir nehmen an, dass es Zahlen gibt, die sich nicht in Primfaktoren zerlegen lassen. Es gibt eine kleinste solche Zahl $n$. Da $n>1$ und keine Primzahl ist, gibt es Teiler von $n=ab$. Für beide gilt $1<a,b<n$. Da $n$ die kleinste Zahl war, die keine Primfaktorzerlegung hat, gibt es Primfaktorzerlegungen für $a$ und $b$. Das bedeutet, $a = \Pi p_i$ und $b=\Pi q_j$ mit $p_i,q_j$ Primzahlen. Damit ist aber auch $\Pi p_i \cdot \Pi q_j$ eine Primfaktorzerlegung von $n$, was der Behauptung widerspricht, dass es Zahlen gäbe, die sich nicht in Primfaktoren zerlegen lassen.

\noindent\textsl{Schritt 2:}
Wir nehmen an, es gibt Zahlen mit unterschiedlichen Primfaktorzerlegungen. Es sei $n=p_1 p_2 \dots p_n = q_1 q_2 \dots q_m$ die kleinste dieser Zahlen mit zwei Zerlegungen in Primfaktoren. Die $p_i$ und $q_j$ müssen alle verschieden sein, denn gäbe es ein $p_k=q_l$, so teilte diese Zahl $n$ und $n/p_k$ hätte wiederum zwei unterschiedliche Zerlegungen, was gegen die Annahme verstößt, dass $n$ die kleinste solche Zahl ist.
Es sei 
\[
n= p_i\cdot \prod_{k\ne i} p_k= q_j\cdot \prod_{k\ne j} q_k
\]
Das Lemma von Euklid besagt nun, dass wenn eine Primzahl ein Produkt teilt ($p_i$ teilt $n$ und somit auch $q_j\cdot \prod_{k\ne j} q_k$), dann auch einen seiner Faktoren. $q_j$ kann nicht durch $p_i$ geteilt werden, da sonst $q_j=p_i$ wäre, entgegen der Annahme. Also teilt $p_i$ den Rest $\prod_{k\ne j} q_k$. Das würde aber bedeuten, dass $p_i$ im $\prod_{k\ne j} q_k$ Produkt enthalten wäre, und das widerspricht wieder der Annahme, dass die Primfaktoren alle unterschiedlich sind. Daraus folgt, dass es keine unterschiedlichen Zerlegungen gibt.

Mit den Schritten 1 und 2 ist nun bewiesen, dass die Zerlegung einer natürlichen Zahl $>1$ in Primfaktoren existiert und eindeutig ist. 

\qed
\end{proof}

Nach diesen ganzen Vorbereitungen kommen wir nun endlich zu dem, was wir eigentlich beweisen wollten, nämlich:

\begin{theorem}
Es gibt unendlich viele Primzahlen.
\end{theorem}
\begin{proof}
Der Beweis wird durch einen Widerspruch geführt. Wir behaupten, dass es nur endlich viele Primzahlen gibt. Wenn dem so wäre, dann gäbe es eine größte Primzahl $N$. Sei

\[ M = 2\cdot 3\cdot 5\dots \cdot N +1 \]
eine Zahl gebildet aus dem Produkt aller Primzahlen (es gibt ja nur endlich viele) addiert mit 1.

Wir wissen, dass $M$ eine eindeutige Zerlegung in Primfaktoren besitzt, nach Lemma \ref{lem:prim}. Da aber $M$ bereits aus dem Produkt aller bekannten Primzahlen erzeugt wurde, und somit nicht durch eine davon teilbar ist (weil M um 1 größer ist, als das Produkt aller Primzahlen), bleiben nur zwei Möglichkeiten: 

\begin{description}
\item[a)] $M$ ist selbst eine Primzahl, es ist aber $M>N$, was gegen die Voraussetzung verstößt, dass $N$ die größte Primzahl ist. Oder
\item[b)] die Zerlegung von $M$ in Primfaktoren enthält eine Primzahl, die größer ist, als $N$, was wiederum gegen die Voraussetzung verstößt.
\end{description}

Also ist die Behauptung, dass es nur endlich viele Primzahlen gibt, falsch und somit stimmt die eigentliche Aussage, die wir beweisen wollten, nämlich dass es unendlich viele Primzahlen gibt. 

\qed
\end{proof}

\section{Nachreichung}

\subsection{Reelle Zahlen}\label{chap:realfinal}

In Arithmetik Teil, Kapitel \ref{chap:realbegin}, wurden die reellen Zahlen nicht wirklich definiert, was hier nachgeholt werden soll. 

TODO



\section{Historische Anmerkung}

Während eines Mathematik-Studiums trifft man immer wieder auf eine Reihe von Namen in verschiedenen Zusammenhängen. Nach einer gewissen Anzahl von Nennungen wird man ganz automatisch interessiert und versucht etwas über die Person hinter einem benannten Theorem oder Lemma herauszufinden. Das was mich am meisten an den Lebensgeschichten beeindruckte war, dass viele dieser Mathematiker sich mit komplizierten mathematischen Sachverhalten zu Zeiten beschäftigten, in denen geschichtliche oder politische Umwälzungen im Gang waren, sowie Weltanschauungen vertreten wurden, die heute so abenteuerlich erscheinen, dass sie (hoffentlich) niemals wieder Eingang in unsere Gesellschaft finden. Jedoch die mathematischen Erkenntnisse aus diesen Zeiten nahezu unverändert heute noch Bestand haben. 

Ganz besonders gilt dies für die Mathematiker jüdischer Abstammung, die mit Hitlers Machtübernahme am 30. Januar 1933 und den darauf folgenden Repressionen gezwungen waren, Deutschland zu verlassen, Hab und Gut zurück zu lassen und ihre wissenschaftliche Karriere abzubrechen. In den meisten Fällen lag einer Suspendierung oder Entlassung das, am 7. April 1933 erlassene, "`Gesetz zur Wiederherstellung des Berufsbeamtentums"' zugrunde. Ein Gesetz das unter anderem zum Ziel hatte, die rassenpolitischen Ideale der NSDAP zu verwirklichen. 

Im ehrenvollen Gedenken an diese Menschen sei hier eine -- gewiss nicht vollständige -- Aufzählung von Mathematikern jüdischer Abstammung dargelegt, deren Arbeit von den Nazis beendet wurde, und deren Schicksal in Einzelfällen auch sehr tragisch endete. Für die nicht aufgelisteten entschuldige ich mich im Vorfeld und bitte, mir diese zur Kenntnis zu bringen, dass ich sie hier mit aufnehmen kann.


\begin{description}
\item[\textbf{Felix Bernstein}] *14. Februar 1878 in Halle (Saale); \ding{61}3. Dezember 1956 in Zürich. Studierte in Göttingen bei Felix Klein und David Hilbert, bei dem er über die Mengenlehre promovierte. 1934 emigrierte er in die USA um dem Nazi Regime zu entkommen.
\item[\textbf{Ludwig Otto Blumenthal}] *20. Juli 1876 in Frankfurt am Main; \ding{61}12. November 1944 im Konzentrationslager Theresienstadt. Blumenthal war der erste Doktorand von David Hilbert. Im April 1943 wurde das Ehepaar Blumenthal ins KZ Herzogenbusch deportiert. Im Durchgangslager Westerbork verstarb seine Frau Mali, Mai 1943. Schließlich wurde er ins KZ Theresienstadt deportiert, wo er dann im November 1943 an einer Lungenentzündung starb.
\item[\textbf{Max Born}] *11. Dezember 1882 in Breslau; \ding{61}5. Januar 1970 in Göttingen. Born wurde aufgrund des Berufsbeamtengesetztes 1933 zwangsbeurlaubt und 1936 entzog man ihm die deutsche Staatsbürgerschaft. Er emigrierte nach England, wo er bis 1953 blieb. Er kehrte dann nach Deutschland zurück. 1954 erhielt er den Nobelpreis für Physik aufgrund seiner Beiträge zur Quantenmechanik.
\item[\textbf{Richard Courant}] *8. Januar 1888 in Lublinitz; \ding{61}27. Januar 1972 in New York. Nach der Machtergreifung verließ er im Sommer 1933 Deutschland. Nach einem Jahr in Cambridge ging er schließlich nach New York. Dort baute er das Institut für Angewandte Mathematik auf, welches zu seinen Ehren seit 1964 "`Courant Institute for Mathematical Sciences"' heißt.
\item[\textbf{Max Wilhelm Dehn}] *13. November 1878 in Hamburg; \ding{61}27. Juni 1952 in Black Mountain, North Carolina. 1935 verlor er seine Stelle in Frankfurt und verließ 1939 Deutschland zunächst in Richtung Kopenhagen, später nach Trondheim. Schließlich ging er in die USA, wo er aufgrund der Vielzahl von emigrierten Wissenschaftlern Schwierigkeiten hatte eine Stelle zu bekommen. Letztlich wurde er vom Black Mountain College eingestellt, wo er der einzige Mathematiker war. 
\item[\textbf{Adolf Abraham Halevi Fraenkel}] *17. Februar 1891 in München; \ding{61}15. Oktober 1965 in Jerusalem. Im Oktober 1933 schaffte er es mit seiner Familie nach Jerusalem auszuwandern, wo er 1938 Rektor der Universität wurde. 
\item[\textbf{Felix Hausdorff}] *8. November 1868 in Breslau; \ding{61}26. Januar 1942 in Bonn. Trotz seiner im letzten Moment noch geglückten Emeritierung 1935, war Hausdorff im weiteren Anfeindungen ausgesetzt. Er versuchte über Courant in die USA zu emigrieren, doch dies schlug fehl. 1942 wurde Hausdorff mit seiner Frau Charlotte und deren Schwester Edith Pappenheim in ein Kloster in Bonn-Endenich deportiert, als Vorbereitung zur Überführung in ein KZ. Dort nahmen sie sich gemeinsam am 26. Januar 1942 das Leben.
\item[\textbf{Ernst David Hellinger}] *30. September 1883 in Striegau; \ding{61}28. März 1950 in Chicago. 1936 von den Nationalsozialisten in den Zwangsruhestand versetzt, weigerte sich Hellinger bis 1939 aus Deutschland zu fliehen. 13. November 1938 wurde er ins Konzentrationslager Dachau deportiert. Aufgrund der Fürsprache von Freunden und unter der Bedingung, dass er emigrieren würde, entließ man Hellinger aus dem KZ. Im Februar 1939 reiste er in die USA und blieb dort bis zu seinem Tode.
\item[\textbf{Edmund Landau}] *14. Februar 1877 in Berlin; \ding{61}19. Februar 1938 in Berlin. 1934 wurde Landau in den Zwangsruhestand versetzt und konnte bis zu seinem Tod nur noch sporadisch in Brüssel und Cambridge lehren. 
\item[\textbf{Emmy Noether}] *23. März 1882 in Erlangen; \ding{61}14. April 1935 in Bryn Mawr, Pennsylvania. Emmy Noether behauptete sich als eine der ganz wenigen Frauen in der Mathematik. Ihr wurde 1917 die Habilitation verweigert aufgrund eines Erlasses, der dies für Frauen unzulässig machte. David Hilbert nahm sie als Assistentin, für den sie Vorlesungen unter dessen Namen hielt. Ab 1933 wurde ihr aber die Lehrerlaubnis entzogen, sie schaffte es glücklicherweise in die USA, wo sie für kurze Zeit am Women's College Bryn Mawr lehrte. Sie verstarb allerdings 1935 an den Folgen einer Operation.
\item[\textbf{Hugo Dionizy Steinhaus}] *4. Januar 1887 in Jas\l o; \ding{61}25. Februar 1972 in Breslau. Steinhaus und seine Frau lebten ab Juli 1941 im Untergrund, so entgingen sie einer Deportation. Sie schafften es, sich bis 1945 durchzuschlagen und zogen dann nach Breslau. Dort konnte er unter seinem Namen wieder arbeiten und nahm eine Professur an der Universität Breslau an.
\item[\textbf{Otto Toeplitz}] *1. August 1881 in Breslau; \ding{61}15. Februar 1940 in Jerusalem. Toeplitz konnte noch bis 1935 in Bonn lehren. Er war unter anderem mit Hausdorff befreundet. Anfang 1939 wurde der Druck auf ihn so groß, dass er nach Palästina emigrierte. Dort verstarb er nur ein Jahr später. 

\end{description}



