%\preface

\chapter{Vorwort}

Dies ist ein sogenanntes "`offenes"' Schulbuch, das bedeutet, dass dieses Buch kostenlos jedem ohne Vorbehalt zur Verfügung gestellt wird. Es enthält Wissen, das von Freiwilligen ohne finanzielle Interessen zusammengetragen wurde. 

Es wird unter einer Commons-Creative-Lizenz veröffentlicht, wie unten angegeben. Lehrer sind herzlich willkommen Inhalte dieses Buches im Unterricht zu verwenden, auch in Teilen und Auszugsweise. Verlage können das Buch als solches Drucken und in gedruckter Form vertreiben. Hierfür ist eine Ausnahme Lizenz zu beantragen. 

\bigskip

Die Mathematik stellt zusammen mit der Sprache, eine Basis allen Wissens dar. Jedoch ist die Sprache grundlegender in dem Sinne, das sie Voraussetzung für die Erklärungen zum Verständnis der Mathematik ist. Folgerichtig ist dieses Buch erst dann zu verwenden, wenn der Lernende bereits ein gewisses Grundverständnis von Sprache besitzt. 

Mathematik ist eine der wenigen Wissenschaften, deren Elemente vollständig aus dem Geist von Menschen entstanden sind. Es gibt kein "`natürliches"' Vorbild für die mathematischen Elemente. Daher ist sie eine der Wissenschaften, deren Inhalte zumindest von einem Menschen bereits verstanden worden sein müssen, nämlich jenem, der die Inhalte aufschrieb und gegebenenfalls bewies.

Das bedeutet, dass die Mathematik vollständig verstehbar ist, sofern der Lernende genug Zeit und Willen aufbringt, sich mit ihr auseinander zusetzen. Diese Erkenntnis steht in direktem Widerspruch zur landläufigen Meinung, dass Mathematik schwer verständlich und unzugänglich ist. Warum dies so ist, wird von Fall zu Fall unterschiedlich sein und es gibt wohl kaum eine allgemeine Begründung. Doch viele Menschen lassen sich speziell von den Formeln abschrecken. So kompliziert diese Formeln auch im Einzelnen aussehen möchten, so sind sie doch unabdingbar in dem Sinne, dass sie vollkommen unzweideutig Sachverhalte darstellen können. Und zwar auf einem Niveau dem natürliche Sprachen weit hinterher hinken. 

Die mathematischen Formalismen\footnote{Im Sinne von Formel und nicht von einer Form-gerechten Vorgehensweise.} werden in Schulen meist auf eine Weise den Schülern beigebracht, dass sie sich davon abgeschreckt fühlen. Auf der anderen Seite sind mathematische Kenntnisse in vielen Bereichen des täglichen Lebens durchaus wichtig, und sei es nur um die Zahlenspielereien der Banken und Versicherungen auf Kredit- und Versicherungsverträgen zu durchschauen. 

Dieses Buch versucht in erster Linie Wissen zu vermitteln. Die Didaktik, wie es vermittelt werden soll, bleibt dabei -- schon aufgrund fehlender didaktischer Fachkenntnis -- auf der Strecke. Es ist also im Sinne einer Referenz zu verstehen, sodass sehr viel tieferes Wissen zur Verfügung steht, als dies in Schulen benötigt wird. Nichts desto trotz erscheint es dem Autor aber sinnvoll das Buch auf diese Weise zu schreiben, da es dann interessierten Lernenden die Möglichkeit gibt, sich auch weit über das Schulwissen hinaus zu bilden. In der Hoffnung, dass es zumindest für einige Menschen von Nutzen sein wird.


\vspace{\baselineskip}
\begin{flushright}\noindent
Koblenz, \today \hfill \textit{Stephan Kesper}
\end{flushright}

\vfill

\noindent Dieser Inhalt ist unter der Creative-Commons-Lizenz vom Typ Namensnennung - Nicht-kommerziell - Weitergabe unter gleichen Bedingungen 3.0 Unported lizenziert. Um eine Kopie dieser Lizenz einzusehen, besuchen Sie

\bigskip
\begin{center}
\texttt{http://creativecommons.org/licenses/by-nc-sa/3.0/}
\end{center}

\bigskip

\noindent oder schreiben Sie einen Brief an Creative Commons, 444 Castro Street, Suite 900, Mountain View, California, 94041, USA.


