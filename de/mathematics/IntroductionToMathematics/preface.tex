\preface

Dies ist ein sogenanntes "`offenes"' Schulbuch, das bedeutet, dass dieses Buch kostenlos jedem ohne Vorbehalt zur Verfügung gestellt wird. Es enthält Wissen, dass von Freiwilligen ohne finanzielle Interessen zusammengetragen wurde. 

Es wird unter einer Commons-Creative-Lizenz veröffentlicht, wie unten angegeben. Lehrer sind herzlich willkommen Inhalte dieses Buches im Unterricht zu verwenden, auch in Teilen und Auszugsweise. Verlage können das Buch als solches Drucken und in gedruckter Form vertreiben, solange die Creative-Commons-Lizenz dadurch nicht beeinträchtigt wird.

Die Mathematik stellt zusammen mit der Sprache, eine Basis allen Wissens dar. Jedoch ist die Sprache grundlegender in dem Sinne, das sie Voraussetzung für die Erklärungen zum Verständnis der Mathematik ist. Folgerichtig ist dieses Buch erst dann zu verwenden, wenn der Lernende bereits ein gewisses Grundverständnis von Sprache besitzt. 


\vspace{\baselineskip}
\begin{flushright}\noindent
Koblenz, \today \hfill {\it Stephan Kesper}
\end{flushright}

\vfill

\noindent Dieser Inhalt ist unter der Creative-Commons-Lizenz vom Typ Namensnennung - Nicht-kommerziell - Weitergabe unter gleichen Bedingungen 3.0 Unported lizenziert. Um eine Kopie dieser Lizenz einzusehen, besuchen Sie

\bigskip
\begin{center}
\texttt{http://creativecommons.org/licenses/by-nc-sa/3.0/}
\end{center}

\bigskip

\noindent oder schreiben Sie einen Brief an Creative Commons, 444 Castro Street, Suite 900, Mountain View, California, 94041, USA.


